\documentclass[twoside, openright]{book}

%glossary and indexing stuff
\usepackage[toc, nopostdot, numberedsection]{glossaries}
\usepackage{datatool}
%\newglossary*{catstatements}{Chapter \ref{chap:catstatements} Key Terms}
%\newglossary*{catsyllogisms}{Key Terms}
\renewcommand{\glsnamefont}[1]{\makefirstuc{#1}}
\makeglossaries

%General packages
\usepackage{answers}
\usepackage{textcomp}
\usepackage{anyfontsize}	
\usepackage{geometry}
\usepackage{url}
\usepackage{changepage}
\usepackage{syntonly}
\usepackage{enumitem}
\usepackage{turnstile}
\usepackage{float}
\usepackage[normalem]{ulem}
\usepackage{fixltx2e}	
\usepackage{wasysym}
\usepackage{tocloft}
\usepackage{fancyhdr}
\usepackage{fancyref}
\usepackage{etoolbox}
\usepackage[utf8]{inputenc}
%\usepackage{amsthm} for some reason, this conflicts with the fitch.sty part of openlogic.sty. 

%%% Bibstuff
\usepackage[authordate,autocite=inline,backend=biber, natbib]{biblatex-chicago}
\bibliography{tex/z-openlogic}
% To typeset the bibliography, you need to run "biber --output-safechars openintroduction" from the command line. You can't use the function within TeXworks, because it doen't have the --output-safechars flag. Without that flag, biber is unable to handle many characters used for Sanskrit words, like the n with a dot under it.




%%%  graphics packages %%%
\usepackage{tikz}
\usetikzlibrary{shapes,backgrounds,matrix,arrows,decorations,positioning,arrows.new}
\usepackage{graphicx}
\usepackage{xcolor}


%%%    Table and figure packages %%%
\usepackage[singlelinecheck=false, skip=0pt]{caption} %left aligns captions for tables, moves them closer to table. 
\usepackage{tabularx}
\usepackage{longtable}
\usepackage{tabu}
\usepackage[framemethod=1]{mdframed} 
\usepackage{wrapfig} %I used this to put a frame around tables.
\tabulinesep=.75ex
\usepackage{colortbl}
\floatstyle{plain} 
\restylefloat{figure}
\usepackage[export]{adjustbox}
\usepackage{multirow}
\usepackage{openintroduction}
\usepackage{rotating}
\usepackage{booktabs}
	

%linking and bookmarks in the pdf.
\usepackage{hyperref}
\hypersetup{pdftex,colorlinks=true,allcolors=blue}
\usepackage{hypcap}	
	
	
\pdfinfo{
  /Title (An Open Introduction to Logic)
  /Author (J. Robert Loftis, Cathal Woods, Robert C Robinson, and P.D. Magnus)
  /Subject (An open access introductory textbook in logic and critical thinking)
  )
}

\begin{document}

%\label{showanswers} %uncomment this tag and typeset twice to show answers
%\label{blank_prob_set} %uncomment this tag and typeset twice to create a blank problem set sheet. Don't use with \label{showanswers} uncommented

\raggedright
\setlength{\parindent}{1em}
\setlength{\parskip}{1em}	

\frontmatter
\pagestyle{plain} %Says there are no running heads, only page numbers centered at the bottom. 
\include{tex/01-cover}
\include{tex/02-frontmatter}	


{
\setlength{\parskip}{0em}
\cftpagenumbersoff{part}
%\cftpagenumbersoff{chapter}

\renewcommand{\cftpartpresnum}{\sf\Large\partname\ }
\tableofcontents
}


\setlength{\parindent}{1em}
\chapter{About this Book}

This book was created by combining two previous books on logic and critical thinking, both made available under a Creative Commons license, and then adding some material so that the
coverage matched that of commonly used logic textbooks.

P.D. Magnus' \textit{For All} X \parencite*{Magnus2008}, 

Cathal Woods' \textit{Introduction to Reasoning}. 
\cite{Woods2014}, 

And J Robert Loftis' ``Loraine County Remix". %https://github.com/rob-helpy-chalk/openintroduction

All subsequent changes have been tracked there at Github: \url{https://github.com/robinson-philo/openintroduction}



 \begin{adjustwidth}{2em}{0em}
 Robert C Robinson \\
\noindent \emph{NY, NY, USA}
\end{adjustwidth}

	
\include{tex/04-acknowledgements}



\mainmatter
\setlength{\parindent}{1em}
\pagestyle{headings} % puts the running heads back.
\label{full_version} %Include this label to make cross references work right when typesetting full text


\part{Basic Concepts} \label{part:basic_concepts}
\chapter{What Is Logic?}
\label{Chap:what_is_logic}
\markright{Ch. \ref{Chap:what_is_logic}: What Is Logic?}

% *****************************
% *		Introduction                      *
% ****************************

\section{Introduction}

Logic is a part of the study of human reason, the ability we have to think abstractly, solve problems, explain the things that we know, and infer new knowledge on the basis of evidence. Traditionally, logic has focused on the last of these items, the ability to make inferences on the basis of evidence. This is an activity you engage in every day. Consider, for instance, the game of Clue. (For those of you who have never played, Clue is a murder mystery game where players have to decide who committed the murder, what weapon they used, and where they were.) A player in the game might decide that the murder weapon was the candlestick by ruling out the other weapons in the game: the knife, the revolver, the rope, the lead pipe, and the wrench. This evidence lets the player know something they did not know previously, namely, the identity of the murderer.

 In logic, we use the word ``argument'' to refer to the attempt to show that certain evidence supports a conclusion. This is very different from the sort of argument you might have when you are mad at someone, which could involve screaming and throwing things. We are going to use the word ``argument'' a lot in this book, so you need to get used to thinking of it as a name for a rational process, and not a word that describes what happens when people disagree.

A logical argument is structured to give someone a reason to believe some conclusion. Here is the argument about a game of Clue written out in a way that shows its structure. 


\label{argClue}
\begin{earg}
\item[P$_1$:] In a game of Clue, the possible murder weapons are the knife, the candlestick, the revolver, the rope, the lead pipe, and the wrench.
\item[P$_2$:] The murder weapon was not the knife.
\item[P$_3$:] The murder weapon was also not the revolver, the rope, the lead pipe, or the wrench.
\vspace{-.5em}
\item [] \rule{0.9\linewidth}{.5pt} 
\item[C:] Therefore, the murder weapon was the candlestick.
\end{earg} 

In the argument above, statements P$_1$--P$_3$ are the evidence. We call these the \emph{premises}. The word ``therefore'' indicates that the final statement, marked with a C, is the \emph{conclusion} of the argument. If you believe the premises, then the argument provides you with a reason to believe the conclusion. You might use reasoning like this purely in your own head, without talking with anyone else. You might wonder what the murder weapon is, and then mentally rule out each item, leaving only the candlestick. On the other hand, you might use reasoning like this while talking to someone else, to convince them that the murder weapon is the candlestick. (Perhaps you are playing as a team.) Either way the structure of the reasoning is the same. 

\newglossaryentry{logic}
{
name=logic,
description={the part of the study of reasoning that focuses on argument.}
}

We can define \textsc{\Gls{logic}}\label{def:logic} then more precisely as the part of the study of reasoning that focuses on argument. In more casual situations, we will follow ordinary practice and use the word ``logic'' to either refer to the business of studying human reason or the thing being studied, that is, human reasoning itself. While logic focuses on argument, other disciplines, like decision theory and cognitive science, deal with other aspects of human reasoning, like abstract thinking and problem solving more generally. Logic, as the study of argument, has been pursued for thousands of years by people from civilizations all over the globe. The initial motivation for studying logic is generally practical. Given that we use arguments and make inferences all the time, it only makes sense that we would want to learn to do these things better.  Once people begin to study logic, however, they quickly realize that it is a fascinating topic in its own right. Thus the study of logic quickly moves from being a practical business to a theoretical endeavor people pursue for its own sake. 

\newglossaryentry{metareasoning}
{
name=metareasoning,
description={Using reasoning to study reasoning. See also \emph{metacognition}.}
}

\newglossaryentry{metacognition}
{
name=metacognition,
description={Thought processes that are applied to other thought processes See also \emph{metareasoning}.}
}

In order to study reasoning, we have to apply our ability to reason to our reason itself. This reasoning about reasoning is called \textsc{\gls{metareasoning}}\label{def:Metareasoning}. It is part of a more general set of processes called \textsc{\gls{metacognition}}\label{def:Metacognition}, which is just any kind of thinking about thinking. When we are pursing logic as a practical discipline, one important part of metacognition will be awareness of your own thinking, especially its weakness and biases, as it is occurring. More theoretical metacognition will be about attempting to understand the structure of thought itself. 


\newglossaryentry{content neutrality}
{
name=content neutrality,
description={the feature of the study of logic that makes it indifferent to the topic being argued about. If a method of argument is considered rational in one domain, it should be considered rational in any other domain, all other things being equal.}
}

Whether we are pursuing logical for practical or theoretical reasons, our focus is on argument. The key to studying argument is to set aside the subject being argued about and to focus on the \emph{way} it is argued \emph{for}. The section opened with an example that was about a game of Clue. However, the kind of reasoning used in that example was just the process of elimination. Process of elimination can be applied to any subject. Suppose a group of friends is deciding which restaurant to eat at, and there are six restaurants in town. If you could rule out five of the possibilities, you would use an argument just like the one above to decide where to eat. Because logic sets aside what an argument is about, and just looks at how it works rationally, logic is said to have \textsc{\gls{content neutrality}}. \label{def:content_neutrality} If we say an argument is good, then the same kind of argument applied to a different topic will also be good.  If we say an argument is good for solving murders, we will also say that the same kind of argument is good for deciding where to eat, what kind of disease is destroying your crops, or who to vote for. 

\newglossaryentry{formal logic}
{
name=formal logic,
description={A way of studying logic that achieves content neutrality by replacing parts of the arguments being studied with abstract symbols. Often this will involve the construction of full formal languages.}
}


When logic is studied for theoretical reasons, it typically is pursued as \textsc{\gls{formal logic}}. \label{def:Formal_logic} In formal logic we get content neutrality by replacing parts of the argument we are studying with abstract symbols. For instance, we could turn the argument above into a formal argument like this:

\label{argClueformal}
\begin{earg}
\item[P$_1$:] There are six possibilities: A, B, C, D, E, and F.
\item[P$_2$:] A is false.
\item[P$_3$:] B, D, E, and F are also false.
\vspace{-.5em}
\item [] \rule{0.6\linewidth}{.5pt} 
\item[C:]  $\therefore$ The correct answer is C.
\end{earg} 

Here we have replaced the concrete possibilities in the first argument with abstract letters that could stand for anything. We have also replaced the English 
word ``therefore'' with the symbol ``$\therefore$,'' which means therefore. This lets us see the formal structure of the argument, which is why it works in 
any domain you can think of. In fact, we can think of formal logic as the method for studying argument that uses abstract notation to identify the formal 
structure of argument.  Formal logic is closely allied with mathematics, and studying formal logic often has the sort of puzzle-solving 
character one associates with mathematics. \iflabelexists{full_version}{You will see this when we get to 
Parts \ref{part:cat_logic} and \ref{part:sent_logic}, which covers formal logic.}
	{\iflabelexists{part:CT}{}
	{\iflabelexists{part:cat_logic}{You will see this when we get to Parts \ref{part:cat_logic} and \ref{part:sent_logic}, which cover formal logic.}
{\iflabelexists{part:quant_logic}{You will see this when we get to Parts \ref{part:sent_logic} and \ref{part:quant_logic}, which cover formal logic.}}	
	{}	}}
 
\newglossaryentry{critical thinking}
{
name=critical thinking,
description={The use of metareasoning to improve our reasoning in practical situations. Sometimes the term is also used to refer to the results of this effort at self improvement, that is, reasoning in practical situations that has been sharpened by reflection and metareasoning.}
}

\newglossaryentry{critical thinker}
{
name=critical thinker,
description={A person who has both sharpened their reasoning abilities using metareasoning and deploys those sharpened abilities in real world situations..}
}

\newglossaryentry{informal logic}
{
name=informal logic,
description={The study of arguments given in ordinary language.}
}


When logic is studied for practical reasons, it is typically called critical thinking. We will define \textsc{\gls{critical thinking}}\label{def:Critical_Thinking}  narrowly as the use of metareasoning to improve our reasoning in practical situations.  Sometimes we will use the term ``critical thinking'' more broadly to refer to the results of this effort at self-improvement.  You are ``thinking critically'' when you reason in a way that has been sharpened by reflection and metareasoning. A \textsc{\gls{critical thinker}}\label{def:critical_thinker} someone who has both sharpened their reasoning abilities using metareasoning and deploys those sharpened abilities in real world situations.

Critical thinking is generally pursued as \textsc{\gls{informal logic}}, rather than formal logic. This means that we will keep arguments in ordinary language and draw extensively on your knowledge of the world to evaluate them. In contrast to the clarity and rigor of formal logic, informal logic is suffused with ambiguity and vagueness. There are problems  with multiple correct answers, or where reasonable people can disagree with what the correct answer is. This is because you will be dealing with reasoning in the real world, which is messy. \label{messiness_warning} \label{ver_var} \iflabelexists{part:CT}{You will learn more about this in the chapters on critical thinking Part \ref{part:CT}}{}
  

You can think of the difference between formal logic and informal logic as the difference between a laboratory science and a field science. \label{lab_vs._field_science} If you are studying, say, mice, you could discover things about them by running experiments in a lab, or you can go out into the field where mice live and observe them in their natural habitat.  Informal logic is the field science for arguments: you go out and study arguments in their natural habitats, like newspapers, courtrooms, and scientific journal articles. Like studying mice scurrying around a meadow, the process takes patience, and often doesn't yield clear answers but it lets you see how things work in the real world. Formal logic takes arguments out of their natural habitat and performs experiments on them to see what they are capable of. The arguments here are like lab mice. They are pumped full of chemicals and asked to perform strange tasks, as it were. They live lives very different than their wild cousins. Some of the arguments will wind up looking like the ``ob/ob mouse'', a genetically engineered obese mouse scientists use to study type II diabetes (See Figure \ref{fig:ob_ob_mouse}). These arguments will be huge, awkward, and completely unable to survive in the wild. But they will tell us a lot about the limits of logic as a process.

\begin{figure}
\begin{mdframed}[style=mytableclearbox]
\begin{center}
\includegraphics*[scale=.8]{img/Fatmouse}
\end{center}
\end{mdframed}
\caption{The ob/ob mouse (left), a laboratory mouse which has been genetically engineered to be obese, and an ordinary mouse (right). Formal logic, which takes arguments out of their natural environment, often winds up studying arguments that look like the ob/ob mouse. They are huge, awkward, and unable to survive in the wild, but they tell us a lot about the limits of logic as a process. Photo from \cite{WikimediaCommons2006}.}
\label{fig:ob_ob_mouse}
\end{figure}


\newglossaryentry{rhetoric}
{
name=rhetoric,
description={The study of effective persuasion.}
}


Our main goal in studying arguments is to separate the good ones from the bad ones. The argument about Clue we saw earlier is a good one, based on the process of elimination.  It is good because it leads to truth. If I've got all the premises right, the conclusion will also be right. The textbook \textit{Logic: Techniques of Formal Reasoning} \citep{Kalish1980} had a nice way of capturing the meaning of logic: ``logic is the study of virtue in argument.'' \label{virtue_in_argument} This textbook will accept this definition, with the caveat that an argument is virtuous if it helps us get to the truth.

Logic is different from \textsc{\gls{rhetoric}}, which is the study of effective persuasion. Rhetoric does not look at virtue in argument. It only looks at the power of arguments, regardless of whether they lead to truth. An advertisement might convince you to buy a new truck by having a gravelly voiced announcer tell you it is ``ram tough'' and showing you a picture of the truck on top of a mountain, where it no doubt actually had to be airlifted. This sort of persuasion is often more effective at getting people to believe things than logical argument, but it has nothing to do with whether the truck is really the right thing to buy. In this textbook we will only be interested in rhetoric to the extent that we need to learn to defend ourselves against the misleading rhetoric of others. \iflabelexists{part:CT}{The sections of this text on critical thinking will emphasize becoming aware of our biases and how others might use misleading rhetoric to exploit them. }{}This will not, however, be anything close to a full treatment of the study of rhetoric.


% ******************************************
% *		Statement, Argument, Premise, Conclusion  *
% ******************************************

\section{Statement, Argument, Premise, Conclusion}
\label{sec:SAPC}

\newglossaryentry{statement}
{
name=statement,
description={A unit of language that can be true or false.}
}

So far we have defined logic as the study of argument and outlined its relationship to related fields. To go any further, we are going to need a more precise definition of what exactly an argument is. We have said that an argument is not simply two people disagreeing; it is an attempt to prove something using evidence. More specifically, an argument is composed of statements. In logic, we define a \textsc{\gls{statement}} \label{def:statement} as a unit of language that can be true or false. To put it another way, it is some combination of words or symbols that have been put together in a way that lets someone agree or disagree with it. All of the items below are statements.

\begin{enumerate}[label=(\alph*)]
\item \label{itm:t.rex_true}\emph{Tyrannosaurus rex} went extinct 65 million years ago. 
\item \label{itm:t.rex_false}\emph{Tyrannosaurus rex} went extinct last week.
\item \label{itm:t.rex_unknown}On this exact spot, 100  million years ago, a \emph{T. rex} laid a clutch of eggs. 
\item \label{itm:silly}George W. Bush is the king of Jupiter. 
\item \label{itm:moral}Murder is wrong. 
\item \label{itm:opinion1}Abortion is murder. 
\item \label{itm:opinion2}Abortion is a woman's right. 
\item \label{itm:opinion3}Lady Gaga is pretty.
\item \label{itm:definition}Murder is the unjustified killing of a person.
\item \label{itm:nonsense}The slithy toves did gyre and gimble in the wabe.
\item \label{itm:history}The murder of logician Richard Montague was never solved. 
\end{enumerate}

Because a statement is something that can be true \emph{or} false, statements include truths like \ref{itm:t.rex_true} and falsehoods like \ref{itm:t.rex_false}. A statement can also be something that that must either be true or false, but we don't know which, like \ref{itm:t.rex_unknown}. A statement can be something that is completely silly, like \ref{itm:silly}. Statements in logic include statements about morality, like \ref{itm:moral}, and things that in other contexts might be called ``opinions,'' like \ref{itm:opinion1} and \ref{itm:opinion2}. People disagree strongly about whether \ref{itm:opinion1} or \ref{itm:opinion2} are true, but it is definitely possible for one of them to be true. The same is true about \ref{itm:opinion3}, although it is a less important issue than \ref{itm:opinion1} and \ref{itm:opinion2}. A statement in logic can also simply give a definition, like \ref{itm:definition}. This sort of statement announces that we plan to use words a certain way, which is different from statements that describe the world, like \ref{itm:t.rex_true}, or statements about morality, like \ref{itm:opinion1}. Statements can include nonsense words like \ref{itm:nonsense}, because we don't really need to know what the statement is about to see that it is the sort of thing that can be true or false. All of this relates back to the content neutrality of logic. The statements we study can be about dinosaurs, abortion, Lady Gaga, and even the history of logic itself, as in statement \ref{itm:history}, which is true.

We are treating statements primarily as units of language or strings of symbols, and most of the time the statements you will be working with will just be words printed on a page. However, it is important to remember that statements are also what philosophers call ``speech acts.'' They are actions people take when they speak (or write). If someone makes a statement they are typically telling other people that they believe the statement to be true, and will back it up with evidence if asked to. When people make statements, they always do it in a context---they make statements at a place and a time with an audience. Often the context statements are made in will be important for us, so when we give examples, statements, or arguments we will sometimes include a description of the context. When we do that, we will give the context in \textit{italics.} See Figure \ref{fig:statements_and_context} for examples. \label{context_marker} For the most part, the context for a statement or argument will be important in the chapters on critical thinking, when we are pursing the study of logic for practical reasons. In the chapters on formal logic, context is less important, and we will be more likely to skip it. 

\begin{figure}
\begin{mdframed}[style=mytableclearbox]
\includegraphics*[width=\linewidth]{img/statement_and_contexts}
\end{mdframed}
\caption{A statement in different contexts, or no context.} \label{fig:statements_and_context}
\end{figure}


``Statements' in this text does \emph{not} include questions, commands, exclamations, or sentence fragments. Someone who asks a \emph{question} like ``Does the grass need to be mowed?'' is typically not claiming that anything is true or false. Generally, \emph{questions} will not count as statements, but \emph{answers} will. ``What is this course about?'' is not a statement. ``No one knows what this course is about,'' is a statement.

For the same reason \emph{commands} do not count as statements for us. If someone bellows ``Mow the grass, now!'' they are not saying whether the grass has been mowed or not. You might infer that they believe the lawn has not been mowed, but then again maybe they think the lawn is fine and just want to see you exercise. 

An exclamation like ``Ouch!'' is also neither true nor false. On its own, it is not a statement. We will treat ``Ouch, I hurt my toe!'' as meaning the same thing as ``I hurt my toe.'' The ``ouch'' does not add anything that could be true or false.

Finally, a lot of possible strings of words will fail to qualify as statements simply because they don't form a complete sentence. In your composition classes, these were probably referred to as sentence fragments. This includes strings of words that are parts of sentences, such as noun phrases like ``The tall man with the hat'' and verb phrases, like ``ran down the hall.'' Phrases like these are missing something they need to make a claim about the world. The class of sentence fragments also includes completely random combinations of words, like ``The up if blender route,'' which don't even have the form of a statement about the world.  

Other logic textbooks describe the components of argument as ``propositions,'' or ``assertions,'' and we will use these terms sometimes as well.  There is actually a great deal of disagreement about what the differences between all of these things are and which term is best used to describe parts of arguments. However, none of that makes a difference for this textbook. We could have used any of the other terms in this text, and it wouldn't change anything. Some textbooks will also use the term ``sentence'' here. We will not use the word ``sentence'' to mean the same thing as ``statement.'' Instead, we will use ``sentence'' the way it is used in ordinary grammar, to refer generally to statements, questions, and commands. 

Sometimes the outward form of a speech act does not match how it is actually being used. A rhetorical question, for instance, has the outward form of a question, but is really a statement or a command. If someone says ``don't you think the lawn needs to be mowed?'' they may actually mean a statement like ``the lawn needs to be mowed'' or a command like ``mow the lawn, now.'' Similarly one might disguise a command as a statement. ``You will respect my authority'' \emph{is} either true or false---either you will or you will not. But the speaker may intend this as an order---''Respect me!''---rather than a prediction of how you will behave.

When we study argument, we need to express things as statements, because arguments are composed of statements. Thus if we encounter a rhetorical question while examining an argument, we need to convert it into a statement. ``Don't you think the lawn needs to be mowed'' will become ``the lawn needs to be mowed.'' Similarly, commands will become should statements. ``Mow the lawn, now!'' will need to be transformed into ``You should mow the lawn.'' 

\newglossaryentry{practical argument}
{
name=practical argument,
description={An argument whose conclusion is a statement that someone should do something.}
}

The latter kind of change will be important in critical thinking, because critical thinking often studies arguments whose goal is to an get audience to do something. These are called \textsc{\glspl{practical argument}}\label{def:practical_argument}. Most advertising and political speech consists of practical arguments, and these are crucial topics for critical thinking.

\newglossaryentry{argument}
{
name=argument,
description={a connected series of statements designed to convince an audience of another statement.}
}

\newglossaryentry{premise}
{
name=premise,
description={a statement in an argument that provides evidence for the conclusion}
}

\newglossaryentry{conclusion}
{
name=conclusion,
description={the statement that an argument is trying to convince an audience of.}
}

 
Once we have a collection of statements, we can use them to build arguments. An \textsc{\gls{argument}} \label{def:Argument} is a connected series of statements designed to convince an audience of another statement. Here an audience might be a literal audience sitting in front of you at some public speaking engagement. Or it might be the readers of a book or article. The audience might even be yourself as you reason your way through a problem. Let's start with an example of an argument given to an external audience. This passage is from an essay by Peter Singer called ``Famine, Affluence, and Morality'' in which he tries to convince people in rich nations that they need to do more to help people in poor nations who are experiencing famine.

\begin{quotation}\noindent \textit{A contemporary philosopher writing in an academic journal} If it is in our power to prevent something bad from happening, without thereby sacrificing anything of comparable moral importance, we ought, morally, to do so. Famine is something bad, and it can be prevented without sacrificing anything of comparable moral importance. So, we ought to prevent famine. \citep{Singer1972} \label{singer_quote} \end{quotation} 

Singer wants his readers to work to prevent famine. This is represented by the last statement of the passage, ``we ought to prevent famine,'' which is called the conclusion of the passage. The \textsc{\gls{conclusion}} \label{def:conclusion} of an argument is the statement that the argument is trying to convince the audience of. The statements that do the convincing are called the \textsc{\glspl{premise}}. \label{def:premise}In this case, the argument has three premises: (1) ``If it is in our power to prevent something bad from happening, without thereby sacrificing anything of comparable moral importance, we ought, morally, to do so''; (2) ``Famine is something bad''; and (3) ``it can be prevented without sacrificing anything of comparable moral importance.''

Now let's look at an example of internal reasoning. 

\begin{quotation}\noindent\textit{Jack arrives at the track, in bad weather.} There is no one here. I guess the race is not happening. \label{racetrack}
\end{quotation}

In the passage above, the words in \textit{italics} explain the context for the reasoning, and the words in regular type represent what Jack is actually thinking to himself. \nix{(We will talk more about his way of representing reasoning in section \ref{sec:arguments_and_context}, below.)} This passage again has a premise and a conclusion. The premise is that no one is at the track, and the conclusion is that the race was canceled. The context gives another reason why Jack might believe the race has been canceled, the weather is bad. You could view this as another premise--it is very likely a reason Jack has come to believe that the race is canceled. In general, when you are looking at people's internal reasoning, it is often hard to determine what is actually working as a premise and what is just working in the background of their unconscious. %[We will talk more about this in section...]


\newglossaryentry{premise indicator}
{
name=premise indicator,
description={a word or phrase such as ``because'' used to indicate that what follows is the premise of an argument.}
}

\newglossaryentry{conclusion indicator}
{
name=conclusion indicator,
description={a word or phrase such as ``therefore'' used to indicate that what follows is the conclusion of an argument.}
}

When people give arguments to each other, they typically use words like ``therefore'' and ``because.'' These are meant to signal to the audience that what is coming is either a premise or a conclusion in an argument. Words and phrases like ``because'' signal that a premise is coming, so we call these \textsc{\glspl{premise indicator}}. Similarly, words and phrases like ``therefore'' signal a conclusion and are called \textsc{\glspl{conclusion indicator}}. The argument from Peter Singer (on page \pageref{singer_quote}) uses the conclusion indicator word, ``so.'' Table \ref{table:Indicators} is an incomplete list of indicator words and phrases in English.


\begin{table}
\begin{mdframed}[style=mytablebox]

\begin{longtabu}{X[1,p]X[2,p]}
\textbf{Premise Indicators:} & because, as, for, since, given that, for the reason that \\
\textbf{Conclusion Indicators:} & therefore, thus, hence, so, consequently, it follows that, in conclusion, as a result, then, must, accordingly, this implies that, this entails that, we may infer that \\
\end{longtabu}
\end{mdframed}
\caption{Premise and Conclusion Indicators.}
\label{table:Indicators}
\end{table}

\newglossaryentry{canonical form}
{
name=canonical form,
description={a method for representing arguments where each premise is written on a separate, numbered, line, followed by a horizontal bar and then the conclusion. Statements in the argument might be paraphrased for brevity and indicator words are removed.}
}


The two passages we have looked at in this section so far have been simply presented as quotations. But often it is extremely useful to rewrite arguments in a way that makes their logical structure clear. One way to do this is to use something called ``canonical form.''   An argument written in \textsc{\gls{canonical form}} \label{def:canonical_form}has each premise numbered and written on a separate line. Indicator words and other unnecessary material should be removed from the premises. Although you can shorten the premises and conclusion, you need to be sure to keep them all complete sentences with the same meaning, so that they can be true or false. The argument from Peter Singer, above, looks like this in canonical form:

\begin{earg}
\item[P$_1$:] If we can stop something bad from happening, without sacrificing anything of comparable moral importance, we ought to do so. 
\item[P$_2$:] Famine is something bad.
\item[P$_3$:] Famine can be prevented without sacrificing anything of comparable moral importance.
\vspace{-.5em}
\item [] \rule{0.9\linewidth}{.5pt} 
\item[C:] We ought to prevent famine.
\end{earg} 

Each statement has been written on its own line and given a number. The statements have been paraphrased slightly, for brevity, and the indicator word ``so'' has been removed. Also notice that the ``it'' in the third premise has been replaced by the word ``famine,'' so that statements reads naturally on its own.  

Similarly, we can rewrite the argument Jack gives at the racetrack, on page \pageref{racetrack}, like this:

\begin{earg}
\item[P:] There is no one at the race track.
\vspace{-.5em}
\item [] \rule{0.4\linewidth}{.5pt} 
\item[C:] The race is not happening. 
\end{earg} 

Notice that we did not include anything from the part of the passage in italics. The italics represent the context, not the argument itself. Also, notice that the ``I guess'' has been removed. When we write things out in canonical form, we write the content of the statements, ignore information about the speaker's mental state, like ``I believe'' or ``I guess.'' 

One of the first things you have to learn to do in logic is to identify arguments and rewrite them in canonical form. This is a foundational skill for everything else we will be doing in this text, so we are going to run through a few examples now, and there will be more in the exercises. The passage below is paraphrased from the ancient Greek philosopher Aristotle. 

\begin{quotation}\noindent \textit{An ancient philosopher, writing for his students} Again, our observations of the stars make it evident that the earth is round. For quite a small change of position to south or north causes a manifest alteration in the stars which are overhead. (\cite{Aristotle:heavens}, 298a2-10)
\label{on_the_heavens} \end{quotation}

The first thing we need to do to put this argument in canonical form is to identify the conclusion. The indicator words are the best way to do this. The phrase ``make it evident that'' is a conclusion indicator phrase. He is saying that everything else is \textit{evidence} for what follows. So we know that the conclusion is that the earth is round. ``For'' is a premise indicator word---it is sort of a weaker version of ``because.''  Thus the premise is that the stars in the sky change if you move north or south. In canonical form, Aristotle's argument that the earth is round looks like this.\\


\begin{earg}
\item[P:] There are different stars overhead in the northern and southern parts of the earth.
\vspace{-.5em}
\item [] \rule{0.9\linewidth}{.5pt} 
\item[C:] The earth is spherical in shape. 
\end{earg} 

That one is fairly simple, because it just has one premise. Here's another example of an argument, this time from the book of Ecclesiastes in the Bible. The speaker in this part of the bible is generally referred to as The Preacher, or in Hebrew, Koheleth. In this verse, Koheleth uses both a premise indicator and a conclusion indicator to let you know he is giving reasons for enjoying life.

\begin{quotation}
\noindent \textit{The words of the Preacher, son of David, King of Jerusalem} There is something else meaningless that occurs on earth: the righteous who get what the wicked deserve, and the wicked who get what the righteous deserve. \ldots So I commend the enjoyment of life, because there is nothing better for a person under the sun than to eat and drink and be glad. (Ecclesiastes 8:14-15, New International Version)
\end{quotation}

Koheleth begins by pointing out that good things happen to bad people and bad things happen to good people. This is his first premise. (Most Bible teachers provide some context here by pointing that that the ways of God are mysterious and this is an important theme in Ecclesiastes.) Then Koheleth gives his conclusion, that we should enjoy life, which he marks with the word ``so.'' Finally he gives an extra premise, marked with a ``because,'' that there is nothing better for a person than to eat and drink and be glad. In canonical form, the argument would look like this.


\begin{earg}
\item[P$_1$:] Good things happen to bad people and bad things happen to good people.
\item[P$_2$:] There is nothing better for people than to eat, to drink and to enjoy life.
\vspace{-.5em}
\item [] \rule{0.8\linewidth}{.5pt} 
\item[C:] You should enjoy life.
\end{earg} 

Notice that in the original passages, Aristotle put the conclusion in the first sentence, while Koheleth put it in the middle of the passage, between two premises. In ordinary English, people can put the conclusion of their argument where ever they want. However, when we write the argument in canonical form, the conclusion goes last.

Unfortunately, indicator words aren't a perfect guide to when people are giving an argument. Look at this passage from a newspaper:

\begin{quotation}
\noindent \textit{From the general news section of a national newspaper} The new budget underscores the consistent and paramount importance of tax cuts in the Bush philosophy. His first term cuts affected more money than any other initiative undertaken in his presidency, including the costs thus far of the war in Iraq. All told, including tax incentives for health care programs and the extension of other tax breaks that are likely to be taken up by Congress, the White House budget calls for nearly \$300 billion in tax cuts over the next five years, and \$1.5 trillion over the next 10 years.  \citep{Toner2006}
\end{quotation}

Although there are no indicator words, this is in fact an argument. The writer wants you to believe something about George Bush: tax cuts are his number one priority. The next two sentences in the paragraph give you reasons to believe this. You can write the argument in canonical form like this.

\begin{earg}
\item[P$_1$:] Bush's first term cuts affected more money than any other initiative undertaken in his presidency, including the costs thus far of the war in Iraq. 
\item[P$_2$:] The White House budget calls for nearly \$300 billion in tax cuts over the next five years, and \$1.5 trillion over the next 10 years. 
\vspace{-.5em}
\item [] \rule{0.9\linewidth}{.5pt} 
\item[C:] Tax cuts are of consistent and paramount importance of in the Bush philosophy.
\end{earg} 

The ultimate test of whether something is an argument is simply whether some of the statements provide reason to believe another one of the statements. If some statements support others, you are looking at an argument. The speakers in these two cases use indicator phrases to let you know they are trying to give an argument.

\newglossaryentry{inference}
{
name=inference,
description={the act of coming to believe a conclusion on the basis of some set of premises.}
}

A final bit of terminology for this section. An \textsc{\gls{inference}} \label{def:Inference} is the act of coming to believe a conclusion on the basis of some set of premises. When Jack in the example above saw that no one was at the track, and came to believe that the race was not on, he was making an inference. We also use the term inference to refer to the connection between the premises and the conclusion of an argument. If your mind moves from premises to conclusion, you make an inference, and the premises and the conclusion are said to be linked by an inference. In that way inferences are like argument glue: they hold the premises and conclusion together. 

%%%% Practice Problems


\practiceproblems
Throughout the book, you will find a series of practice problems that review and explore the material covered in the chapter. There is no substitute for actually working through some problems, because logic is more about a way of thinking than it is about memorizing facts. %The answers to some of the problems are provided at the end of the book in Appendix \ref{app.solutions}; the problems that are solved in the appendix are marked with a star (\solutions.)

\noindent\problempart Decide whether the following passages are statements in the logical sense and give reasons for your answers.

\begin{longtabu}{p{.1\linewidth}p{.9\linewidth}}
\textbf{Example}: & Did you follow the instructions? \\
\textbf{Answer}: & Not a statement, a question. \\
\end{longtabu}


\begin{exercises}
\item England is smaller than China. \answerblank{\underline{Statement}}{\vspace{.25in}}
\item Greenland is south of Jerusalem. \answerblank{\underline{Statement}}{\vspace{.25in}}
\item Is New Jersey east of Wisconsin? \answerblank{\underline{A question, not a Statement.}}{\vspace{.25in}}
\item The atomic number of helium is 2. \answerblank{\underline{Statement}}{\vspace{.25in}}
\item The atomic number of helium is $\pi$. \answerblank{\underline{Statement}}{\vspace{.25in}}
\item I hate overcooked noodles. \answerblank{\underline{Statement}}{\vspace{.25in}}
\item Blech! Overcooked noodles! \answerblank{\underline{An exclamation, not a statement.}}{\vspace{.25in}}
\item Overcooked noodles are disgusting.\answerblank{\underline{Statement}}{\vspace{.25in}}
\item Take your time. \answerblank{\underline{A command, not a Statement}}{\vspace{.25in}}
\item This is the last question. \answerblank{\underline{Statement}}{\vspace{.25in}}
\end{exercises}


\noindent\problempart Decide whether the following passages are statements in the logical sense and give reasons for your answers.
\answer{Answers from Ben Sheredos.}
\begin{exercises}
\item Is this a question? \answer{\underline{Question, not a statement.}}
\item Nineteen out of the 20 known species of Eurasian elephants are extinct. \answer{\underline{Statement; has to be true or false (might be false bc 20 is the wrong number, or because they are not extinct, etc.)}}
\item The government of the United Kingdom has formally apologized for the way it treated the logician Alan Turing. \answer{\underline{ Statement: has to be true or false; they either have or have not apologized}} 

\item Texting while driving \answer{\underline{Not a statement, but a sentence fragment}}
\item Texting while driving is dangerous. \answer{\underline{Statement; has to be true or false.}}
\item Insanity ran in the family of logician Bertrand Russell, and he had a life-long fear of going mad. \answer{\underline{Complex, but a statement: both halves are true or false, so is the whole.}}
\item For the love of Pete, put that thing down before someone gets hurt!  \answer{\underline{Not a statement: First bit is an exclamation, second is a command.}}
\item Don't try to make too much sense of this. \answer{\underline{Not a statement, a command.}}
\item Never look a gift horse in the mouth.  \answer{\underline{Not a statement, a command.}}
\item The physical impossibility of death in the mind of someone living  \answer{\underline{ Not a statement, sentence fragment.}}
\end{exercises}

\noindent\problempart Rewrite each of the following arguments in canonical form. Be sure to remove all indicator words and keep the premises and conclusion as complete sentences. Write the indicator words and phrases separately and state whether they are premise or conclusion indicators. 

%NTS: when writing these problems, be sure to include a mix of conclusion-first, conclusion-last and conclusion middle, as well as a mix of arguments with true and false premises and a variety of indicator words (or lack thereof).

\begin{longtabu}{p{.1\linewidth}p{.9\linewidth}}	
\textbf{Example}: & \textit{An ancient philosopher writes} We should not be distressed or concerned by the thought of our our own death in any way. Why? Look back on the time before you were born: It is a time you did not exist, but it does not trouble you in any way. The time after you die is also a time when you will not exist, so it shouldn't trouble you either. (Based on Lucretius \citetitle{Lucretius2001} 3.972--75)\\
\textbf{Answer}: & 
\vspace{-16pt}
\begin{earg}
\item[P$_1$:] The time before you were born is a time you did not exist.
\item[P$_2$:] You are not troubled by the time before you were born. 
\item[P$_3$:] The time after you die is also a time you will not exist.
\vspace{-.5em}
\item [] \rule{0.6\linewidth}{.5pt} 
\item[C:] We should not be distressed or concerned by the thought of our our own death. 
\end{earg} 
Premise indicator: So
\\
\end{longtabu}
	
\begin{exercises}

\item \textit{A detective is speaking: }Henry's finger-prints were found on the stolen computer. So, I infer that Henry stole the computer.  

\answer{
\begin{earg*}
\item Henry's finger-prints were found on the stolen computer
\itemc Henry stole the computer.  
\end{earg*}
Conclusion indicator word: So, I infer that}


\item \textit{Monica is wondering about her co-workers political opinions} You cannot both oppose abortion and support the death penalty, unless you think there is a difference between fetuses and felons. Steve opposes abortion and supports the death penalty. Therefore Steve thinks there is a difference between fetuses and felons. 
		%Conclusion-last

\answer{
\begin{earg*}
\item You cannot both oppose abortion and support the death penalty, unless you think there is a difference between fetuses and felons. 
\item Steve opposes abortion and supports the death penalty. 
\itemc Steve thinks there is a difference between fetuses and felons. 
\end{earg*}
Conclusion Indicator: Therefore}


\item \textit{The Grand Moff of Earth defense considers strategy} We know that whenever people from one planet invade another, they always wind up being killed by the local diseases, because in 1938, when Martians invaded the Earth, they were defeated because they lacked immunity to Earth's diseases. Also, in 1942, when Hitler's forces landed on the Moon, they were killed by Moon diseases.
		%Conclusion-first

\answer{
\begin{earg} 
\item[1.] In 1938, when Martians invaded the Earth, they were defeated because they lacked immunity to Earth's diseases. 
\item[2.] In 1942, when Hitler's forces landed on the Moon, they were killed by Moon diseases.
\item [] \noindent\hrulefill 
\item[$\therefore$] Whenever people from one planet invade another, they always wind up being killed by the local diseases, 
\end{earg}
Premise indicator: Because }


\item If you have slain the Jabberwock, my son, it will be a frabjous day. The Jabberwock lies there dead, its head cleft with your vorpal sword. This is truly a fabjous day. 
%Conclusion-last
\answer{ 
\begin{earg*} 
\item  If you have slain the Jabberwock, my son, it will be a frabjous day. 
\item The Jabberwock lies there dead
 
\itemc This is truly a fabjous day 
\end{earg*}
Indicators: none		
}	

\item \textit{A detective trying to crack a case thinks to herself} Miss Scarlett was jealous that Professor Plum would not leave his wife to be with her. Therefore she must be the killer, because she is the only one with a motive. 
%Conclusion-middle
\answer{
\begin{earg*} 
\item Miss Scarlett was jealous that Professor Plum would not leave his wife to be with her. 
\item Miss Scarlett is the only one with a motive. 
 
\itemc Miss Scarlett must be the killer
\end{earg*}

Premise Indicator: Because \\
Conclusion Indicator: Therefore}
\end{exercises}



\noindent\problempart Rewrite each of the following arguments in canonical form. Be sure to remove all indicator words and keep the premises and conclusion as complete sentences. Write the indicator words and phrases separately and state whether they are premise or conclusion indicators. 

\answer{Answers from Ben Sheredos.}

\begin{enumerate}[label=\arabic*), topsep=0pt, parsep=0pt, itemsep=6pt]
\item \textit{A pundit is speaking on a Sunday political talk show} Hillary Clinton should drop out of the race for Democratic Presidential nominee. For every day she stays in the race, McCain gets a day free from public scrutiny and the members of the Democratic party get to fight one another.  
			%Conclusion-first

\answer{ 
	\begin{earg*} 
		\item For every day Hillary stays in the race, McCain gets a day free from public scrutiny and the members of the Democratic party get to fight one another.
		\itemc Hillary Clinton should drop out of the race for Democratic Presidential Nominee.
	\end{earg*}
	"For" could be a premise-indicator, functioning like "since."
}


\item You have to be smart to understand the rules of Dungeons and Dragons. Most smart people are nerds. So, I bet most people who play D\&D are nerds.  
			%Conclusion-last

\answer{ 
			\begin{earg*} 
				\item You have to be smart to understand the rules of D\&D.
				\item Most smart people are nerds.
				\itemc $[I bet]$ most people who play D\&D are nerds.
			\end{earg*}
			"So" is definitely a conclusion-indicator; "I bet" is probably part of a conclusion-indicator as well, with the speaker indicating that they think this argument is a bit weak.
		}

\item Any time the public receives a tax rebate, consumer spending increases. Since the public just received a tax rebate, consumer spending will increase. 
		%Conclusion-last

\answer{ 
	\begin{earg*} 
		\item Any time the public receives a tax rebate, consumer spending increases. 
		\item The public just received a tax rebate.
		\itemc Consumer spending will increase.
	\end{earg*}
	"Since" is a premise-indicator, but the last sentence needs to be split up into premise and conclusion. This would be more clear if the speaker said "\underline{Since} the public just received a tax rebate, \underline{it follows that} consumer spending will increase." Our speaker is lazy.
}

\item Isabelle is taller than Jacob. Kate must also be taller than Jacob, because she is taller than Isabelle. 
%conclusion-middle

\answer{ 
	\begin{earg*} 
		\item Isabelle is taller than Jacob.
		\item Kate is taller than Isabelle.
		\itemc Kate is taller than Jacob.
	\end{earg*}
	"Must" is a conclusion indicator, "because" is a premise-indicator, and so the last sentence has to be split up to put this argument into canonical form.
}
\end{enumerate}

% * **********************************
% *     Arguments and Nonarguments          *
% ************************************

\section{Arguments and Nonarguments}
\label{sec:arguments_and_nonarguments}

We just saw that arguments are made of statements. However, there are lots of other things you can do with statements. Part of learning what an argument is involves learning what an argument is not, so in this section and the next we are going to look at some other things you can do with statements besides make arguments. 

The list below of kinds of nonarguments is not meant to be exhaustive: there are all sorts of things you can do with statements that are not discussed. Nor are the items on this list meant to be exclusive. One passage may function as both, for instance, a narrative and a statement of belief. Right now we are looking at real world reasoning, so you should expect a lot of ambiguity and imperfection. If your class is continuing on into the critical thinking portions of this textbook, you will quickly get used to this. 

\subsection{Simple Statements of Belief}

\newglossaryentry{simple statement of belief}
{
name=simple statement of belief,
description={A kind of nonargumentative passage where the speaker simply asserts what they believe without giving reasons. }
}

An argument is an attempt to persuade an audience to believe something, using reasons. Often, though, when people try to persuade others to believe something, they skip the reasons, and give a \textsc{\gls{simple statement of belief}}. \label{def:simple_statement_of_belief} This is a kind of nonargumentative passage where the speaker simply asserts what they believe without giving reasons. Sometimes simple statements of belief are prefaced with the words ``I believe,'' and sometimes they are not. A simple statements of belief can be a profoundly inspiring way to change people's hearts and minds. Consider this passage from Dr. Martin Luther King's Nobel acceptance speech.

\begin{quotation} \noindent I believe that even amid today's mortar bursts and whining bullets, there is still hope for a brighter tomorrow. I believe that wounded justice, lying prostrate on the blood-flowing streets of our nations, can be lifted from this dust of shame to reign supreme among the children of men. I have the audacity to believe that peoples everywhere can have three meals a day for their bodies, education and culture for their minds, and dignity, equality and freedom for their spirits. \citep{King2001} \end{quotation}

This actually is a part of a longer passage that consists almost entirely of statements that begin with some variation of ``I believe.''It is incredibly powerful oration, because the audience, feeling the power of King's beliefs, comes to share in those beliefs. The language King uses to describe how he believes is important, too. He says his belief in freedom and equality requires audacity, making the audience feel his courage and want to share in this courage by believing the same things. 

These statements are moving, but they do not form an argument. None of these statements provide evidence for any of the other statements. In fact, they all say roughly the same thing, that good will triumph over evil. So the study of this kind of speech belongs to the discipline of rhetoric, not of logic.  
  
\subsection{Expository Passages}

Perhaps the most basic use of a statement is to convey information. Often if we have a lot of information to convey, we will sometimes organize our statements around a theme or a topic. Information organized in this fashion can often appear like an argument, because all of the statements in the passage relate back to some central statement. However, unless the other statements are given as reasons to believe the central statement, the passage you are looking at is not an argument. Consider this passage:

\begin{quotation}\noindent\textit{From a college psychology textbook.} Eysenck advocated three major behavior techniques that have been used successfully to treat a variety of phobias. These techniques are modeling, flooding, and systematic desensitization. In \textbf{modeling} phobic people watch nonphobics cope successfully with dreaded objects or situations.In \textbf{flooding} clients are exposed to dreaded objects or situations for prolonged periods of time in order to extinguish their fear. In contrast to flooding, \textbf{systematic desensitization} involves gradual, client-controlled exposure to the anxiety eliciting object or situation. (Adapted from Ryckman \cite*{Ryckman2007}) \end{quotation}

\newglossaryentry{expository passage}
{
name=expository passage,
description={A nonargumentative passage that organizes statements around a central theme or topic statement.}
}

We call this kind of passage an expository passage. In an \textsc{\gls{expository passage}}, \label{def:expository_passage} statements are organized around a central theme or topic statement. The topic statement might look like a conclusion, but the other statements are not meant to be evidence for the topic statement. Instead, they elaborate on the topic statement by providing more details or giving examples. In the passage above, the topic statement is ``Eysenck advocated three major behavioral techniques \ldots.'' The statements describing these techniques elaborate on the topic statement, but they are not evidence for it. Although the audience may not have known this fact about Eysenk before reading the passage, they will typically accept the truth of this statement instantly, based on the textbook's authority. Subsequent statements in the passage merely provide detail. 

Deciding whether a passage is an argument or an expository passage is complicated by the fact that sometimes people argue by example: 

\begin{adjustwidth}{2em}{0em}
\begin{longtabu}{p{.1\linewidth}p{.8\linewidth}}
\textbf{Steve:} & Kenyans are better distance runners than everyone else. \\
\textbf{Monica:} & Oh come on, that sounds like an exaggeration of a stereotype that isn't even true.\\
\textbf{Steve:} & What about Dennis Kimetto, the Kenyan who set the world record for running the marathon? And you know who the previous record holder was: Emmanuel Mutai, also Kenyan. \\
\end{longtabu}
\end{adjustwidth}
\vspace{-1.5cm}

Here Steve has made a general statement about all Kenyans. Monica clearly doubts this claim, so Steve backs it up with some examples that seem to match his generalization. This isn't a very strong way to argue: moving from two examples to statement about all Kenyans is probably going to be a kind of bad argument known as a hasty generalization. (This mistake is covered in the complete version of this text in the chapter on induction\nix{Chapter \ref{chap:induction} on induction.}\label{ver_var}) The point here however, is that Steve is just offering it as an argument. 

The key to telling the difference between expository passages and arguments by example is whether there is a conclusion that they audience needs to be convinced of. In the passage from the psychology textbook, ``Eysenck advocated three major behavioral techniques'' doesn't really work as a conclusion for an argument. The audience, students in an introductory psychology course, aren't likely to challenge this assertion, the way Monica  challenges Steve's overgeneralizing claim. 

Context is very important here, too. The Internet is a place where people argue in the ordinary sense of exchanging angry words and insults. In that context, people are likely to actually give some arguments in the logical sense of giving reasons to believe a conclusion. 

\subsection{Narratives} 

Statements can also be organized into descriptions of events and actions, as in this snippet from book V of \textit{Harry Potter}.

\begin{quotation} \noindent But she [Hermione] broke off; the morning post was arriving and, as usual, the \textit{Daily Prophet} was soaring toward her in the beak of a screech owl, which landed perilously close to the sugar bowl and held out a leg. Hermione pushed a Knut into its leather pouch, took the newspaper, and scanned the front page critically as the owl took off again. \citep{Rowling2003} \end{quotation} 

\newglossaryentry{narrative}
{
name=narrative,
description={A nonargumentative passage that describes a sequence of events or actions.}
}

We will use the term \textsc{\gls{narrative}} \label{def:narrative} loosely to refer to any passage that gives a sequence of events or actions. A narrative can be fictional or nonfictional. It can be told in regular temporal sequence or it can jump around, forcing the audience to try to reconstruct a temporal sequence. A narrative can describe a short sequence of actions, like Hermione taking a newspaper from an owl, or a grand sweep of events, like this passage about the  rise and fall of an empire in the ancient near east:

\begin{quotation}\noindent The Guti were finally expelled from Mesopotamia by the Sumerians of Erech (\textit{c}. 2100), but it was left to the kings of Ur's famous third dynasty to re-establish the Sargonoid frontiers and write the final chapter of the Sumerian History. The dynasty lasted through the twenty first century at the close of which the armies of Ur were overthrown by the Elamites and Amorites \citep{McEvedy1967}. \end{quotation} 

This passage does not feature individual people performing specific actions, but it is still united by character and action. Instead of Hermione at breakfast, we have the Sumarians in Mesopotamia. Instead of retrieving a message from an owl, the conquer the Guti, but then are conquered by the Elamites and Amorites. The important thing is that the statements in a narrative are not related as premises and conclusion. Instead, they are all events which are united common characters acting in specific times and places. 

%%%%%%% Practice Problems

\practiceproblems
\problempart Identify each passage below as an argument or a nonargument, and give reasons for your answers. If it is a nonargument, say what kind of nonargument you think it is. If it is an argument, write it out in canonical form.

\begin{longtabu}{p{.1\linewidth}p{.9\linewidth}}
\textbf{Example}: & \textit{One student speaks to another student who has missed class:} The instructor passed out the syllabus at 9:30. Then he went over some basic points about reasoning, arguments and explanations. Then he said we should call it a day. \\
\textbf{Answer}: & Not an argument, because none of the statements provide any support for any of the others. This is probably better classified as a narration because the events are in temporal sequence. \\
\end{longtabu}

\begin{exercises}
\item \textit{An anthropology teacher is speaking to her class }Different gangs use different colors to distinguish themselves. Here are some illustrations: biologists tend to wear some blue, while the philosophy gang wears black. 


\answer{Not an argument. Expository passage. The students probably will believe the teacher as soon as she makes an assertion. The word ``illustration'' is also a clue.}

\item The economy has been in trouble recently. And it's certainly true that cell phone use has been rising during that same period. So, I suspect increasing cell phone use is bad for the economy. 


\answer{Argument. The indicator ``so'' is a clue. 

\begin{earg*}
\item The economy has been in trouble recently. 
\item Cell phone use has been rising during that same period. 
\itemc Cell phone use is bad for the economy. 
\end{earg*}
}


\item \textit{At Widget-World Corporate Headquarters:} We believe that our company must deliver a quality product to our customers. Our customers also expect first-class customer service. At the same time, we must make a profit. 

%\vspace{6pt}
\answer{Not an argument. The speaker is not using any of the propositions as reasons to believe or explain any of the others; rather she is simply asserting various things.}
      
\item \textit{Jack is at the breakfast table and shows no sign of hurrying. Gill says:} You should leave now. It's almost nine a.m. and it takes three hours to get there.

\answer{Arguing. Jack's inaction suggests that he does believe that he needs to leave now and so Gill provides reasons that might convince him. Notice that there are no argument flag words or phrases.

This example also includes the word ``should'' in its conclusion. Words such as ``ought'' and ``should'' indicate that the speaker is trying to get the audience to do or believe something that they are not currently doing or believing.

\begin{earg*}
\item It's almost nine a.m. 
\item It takes three hours to get there.
\itemc  You should leave now.
\end{earg*}
}
      
\item \textit{In a text book on the brain:} Axons are distinguished from dendrites by several features, including shape (dendrites often taper while axons usually maintain a constant radius), length (dendrites are restricted to a small region around the cell body while axons can be much longer), and function (dendrites usually receive signals while axons usually transmit them).

\answer{Not an argument. Expository passage. The features named just fill in the first statement.}

\end{exercises}
%
\problempart Identify each passage below as an argument or a nonargument, and give reasons for your answers. If it is a nonargument, say what kind of nonargument you think it is. If it is an argument, write it out in canonical form.

%
\begin{exercises}
\item \textit{Suzi doesn't believe she can quit smoking. Her friend Brenda says} Some people have been able to give up cigarettes by using their will-power. Everyone can draw on their will-power. So, anyone who wants to give up cigarettes can do so.

\item \textit{The words of the Preacher, son of David, King of Jerusalem} I have seen something else under the sun: The race is not to the swift or the battle to the strong, nor does food come to the wise or wealth to the brilliant or favor to the learned; but time and chance happen to them all. (Ecclesiastes 9:11, New International Version)

\item \textit{An economic development expert is speaking.} The introduction of cooperative marketing into Europe greatly increased the prosperity of the farmers, so we may be confident that a similar system in Africa will greatly increase the prosperity of our farmers.

\item \textit{From the CBS News website, US section.} Headline: ``FBI nabs 5 in alleged plot to blow up Ohio bridge.'' Five alleged anarchists have been arrested after a months-long sting operation, charged with plotting to blow up a bridge in the Cleveland area, the FBI announced Tuesday. CBS News senior correspondent John Miller reports the group had been involved in a series of escalating plots that ended with their arrest last night by FBI agents. The sting operation supplied the anarchists with what they thought were explosives and bomb-making materials. At no time during the case was the public in danger, the FBI said. \citep{CBSNews2012}


\item \textit{At a school board meeting.} Since creationism can be discussed effectively as a scientific model, and since evolutionism is fundamentally a religious philosophy rather than a science, it is unsound educational practice for evolution to be taught and promoted in the public schools to the exclusion or detriment of special creation. (Kitcher \cite*{Kitcher1982}, p. 177, citing Morris \cite*{Morris1975}.)

\end{exercises}

% * **********************************
% *     Arguments and Explanations          *
% ************************************

\section{Arguments and Explanations}
\label{arguments_and_explanations}

Explanations are are not arguments, but they they share important characteristics with arguments, so we should devote a separate section to them. Both explanations and arguments are parts of reasoning, because both feature statements that act as reasons for other statements. The difference is that explanations are not used to convince an audience of a conclusion.  

Let's start with workplace example. Suppose you see your co-worker, Henry, removing a computer from his office. You think to yourself ``Gosh, is he stealing from work?'' But when you ask him about it later, Henry says, ``I took the computer because I believed that it was scheduled for repair.'' Henry's statement looks like an argument. It has the indicator word ``because'' in it, which would mean that the statement ``I believed it was scheduled for repairs'' would be a premise. If it was, we could put the argument in canonical form, like this: 

\begin{earg}
\item[P:] I believed the computer was scheduled for repair
\vspace{-.5em}
\item [] \rule{0.6\linewidth}{.5pt} 
\item[C:] I took the computer from the office. 
\end{earg} 

But this would be awfully weird as an argument. If it were an argument, it would be trying to convince us of the conclusion, that Henry took the computer from the office. But you don't need to be convinced of this. You already know it---that's why you were talking to him in the first place. 
  
Henry is giving reasons here, but they aren't reasons that try to \textit{prove} something. They are reasons that \textit{explain} something. When you explain something with reasons, you increase your understanding of the world by placing something you already know in a new context. You already knew that Henry took the computer, but now you know \textit{why} Henry took the computer, and can see that his action was completely innocent (if his story checks out). 


\newglossaryentry{explanation}
{
name=explanation,
description={A kind of reasoning where reasons are used to provide a greater understanding of something that is already known.}
}

\newglossaryentry{explainer}
{
name=explainer,
description={The part of an explanation that provides greater understanding of the explainee.}
}

\newglossaryentry{explainee}
{
name=explainee,
description={The part of an explanation that one gains a greater understanding of as a result of the explainer.}
}

\newglossaryentry{reason}
{
name=reason,
description={The premise of an argument or the explainer in an explanation; the part of reasoning that provides logical support for the target proposition.}
}

\newglossaryentry{target proposition}
{
name=target proposition,
description={The conclusion of an argument or the explainee in an explanation; the part of reasoning that is logically supported by the reasons.}
}



Both arguments and explanations both involve giving reasons, but the reasons function differently in each case. An \textsc{\gls{explanation}} \label{def:explanation}is defined as a kind of reasoning where reasons are used to provide a greater understanding of something that is already known.  

Because both arguments and explanations are parts of reasoning, we will use parallel language to describe them. In the case of an argument, we called the reasons ``premises.'' In the case of an explanation, we will call them \textsc{\glspl{explainer}}. \label{def:explainer} Instead of a ``conclusion,'' we say that the explanation has an \textsc{\gls{explainee}}.  \label{def:explainee} We can use the generic term \textsc{\glspl{reason}} \label{def:reason} to refer to either premises or explainers and the generic term \textsc{\gls{target proposition}} \label{def:target_proposition} to refer to either conclusions or explainees. Figure \ref{fig:arguments_explanations} shows this relationship. 


\begin{figure}
\begin{mdframed}[style=mytableclearbox, userdefinedwidth=.65\textwidth]
\begin{tikzpicture}

\tikzstyle{mynode} = [rectangle, draw, fill=light-gray, rounded corners=3pt,] 

\path
	(0,0) node [mynode] (premises) {Premises}
	(0,-2) node [mynode] (conclusion) {Conclusion}
	(3,0) node  [mynode] (explainers) {Explainers}
	(3,-2) node [mynode] (explainee) {Explainee}
	(5,0) node  [anchor=west](reasons) {Reasons}
	(5,-2) node [anchor=west](target) {Target Proposition};

\tikzstyle{myblockarrow} = [thick, fill=light-gray,decoration={markings,mark=at position
   1 with {\arrow[semithick]{open triangle 60}}},
   double distance=2pt, shorten >= 5.5pt,
   preaction = {decorate},
   postaction = {draw,line width=2pt, white,shorten >= 4.5pt}]


\draw [myarrow2] (premises) to node [left] {\color{black}Prove} (conclusion);
\draw [myarrow2] (explainers) to node [right] {\color{black}Clarify} (explainee);
\draw [myarrow1, shorten >=.5cm] (reasons) to (explainers);
\draw [myarrow1, shorten >=.5cm] (target) to (explainee);


\end{tikzpicture}
\end{mdframed}
\caption{Arguments vs. Explanations.} \label{fig:arguments_explanations}
\end{figure}

We can put explanations in canonical form, just like arguments, but to distinguish the two, we will simply number the statements, rather than writing Ps and Cs, and we will put an E next to the line that separates explainers and exaplainee, like this:

\begin{tikzpicture}
\path
	(0,0) node [anchor=west] {1. Henry believed the computer was scheduled for repair}
	(9,-8pt) node [anchor=west]{E}
	(0,-20pt) node[anchor=west] {2. Henry took the computer from the office.};
\draw (.5,-8pt) -- (9,-8pt);
\end{tikzpicture}

Cases where the target proposition is something that is completely common sense are clearcut cases of explanation. Consider the following passage.

\begin{quotation}
\noindent\textit{From Livescience, a science education website, under the headline “Why is grass green?”} Like many plants, most species of grass produce a bright pigment called chlorophyll. Chlorophyll absorbs blue light (high energy, short wavelengths) and red light (low energy, longer wavelengths) well, but mostly reflects green light, which accounts for your lawn's color. \citep{Mauk2013}
\end{quotation}

The passage contains reasoning. The nature of chlorophyll ``accounts for'' the color of grass. But in this case the audience does not need to be convinced that grass is green. Everyone knows that. The audience went to the Livescience website because they wanted an \emph{explanation} for why grass was green. 

Often the same piece of reasoning can work as either an argument or an explanation, depending on the situation where it is used. Consider this short dialogue

\begin{adjustwidth}{2em}{0em}
\begin{longtabu}{p{.1\linewidth}p{.8\linewidth}}
\multicolumn{2}{p{.9\linewidth}}{\textit{Monica visits Steve's cubical}.}\\
\textbf{Monica:} &All your plants are dead.\\
\textbf{Steve:} & It's because I never water them.
\end{longtabu}
\end{adjustwidth}
\vspace{-1cm}


In the passage above, Steve uses the word ``because,'' which we've seen in the past is a premise indicator word. But if it were a premise, the conclusion would be ``All Steve's plants are dead.'' But Steve can't possibly be trying to convince Monica that all his plants are dead. It is something that Monica herself says, and that they both can see. The ``because'' here indicates a reason, but here Steve is giving an explanation, not an argument. He takes something that Steve and Monica already know---that the plants are dead---and puts it in a new light by explaining how it came to be. In this case, the plants died because they didn't get water, rather than dying because they didn't get enough light or were poisoned by a malicious co-worker. The reasoning is best represented like this:

\begin{tikzpicture}
\path
	(0,0) node [anchor=west] {1. Steve never waters his plants.}
	(5.5,-8pt) node [anchor=west]{E}
	(0,-20pt) node[anchor=west] {2. All the plants are dead.};
\draw (.5,-8pt) -- (5.5,-8pt);
\end{tikzpicture}

But the same piece of reasoning can change form an explanation into an argument simply by putting it into a new situation:


\begin{adjustwidth}{2em}{0em}
\begin{longtabu}{p{.1\linewidth}p{.8\linewidth}}
\multicolumn{2}{p{.9\linewidth}}{\textit{Monica and Steve are away from the office}.}\\
\textbf{Monica:} &Did you have someone water your plants while you were away?\\
\textbf{Steve:}& No.\\
\textbf{Monica:}& I bet they are all dead.
\end{longtabu}
\end{adjustwidth}
\vspace{-1cm}

Here Steve and Monica do not know that Steve's plants are dead. Monica is inferring this idea based on the premise which she learns from Steve, that his plants are not being watered. This time ``Steve's plants are not being watered'' is a premise and ``The plants are dead'' is a conclusion. We represent the argument like this:

\begin{earg}
\item[P.] Steve never waters his plants. 
\vspace{-.5em}
\item [] \rule{0.3\linewidth}{.5pt} 
\item[C.] All the plants are dead. 
\end{earg}

In the example of Steve's plants, the same piece of reasoning can function either as an argument or an explanation, depending on the context where it is given. This is because the reasoning in the example of the plants is causal: the \textit{causes} of the plants dying are given as reasons for the death, and we can appeal to causes either to explain something that we know happened or to predict something that we think might have happened. 

Not all kinds of reasoning are flexible like that, however. Reasoning from authority can be used in some kinds of argument, but often makes a lousy explanation. Consider another conversation between Steve and Monica:

\begin{adjustwidth}{2em}{0em}
\begin{longtabu}{p{.1\linewidth}p{.8\linewidth}}
\textbf{Monica:} & I saw on a documentary last night that the universe is expanding and probably will keep expanding for ever. \\
\textbf{Steve:} & Really?\\
\textbf{Monica:} &Yeah, Steven Hawking said so. \\
\end{longtabu}
\end{adjustwidth}
\vspace{-1cm}

There aren't any indicator words here, but it looks like Monica is giving an argument. She states that the universe is expanding, and Steve gives a skeptical ``really?'' Monica then replies by saying that she got this information from the famous physicist Steven Hawking. It looks like Steve is supposed to believe that the universe will expand indefinitely because Hawking, an authority in the relevant field, said so. This makes for an ok argument: 

 \begin{earg}
\item[P:] Steven Hawking said that the universe is expanding and will continue to do so indefinitely.
\vspace{-.5em}
\item [] \rule{\linewidth}{.5pt} 
\item[C:] The universe is expanding and will continue to do so indefinitely.
\end{earg} 

Arguments from authority aren't very reliable, but for very many things they are all we have to go on. We can't all be experts on everything. But now try to imagine this argument as an explanation. What would it mean to say that the expansion of the universe can be \textit{explained} by the fact that Steven Hawking said that it should expand. It would be as if Hawking were a god, and the universe obeyed his commands! Arguments from authority are acceptable, but not ideal. Explanations from authority, on the other hand, are completely illegitimate. \label{no_exp_from_authority}

In general, arguments that appeal to how the world works are more satisfying than ones which appeals to the authority or expertise of others. Compare the following pair of arguments:

\begin{enumerate}[label=(\alph*)]
\item Jack says traffic will be bad this afternoon. So, traffic will be bad this afternoon. 
\item Oh no! Highway repairs begin downtown today. And a bridge lift is scheduled for the middle of rush hour. Traffic is going to be terrible \end{enumerate}

Even though the second passage is an argument, the reasons used to justify the conclusion could be used in an explanation. Someone who accepts this argument will also have an explanation ready to offer if someone should later ask ``Traffic was terrible today! I wonder why?''. This is not true of the first passage: bad traffic is not explained by saying ``Jack said it would be bad.'' The argument that refers to the drawbridge going up is appealing to a more powerful sort of reason, one that works in both explanations and arguments. This simply makes for a more satisfying argument, one that makes for a deeper understanding of the world, than one that merely appeals to authority. 

Although arguments based on explanatory premises are preferred, we must often rely on other people for our beliefs, because of constraints on our time and access to evidence. But the other people we rely on should hopefully hold the belief on the basis of an empirical understanding. And if \textit{those} people are just relying on authority, then we should hope that at some point the chain of testimony ends with someone who is relying on something more than mere authority. In [cross ref] we'll look more closely at sources and how much you should trust them.

We just have seen that they same set of statements can be used as an argument or an explanation depending on the context. This can cause confusion between speakers as to what is going on. Consider the following case:

\begin{adjustwidth}{2em}{0em}
\begin{longtabu}{p{.1\linewidth}p{.8\linewidth}}
\multicolumn{2}{p{.9\linewidth}}{\textit{Bill and Henry have just finished playing basketball.}}\\
\textbf{Bill:} & Man, I was terrible today. \\
\textbf{Henry:} & I thought you played fine. \\
\textbf{Bill:} & Nah. It's because I have a lot on my mind from work. \\
\end{longtabu}
\end{adjustwidth}
\vspace{-1cm}

Bill and Henry disagree about what is happening---arguing or explaining. Henry doubts Bill's initial statement, which should provoke Bill to argue. But instead, he appears to plough ahead with his explanation. What Henry can do in this case, however, is take the reason that Bill offers as an explanation (that Bill is preoccupied by issues at work) and use it as a premise in an argument for the conclusion ``Bill played terribly.'' Perhaps Henry will argue (to himself) something like this: ``It's true that Bill has a lot on his mind from work. And whenever a person is preoccupied, his basketball performance is likely to be degraded. So, perhaps he did play poorly today (even though I didn't notice).''

In other situations, people can switch back and forth between arguing and explaining. Imagine that Jones says ``The reservoir is at a low level because of several releases to protect the down-stream ecology.'' Jones might intend this as an explanation, but since Smith does not share the belief that the reservoir's water level is low, he will first have to be given reasons for believing that it is low. The conversation might go as follows:

\begin{adjustwidth}{2em}{0em}
\begin{longtabu}{p{.1\linewidth}p{.8\linewidth}}
\textbf{Jones:} & The reservoir is at a low level because of several releases to protect the down-stream ecology. \\
\textbf{Smith:} & Wait. The reservoir is low?\\
\textbf{Jones:} & Yeah. I just walked by there this morning. You haven't been up there in a while? \\
\textbf{Smith:} & I guess not. \\
\textbf{Jones:} & Yeah, it's because they've been releasing a lot of water to protect the ecology lately. \\
\end{longtabu}
\end{adjustwidth}
\vspace{-1cm}

When challenged, Smith offers evidence from his memory: he saw the reservoir that morning. Once Smith accepts that the water level is low, Jones can restate his explanation.

Some forms of explanation overlap with other kinds of nonargumentative passages. We are dealing right now with thinking in the real world, and as we mentioned on page \pageref{messiness_warning} the real world is full of messiness and ambiguity. One effect of this is that all the categories we are discussing will wind up overlapping. Narratives and expository passages, for instance, can also function as explanations. Consider this passage

\begin{quotation} \noindent\textit{From the sports section} Duke beat Butler 61-59 for the national championship Monday night. Gordon Hayward's half-court, 3-point heave for the win barely missed to leave tiny Butler one cruel basket short of the Hollywood ending. (Based on \cite{AP2010}) \end{quotation}

On the one hand, this is clearly a narrative---retelling a sequence of events united by time, place, and character. But it also can work as an explanation about how Duke won, if the audience immediately accepts the result. 'The last shot was a miss\textit{ }and then Duke won' can be understood as 'the last shot was a miss and so Duke won'.


%%%%%% Practice problems


\practiceproblems 
\problempart Identify each of the passages below as an argument, an explanation, or neither, and justify your answer. If the passage is an argument write it in canonical form, with premises marked P$_1$ etc., then a line, and then the conclusion marked with a C. If the argument is an explanation, write it in the canonical form for an explanation, with the explainers numbered and an ``E'' after the line that separates the explainers and the explainee. If the argument is neither an argument nor an explanation, state what kind of nonargument you think it is, such as a narrative or an expository passage.
 
\begin{longtabu}{p{.1\linewidth}p{.9\linewidth}}
\textbf{Example}: & \textit{Henry arrives at work late: }Bill is not here. He very rarely arrives late. So, he is not coming in today. \\
\textbf{Answer}: & \textit{Argument} You can tell Henry is giving an argument to himself here because the conclusion is something that he did not already believe. \\
&\begin{earg}
\item[P$_1$:] Bill is not here. 
\item[P$_2$:] Bill very rarely arrives late. 
\vspace{-.5em}
\item [] \rule{0.6\linewidth}{.5pt} 
\item[C:] Bill is not coming in today
\end{earg} 
\end{longtabu}

\begin{exercises}

\item \textit{From a science education website run by NASA, also promoted by Google as the answer to the question “Why is the sky blue?”}  Blue light is scattered in all directions by the tiny molecules of air in Earth's atmosphere. Blue is scattered more than other colors because it travels as shorter, smaller waves. This is why we see a blue sky most of the time. \citep{NASA2015}

\answer{\vspace{6pt}This is an explanation, because the target proposition is common knowledge, as in the ``Grass is green'' example in your text. \vspace{6pt}

\begin{tikzpicture}
\path
	(0,0) node [anchor=west] {1. Blue travels in shorter wavelengths. }
	(0,-11pt) node [anchor=west] {2. Blue is scattered more than other colors.}
	(0,-22pt)  node [anchor=west] {3. Light is scattered by molecules in the atmosphere.}
	(9, -33pt) node [anchor=west] {E}
	(0, -44 pt) node[anchor=west] {4. We see a blue sky most of the time.};
\draw (.5,-33pt) -- (9,-33pt);
\end{tikzpicture}

\vspace{6pt}There are actually two separate levels of explanation in this passage, each marked with separate indicator words. The first ``because'' relates the sentence ``Blue travels in shorter wavelengths'' to the sentence, ``Blue is scattered more than other colors.'' The short wavelengths explain why blue is scattered more. The fact that blue is scattered more and that light is scattered when it enters the atmosphere in turn explains why we see the sky as blue. If you wanted to be very precise, you would represent the explanations with two separate diagrams. First

\begin{tikzpicture}
\path
	(0,0) node [anchor=west] {1. Blue travels in shorter wavelengths. }
	(9, -11pt) node [anchor=west] {E}
	(0,-22pt) node [anchor=west] {2. Blue is scattered more than other colors.};
\draw (.5,-11pt) -- (9,-11pt);
\end{tikzpicture}

and then \vspace{6pt}

\begin{tikzpicture}
\path
	(0,0) node [anchor=west] {1. Blue is scattered more than other colors.}
	(0,-11pt) node [anchor=west] {2. Light is scattered by molecules in the atmosphere.}
	(9,-22pt)  node [anchor=west] {E}
	(0, -33 pt) node[anchor=west] {3. We see a blue sky most of the time.};
\draw (.5,-22pt) -- (9,-22pt);
\end{tikzpicture}

}



\item \textit{Jack is reading a popular science magazine. Analyze Jack's reasoning. The magazine says: }Recent research has shown that people who rate themselves as ``very happy'' are less successful financially than those who rate themselves as ``moderately happy.'' \textit{Jack says,} ``Huh! It seems that a little unhappiness is good in life.''  


\answer{
\begin{earg*}
\item People who rate themselves as ``very happy'' are less successful financially than those who rate themselves as ``moderately happy.''
\itemc A little unhappiness is good in life.
\end{earg*}

Jack is arguing to himself. He makes an inference about happiness based on the premise he read in the magazine. Other people may have drawn a different conclusion from that premise. }

\item \textit{An anthropologist is speaking. }People get nicknames based on some distinctive feature they possess. And so, Mark, for example, who is 6'6'' is  (ironically) called ``Smalls'', while Matt, who looks young, is called ``Baby Face.'' John looks just like his dad, and is called ``Chip.'' 

\answer{\vspace{6pt}
Neither. Mark, Matt and John are examples of the general proposition stated in the first statement. However, the examples aren't being used to prove the first statement, nor do they explain why the first statement is true.} 




\item \textit{Two teenaged friends are talking. Analyze Saida's reasoning.}
\vspace{-6pt}
\begin{adjustwidth}{2em}{0em}
\begin{longtabu}{p{.1\linewidth}p{.8\linewidth}}
\textbf{Saida}: &I can't go to the show tonight. \\
\textbf{Jordan}:& Bummer. \\
\textbf{Saida}: &I know! My mother wouldn't let me go out when I asked. 
\end{longtabu}
\end{adjustwidth}
\vspace{-.9cm}

\answer{\vspace{6pt} Explanation or expository passage. It is not an argument because Jordan is going to believe Saida right away because they are friends. So Saida doesn't need to prove that she can't go to the show. Most likely this is an explanation. "My mother won't let me," \textit{explains} why she can't go. \\

\begin{tikzpicture}
\path
	(0,0) node [anchor=west] {1. My mother won't let me go to the show.}
	(9,-8pt) node [anchor=west]{E}
	(0,-19pt) node[anchor=west] {2. I can't go to the show.};
\draw (.5,-8pt) -- (9,-8pt);
\end{tikzpicture}}


\item \textit{A mother is speaking to her teenage son. }You should always listen 
to your mother. I say ``no\texttt.'' So, you have to stay in tonight. 

\answer{

\begin{earg*}
\item You should always listen to your mother. 
\item Your mother says ``no.'' 
\itemc You have to stay in tonight. 
\end{earg*}
Conclusion indicator word: So\\

Arguing. The mother is giving her son a reason to stay home--her authority. We will talk more about arguments from authority later.
}

\item \textit{An economist is speaking. }Any time the public receives a tax rebate, consumer spending increases. Since the public just received a tax rebate, consumer spending will increase. 


\answer{ 
\begin{earg*}
\item Any time the public receives a tax rebate, consumer spending increases. 
\item The public just received a tax rebate
\itemc Consumer spending will increase.
\end{earg*}

Premise indicator word: Since\\
\vspace{6pt}
Arguing. ``Since'' is a flag work here that shows some kind of inference is being made. But the increase in spending hasn't happened yet, so the audience needs to be convinced by the economist that it will happen. Therefore the passage is an argument. If the increase in spending had already happened, and the audience already believe it, then the economist would be explaining. In general, if the target proposition is a prediction, then the passage is likely to be an argument, because the audience doesn't already know the future. % (JRL)
}


\item \textit{In a letter to the editor. }Today's kids are all slackers. American 
society is doomed. 

\answer{
\begin{earg*}
\item Today's kids are all slackers. 
\itemc American society is doomed. 
\end{earg*}
Argument. This is an argument for the same reason as the last one. It is a prediction about the future.}

\item  \textit{On Monday, Jack is told that his unit ships to Iraq in two days: }I 
was hoping to go to Henry's birthday party next weekend. But I'm shipping out on Wednesday. So, I will miss it. 

\answer{
\begin{earg*}
\item I was hoping to go to Henry's birthday party next weekend. 
\item I'm shipping out on Wednesday. 
\itemc I will miss it.
\end{earg*}

Arguing. Jack learns a fact, in this case that he is shipping out, and then infers another fact, that he will miss the party. I have no idea why missing a party is the first thing on his mind when he is given this news. }%(JRL) 



\item \textit{A student is speaking to her instructor: }I was late for class because the battery in my mobile phone, which I was using as an alarm clock, ran out.
\answer{Explaining. The instructor already knows the student is late for class. \\

\begin{tikzpicture}
\path
	(0,0) node [anchor=west] {1. I use my mobile phone for an alarm clock.}
	(0,-11pt) node [anchor=west] {2. The battery on my phone ran out.}
	(9,-19pt) node [anchor=west]{E}
	(0,-30pt) node[anchor=west] {3. I was late to class.};
\draw (.5,-19pt) -- (9,-19pt);
\end{tikzpicture}
}
\item There is a lot of positive talk concerning parenthood because people tend to think about the positive effects that have a child brings and they tend to exclude the numerous negatives that it brings.
\answer{Explaining. The flag word ``because'' indicates reasoning and that the target comes first. The kind of reasoning here is likely explaining, because the target is a commonly held belief. \\

\begin{tikzpicture}
\path
	(0,0) node [anchor=west] {1. People tend to think about the positive effects that have a child brings.}
	(0,-11pt) node [anchor=west] {2. People tend to tend to exclude the numerous negatives that it brings.}
	(9,-19pt) node [anchor=west]{E}
	(0,-30pt) node[anchor=west] {3. There is a lot of positive talk concerning parenthood.};
\draw (.5,-19pt) -- (9,-19pt);
\end{tikzpicture}
}

\end{exercises}



\noindent\problempart Identify each of the passages below as an argument, an explanation, or neither, and justify your answer. If the passage is an argument write it in canonical form, with premises marked P$_1$ etc., then a line, and then the conclusion marked with a C. If the argument is an explanation, write it in the canonical form for an explanation, with the explainers numbered and an ``E'' after the line that separates the explainers and the explainee. If the argument is neither an argument nor an explanation, state what kind of nonargument you think it is, such as a narrative or an expository passage.

 
\begin{exercises}

\item You have to be smart to understand the rules of Dungeons and Dragons. Most smart people are nerds. So, I bet most people who play D\&D are nerds.

% use this passage as a basis for some problems
%Notice that knowledge of an explanation can be used (on a different occasion) to make an argument for the truth of a conclusion. For example, if extremely cold weather in Europe is explained by the movement of air from Siberia, on a future occasion the movement of air from Siberia can be used to argue that it is or will be extremely cold. 

%also this one
%
%\begin{quote} The IPCC, a panel of experts from various countries, has stated that human activity has an impact on climate. So, that's how it is.\end{quote}

%In this passage, a speaker provides a reason for believing \textit{that} human activity has an impact on climate, namely, that an international panel believes so. That is, the speaker provides a premise which might justify adopting the conclusion as a belief. This premise, however, it does not explain \textit{why }or \textit{how}human activity impacts climate. It might thus be a justification, but it could not be used as an explanation. If a speaker says something is so because some source says it, you are looking at an argument. 



\item \textit{A coach is emailing parents in a neighborhood youth soccer league.} The game is canceled since it is raining heavily.


\item  \textit{At the market. }You know, granola bars generally aren't healthy. The ingredients include lots of processed sugars.

\item  \textit{At the pet store.}
\vspace{-8pt}
\begin{adjustwidth}{2em}{0em}
\begin{longtabu}{p{.1\linewidth}p{.8\linewidth}}
\textbf{Salesman}:     &A small dog makes just as effective a guard dog for your 
home as a big dog.\\
\textbf{Henry}:        &   No way!\\
\textbf{Salesman}: &    It might seem strange. But smaller ``yappy'' dogs bark readily and they also generate distinctive higher-pitched sounds. Most of a dog's effectiveness as a guard is due to making a sound, not physical size. \end{longtabu}
\end{adjustwidth}
\vspace{-.9cm}

\item \textit{A child is thinking out loud. }I think my cat must be dead. It isn't in any of its usual places. And when I asked my mother if she had seen it, she couldn't look me in the eyes. 

\item {\color{white}flurm}
\vspace{-24pt}
\begin{adjustwidth}{2em}{0em}
\begin{longtabu}{p{.1\linewidth}p{.8\linewidth}}
\textbf{Smith:} & I can solve any puzzle more quickly than you.\\
\textbf{Jones:}& Get out of here. \\
\textbf{Smith:} & It's true! I'm a member of MENSA, and you're not. 
\end{longtabu}
\end{adjustwidth}
\vspace{-.9cm}

\item \textit{In the comments on a biology blog: }According to Darwin's theory, my ancestors were monkeys. But since that's ridiculous, Darwin's theory is false. 

\item If you believe in [the Christian] God and turn out to be incorrect, you have lost nothing. But if you don't believe in God and turn out to be incorrect, you will go to hell. Believing in God is better in both cases. One should therefore believe in God. (A formulation of ``Pascal's Wager'' by Blaise Pascal.) 

\item \textit{Bill and Henry are in Columbus.}
\vspace{-8pt}
\begin{adjustwidth}{2em}{0em}
\begin{longtabu}{p{.1\linewidth}p{.8\linewidth}}
\textbf{Bill:} & Good news---I just accepted a job offer in Omaha. \\
\textbf{Henry:} & That's great. Congratulations! I suppose this means you'll be leaving us, then?\\
\textbf{Bill:} & Yes, I'll need to move sometime before September.  \\
\end{longtabu}
\end{adjustwidth}
\vspace{-.9cm}

\item You already know that God kicked humanity out of Eden before they could eat of the tree of life but only after they had eaten of the tree of knowledge of good and evil. That was because Satan wanted to take over God's throne and was responsible for their eating from the tree. If humans had eaten of both trees they could have been a threat to God. 

\end{exercises}

% *****************************************
% *  		Recognizing Arguments in Wild		*
% *****************************************

% a section for working with newspapers and field projects.

%
%\section{Recognizing Arguments in Wild}
%%When faced with a passage or dialogue, you must first determine whether or not it contains reasoning, and in particular whether the reasons involved are reasons-to-believe or reasons-which-explain. 
%
%%reiterate flag words. 
% 
%       
%%      There are an infinitely large number of flag words and phrases.
%%
%%      These flag words and phrases indicate reasoning because they indicate premises or target, but they do not distinguish between arguing and explaining. Moreover, passages sometimes do not have any flag words. So, we need other ways of telling whether and what kind of reasoning is taking place.
%
%
%Discuss replacing pronouns, making each statement stand on its own, paraphrasing for length. Use lots of real world examples.
%
% 
%   Discuss reports of arguments here. 
%
%Arguments vs. explanations (again)
%
%``Should'' statements in the conclusion are generally a sign of arguing. Predictions are a sign of arguing.
%
%
%\begin{quote}Highway repairs begin downtown today. And a bridge lift is scheduled for the middle of rush hour. I predict that traffic is going to be terrible.\end{quote}
%
%\begin{quote}Yeah, I know traffic is going to be terrible. It's because repairs begin downtown today. And a bridge lift is scheduled for the middle of rush hour.\end{quote}
%
%The words ``I predict'' in the first passage suggest the conclusion is a novel belief, (in fact, it's novel even to the speaker). The second passage starts out with the speaker saying ``I know'' about what is clearly the target, because of the reasons offered subsequently. In the first, therefore, the speaker is making an inference and trying to convince someone (perhaps herself) that the proposition ``Traffic is going to be terrible.'' is true. The second, on the other hand, is an explanation. The speaker is not trying to increase her (or anyone else's) store of knowledge; she is trying to describe connections between states of affairs.
%
%      Sometimes you need to use your knowledge of what various specific people know, as well as your general knowledge of the knowledge that people have, including your knowledge of what you can reasonably expect people or different ages (children, teens, adults) or different backgrounds (people from your own country or region as opposed to foreigners) and so on. This is called epistemic score-keeping.
%      




\section*{Key Terms}
\begin{multicols}{2}
\begin{sortedlist}
\sortitem{Logic}{}
\sortitem{Metareasoning}{}
\sortitem{Metacognition}{} 	
\sortitem{Content neutrality}{}
\sortitem{Formal logic}{}
\sortitem{Critical thinking}{}
\sortitem{Informal logic}{}
\sortitem{Rhetoric}{}
\sortitem{Canonical form}{}
\sortitem{Conclusion indicator}{} 
\sortitem{Premise indicator}{}
\sortitem{Statement}{}
\sortitem{Argument}{}
\sortitem{Conclusion}{}
\sortitem{Premise}{}
\sortitem{Inference}{}
\sortitem{Simple statement of belief}{}
\sortitem{Expository passage}{}
\sortitem{Narrative}{}
\sortitem{Explanation}{}
\sortitem{Explainer}{}
\sortitem{Explainee}{}
\sortitem{Reason}{}
\sortitem{Target proposition}{}
\sortitem{Critical thinker}{}
\sortitem{Practical argument}{}
\end{sortedlist}
\end{multicols}
	
\include{tex/ch02-basicevaluation}

%\part{Formal Logic} \label{part:formal_logic}
\chapter{What is Formal Logic?}
\label{chap:whatisformallogic}
\markright{Ch. \ref{chap:whatisformallogic}: What is Formal Logic?}
\setlength{\parindent}{1em}

% **************************************************
% *	3.1 Formal as in  Concerned with the Form of Things             *
% **************************************************

\section{Formal as in Concerned with the Form of Things}


The chapters in 
\iflabelexists{part:formal_logic}{Part \ref{part:formal_logic}} %This prints ``Part $N$ if there is a single section for all of formal logic
{Parts \iflabelexists{part:cat_logic}{\ref{part:cat_logic} and \ref{part:sent_logic}} %this prints ``Parts $N$ and $M'$' if there is a section for cat logic, where $N$ and $M$ are the part numbers for cat and sent logic.
{\ref{part:sent_logic} and \ref{part:quant_logic}}} %this prints ``Parts $N$ and $M'$' if there is not a section for cat logic, where $N$ and $M$ are the part numbers for sent and quant logic.
deal with formal logic. Formal logic is distinguished from other branches of logic by the way it achieves content neutrality. Back on page \pageref{def:content_neutrality}, we said that a distinctive feature of logic is that it is neutral about the content of the argument it evaluates. If a kind of argument is strong---say, a kind of statistical argument---it will be strong whether it is applied to sports, politics, science or whatever. Formal logic takes radical measures to ensure content neutrality: it removes the parts of a statement that tie it to particular objects in the world and replaces them with abstract symbols. 

Consider the two arguments from Figure \ref{fig:valid_sound} again:
\begin{multicols}{2}
\begin{earg*}
\item Socrates is a person.
\item All persons are mortal.
\itemc Socrates is mortal.
\end{earg*}

\begin{earg*}
\item Socrates is a person.
\item All people are carrots.
\itemc Socrates is a carrot.
\end{earg*}

\end{multicols}

These arguments are both valid. In each case, if the premises were true, the conclusion would have to be true. (In the case of the first argument, the premises are actually true, so the argument is sound, but that is not what we are concerned with right now.) What makes these arguments valid is that they are put together the right way. Another way of thinking about this is to say that they have the same logical form. Both arguments can be written like this:

\begin{earg*}
\item $S$ is $M$.
\item All $M$ are $P$.
\itemc[.2] $S$ is $P$.
\end{earg*}

In both arguments $S$ stands for Socrates and $M$ stands for person. In the first argument, $P$ stands for mortal; in the second, $P$ stands for carrot. \iflabelexists{chap:catstatements}{(The reason we chose these letters will become clear in Chapters \ref{chap:catstatements} and \ref{chap:cat_syllogisms}.)}{} The letters `S', `M', and `P' are variables. They are just like the variables you may have learned about in algebra class. In algebra, you had equations like $y = 2x + 3$, where $x$ and $y$ were variables that could stand for any number. Just as $x$ could stand for any number in algebra, `S' can stand for any name in logic. In fact, this is one of the original uses of variables. Long before variables were used to stand for numbers in algebra, they were used to stand for classes of things, like people or carrots, by Aristotle in his book the \cite*{Aristotle:prior}. At about the same time, over in India, the ancient grammarian and linguist P\={a}\d{n}ini was also using variables to represent possible sounds that could be used in different forms of a word \citep{Panini2015}. Both thinkers introduce their variables fairly causally, as if their readers would be familiar with the idea, so it may be that people prior to them actually invented the variable.

Whoever invented it, the variable was one of the most important conceptual innovations in human history, right up there with the invention of the zero, or alphabetic writing. The importance of the variable for the history of mathematics is obvious. But it was also incredibly important in one of its original fields of application, logic. For one thing, it allows logicians to be more content neutral. We can set aside any associations we have with people, or carrots, or whatever, when we are analyzing an argument. More importantly, once we set aside content in this way, we discover that something incredibly powerful is left over, the logical structure of the sentence itself. This is what we investigate when we study formal logic. In the case of the two arguments above, identifying the logical structure of statements reveals not only that the two arguments have the same logical form, but they have an impeccable logical form. Both arguments are valid, and any other arguments that have this form will be valid. 

When Aristotle introduced the variable to the study of logic he used it the way we did in the argument above. His variables stood for names and categories in simple two-premise arguments called syllogisms. The system of logic Aristotle outlined became the dominant logic in the Western world for more than two millennia. It was studied and elaborated on by philosophers and logicians from Baghdad to Paris. The thinkers that carried on Aristotelian tradition were divided by language and religion. They were pagans, Muslims, Jews, and Christians writing typically in Greek, Latin or Arabic. But they were all united by the sense that the tools Aristotle had given them allowed them to see something profound about the nature of reality. They were looking at abstract structures which somehow seemed to be at the foundation of things. As the philosopher and historian of logic Catarina Dutilh Novaes points out, the logic that the thinkers of all these religious traditions were pursuing was formal in that it concerned the \textit{forms} of things \citep{DutilhNovaes2011}. As formal logic evolved, however, the idea of being ``formal'' would take on an additional meaning. 

\vfill

% *********************************************************
% *				3.2 Formal meaning strictly following rules              *
% ********************************************************


\section{Formal as in Strictly Following Rules}


\newglossaryentry{artificial language}
{
name=artificial language,
description={A language that was consciously developed by identifiable individuals for some purpose.}
}

\newglossaryentry{natural language}
{
name=natural language,
description={A language that develops spontaneously and learned by infants as their first language.}
}

\newglossaryentry{formal language}
{
name=formal language,
description={An artificial language designed to bring out the logical structure of  ideas and remove all the ambiguity and vagueness that plague natural languages like English. Sometimes, formal languages are also said to be languages that can be implemented by a machine.}
}

Despite its historical importance, Aristotelean logic has largely been superseded. Starting in the 19th century people learned to do more than simply replace categories with variables. They learned to replicate the whole structure of sentences with a formal system that brought out all sorts of features of the logical form of arguments. The result was the creation of entire artificial languages. An \textsc{\gls{artificial language}} \label{def:artificial_language} is a language that was consciously developed by identifiable individuals for some purpose. Esperanto, for instance, is an artificial language developed by Ludwig Lazarus Zamenhof in the 19th century with the hope of promoting world peace by creating a common language for all. J.R.R. Tolkien invented several languages to flesh out the fictional world of his fantasy novels, and even created timelines for their evolution. For Tolkien, the creation of languages was an art form in itself, ``An art for which life is not long enough, indeed: the construction of imaginary languages in full or outline for amusement, for the pleasure of the constructor or even conceivably of any critic that might occur'' (\cite*{Tolkien1931}). And it is an art that is really beginning to catch on, especially with Hollywood commissioning languages to be constructed for blockbuster films. 

Artificial languages contrast with \textsc{\glspl{natural language}}, \label{def:natural_language} which develop spontaneously and are learned by infants as their first language. Natural languages include all the well-known languages spoken around the world, like English or Japanese or Arabic. It also includes more recently developed languages and evolved spontaneously amongst groups of people. For instance, whenever you put deaf children together, for instance in a boarding school, they will spontaneously develop their own sign language. This phenomenon was important for the development of American Sign Language (ASL) and is part of why ASL counts as a \textit{natural} language. For similar reasons Nicaraguan Sign Language counts as a natural language, even though it emerged very recently---in the late 1970s and 80s, when the new Sandinista government set up schools for the deaf for the first time. Natural languages can also develop by creolization, when languages merge and children grow up speaking the merged language as their first language. Haitian Creole is the most famous example of this.   

The languages developed by logicians are artificial, not natural. Their goal is not to promote global harmony, like Zamenhof's Esperanto. Nor are they creating art for art's sake, as Tolkein was, although logical languages can have a great deal of beauty. When the languages first started being developed in the late 19th and early 20th centuries, the goal was, in fact, to have a logically pure language, free of the irrationalities the plague natural languages. More specifically, they had two distinct goals: first, remove all ambiguity and vagueness, and second, to make the logical structure of the language immediately apparent, so that the language wore its logical structure on its face, as it were. If such a language could be developed, it would help us solve all kinds of problems. The logician and philosopher Rudolf Carnap, for instance, felt that the right artificial language could simply make philosophical problems disappear \citep{Carnap1928}.

The languages developed by logicians in the late 19th and early 20th centuries got labeled formal languages, in part because the logicians in question were working in the tradition of formal logic that was already established. A shift began to happen here with the meaning of formal, however, a change which is well documented by Dutilh Novaes  (\cite*{DutilhNovaes2011}). Logicians began to hope that the languages that were being developed were so logical that everything about them could be characterized by a machine. A machine could be used to create sentences in this language, and then again to identify all the valid arguments in this language. This brings out another sense of the word ``formal.'' As Dutilh Novaes puts it (\cite*{DutilhNovaes2011}) instead of being ``formal'' in the sense of concerning the forms of things, logic was formal in the sense that it followed rules perfectly precisely. You might compare this to the way a ``formal hearing'' in a court of law follows the rule of law to the letter. 

For the purposes of this textbook, we will say that the core idea of a  \textsc{\gls{formal language}} \label{def:formal_language} is that it is an artificial language designed to bring out the logical structure of ideas and remove all the ambiguity and vagueness that plague natural languages like English. We will further add that sometimes, formal languages are languages that can be implemented by a machine. Creating formal languages always involves all kinds of trade offs. On the one hand, we are trying to create a language that makes a logical structure clear and obvious. This will require simplifying things, removing excess baggage from the language. On the other hand, we want to make the language perfectly precise, free of vagueness and ambiguity. This will mean adding complexity to the language. The other thing was that it was very important for the people developing these languages that you be able to prove the all the truths of mathematics in them. This meant that the languages had to have a certain scope.

This was a trade off no logician was ever able to get perfectly correct, because, as it turns out, a logically pure language is impossible. No formal language can do everything that a natural language can do. Logicians became convinced of this, naturally enough, because of a pair of logical proofs. In 1931, the logician Kurt G\"{o}del showed that you couldn't do all of mathematics in a consistent logical system, which was enough to persuade most of the logicians engaged in the project to drop it. There is a more general problem with the idea of a purely logical language, though, which is that that many of the features logicians were trying to remove from language were actually necessary to make it function. Arika Okrent puts the point quite well. For Okrent, the failure of artificial languages is precisely what illuminates the virtues of natural language. 

\begin{quotation}\noindent [By studying artificial languages we] gain a deeper appreciation of natural language and the messy qualities that give it so much flexibility and power and that a simple communication device. The ambiguity and lack of precision allow to serve as a instrument of thought \textit{formation}, of experimentation and discovery. We don't know exactly what we mean before we speak; we can figure it out as we go along,. We can talk just to talk, to be social, to feel connected, to participate. At the same time natural language still works as an instrument of thought transmission, one that can be \textit{made} extremely precise and reliable when we need it to be, or left loose and sloppy when we can't spare the time or effort \citep{Okrent2009} \end{quotation} 

The languages developed in the late 19th and early 20th centuries had goals that were theoretical, rather than practical. They languages were meant to improve our understanding of the world for the sake of improving our understanding of the world. They failed at this theoretical goal, but they wound up having a practical spin-off of world-historical proportions, which is why formal logic is a thriving discipline to this day. Remember that in this period people started thinking of formal languages as languages that could be implemented mechanically. At first, the idea of a a mechanistic language was a metaphor. The rules that were being followed to the letter were to be followed by a human being actually writing down symbols. This human being was generally referred to as a ``computer,'' because they were computing things. The world changed when a logician named Alan Turing started using literal machines to be computers.

In the 1930s, Turing developed the idea of a reasoning machine that could compute any function. At first, this was just an abstract idea: it involved an infinite stretch of tape. But during World War II, Turing went to work the British code breaking effort at Bletchley Park. The Nazis encoded messages using a device called the Enigma Machine. The Allies had captured one, but since they settings on the machine were reshuffled for each message, it didn't do them much good. Turing, together with people like the mathematicians Gordon Welchman and Joan Clarke, managed to build another machine that could test Enigma settings rapidly to identify the configuration being used. People had made computing machines before, but now the science of logic was so much more advanced that they real power of mechanical computing could be exploited. The human computers became the fully programmable machines we know today, and the formal languages logicians created for theoretical reasons came the computer languages the world of the 21st century depends on. (All of this information, plus lots of fascinating pictures and diagrams, is available at www.turing.org.uk.)



%This version is for the complete text, where all formal sections are covered in a unified Part II.
\iflabelexists{part:formal_logic}{Part \ref{part:formal_logic} of this book begins by exploring the world of Aristotelian logic, where logic is ``formal'' in the sense of being about the forms of things. Chapter \ref{chap:catstatements} looks at the logical structure of the individual statements studied by the Aristotelian tradition. Chapter \ref{chap:cat_syllogisms} then builds these into valid arguments. After we study Aristotelian logic, we will develop two formal languages, called SL and QL.  Chapters \ref{chap:SL} through \ref{chap:proofsinSL} develop SL. In SL, the smallest units are individual statements. Simple statements are represented as letters and connected with {logical connectives} like \emph{and} and \emph{not} to make more complex statements. Chapters \ref{chap:QL} through \ref{chap:proofsinQL} develop QL. In QL, the basic units are objects, properties of objects, and relations between objects.}{
%% This version of the paragraph is for texts that just do cat and sent.
\iflabelexists{part:cat_logic}{Part \ref{part:cat_logic} of this book explores the world of Aristotelian logic. Chapter \ref{chap:catstatements} looks at the logical structure of the individual statements studied by the Aristotelian tradition. Chapter \ref{chap:cat_syllogisms} then builds these into valid arguments. Part \ref{part:sent_logic} develops a full-blown formal language, called Sentential logic, or SL. In SL Simple statements are represented as letters and connected with logical connectives like \emph{and} and \emph{not} to make more complex statements.}{
%This version of the paragraph is for texts that just do sent and quant
In this book we will be developing two formal languages, called SL and QL. Part \ref{part:sent_logic} develops SL.In SL, the smallest units are individual statements. Simple statements are represented as letters and connected with logical connectives like \emph{and} and \emph{not} to make more complex statements. Part \ref{part:quant_logic} develops QL. In QL, the basic units are objects, properties of objects, and relations between objects.} }




%\iflabelexists{part:formal_logic}{ and QL.  Chapters \ref{chap:SL} through \ref{chap:proofsinSL} develop SL. In SL, the smallest units are individual statements. Several chapters in the complete version of this text \label{ver_var}\nix{Chapters \ref{chap:QL} through \ref{chap:proofsinQL}} develop QL. In QL, the basic units are objects, properties of objects, and relations between objects.


%% **********************************************
%% *			On Learning a Formal Logical System        *
%% **********************************************
%\section{On Learning a Formal Logical System}
%\label{sec:On_learning_a_formal_logical_system}
%
%You may be reading this book because you have a keen interest in logic and are excited to learn more about it. You may also be reading this book because it was assigned in a class that you need to fulfill a distribution requirement. As the chapters on formal logic roll on, and the pages begin to fill up with unfamiliar squiggles, you may even begin to question whether the study of logic is for you. Rest assured, if you have a human brain capable of reading this sentence, you are also capable of doing formal logic---and you can benefit from doing so, too. In this section, we are going to talk about why you can be confident in your ability to do logic, even if you are new to it. We are also going to offer some strategies for studying formal logic, so even if you are already quite confident in your abilities, it will be worth reading the rest of this section. 
%
%All of the basic mental skills used in a formal logical system are just that: basic mental skills. They are things you do whenever you use language. A basic part of formal logic is using abstract symbols to refer to a group of things that aren't specified. So earlier we used ``$P$'' in place of the words ``mortal'' and ``carrot'' and a whole bunch of other words that might occupy that spot in an argument. This is the same thing you do when you use a word like ``dog'' to refer to Spot and Fido and a whole bunch of other dogs that you don't know about. We are also going to spend time transforming things in one logical form into another. Again, this is something you already do when you speak. You know that ``Jane gave the ball to Sally'' can be changes to ``Sally was given the ball by Jane'' without changing its meaning. The kinds of things we are doing in this text are no different. 
%


% **********************************************
% *			More Logical Notions for Formal Logic      *
% **********************************************
\section{More Logical Notions for Formal Logic}
\label{sec:other_logical_notions}
\setlength{\parindent}{1em}

Part \ref{part:basic_concepts} covered the basic concepts you need to study any kind of logic. When we study formal logic, we will be interested in some additional logical concepts, which we will explain here. 

%1.5.1 Truth values

\subsection{Truth values}

\newglossaryentry{truth value}
{
  name=truth value,
  description={The status of a statement with relationship to truth. For this textbook, this means the status of a statement as true or false.}
}

\newglossaryentry{bivalent}
{
  name=bivalent,
  description={A property of logical systems which is present when the system only has two truth values, generally ``true'' and ``false.''}
}



A truth value is the status of a statement as true or false. Thus the truth value of the sentence ``All dogs are mammals'' is ``True,'' while the truth value of ``All dogs are reptiles'' 
is false. More precisely, a \textsc{\gls{truth value}} \label{def:Truth_value} is the status of a statement with relationship to truth. We have to say this, because there are systems of 
logic that allow for truth values besides ``true'' and ``false,'' like ``maybe true,'' or ``approximately true,'' or ``kinda sorta true.'' For instance, some philosophers have claimed 
that the future is not yet determined. If they are right, then statements about \emph{what will be the case} are not yet true or false. Some systems of logic accommodate this by having an 
additional truth value. Other formal languages, so-called paraconsistent logics, allow for statements that are both true \emph{and} false. We won't be dealing with those in this textbook, 
however. For our purposes, there are two truth values, ``true'' and ``false,'' and every statement has exactly one of these. Logical systems like ours are called \textsc{\gls{bivalent}}. 
\label{def:bivalent}







%1.5.2 Tautology, Contingent Statement, Contradiction

\subsection{Tautology, contingent statement, contradiction}

In considering arguments formally, we care about what would be true \emph{if} the premises were true. Generally, we are not concerned with the actual truth value of any particular statements--- whether they are \emph{actually} true or false. Yet there are some statements that must be true, just as a matter of logic.

Consider these statements:
\begin{enumerate}[label=(\alph*)]
\item \label{itm:ex_contingent} It is raining.
\item \label{itm:ex_tautology} Either it is raining, or it is not.
\item \label{itm:ex_contradiction} It is both raining and not raining.
\end{enumerate}
In order to know if statement \ref{itm:ex_contingent} is true, you would need to look outside or check the weather channel. Logically speaking, it might be either true or false. Statements like this are called \emph{contingent} statements.


\newglossaryentry{tautology}
{
name=tautology,
description={A statement that must be true, as a matter of logic.}
}

Statement \ref{itm:ex_tautology} is different. You do not need to look outside to know that it is true. Regardless of what the weather is like, it is either raining or not. If it is drizzling, you might describe it as partly raining or in a way raining and a way not raining. However, our assumption of bivalence means that we have to draw a line, and say at some point that it is raining. And if we have not crossed this line, it is not raining. Thus the statement ``either it is raining or it is not'' is always going to be true, no matter what is going on outside. A statement that has to be true, as a matter of logic is called a \textsc{\gls{tautology}} \label{def:tautology} or logical truth. 

\newglossaryentry{contradiction}
{
name=contradiction,
description={A statement that must be false, as a matter of logic.}
}

You do not need to check the weather to know about statement \ref{itm:ex_contradiction}, either. It must be false, simply as a matter of logic. It might be raining here and not raining across town, it might be raining now but stop raining even as you read this, but it is impossible for it to be both raining and not raining here at this moment. The third statement is \emph{logically false}; it is false regardless of what the world is like. A logically false statement is called a \textsc{\gls{contradiction}}. \label{def:contradiction}

\newglossaryentry{contingent statement}
{
name=contingent statement,
description={A statement that is neither a tautology nor a contradiction.}
}

We have already said that a contingent statement is one that could be true, or could be false, as far as logic is concerned. To be more precise, we should define a \textsc{\gls{contingent statement}}  \label{def:contingent_statement} as a statement that is neither a tautology nor a contradiction. This allows us to avoid worrying about what it means for something to be logically possible. We can just piggyback on the idea of being logically necessary or logically impossible. 

A statement might \emph{always} be true and still be contingent. For instance, it may be the case that in no time in the history of the universe was there ever an elephant with tiger stripes. Elephants only ever evolved on Earth, and there was never any reason for them to evolve tiger stripes. The statement ``Some elephants have tiger stripes,'' is therefore always false. It is, however, still a contingent statement. The fact that it is always false is not a matter of logic. There is no contradiction in considering a possible world in which elephants evolved tiger stripes, perhaps to hide in really tall grass. The important question is whether the statement \emph{must} be true, just on account of logic.

When you combine the idea of tautologies and contradictions with the notion of deductive validity, as we have defined it, you get some curious results. For one thing, any argument with a tautology in the conclusion will be valid, even if the premises are not relevant to the conclusion. This argument, for instance, is valid.

\begin{earg*}
\item There is coffee in the coffee pot.
\item There is a dragon playing bassoon on the armoire.
\itemc All bachelors are unmarried men.
\end{earg*}

The statement ``All bachelors are unmarried men'' is a tautology. No matter what happens in the world, all bachelors have to be unmarried men, because that is how the word ``bachelor'' is defined. But if the conclusion of the argument is a tautology, then there is no way that the premises could be true and the conclusion false. So the argument must be valid.

Even though it is valid, something seems really wrong with the argument above. The premises are not relevant to the conclusion. Each sentence is about something completely different. This notion of relevance, however, is something that we don't have the ability to capture in the kind of simple logical systems we will be studying. The logical notion of validity we are using here will not capture everything we like about arguments.

Another curious result of our definition of validity is that any argument with a contradiction in the premises will also be valid. In our kind of logic, once you assert a contradiction, you can say anything you want. This is weird, because you wouldn't ordinarily say someone who starts out with contradictory premises is arguing well. Nevertheless, an argument with contradictory premises is valid.

%1.5.3 Logical equivalence. 

\subsection{Logically Equivalent and Contradictory Pairs of Sentences}

We can also ask about the logical relations \emph{between} two statements. For example:

\begin{enumerate}[label=(\alph*)]
\item John went to the store after he washed the dishes.
\item John washed the dishes before he went to the store.
\end{enumerate}

\newglossaryentry{logical equivalence}
{
name={logical equivalence},
text={logically equivalent},
description={A property held by a pair of sentences that must always have the same truth value.}
}

These two statements are both contingent, since John might not have gone to the store or washed dishes at all. Yet they must have the same truth value. If either of the statements is true, then they both are; if either of the statements is false, then they both are. When two statements necessarily have the same truth value, we say that they are \textsc{\gls{logical equivalence}}. \label{def:logical_equivalence}

\newglossaryentry{contradictories}
{
name=contradictories,
description={Two statements that must have opposite truth values, so that one must true and the other false.}
}

On the other hand, if two sentences must have opposite truth values, we say that they are \textsc{\gls{contradictories}}. \label{def:contradictory}Consider these two sentences 

\begin{enumerate}[label=(\alph*)]
\item Susan is taller than Monica.
\item Susan is shorter or the same height as Monica.
\end{enumerate}

One of these sentences must be true, and if one of the sentences is true, the other one is false. It is important to remember the difference between a single sentence that is a \emph{contradiction} and a pair of sentences that are \emph{contradictory}. A single sentence that is a contradiction is in conflict with itself, so it is never true. When a pair of sentences is contradictory, one must always be true and the other false.

%%%%%%%%%%%%%%  consistency

\subsection{Consistency}
\label{sec:consistency}
Consider these two statements:

\begin{enumerate}[label=(\alph*)]
\item \label{itm:taller} My only brother is taller than I am.
\item \label{itm:shorter} My only brother is shorter than I am.
\end{enumerate}

Logic alone cannot tell us which, if either, of these statements is true. Yet we can say that \emph{if} the first statement \ref{itm:taller} is true, \emph{then} the second statement \ref{itm:shorter} must be false. And if \ref{itm:shorter}  is true, then \ref{itm:taller} must be false. It cannot be the case that both of these statements are true. It is possible, however that both statements can be false. My only brother could be the same height as I am. 

\newglossaryentry{inconsistency}
{
name=inconsistency,
text={inconsistent},
description={A property possessed by a set of sentences when they cannot all be true at the same time, but they may all be false at the same time.}
}

\newglossaryentry{consistency}
{
name=consistency,
text={consistent},
description={A property possessed by a set of sentences when they can all be true at the same time, but are not necessarily so.}
}

If a set of statements could not all be true at the same time, they are said to be \textsc{\gls{inconsistency}}. \label{def:inconsistency} Otherwise, they are \textsc{\gls{consistency}}. \label{def:consistency} 

We can ask about the consistency of any number of statements. For example, consider the following list of statements:

\label{MartianGiraffes}
\begin{enumerate}[label=(\alph*)]
\item \label{itm:at_least_four}There are at least four giraffes at the wild animal park.
\item \label{itm:exactly_seven} There are exactly seven gorillas at the wild animal park.
\item \label{itm:not_more_than_two} There are not more than two Martians at the wild animal park.
\item \label{itm:martians} Every giraffe at the wild animal park is a Martian.
\end{enumerate}

Statements \ref{itm:at_least_four} and \ref{itm:martians} together imply that there are at least four Martian giraffes at the park. This conflicts with \ref{itm:not_more_than_two}, which implies that there are no more than two Martian giraffes there. So the set of statements \ref{itm:at_least_four}--\ref{itm:martians} is inconsistent. Notice that the inconsistency has nothing at all to do with \ref{itm:exactly_seven}. Statement \ref{itm:exactly_seven} just happens to be part of an inconsistent set.

Sometimes, people will say that an inconsistent set of statements ``contains a contradiction.'' By this, they mean that it would be logically impossible for all of the statements to be true at once. A set can be inconsistent even when all of the statements in it are either contingent or tautologous. When a single statement is a contradiction, then that statement alone cannot be true.

%%%%%%%%%%%  Practice Problems %%%%%%%%%%%


\practiceproblems
\noindent \problempart \label{pr.EnglishTautology} Label the following tautology, contradiction, or contingent statement.

\begin{longtabu}{p{.1\linewidth}p{.9\linewidth}}
\textbf{Example}: & Caesar crossed the Rubicon. \\
\textbf{Answer}: & Contingent statement. \\
&The Rubicon is a river in Italy. When General Julius Caesar took his army across it, he was committing to a revolution against the Roman Republic. Since that time, ``crossing the Rubicon'' has been a expression referring to making an irreversible decision. This kind of decision certainly seems to be contingent. Caesar could have decided otherwise.\\
\end{longtabu}

\begin{exercises}
\item Someone once crossed the Rubicon. \answer{\underline{Contingent statement}}
\item No one has ever crossed the Rubicon. \answer{\underline{Contingent  statement}}
\item If Caesar crossed the Rubicon, then someone has. \answer{\underline{Tautology}}
\item Even though Caesar crossed the Rubicon, no one has ever crossed the Rubicon. \answer{\underline{Contradiction}}
\item If anyone has ever crossed the Rubicon, it was Caesar. \answer{\underline{Contingent statement}}
\end{exercises}

\noindent \problempart Label the following tautology, contradiction, or contingent statement.
\begin{exercises}
\item Elephants dissolve in water. \answer{\underline{Contingent}}
\item Wood is a light, durable substance useful for building things. \answer{\underline{Contingent}}
\item If wood were a good building material, it would be useful for building things. \answer{\underline{Tautology}}
\item I live in a three story building that is two stories tall. \answer{\underline{Contradiction}}
\item If gerbils were mammals they would nurse their young. \answer{\underline{Tautology}}
\end{exercises}

\noindent \problempart Label the following logically equivalent, contradictory, or neither. 

\begin{longtabu}{p{.1\linewidth}p{.9\linewidth}}
\textbf{Example}: &  All students who study will pass the test. \\
& If Jeremy studies, he will pass the test. \\
\textbf{Answer}: & Neither. \\
&If the first statement is true, then the second statement has to be true, but the reverse is not the case. It might be that Jeremy will pass the test if he studies, but some other students are going to fail no matter what.\\
\end{longtabu}

 
\begin{exercises}
\item Elephants dissolve in water.	\\
	If you put an elephant in water, it will dissolve.
\answer{\\\underline{Logically equivalent}}	

\item All mammals dissolve in water.\\		
	If you put an elephant in water, it will dissolve. 
\answer{\\ \underline{Neither}}

\item Elephants are bigger than lions. \\                                                                                        
Elephants are smaller or the same size as lions.
\answer{\\ \underline{Contradictory}}

\item The Eurasian elephant is an herbivore \\
All the Eurasian elephant sometimes eats meat
\answer{\\ \underline{Contradictory}}

\item Elephants dissolve in water. 	\\	
	All mammals dissolve in water. 
\answer{\\ \underline{Neither}}

\end{exercises}


\noindent \problempart Label the following logically equivalent, contradictory, or neither. 

\begin{exercises}
\item  Thelonious Monk played piano.	\\
John Coltrane played tenor sax. 
\answer{\\ \underline{Neither}}

\item  Thelonious Monk played gigs with John Coltrane.	\\
	John Coltrane played gigs with Thelonious Monk.
\answer{\\ \underline{Logically equivalent}}

\item  All professional piano players have big hands.	\\
	Piano player Bud Powell had big hands.
	\answer{\\ \underline{Neither}}

\item  Bud Powell suffered from severe mental illness.	 \\
	All piano players suffer from severe mental illness.
	\answer{\\ \underline{Neither}}

\item John Coltrane was deeply religious.	 \\
John Coltrane was moderately or not at all religious 
\answer{\\ \underline{Contradictory}}
\end{exercises}


\noindent \problempart Consider again the statements on p.\pageref{MartianGiraffes}: 
\begin{enumerate}[label=(\alph*)]
\item \label{itm:at_least_four}There are at least four giraffes at the wild animal park.
\item \label{itm:exactly_seven} There are exactly seven gorillas at the wild animal park.
\item \label{itm:not_more_than_two} There are not more than two Martians at the wild animal park.
\item \label{itm:martians} Every giraffe at the wild animal park is a Martian.
\end{enumerate}
Now mark each of the following sets of statements consistent or inconsistent.
\begin{longtabu}{p{.1\linewidth}p{.9\linewidth}}
\textbf{Example}: & Statements \ref{itm:at_least_four}, \ref{itm:not_more_than_two}, and \ref{itm:martians}\\
\textbf{Answer}: & Inconsistent. If there are at least four giraffes, and every one of them is Martian, there can't be no more than two Martians in the park.\\
\end{longtabu}



\begin{exercises}
\item Statements \ref{itm:exactly_seven}, \ref{itm:not_more_than_two}, and \ref{itm:martians} \answer{\underline{consistent}}
\item Statements \ref{itm:at_least_four}, \ref{itm:exactly_seven}, \ref{itm:not_more_than_two}, and \ref{itm:martians} \answer{\underline{inconsistent}}
\item Statements \ref{itm:at_least_four}, \ref{itm:exactly_seven}, and \ref{itm:martians}\answer{\underline{consistent}}
\item Statements \ref{itm:at_least_four}, \ref{itm:exactly_seven}, and \ref{itm:not_more_than_two} \answer{\underline{consistent}}
\end{exercises}

\noindent \problempart Consider the following set of statements.
\begin{enumerate}[label=(\alph*)]
\item \label{itm:allmortal} All people are mortal.
\item \label{itm:socperson} Socrates is a person.
\item \label{itm:socnotdie} Socrates will never die.
\item \label{itm:socmortal} Socrates is mortal.
\end{enumerate}
Which combinations of statements form consistent sets? Mark each “consistent” or “inconsistent.”
\begin{exercises}
\item Statements \ref{itm:allmortal}, \ref{itm:socperson}, and \ref{itm:socnotdie}  \answer{\underline{Inconsistent}}
\item Statements \ref{itm:socperson}, \ref{itm:socnotdie}, and \ref{itm:socmortal} \answer{\underline{Inconsistent}}
\item Statements \ref{itm:socperson} and \ref{itm:socnotdie} \answer{\underline{Consistent}}
\item Statements \ref{itm:allmortal} and \ref{itm:socmortal} \answer{\underline{Consistent}}
\item Statements \ref{itm:allmortal}, \ref{itm:socperson}, \ref{itm:socnotdie}, and \ref{itm:socmortal} \answer{\underline{Inconsistent}} 
\end{exercises}

\noindent \problempart \label{pr.EnglishCombinations} Which of the following is possible? If it is possible, give an example. If it is not possible, explain why.


\begin{longtabu}{p{.1\linewidth}p{.9\linewidth}}
\textbf{Example}: & A valid argument that has one false premise and one true premise.\\
\textbf{Answer}: & Possible: Example: If Taylor Swift were a kangaroo, she would be a marsupial (true). Taylor Swift is a kangaroo. (False.) Therefore Taylor Swift is a marsupial (false.)\\ &Remember, if an argument is valid, the only thing that can't happen is for it to have all true premises and a false conclusion. So if you don't specify a false conclusion anything is possible.\\
\end{longtabu}



\begin{exercises}
\item A false tautology. 

\answer{Impossible. Tautologies, by definition, are always true.}

\item A valid argument that has a false conclusion

\answer{\underline{Possible}. Example: If grass is green, then I am the pope. (False) Grass is green. (True) \therefore  I am the pope. (False)}

\item A valid argument, the conclusion of which is a contradiction

\answer{\underline{Possible}. The conclusion is always false, but if the premises are also always false, you are fine. Example: If A, then not A. \therefore If B, then not B. \\}

\item An invalid argument, the conclusion of which is a tautology

\answer{\underline{Impossible}. If the conclusion is always true, then the there is no way for all the premises to be true and conclusion false.\\}

\item A tautology that is contingent

\answer{\underline{Impossible}. Contradictions, contingencies, and tautologies are exclusive categories. If you are one, you can't be either of the others. \\}


\item Two logically equivalent sentences, both of which are tautologies

\answer{\underline{Possible} In fact, all tautologies are logically equivalent. Logically equivalent sentences always have the same truth value, and all tautologies are always true. \\}


\item Two logically equivalent sentences, one of which is a tautology and one of which is contingent

\answer{\underline{Impossible}. A tautology is always true, but contingent sentences can be false. Therefore they can have different truth values. \\}


\item Two logically equivalent sentences that together are an inconsistent set

\answer{\underline{Possible} Two contradictions are logically equivalent, however it is impossible for them to both be true, because it is impossible for either one to be true. \\}


\item A consistent set of sentences that contains a contradiction

\answer{\underline{Impossible}. The contradiction can never be true, so the whole set cannot never all be true. \\}


\item An inconsistent set of sentences that contains a tautology
\answer{\underline{Possible}. Example: A, Not A, If A then A.} 
\end{exercises}

\noindent \problempart Which of the following is possible? If it is possible, give an example. If it is not possible, explain why.
\answer{All answers, except for the last question, are by Ben Sheredos}
\begin{exercises}
\item A valid argument, whose premises are all tautologies, and whose conclusion is contingent
\answer{Not Possible. If the argument is valid, then the conclusion must be true if the premises are true. If the premises are \textit{tautologies}, then the premises are \textit{always} true, and so the conclusion also must always be true.}

\item A valid argument with true premises and a false conclusion
\answer{ \textit{Absolutely not!} This contradicts the very definition of a valid argument.
}
\item A consistent set of sentences that contains two sentences that are not logically equivalent
\answer{ Most definitely. Here are two sentences that are consistent but not logically equivalent: ``Today is a Wednesday'' and ``I like pie.''
}
\item A consistent set of sentences, all of which are contingent
\answer{For sure. See the examples given in the previous answer. Both are contingent (sometimes it's not Wednesday today, and I might've hated pie.)
}
\item A false tautology
\answer{Not possible. By definition, a tautology is always true.
}
\item A valid argument with false premises
\answer{ Yup. Because validity only requires that \textit{if} the premises are true, \textit{then} the conclusion must be true. But all of them could be false, and the argument would remain valid. 
}
\item A logically equivalent pair of sentences that are not consistent
\answer{ Careful here. Our definition of consistency is that a set of statements are consistent if they could all be true at the same time. Well, consider the case of 2 statements which are logically equivalent, and which are both \textit{contradictions}. Neither can be true. So they cannot \textit{both} be true. So they are not consistent. 
}
\item A tautological contradiction
\answer{ Impossible. This is gibberish-nonsense.
}
\item A consistent set of sentences that are all contradictions
\answer{ Nope: see again \#7 above. If a set of statements contains nothing but contradictions, then none of them can be true. But if none of them can be true, then they cannot be true together, and so they cannot be consistent.
}

\item A valid argument, whose premises are all tautologies, and whose conclusion is contingent.
\answer{Impossible. If the conclusion is contingent, then it could be false, in which case you would have true premises and a false conclusion, which would make the argument invalid.}

\end{exercises}

\section*{Key Terms}
\begin{sortedlist}
\sortitem{Truth value}{} 	
\sortitem{Natural language}{}
\sortitem{Artificial language}{}
\sortitem{Formal language}{}
\sortitem{Tautology}{}
\sortitem{Contradiction}{}
\sortitem{Contingent statement}{}
\sortitem{Logically equivalent}{}
\sortitem{Contradictories}{}
\sortitem{Consistent}{}
\sortitem{Inconsistent}{}
\sortitem{Formal logic as concern for logical form}{}
\sortitem{Formal logic as strictly following rules}{}
\sortitem{Bivalent}{}
\end{sortedlist}

\part{Categorical Logic}
\label{part:cat_logic}
\include{tex/ch04-categoricalstatements}
\include{tex/ch05-categoricalsyllogisms} %Label for typesetting full chapter is at the start of the file. Uncomment to get the whole thing. 
\part{Sentential Logic} \label{part:sent_logic}
\chapter{Sentential Logic}
\markright{Chap. \ref{chap:SL}: Sentential Logic}
\label{chap:SL}


\iflabelexists{part:cat_logic} %There are two versions of the preamble for this chapter, one for books that include the chapters on categorical logic, and a generic one
{In Part \ref{part:cat_logic}, we introduced a system of logic that dealt with categorical statements, statements like ``All people are mortal'' or ``Some dogs have  fleas.'' The system developed there was somewhat formal, because it replaced some of the contents of ordinary English sentences with abstract symbols. In Part \ref{part:sent_logic}, we go the rest of the way, and replace all of ordinary English with abstract symbols, thus creating a fully artificial language. In the previous system, capital letters like $S$ and $P$ stood for categories, like ``dogs'' or ``things that have fleas.'' In the new system individual letters will stand for whole sentences, like ``Tom wants to go to the bookstore'' or ``The sky is blue.'' Because individual letters stand for sentences this kind of system is known as \textit{sentential logic}, a term which we will be able to precisely define on page \pageref{def:sentential_logic}. We will call the specific version of sentential logic we will be developing SL.}%this is the preamble for texts that include categorical logic
{This chapter introduces a logical language called SL. It is a version of \emph{sentential logic}, because the basic units of the language will represent statements, and a statement is usually given by a complete sentence in English.} %this  is the generic preamble







% ******************************************
%  * Section 6.1  Sentence Letters                          *
% ******************************************

\section{Sentence Letters}


\newglossaryentry{sentence letter}
{
name=sentence letter,
description={A single capital letter, used in SL to represent a statement.}
}

The most basic unit in our formal language SL is an individual capital letter---$A, B, C, D$, etc. These letters, called \textsc{\glspl{sentence letter}}, \label{def:sentence_letter} are used to represent individual statements. Remember in section \ref{def:statement}, we defined a statement as some bit of language that can be true or false, and listed all kinds of things that count as statements in English, from ``\emph{Tyrannosaurus rex} went extinct 65 million years ago'' to ``Lady Gaga is pretty.'' In SL, all these statements are reduced to single capital letters.

\newglossaryentry{translation key}
{
name=translation key,
description={A list that assigns English phrases or sentences to variable names. Also called a ``symbolization key''  or simply a ``dictionary.''}
}

Considered only as a symbol of SL, the letter $A$ could mean any statement. \iflabelexists{part:cat_logic}{%text if the term was defined in the cat logic section
So when translating from English into SL, it is important to provide a \gls{translation key}\label{def:translation_key}. Previously, we used translation keys to say assign the variables $S$, $M$, and $P$ to terms. (See page \pageref{def:translation_key}.) Now we will use them to assign sentences to sentence letters.}{%text if term is being defined for the first time
In order to specify what we mean, we need to provide a key saying what the sentence letters represent. We will call a list that assigns English phrases or sentences to variable names a \textsc{\gls{translation key}}.\label{def:translation_key} These are sometimes also called ``symbolization keys'' or simply just ``dictionaries.'' }

Consider this argument (recall that the portion of the passage in italics establishes the context, and is not part of the passage):

\begin{quotation}
\noindent \textit{A teacher is looking to see who has come to class} There is an apple on the desk. If there is an apple on the desk, then Jenny made it to class. Therefore, Jenny made it to class.
\end{quotation}

In canonical form, the argument would look like this:

\begin{earg}
\item[1.] There is an apple on the desk.
\item[2.] If there is an apple on the desk, then Jenny made it to class.
\item[] \textcolor{white}{.}\sout{\hspace{.8\linewidth}}\textcolor{white}{.} 
\item[$\therefore$] Jenny made it to class.
\end{earg}

A good symbolization key for this passage would look like this:

\begin{ekey}
\item[A:]There is an apple on the desk.
\item[B:]Jenny made it to class.
\end{ekey}

Why do the symbolization key this way? The argument we are looking at is obviously valid in English. In symbolizing it, we want to preserve the structure of the argument that makes it valid. We could have made each sentence in the original argument into its own letter. Then the symbolization key would look like this: 

\begin{ekey}
\item[A:]There is an apple on the desk.
\item[B:]If there is an apple on the desk, then Jenny made it to class.
\item[C:]Jenny made it to class.
\end{ekey}
But that would mean the argument would look like this:
\begin{earg}
\item[1.] $A$
\item[2.] $B$
\item[] \textcolor{white}{.}\sout{\hspace{.05\linewidth}}\textcolor{white}{.} 
\item[$\therefore$] $C$
\end{earg}
There is no necessary connection between some sentence $A$, which could be any statement, and some other sentences $B$ and $C$, which could also be anything. The structure of the argument has been completely lost in this translation.

The important thing about the argument is that the second premise is not merely \emph{any} statement, logically divorced from the other statement in the argument. The second premise contains the first premise and the conclusion \emph{as parts}. Our original symbolization key allows us to write the argument like this.

\begin{earg}
\item[1.] $A$
\item[2.] If $A$, then $B$.
\item[] \textcolor{white}{.}\sout{\hspace{.2\linewidth}}\textcolor{white}{.} 
\item[$\therefore$] $B$
\end{earg}
This preserves the structure of the argument that makes it valid, but it still makes use of the English expression ``If$\ldots$ then$\ldots$.'' Although we ultimately want to replace all of the English expressions with logical notation, this is a good start.

\newglossaryentry{atomic statement}
{
name=atomic statement,
description={A statement that does not have any other statements as proper parts.}
}

The individual sentence letters in SL are called atomic statements, because they are the basic building blocks out of which more complex sentences can be built. We can identify atomic statements in English as well. An \textsc{\gls{atomic statement}} \label{def:atomic_statement} is one that cannot be broken into parts that are themselves sentences. ``There is an apple on the desk'' is an atomic statement in English, because you can't find any proper part of it that forms a complete statement. For instance ``an apple on the desk'' is a noun phrase, not a complete statement. Similarly ``on the desk'' is a prepositional phrase, and not a statement, and ``is an'' is not any kind of phrase at all. This is what you will find no matter how you divide ``There is an apple on the desk.'' On the other hand you can find two proper parts of ``If there is an apple on the desk, then Jenny made it to class'' that are complete sentences: ``There is an apple on the desk'' and ``Jenny made it to class.'' As a general rule, we will want to use atomic sentences in SL (that is, the sentence letters) to represent atomic statement in English. Otherwise, we will lose some of the logical structure of the English sentence, as we have just seen. 

There are only 26 letters of the alphabet, but there is no logical limit to the number of atomic statement. We can use the same letter to symbolize different atomic statement by adding a subscript, a small number written after the letter. We could have a symbolization key that looks like this:
\begin{ekey}
\item[A$_1$:] The apple is under the armoire.
\item[A$_2$:] Arguments in SL always contain atomic sentences.
\item[A$_3$:] Adam Ant is taking an airplane from Anchorage to Albany.
\item[$\vdots$]
\item[A$_{294}$:] Alliteration angers otherwise affable astronauts.
\end{ekey}
Keep in mind that each of these is a different sentence letter. When there are subscripts in the symbolization key, it is important to keep track of them.


% ******************************************
%  * Sentential Connectives		                          *
% ******************************************

\section{Sentential Connectives}


The previous section introduced the basic elements of SL, the sentence letters. But when we were looking at the argument involving Jenny and the apple, we saw that the best way to write a dictionary for the argument left the words ``if'' and ``then'' in English. In this section we will introduce ways to connect the sentence letters together that will allow us to form a complete artificial language.   

\newglossaryentry{sentential connective}
{
name=sentential connective,
description={A logical operator in SL used to combine sentence letters into larger sentences.}
}

\newglossaryentry{logical constant}
{
name=logical constant,
description={A symbol whose meaning is fixed by a formal language. Sometimes these are just called ``logical symbols.'' They are contrasted with \textsc{non-logical symbols}.}
}

\newglossaryentry{nonlogical symbol}
{
name=nonlogical symbol,
description={A symbol whose meaning is not fixed by a formal language.}
}

The symbols used to connect sentence letters are called \textsc{\glspl{sentential connective}} \label{def:sentential_connective}, naturally enough. SL uses five sentential connectives: \eand, \eor, \enot, \eif, and \eiff. To write the sentence about Jenny and the apple we use the symbol ``\eif.'' Using the dictionary above, ``If there is an apple on the desk, then Jenny made it to class'' becomes $A \eif B$. Table \ref{table:sentential_connectives} summarizes the meaning of the five sentential connectives.

The sentential connectives are a kind of \textsc{\gls{logical constant}},\label{def:logical_constant} because their meaning is fixed by the formal language that we have chosen. The other logical constants in SL are the parentheses. These are the things we cannot change in the symbolization key. The sentence letters, by contrast, are \textsc{\glspl{nonlogical symbol}}, \label{def:nonlogical_symbol} because their meaning can change as we change the symbolization key. We can decide that $A$ stands for ``Arthur is an aardvark'' in one translation key and ``Apu is an anthropologist'' in the next. But we can't say that the  the $\enot$ symbol will mean ``not'' in one argument and ``perhaps'' in another.

The subsections below describe each connective in more detail.

\begin{table}
\begin{mdframed}[style=mytablebox]
\begin{tabu}{p{.1\linewidth}p{.3\linewidth}p{.3\linewidth}}
\underline{Symbol}&\underline{What it is called}&\underline{What it means}\\
\enot&negation&``It is not the case that$\ldots$''\\
\eand&conjunction&``Both $\ldots$\ and $\ldots$''\\
\eor&disjunction&``Either $\ldots$\ or $\ldots$''\\
\eif&conditional&``If $\ldots$\ then $\ldots$''\\
\eiff&biconditional&``$\ldots$ if and only if $\ldots$''\\
\end{tabu}
\end{mdframed}
\caption{The Sentential Connectives.}
\label{table:sentential_connectives}
\end{table}

%%%%%%%%%%%%%%%%%% 2.2.1 Negation

\subsection{Negation}
Consider how we might symbolize these sentences:
\begin{earg}
\item[\ex{not1}] Mary is in Barcelona.
\item[\ex{not2}] Mary is not in Barcelona.
\item[\ex{not3}] Mary is somewhere other than Barcelona.
\end{earg}

In order to symbolize sentence \ref{not1}, we will need one sentence letter. We can provide a symbolization key:

\begin{ekey}
\item[B:]Mary is in Barcelona.
\end{ekey}

Note that here we are giving $B$ a different interpretation than we did in the previous section. The symbolization key only specifies what $B$ means \emph{in a specific context}. It is vital that we continue to use this meaning of $B$ so long as we are talking about Mary and Barcelona. Later, when we are symbolizing different sentences, we can write a new symbolization key and use $B$ to mean something else.

\newglossaryentry{negation}
{
name=negation,
description={The symbol \enot, used to represent words and phrases that function like the English word ``not''.}
}

Now, sentence \ref{not1} is simply $B$. Sentence \ref{not2} is obviously related to sentence \ref{not1}: it is basically \ref{not1} with a ``not'' added. We could put the sentence partly our symbolic language by writing ``Not $B$.'' This means we do not want to introduce a different sentence letter for \ref{not2}. We just need a new symbol for the ``not'' part. Let's use the symbol `\enot,' which we will call \textsc{\gls{negation}}. \label{def:negation} Now we can translate `Not $B$' to $\enot B$. 

Sentence \ref{not3} is about whether or not Mary is in Barcelona, but it does not contain the word ``not.'' Nevertheless, it is obviously logically equivalent to sentence \ref{not2}. They both say that if you are looking for Mary, you shouldn't look in Barcelona. Remember that in section \ref{def:logical_equivalence}, we said that two sentences in English are logically equivalent if they always have the same truth value. For our purposes, this means that they basically say the same thing. It is clear then that \ref{not2} and \ref{not3} are logically equivalent, so we can translate them both as $\enot B$.



Consider these further examples:
\begin{earg}
\item[\ex{not4}] The widget can be replaced if it breaks.
\item[\ex{not5}] The widget is irreplaceable.
\item[\ex{not5b}] The widget is not irreplaceable.
\end{earg}


If we let $R$ mean ``The widget is replaceable'', then sentence \ref{not4} can be translated as $R$. Sentence \ref{not5} means the opposite of sentence \ref{not4}, so we can translate it 
$\enot R$. Sentence \ref{not5b} adds another negation to sentence \ref{not5}. We know, as competent English speakers, that the two negations cancel each other out, so that sentence 
\ref{not5b} is equivalent to sentence \ref{not4}. But the fact that two negations cancel each other out is a part of the logic of English that we actually want to capture with our formal 
language SL. So we will represent the two negations in sentence \ref{not5b} as two negations in SL: $\enot \enot R$. We will now have to be sure that in SL the sentences $R$ and $\enot 
\enot R$ mean the same thing.

As the above examples begin to indicate, English has all kinds of ways to negate a sentence.  Sometimes we use an explicit ``not.'' Sometimes we use a prefix like the ``ir-'' in ``irreplaceable.'' SL has just one way to form a negation: slap a \enot in front of the sentence. There is an English expression, however, that always occurs in the same place in an English sentence as the \enot occurs in the sentence SL. The English phrase is ``It is not the case that.'' Although this phrase sounds awkward, it always occurs in front of the sentence it is negating, just as the symbol \enot does. This makes it useful in translating sentences from SL back into English. $\enot R$ can be translated ``it is not the case that this widget is replaceable.'' In the earlier example, $\enot B$ can be translated ``It is not the case that Mary is in Barcelona.'' 

\factoidbox{
A sentence can be symbolized as $\enot\script{A}$ can always be paraphrased in English as ``It is not the case that \script{A}.''
}

Sometimes negations in English do not function as neatly as the \enot does in SL, because two things aren't perfect opposites. Consider these sentences:

\begin{earg}
\item[\ex{not6}] Elliott is happy.
\item[\ex{not7}] Elliott is unhappy.
\end{earg}


If we let $H$ mean ``Elliot is happy'', then we can symbolize sentence \ref{not6} as $H$, but does \ref{not7} really mean the same thing as $\enot H$? Saying ``Elliott is unhappy'' 
indicates that Elliott is actively sad. But $\enot H$ can be paraphrase as simply ``It is not the case that Elliott is happy,'' which might merely mean that Elliott is just feeling 
neutral. As we saw on page \pageref{def:bivalent}, the logics we discuss in this textbook are \emph{bivalent}. Statements are only either true or false. Everything is in black and white, 
and issues like Elliott's fine gradations in mood cannot be directly represented in our system. So in SL, sentences \ref{not6} and \ref{not7} would generally be represented by separate 
sentence letters.

One way of capturing the meaning of a sentential connective is to make a table which shows how the connective changes the meaning of the sentences it is applied to. The negation simply 
reverses the truth value of any sentence it is put in front of. For any sentence \script{A}: If \script{A} is true, then \enot\script{A} is false. If \enot\script{A} is true, then 
\script{A} is false. Using T for true and F for false, we can summarize this in a \emph{characteristic truth table} for negation:

\begin{center}
\begin{tabular}{c|c}
\script{A} & \enot\script{A}\\
\hline
T & F\\
F & T 
\end{tabular}
\end{center}
We will discuss truth tables at greater length in the next chapter.

%%%%%%%%%%%%%%%%%% 2.2.2 Conjunction

\subsection{Conjunction}
Consider these sentences:
\begin{earg}
\item[\ex{and1}]Adam is athletic.
\item[\ex{and2}]Barbara is athletic.
\item[\ex{and3}]Adam is athletic, and Barbara is also athletic.
\end{earg}

We will need separate sentence letters for \ref{and1} and \ref{and2}, so we define this symbolization key:
\begin{ekey}
\item[A:] Adam is athletic.
\item[B:] Barbara is athletic.
\end{ekey}


\newglossaryentry{conjunction}
{
name=conjunction,
description={The symbol \eand, used to represent words and phrases that function like the English word ``and.''}
}

\newglossaryentry{conjunct}
{
name=conjunct,
description={A sentences joined to another by a conjunction.}
}

Sentence \ref{and1} can be symbolized as $A$. Sentence \ref{and2} can be symbolized as $B$. Sentence \ref{and3} can be paraphrased as ``$A$ and $B$.'' In order to fully symbolize this sentence, we need another symbol. We will use \eand. We translate ``$A$ and $B$'' as $A\eand B$. The logical connective \eand is called the \textsc{\gls{conjunction}}, \label{def:conjunction} and $A$ and $B$ are each called \textsc{\glspl{conjunct}}. \label{def:conjunct}

Notice that we make no attempt to symbolize ``also'' in sentence \ref{and3}. Words like ``both'' and ``also'' function to draw our attention to the fact that two things are being conjoined. They are not doing any further logical work, so we do not need to represent them in SL.

Some more examples:
\begin{earg}
\item[\ex{and4}]Barbara is athletic and energetic.
\item[\ex{and5}]Barbara and Adam are both athletic.
\item[\ex{and6}]Although Barbara is energetic, she is not athletic.
\item[\ex{and7}]Barbara is athletic, but Adam is more athletic than she is.
\end{earg}

Sentence \ref{and4} is obviously a conjunction. The sentence says two things about Barbara, that she is athletic and engergetic. In English, it is acceptable to only say ``Barbara'' once, even though two statements are being made about her. Because of this, you might be tempted just to translate the first part of the English sentence with a sentence letter and leave the second part dangling.  $B$ would then stand for ``Barbara is athletic,'' and the full sentence would be ``$B$ and energetic.'' But this doesn't work, because ``and energetic'' isn't a statement. On its own, it can't be true or false. We should instead paraphrase the sentence as ``$B$ and Barbara is energetic.'' Now we need to add a sentence letter to the symbolization key. Let $E$ mean ``Barbara is energetic.'' Now the sentence can be translated as $B \eand E$.

\factoidbox{
A sentence can be symbolized as $\script{A} \eand \script{B}$ if it can be paraphrased in English as `Both \script{A}, and \script{B}.' Each of the conjuncts must be a sentence.
}

Sentence \ref{and5} says one thing about two different subjects. It says of both Barbara and Adam that they are athletic, and in English we use the word ``athletic'' only once. In translating to SL, it is important to realize that the sentence can be paraphrased as, ``Barbara is athletic, and Adam is athletic.'' This translates as $B \eand A$.

Sentence \ref{and6} is a bit more complicated. The word ``although'' sets up a contrast between the first part of the sentence and the second part. Nevertheless, the sentence says both that Barbara is energetic and that she is not athletic. In order to make each of the conjuncts an atomic statement, we need to replace ``she'' with ``Barbara.''

So we can paraphrase sentence \ref{and6} as, ``\emph{Both} Barbara is energetic, \emph{and} Barbara is not athletic.'' The second conjunct contains a negation, so we paraphrase further: ``\emph{Both} Barbara is energetic \emph{and} \emph{it is not the case that} Barbara is athletic.'' This translates as $E \eand \enot B$.

Sentence \ref{and7} contains a similar contrastive structure. It is irrelevant for the purpose of translating to SL, so we can paraphrase the sentence as ``\emph{Both} Barbara is athletic, \emph{and} Adam is more athletic than Barbara.'' (Notice that we once again replace the pronoun ``she'' with her name.) How should we translate the second conjunct? We already have the sentence letter $A$ which is about Adam's being athletic and $B$ which is about Barbara's being athletic, but neither is about one of them being more athletic than the other. We need a new sentence letter. Let $R$ mean ``Adam is more athletic than Barbara.'' Now the sentence translates as $B \eand R$.

\factoidbox{Sentences that can be paraphrased ``\script{A}, but \script{B}'' or ``Although \script{A}, \script{B}'' are best symbolized using conjunction  \script{A} \eand \script{B}.}

It is important to keep in mind that the sentence letters $A$, $B$, and $R$ are atomic statements. Considered as symbols of SL, they have no meaning beyond being true or false. We have used them to symbolize different English language sentences that are all about people being athletic, but this similarity is completely lost when we translate to SL. No formal language can capture all the structure of the English language, but as long as this structure is not important to the argument there is nothing lost by leaving it out.

As with the negation, we can understand the meaning of the conjunction by making a table that shows how the conjunction affects the truth value of the  sentences it is bringing together. 
For any sentences \script{A} and \script{B}, \script{A} \eand \script{B} is true if and only if both \script{A} and \script{B} are true. We can summarize this in the {characteristic truth table} for conjunction:
\begin{center}
\begin{tabular}{c|c|c}
\script{A} & \script{B} & \script{A} \eand \script{B}\\
\hline
T & T & T\\
T & F & F\\
F & T & F\\
F & F & F
\end{tabular}
\end{center}

Conjunction is symmetrical because we can swap the conjuncts without changing the truth value of the sentence. Regardless of what \script{A} and \script{B} are, \script{A}\eand\script{B} is logically equivalent to \script{B} \eand \script{A}.


%%%%%%%%%%%%%%%%%%%% 2.2.3 disjunction

\subsection{Disjunction}
Consider these sentences:
\begin{earg}
\item[\ex{or1}]Either Denison will play golf with me, or he will watch movies.
\item[\ex{or2}]Either Denison or Ellery will play golf with me. 
\end{earg}

For these sentences we can use this symbolization key:

\begin{ekey}
\item[D:] Denison will play golf with me.
\item[E:] Ellery will play golf with me.
\item[M:] Denison will watch movies.
\end{ekey}

\newglossaryentry{disjunction}
{
name=disjunction,
description={The symbol \eor, used to represent words and phrases that function like the English word ``or'' in its inclusive sense.}
}

\newglossaryentry{disjunct}
{
name=disjunct,
description={A sentences joined to another by a disjunction.}
}



Sentence \ref{or1} is ``Either $D$ or $M$.'' To fully symbolize this, we introduce a new symbol. The sentence becomes $D \eor M$. The $\eor$ connective is called \textsc{\gls{disjunction}}, \label{def:disjunction} and $D$ and $M$ are called \textsc{\glspl{disjunct}}. \label{def:disjunct}

Sentence \ref{or2} is only slightly more complicated. There are two subjects, but the English sentence only gives the verb once. In translating, we can paraphrase it as ``Either Denison will play golf with me, or Ellery will play golf with me.'' Now it obviously translates as $D \eor E$.


\factoidbox{
A sentence can be symbolized as $\script{A}\eor\script{B}$ if it can be paraphrased in English as ``Either \script{A} or \script{B}.'' Each of the disjuncts must be a sentence.
}


\newglossaryentry{exclusive or}
{
name=exclusive or,
description={A kind of disjunction that excludes the possibility that both disjuncts are true. The exclusive or says ``This or that, but not both.''}
}

\newglossaryentry{inclusive or}
{
name=inclusive or,
description={A kind of disjunction that allows for the possibility that both disjuncts are true. The inclusive or says ``This or that, or both.''}
}


The English word ``or'' is somewhat ambiguous. Sometimes in English, when we say ``this or that,'' we mean that either option is possible, but not both. For instance, if a  restaurant menu says, ``Entr\'ees come with either soup or salad'' we naturally assume you can have soup, or you can have salad; but, if you want \emph{both} soup \emph{and} salad, then you will have to pay extra. This kind of disjunction is called an \textsc{\gls{exclusive or}} \label{def:exclusive_or}, because it excludes the possibility that both disjuncts are true. 
 
At other times, the word ``or'' allows for the possibility that both disjuncts might be true. This is probably the case with sentence \ref{or2}, above. I might play with Denison, with 
Ellery, or with both Denison and Ellery. Sentence \ref{or2} merely says that I will play with \emph{at least} one of them. The \textsc{\gls{inclusive or}}\label{def:inclusive_or} is the 
kind of disjunction that allows for the possibility that both disjuncts are true. The inclusive or says ``This or that, or both.''

To goal of a formal language is to remove ambiguity, so we need to pick one of these ors. SL follows tradition and uses the symbol $\eor$ to represent an \emph{inclusive or}. This winds up being reflected in the characteristic truth table for the $\eor$. The sentence $D \eor E$ is true if $D$ is true, if $E$ is true, or if both $D$ and $E$ are true. It is false only if both $D$ and $E$ are false. The truth table looks like this:

\begin{center}
\begin{tabular}{c|c|c}
\script{A} & \script{B} & \script{A} \eor \script{B} \\
\hline
T & T & T\\
T & F & T\\
F & T & T\\
F & F & F
\end{tabular}
\end{center}

Like conjunction, disjunction is symmetrical. \script{A} \eor \script{B} is logically equivalent to \script{B} \eor \script{A}.


%%%%%%%%%%%%%%%%% 2.2.4 conditional

\subsection{Conditional}

\newglossaryentry{conditional}
{
name=conditional,
description={The symbol \eif, used to represent words and phrases that function like the English phrase ``if \ldots then.''}
}

\newglossaryentry{antecedent}
{
name=antecedent,
description={The sentence to the left of a conditional..}
}

\newglossaryentry{consequent}
{
name=consequent,
description={The sentence to the right of a conditional.}
}

We already met the conditional at the start of this section, when we were discussing the sentence ``If there is an apple on the table, Jenny made it to class,'' which became $A \eif B$. The symbol $\eif$ is called a \textsc{\gls{conditional}}. \label{def:conditional} The sentence on the left-hand side of the conditional ($R$ in this example) is called the \textsc{\gls{antecedent}}. \label{def:antecedent}.  The sentence on the right-hand side ($B$) is called the \textsc{\gls{consequent}}. \label{def:consequent} 
	
Like the English word ``or,'' the English phrase ``if\ldots then\ldots'' has some ambiguity. Consider our original example, ``If there is an apple on the table, Jenny made it to class.'' The statements tells us what we should infer if there is an apple on the table, but what if there \emph{isn't} an apple on the table. Does that guarantee that Jenny did not make it to class? It could be that an apple on the table is a clear sign that Jenny made it to class, because no one else would put an apple on the table, but nevertheless Jenny sometimes comes to class without putting an apple on the table. 

We can get a good sense of the decision we face if we try to write up the characteristic truth table for the conditional. The first two lines are easy. The sentence``If \script{A}, then \script{B}'' means that if \script{A} is true, then so is \script{B}. This would be confirmed by the situation where both \script{A} and \script{B} are true, but falsified by the situation where \script{A} is true and \script{B} is false. In terms of our example, if we came to class and found the apple there, but Jenny absent, we would know that the statement ``If there is an apple on the table, Jenny made it to class'' is false. But if we came to class and found both Jenny and the apple present, we could say that the statement ``If there is an apple on the table, Jenny made it to class'' is true. That gives us this much of a truth table.


\begin{center}
\begin{tabular}{c|c|c}
\script{A} & \script{B} & \script{A}\eif\script{B}\\
\hline
T & T & T\\
T & F & F\\
F & T & ?\\
F & F & ?
\end{tabular}
\end{center}

How do we fill in the question marks in the last two lines?  In real life, we would generally make judgments on a case by case basis, relying heavily on the context we are in. But for a formal language we just want to lay down a simple rule. The traditional solution for sentential logic is to say that the conditional is what logicians call a ``material conditional.'' If the antecedent of a material conditional is false, then the whole statement is automatically true, regardless of the truth value of \script{B}. In short, \script{A} \eif \script{B} is false if and only if \script{A} is true and \script{B} is false. We can summarize this with a characteristic truth table for the conditional.

\begin{center}
\begin{tabular}{c|c|c}
\script{A} & \script{B} & \script{A}\eif\script{B}\\
\hline
T & T & T\\
T & F & F\\
F & T & T\\
F & F & T
\end{tabular}
\end{center}

The conditional is asymmetrical. You cannot swap the antecedent and consequent without changing the meaning of the sentence, because \script{A} \eif \script{B} and \script{B} \eif \script{A} are not logically equivalent.

%\begin{earg}
%\item[\ex{if3}] Everytime a bell rings, an angel earns its wings.
%\item[\ex{if4}] Bombs always explode when you cut the red wire.
%\end{earg}

Not all sentences of the form ``If$\ldots$, then$\ldots$'' are conditionals. Consider this sentence:

\begin{earg}
\item[\ex{if5}] If anyone wants to see me, then I will be on the porch.
\end{earg}

When I say this, it means that I will be on the porch, regardless of whether anyone wants to see me or not---but if someone did want to see me, then they should look for me there. If we let $P$ mean ``I will be on the porch,'' then sentence \ref{if5} can be translated simply as $P$.

%%%%%%%%%%%%%%%% 6.2.5 Biconditional

\subsection{Biconditional}

\newglossaryentry{biconditional}
{
name=biconditional,
description={A sentential connective, written as double headed arrow, $\eiff$, used to represent a situation where $A$ implies  $B$ and $B$ implies $A$. This is also the situation where $A$ and $B$ are logically equivalent. This is often expressed by the English phrase ``if and only if.''}
}

The conditional was an asymmetric connective. The sentence $A \eif B$ does not mean the same thing as the sentence $B \eif A$. It is convenient to have a single symbol that combines the meaning of these two sentences. The \textsc{\gls{biconditional}}\label{def:bicondional}---written as double headed arrow, $\eiff$---is a sentential connective used to represent a situation where $A$ implies $B$ and $B$ implies $A$. 

To draw up the characteristic truth table for the biconditional, we need to think about the situations where $A \eif B$ and $B \eif A$ are false. The sentence $A \eif B$ is only false when $A$ is true and $B$ is false. For $B \eif A$ the reverse is true. It is false when $B$ is true and $A$ is false. Our biconditional $A \eiff B$ needs to avoid both of these situations to be true, because it is only true when $A \eif B$ and $B \eif A$ are true. This, then, is the characteristic truth table for the biconditional. It says that the biconditional is true when the truth values of the two sides match.

\begin{center}
\begin{tabular}{c|c|c}
\script{A} & \script{B} & \script{A} \eiff \script{B}\\
\hline
T & T & T\\
T & F & F\\
F & T & F\\
F & F & T
\end{tabular}
\end{center}

If the bioconditional holds between two sentences, we can that the two sentences are logically equivalent. Back on page \pageref{def:logical_equivalence}, we said that two sentences were logically equivalent if they always  have the same truth value. That is exactly what is happening here. 


% ******************************************
%  * 		More Complicated Translations                     *
% ******************************************

\section{More Complicated Translations}

\iflabelexists{part:cat_logic} %There are two versions of this passage, one for books that include the chapters on categorical logic, and a generic one
{Back in section \ref{sec:transformation}, we saw that the system of categorical logic we were studying at the time could actually represent a large range of sentences in ordinary English, even though it only had the quantifiers ``All'' and ``some'' plus negation. In this section, we will see that something similar happens with SL. There is actually a lot we can cover, even though we only have five connectives. }%this is the preamble for texts that include categorical logic
{The previous section introduced the five sentential connectives. Now we will look at some trickier translations involving those connectives}%this is the generic preamble


\subsection{Combining connectives}

\iflabelexists{part:cat_logic} %There are two versions of this passage, one for books that include the chapters on categorical logic, and a generic one
{In our system of categorical logic, we just had four kinds of sentences---A, E, I, and O---and if we wanted to combine them, the only way to do that would be to form a syllogism. In SL, we can combine an unlimited number of connectives together into a single sentence to express complicated ideas that couldn't be represented by Aristotelean logic. }%this is the preamble for texts that include categorical logic
{A single sentence in SL can use multiple connectives.} %this is the generic preamble

Consider the English sentence ``If it is not raining, we will have a picnic.'' There are two aspects of this sentence we will want to represent with sentential connectives in SL, the ``if\ldots then\ldots'' structure and the negation in the first part of the sentence. The rest of the sentence can be represented by these sentence letters

\begin{ekey}
\item[$A$:] It is raining.
\item[$M$:] We will have a picnic.
\end{ekey}

We can then translate the whole sentence into SL like this: $\enot A \eif B$. We can make sentences as complicated as we want this way, even to the point where the equivalent English sentence would be impossible to follow. The sentence $\enot (P \eand Q) \eif  [(R \eor S) \eiff \enot (T \eand U)]$ is perfectly acceptable in SL, even if any English sentence it translates into would be a monster. This is part of the power of a complete formal language like SL, but it is also why arguments in SL begin to resemble the ob/ob mouse more than they resemble any argument you might encounter in the wild. (See page \pageref{fig:ob_ob_mouse}) 

Although sentences in SL can be as long as you like, you can't just combine symbols any old way. There is a specific set of rules you have to follow. These are outlined in section \ref{recursive_syntax_for_SL}, below.

The fact that we can write these more complicated sentences means we can actually do without some of the connectives we have given ourselves in SL. For instance, we don't really need the biconditional. Any sentence of the form $\script{A} \eiff \script{B}$ is going to be equivalent to the sentence $(\script{A} \eif \script{B}) \eand (\script{B} \eif \script{A})$. This just follows from the way we defined the biconditional earlier. Nevertheless, tradition and convenience mandate that we give the biconditional a separate symbol.

\subsection{Unless}

Because our connectives can be put together in different ways, some English sentences can be represented equally well by multiple sentences in SL. English sentences involving the word ``unless'' are a case in point. 

\begin{earg}
\item[\ex{unless1}] Unless you wear a jacket, you will catch cold. 
\item[\ex{unless2}] You will catch cold unless you wear a jacket. 
\end{earg}

These are basically two different version of the same English sentence. The only difference is that in one case, the ``unless'' clause comes first, and in the other it comes second. Let $J$ mean ``You will wear a jacket'' and let $C$ mean ``You will catch a cold.'' We can paraphrase sentence \ref{unless1} as ``Unless $J$, $C$.'' This means that if you do not wear a jacket, then you will catch cold. With this in mind, we might translate it as $\enot J \eif C$. It also means that if you do not catch a cold, then you must have worn a jacket; with this in mind, we might translate it as $\enot C \eif J$.

Which of these is the correct translation of sentence \ref{unless1}? Both translations are correct, because the two translations are logically equivalent in SL. Sentence \ref{unless2}, in English, is logically equivalent to sentence \ref{unless1}. So, it also can be translated as either $\enot J \eif D$ or $\enot D \eif J$.

When symbolizing sentences like sentence \ref{unless1} and sentence \ref{unless2}, it is easy to get turned around. We have two different versions of the English sentence and two different versions of the sentence in SL. The important thing to see here is that none of these sentences are equivalent to $J \eif \enot D$. The negated statement must be the antecedent to the conditional. 

If this is too many options to keep track of, there is a simpler alternative. It turns out that any ``unless'' statement is actually equivalent to an ``or'' statement. Both statements \ref{unless1} and  \ref{unless2} mean that you will wear a jacket or---if you do not wear a jacket---then you will catch a cold. So we can translate them as $J \eor D$. (You might worry that the ``or'' here should be an \emph{exclusive or}. However, the sentences do not exclude the possibility that you might \emph{both} wear a jacket \emph{and} catch a cold; jackets do not protect you from all the possible ways that you might catch a cold.)


\factoidbox{
If a sentence can be paraphrased as ``Unless \script{A}, \script{B},'' then it can be symbolized as $\script{A}\eor\script{B}$.
}

\subsection{Only}

\iflabelexists{part:cat_logic} %There are two versions of this passage, one for books that include the chapters on categorical logic, and a generic one
{In section \ref{sec:transformation}, we saw that the word ``only'' could reverse the meaning of a statement in Mood A. ``All dogs are mammals'' means something different than ``Only dogs are mammals,'' the first one is true but the second one is false. Something similar happens with conditional statements in SL.}%this is the preamble for texts that include categorical logic
{[The word ``only'' can reverse the meaning of a conditional sentence in SL.]}%this is the generic preamble
For the following sentences, let $R$ mean ``You will cut the red wire'' and $B$ mean ``The bomb will explode.''

\begin{earg}
\item[\ex{if1}] If you cut the red wire, then the bomb will explode.
\item[\ex{if2}] The bomb will explode only if you cut the red wire.
\end{earg}

Sentence \ref{if1} can be translated partially as ``If $R$, then $B$.'' Sentence \ref{if2} is also a conditional. Since the word ``if'' appears in the second half of the sentence, it might be tempting to symbolize this in the same way as sentence \ref{if1}. That would be a mistake.

The conditional $R\eif B$ says that \emph{if} $R$ were true, \emph{then} $B$ would also be true. It does not say that you cutting the red wire is the \emph{only} way that the bomb could explode. Someone else might cut the wire, or the bomb might be on a timer. The sentence $R\eif B$ does not say anything about what to expect if $R$ is false. Sentence \ref{if2} is different. It says that the only conditions under which the bomb will explode involve you having cut the red wire; i.e., if the bomb explodes, then you must have cut the wire. As such, sentence \ref{if2} should be symbolized as $B \eif R$.

It is important to remember that the connective $\eif$ says only that, if the antecedent is true, then the consequent is true. It says nothing about the \emph{causal} connection between the two events. Translating sentence \ref{if2} as $B \eif R$ does not mean that the bomb exploding would somehow have caused you cutting the wire. Both sentence \ref{if1} and \ref{if2} suggest that, if you cut the red wire, you cutting the red wire would be the cause of the bomb exploding. They differ on the \emph{logical} connection. If sentence \ref{if2} were true, then an explosion would tell us---those of us safely away from the bomb---that you had cut the red wire. Without an explosion, sentence \ref{if2} tells us nothing.

\factoidbox{
The paraphrased sentence ``\script{A} only if \script{B}'' is logically equivalent to ``If \script{A}, then \script{B}.''
}

Things can get a bit more complicated, because English also allows you to reverse the order of the clauses. Think about this sentence

\begin{earg}
\item[\ex{if3}] The bomb will explode, if you cut the red wire
\end{earg}

This is just sentence \ref{if1} with the order of the clauses reversed, so it still means $R \eif B$. Changing the order of the English clauses does not change the sentence in SL, but adding the word ``only'' does.

If this gets confusing, just remember this rule: 

\factoidbox{
``If\ldots'' introduces the antecedent. ``Only if\ldots'' introduces the consequent. 
}

Because ``if'' and ``only if'' have opposite meanings, when we put them together, we get the biconditional. Consider these sentences:
\begin{earg}
\item[\ex{iff1}] The figure on the board is a triangle only if it has exactly three sides.
\item[\ex{iff2}] The figure on the board is a triangle if it has exactly three sides.
\item[\ex{iff3}] The figure on the board is a triangle if and only if it has exactly three sides.
\end{earg}

Let $T$ mean ``The figure is a triangle'' and $S$ mean ``The figure has three sides.'' Sentence \ref{iff1}, for reasons discussed above, can be translated as $T\eif S$. Sentence \ref{iff2} is importantly different. It can be paraphrased as ``If the figure has three sides, then it is a triangle.'' So it can be translated as $S\eif T$.

Sentence \ref{iff3} says that $T$ is true \emph{if and only if} $S$ is true; we can infer $S$ from $T$, and we can infer $T$ from $S$.  In other words, \ref{iff3} is equivalent to $T\eif S$ and $S\eif T$, which is the same as $T \eiff S$

A final way to think about the way ``only'' effects a conditional sentence is to think about the  difference between necessary and sufficient conditions. In a way, the terms are pretty much self explanatory. Nevertheless, it is really easy to get them confused, to the extent that even professional logicians and trained philosophers can get them mixed up. 


\newglossaryentry{necessary condition}
{
name=necessary condition,
description={A condition that must be true in order for something else to be, generally contrasted with a \textit{sufficient condition}.}
}

A \textsc{\gls{necessary condition}}\label{def:necessary_condition} is one that is needed for something else to be true, just like the name says. Having gas in the tank is a \textit{necessary} condition for the car to move. It just doesn't go anywhere without gas. However, having gas in the tank isn't \textit{all you need} to get the car moving. You also have to put the key in the  ignition and turn it. 

\newglossaryentry{sufficient condition}
{
name=sufficient condition,
description={A condition that is all you need for something to be true, generally contrasted with a \textit{necessary condition}.}
}

A \textsc{\gls{sufficient condition}}\label{def:sufficient_condition}, on the other hand, is \textit{all you need} for something else to be true. If something is a dog, that is a \textit{sufficient} condition for it to be a mammal. Once you know Cupcake (Fig. \ref{fig:cupcake}) is a dog, you have enough information to infer that she is a mammal. Being a dog is not a necessary condition for being a mammal however. You can also be a mammal being being a cat, or a human, or a wombat. 

\begin{figure}
\begin{mdframed}[style=mytableclearbox]
\begin{center}
\includegraphics*{img/cupcake}
\end{center}
\end{mdframed}
\caption{This is Cupcake. The fact that she is a dog is a \textit{sufficient} condition for her to be a mammal. She also likes socks.}
\label{fig:cupcake}
\end{figure}

The conditional symbol in SL represents a sufficient condition, at least when read forward. That is, the antecedent is a sufficient condition for the consequent. If you have the antecedent, that is all you need to know to infer the consequent. So if $D$ is ``Cupcake is a dog'' and $M$ is ``Cupcake is a mammal, then $D \eif C$ is true. Being a dog is sufficient for being a mammal. As it turns out, if the relationship is sufficient going one direction, it is necessary going the other. So being a mammal is a necessary condition for being a mammal. If cupcake weren't a mammal, there would be no way for her to be a dog. Figure \ref{fig:necessary_and_sufficient} shows this relationship.

\begin{figure}
\begin{mdframed}[style=mytableclearbox, userdefinedwidth=.5\textwidth]
\begin{center}
\includegraphics*{img/necessaryandsufficient.png}
\end{center}
\end{mdframed}
\caption{The antecedent of a material conditional is a sufficient condition for the consequent, while the consequent is a necessary condition for the antecedent.}
\label{fig:necessary_and_sufficient}
\end{figure}


\subsection{Combining negation with conjunction and disjunction}

Tricky things happen when you combine a negation with a conjunction or disjunction, so it is worth taking a closer look here. Consider these sentences

\begin{earg}
\item[\ex{or3}] Either you will not have soup, or you will not have salad.
\item[\ex{or4}] You will have neither soup nor salad.
\end{earg}

We let $S_1$ mean that you get soup and $S_2$ mean that you get salad. Sentence \ref{or3} can be paraphrased in this way: ``Either \emph{it is not the case that} you get soup, or \emph{it is not the case that} you get salad.'' Translating this requires both disjunction and negation. It becomes $\enot S_1 \eor \enot S_2$.

Sentence \ref{or4} also requires negation. It can be paraphrased as, ``\emph{It is not the case that} either you get soup or you get salad.'' We need some way of indicating that the negation does not just negate the right or left disjunct, but rather negates the entire disjunction. In order to do this, we put parentheses around the disjunction: ``It is not the case that $(S_1 \eor S_2)$.'' This becomes simply $\enot (S_1 \eor S_2)$. Notice that the parentheses are doing important work here. The sentence $\enot S_1 \eor S_2$ would mean ``Either you will not have soup, or you will have salad.''

Something similar happens with negation and conjunction. Consider these sentences

\begin{earg}
\item[\ex{notand1}] You can't have soup and you can't have salad.
\item[\ex{notand2}] You can't have both soup and salad. 
\end{earg}

In sentence \ref{notand1}, the two parts of the sentence are negated individually. We would translate it into SL like this: $\enot S_1 \eand \enot S_2$. In sentence \ref{notand2}, the negation applies to soup and salad taken together. You are allowed to have soup only, or salad only. You just can't have both together. We would translate sentence \ref{notand2} like this: $\enot(S_1 \eand S_2)$. 

You can combine disjunction, conjunction, and negation to represent the exclusive or, as in this sentence. 

\begin{earg}
\item[\ex{or.xor}] You get either soup or salad, but not both.
\end{earg}

Remember on page \pageref{def:inclusive_or}, we said that the $\eor$ in SL represented an inclusive or. It said ``this or that or both.'' If we want to represent an exclusive or, we need to combine disjunction, conjuction and negation. We can break the sentence into two parts. The first part says that you get one or the other. We translate this as $(S_1 \eor S_2)$. The second part says that you do not get both. We can paraphrase this as ``It is not the case both that you get soup and that you get salad.'' Using both negation and conjunction, we translate this as $\enot(S_1 \eand S_2)$. Now we just need to put the two parts together. As we saw above, ``but'' can usually be translated as a conjunction. Sentence \ref{or.xor} can thus be translated as $(S_1 \eor S_2) \eand \enot(S_1 \eand S_2)$.

% ******************************************
%  * 		Recursive Syntax for  SL                		    *
% ******************************************
\label{recursive_syntax_for_SL}

\section{Recursive Syntax for SL} % I reworked this section to focus on the idea of recursive syntax

The previous two sections gave you a rough, informal sense of how to create sentences in SL. If I give you an English sentence like ``Grass is either green or brown,'' you should be able to write a corresponding sentence in SL: ``$A \eor B$.'' In this section we want to give a more precise definition of a sentence in SL.  When we defined statements in English, we did so using the concept of truth: Sentences were units of language that can be true or false. (See page \pageref{def:statement}.) When we talk abotu strings of symbols in SL, we can actually say whether they are actually parts of SL without talking about their truth. So we are going to call them sentences rather than statements, and we are going to define a sentence in SL just by looking at its structure. This is one respect in which a formal language like SL is more precise than a natural language like English.

\newglossaryentry{syntax}
{
name=syntax,
description={The structure of a bit of language, considered without reference to truth, falsity, or meaning.}
}

\newglossaryentry{semantics}
{
name=semantics,
description={The meaning of a bit of language is its meaning, including truth and falsity.}
}

The structure of a sentence in SL considered without reference to truth or falsity is called its syntax. More generally \textsc{\gls{syntax}} \label{def:syntax} refers to the study of the properties of language that are there even when you don't consider meaning. Whether a sentence is true or false is considered part of its meaning. In this chapter, we will be giving a purely syntactical definition of a sentence in SL.  The contrasting term is \textsc{\gls{semantics}} \label{def:semantics} the study of aspects of language that relate to meaning, including truth and falsity. (The word ``semantics'' comes from the Greek word for ``mark'')

\newglossaryentry{object language}
{
name=object language,
description={A language that is constructed and studied by logicians. In this textbook, the object \iflabelexists{part:quant_logic}{languages are SL and QL.}{language is SL.}}
}


\newglossaryentry{metalanguage}
{
name=metalanguage,
description={The language logicians use to talk about the object language. In this textbook, the metalanguage is English, supplemented by certain symbols like metavariables and technical terms like ``valid.''}
}

If we are going to define a sentence in SL just using syntax, we will need to carefully distinguish SL from the language that we use to talk about SL. When you create an artificial language like SL, the language that you are creating is called the \textsc{\gls{object language}}. \label{def:object_language} The language that we use to talk about the object language is called the \textsc{\gls{metalanguage}}. \label{def:metalanguage} Imagine building a house. The object language is like the house itself. It is the thing we are building. While you are building a house, you might put up scaffolding around it. The scaffolding isn't part of the the house. You just use it to build the house. The metalanguage is like the scaffolding. 

The object language in this chapter is SL. For the most part, we can build this language just by talking about it in ordinary English. However we will also have to build some special scaffolding that is not a part of SL, but will help us build SL. Our metalanguage will thus be ordinary English plus this scaffolding.

\newglossaryentry{metavariables}
{
name=metavariables,
description={A variable in the metalanguage that can represent any sentence in the object language.}
}



%rob: Paragraph on metavariables added.
An important part of the scaffolding are the \textsc{\gls{metavariables}} \label{def:metavariables} These are the fancy script letters we have been using in the characteristic truth tables for the connectives: \script{A}, \script{B}, \script{C}, etc. These are letters that can refer to any sentence in SL. They can represent sentences like $P$ or $Q$, or they can represent longer sentences, like $(((A \eor B) \eand G) \eif (P \eiff Q))$. Just as the sentence letters $A$, $B$, etc. are variables that range over any English sentence, the metavariables \script{A}, \script{B}, etc. are variables that range over any sentence in SL, including the sentence letters $A$, $B$, etc. 

As we said, in this chapter we will give a syntactic definition for ``sentence of SL.'' The definition itself will be given in mathematical English, the metalanguage. Table \ref{tab:basic_elements_of_SL} gives the basic elements of SL.


\begin{table}
\begin{mdframed}[style=mytablebox, userdefinedwidth=.75\textwidth]
\begin{tabu}{p{.3\linewidth}p{.4\linewidth}}
\underline{Element}& \underline{Symbols} \\ 
sentence letters & $A,B,C,\ldots,Z$ $A_1, B_1,Z_1,A_2,A_{25},J_{375},\ldots$\\
connectives & \enot,\eand,\eor,\eif,\eiff\\
parentheses&( , )\\\end{tabu}
\end{mdframed}
\caption{The basic elements of SL} \label{tab:basic_elements_of_SL}
\end{table}


Most random combinations of these symbols will not count as sentences in SL. Any random connection of these symbols will just be called a ``string'' or ``expression'' Random strings only become meaningful sentences when the are structured according to the rules of syntax. We saw from the earlier two sections that individual sentence letters,  like $A$ and $G_{13}$ counted as sentences. We also saw that we can put these sentences together using connectives so that  $\enot A$ and $\enot G_{13}$ is a sentence.  The problem is, we can't simply list all the different sentences we can put together this way, because there are infinitely many of them. Instead, we will define a sentence in SL by specifying the process by which they are constructed.

Consider negation: Given any sentence \script{A} of SL, $\enot\script{A}$ is a sentence of SL. It is important here that \script{A} is not the sentence letter $A$. Rather, it is a metavariable: part of the metalanguage, not the object language. Since \script{A} is not a symbol of SL, $\enot\script{A}$ is not an expression of SL. Instead, it is an expression of the metalanguage that allows us to talk about infinitely many expressions of SL: all of the expressions that start with the negation symbol. 


\newglossaryentry{sentence of SL}
{
name=sentence of SL,
description={A string of symbols in SL that can be built up according to the recursive rules given on page} % NB: The page number is inserted by the indexing function
}



We can say similar things for each of the other connectives. For instance, if \script{A} and \script{B} are sentences of SL, then $(\script{A}\eand\script{B})$ is a sentence of SL. Providing clauses like this for all of the connectives, we arrive at the following formal definition for a \textsc{\gls{sentence of SL}}: \label{def:sentence_of_SL}

\begin{enumerate}
\item Every atomic statement is a sentence.
\item If \script{A} is a sentence, then $\enot\script{A}$ is a sentence of SL.
\item If \script{A} and \script{B} are sentences, then $(\script{A}\eand\script{B})$ is a sentence.
\item If \script{A} and \script{B} are sentences, then $(\script{A}\eor\script{B})$ is a sentence.
\item If \script{A} and \script{B} are sentences, then $(\script{A}\eif\script{B})$ is a sentence.
\item If \script{A} and \script{B} are sentences, then $(\script{A}\eiff\script{B})$ is a sentence.
\item All and only sentences of SL can be generated by applications of these rules.
\end{enumerate}

We can apply this definition to see whether an arbitrary string is a sentence. Suppose we want to know whether or not $\enot \enot \enot D$ is a sentence of SL. Looking at the second clause of the definition, we know that $\enot \enot \enot D$ is a sentence \emph{if} $\enot \enot D$ is a sentence. So now we need to ask whether or not $\enot \enot D$ is a sentence. Again looking at the second clause of the definition, $\enot \enot D$ is a sentence \emph{if} $\enot D$ is. Again, $\enot D$ is a sentence \emph{if} $D$ is a sentence. Now $D$ is a sentence letter, an atomic statementof SL, so we know that $D$ is a sentence by the first clause of the definition. So for a compound formula like $\enot \enot \enot D$, we must apply the definition repeatedly. Eventually we arrive at the atomic statement from which the sentence is built up.

\newglossaryentry{recursive definition}
{
name=recursive definition,
description={A definition that defines a term by identifying base class and rules for extending that class. Also called an ``inductive definition.''}
}

Definitions like this are called recursive. \textsc{\Glspl{recursive definition}}\label{def:recursive_definition} begin with some specifiable base elements and define ways to indefinitely compound the base elements. Just as the recursive definition allows complex sentences to be built up from simple parts, you can use it to decompose sentences into their simpler parts. To determine whether or not something meets the definition, you may have to refer back to the definition many times. Recursive definitions are also sometimes called ``inductive definitions.''

\newglossaryentry{sentential logic}
{
name=sentential logic,
description={A system of logic in which statements can be defined using a recursive definition with only sentences in the base class.}
}


We are now in a position to define what it means for a system of logic to be a system of sentential logic. A \textsc{\gls{sentential logic}} \label{def:sentential_logic} is a system of logic in which statements can be defined using a recursive definition with only sentences in the base class. This book defines on system of sentential logic, which we call SL. Other books use other systems.


\newglossaryentry{scope}
{
name=scope,
description={The sentences that are joined by a connective. These are the sentences the connective was applied to when the sentence was assembled using a recursive definition.}
}

When you use a connective to build a longer sentence from shorter ones, the shorter sentences are said to be in the \textsc{\gls{scope}} \label{def:scope} of the connective. So in the sentence $(A \eand B) \eif C$, the scope of the connective $\eif$ includes $(A \eand B)$ and C. In the sentence $\enot(A \eand B)$ the scope of the $\enot$ is $(A \eand B)$. On the other hand, in the sentence $\enot A \eand B$ the scope of the $\enot$ is just A.

\newglossaryentry{main connective}
{
name=main connective,
description={The last connective that you add when you assemble a sentence using the recursive definition.}
}

The last connective that you add when you assemble a sentence using the recursive definition is the \textsc{\gls{main connective}} \label{def:main_connective} of that sentence. For example: The main logical operator of $\enot (E \eor (F \eif G))$ is negation, \enot. The main logical operator of $(\enot E \eor (F \eif G))$ is disjunction, \eor. The main connective of any sentence will have all the rest of the sentence in its scope.

\newglossaryentry{unique readability}
{
name=unique readability,
description={A property of formal languages which is present when each \iflabelexists{part:quant_logic}{well formed formula}{statement} is the product of a unique process of recursive construction.}
}

Because statement in our language is defined recursively, we can say it is ``uniquely readable.'' \textsc{\Gls{unique readability}}\label{def:unique_readability} is a property of formal languages which is present when each \iflabelexists{part:quant_logic}{well formed formula}{statement} can only be constructed in a single way. Every process of building up a sentence recursively yields a unique sentence, and every sentence is the product of a unique process of recursive definitions. This means that in an important sense our language SL is free of ambiguity, which is a key goal in the construction of any formal language. Every sentence in SL will have a unambiguous main connective and every connective in a sentence will have an unambiguous scope. This makes logicians happy.


%The recursive structure of sentences in SL will be important when we consider the circumstances under which a particular sentence would be true or false. The sentence $\enot \enot \enot D$ is true if and only if the sentence $\enot \enot D$ is false, and so on through the structure of the sentence until we arrive at the atomic components: $\enot \enot \enot D$ is true if and only if the atomic sentence $D$ is false. We will return to this point in the next chapter.
%restore when you restore the recursive part of chap. 3.

\subsection{Notational conventions}
\label{SLconventions}
A sentence like $(Q \eand R)$ must be surrounded by parentheses, because we might apply the definition again to use this as part of a more complicated sentence. If we negate $(Q \eand R)$, we get $\enot(Q \eand R)$. If we just had $Q \eand R$ without the parentheses and put a negation in front of it, we would have $\enot Q \eand R$. It is most natural to read this as meaning the same thing as $(\enot Q \eand R)$, something very different than $\enot(Q\eand R)$. The sentence $\enot(Q \eand R)$ means that it is not the case that both $Q$ and $R$ are true; $Q$ might be false or $R$ might be false, but the sentence does not tell us which. The sentence $(\enot Q \eand R)$ means specifically that $Q$ is false and that $R$ is true. As such, parentheses are crucial to the meaning of the sentence.

So, strictly speaking, $Q \eand R$ without parentheses is \emph{not} a sentence of SL. When using SL, however, we will often be able to relax the precise definition so as to make things easier for ourselves. We will do this in several ways.

First,  we understand that $Q \eand R$ means the same thing as $(Q \eand R)$. As a matter of convention, we can leave off parentheses that occur \emph{around the entire sentence}.

Second, it can sometimes be confusing to look at long sentences with many nested pairs of parentheses. We adopt the convention of using square brackets [ and ] in place of parentheses. There is no logical difference between $(P\eor Q)$ and $[P\eor Q]$, for example. The unwieldy sentence
$$(((H \eif I) \eor (I \eif H)) \eand (J \eor K))$$
could be written in this way:
$$\bigl[(H \eif I) \eor (I \eif H)\bigr] \eand (J \eor K)$$


Third, we will sometimes want to translate the conjunction of three or more sentences. For the sentence ``Alice, Bob, and Candice all went to the party,'' suppose we let $A$ mean ``Alice went,'' $B$ mean ``Bob went,'' and $C$ mean ``Candice went.'' The definition only allows us to form a conjunction out of two sentences, so we can translate it as $(A \eand B) \eand C$ or as $A \eand (B \eand C)$. There is no reason to distinguish between these, since the two translations are logically equivalent. There is no logical difference between the first, in which $(A \eand B)$ is conjoined with $C$, and the second, in which $A$ is conjoined with $(B \eand C)$.  So we might as well just write $A \eand B \eand C$. As a matter of convention, we can leave out parentheses when we conjoin three or more sentences.

Fourth, a similar situation arises with multiple disjunctions. ``Either Alice, Bob, or Candice went to the party'' can be translated as $(A \eor B) \eor C$ or as $A \eor (B \eor C)$. Since these two translations are logically equivalent, we may write $A \eor B \eor C$.

These latter two conventions only apply to multiple conjunctions or multiple  disjunctions. If a series of connectives includes both disjunctions and conjunctions, then the parentheses are essential; as with $(A \eand B) \eor C$ and $A \eand (B \eor C)$. The parentheses are also required if there is a series of conditionals or biconditionals; as with $(A \eif B) \eif C$ and $A \eiff (B \eiff C)$.

We have adopted these four rules as notational conventions, not as changes to the definition of a sentence. Strictly speaking, $A \eor B \eor C$ is still not a sentence. Instead, it is a kind of shorthand. We write it for the sake of convenience, but we really mean the sentence $(A \eor (B \eor C))$.

If we had given a different definition for a sentence, then these could count as sentences. We might have written rule 3 in this way: ``If \script{A}, \script{B}, $\ldots$ \script{Z} are sentences, then $(\script{A}\eand\script{B}\eand\ldots\eand\script{Z})$, is a sentence .'' This would make it easier to translate some English sentences, but would have the cost of making our formal language more complicated. We would have to keep the complex definition in mind when we develop truth tables and a proof system. We want a logical language that is expressively simple and allows us to translate easily from English, but we also want a formally simple language. Adopting notational conventions is a compromise between these two desires.


\practiceproblems
\noindent\problempart Using the symbolization key given, translate each English-language sentence into SL.
\label{pr.monkeysuits}
\begin{ekey}
\item[M:] Those creatures are men in suits. 
\item[C:] Those creatures are chimpanzees. 
\item[G:] Those creatures are gorillas.
\end{ekey}

\begin{longtabu}{p{.1\linewidth}p{.9\linewidth}}
\textbf{Example}: & If those creatures are not men in suits, they are gorillas. \\
\textbf{Answer}: & $\enot M \eif G$ \\
\end{longtabu}


\begin{exercises}
\item Those creatures are not men in suits. \answer{$\enot M$} 
\item Those creatures are men in suits, or they are not. \answer{$M \eor \enot M$} 
\item Those creatures are either gorillas or chimpanzees. \answer{$G \eor C$} 
\item Those creatures are not gorillas, but they are not chimpanzees either. \answer{$\enot G \eand \enot C$} 
\item Those creatures cannot be both gorillas and men in suits. \answer{$\enot(G \eand M)$} 
\item If those creatures are not gorillas, then they are men in suits \answer{$\enot G \eif M$} 
\item Those creatures are men in suits only if they are not gorillas. \answer{$M \eif \enot G$} 
\item Those creatures are chimpanzees if and only if they are not gorillas. \answer{$C \eiff \enot G$} 
\item Those creatures are neither gorillas nor chimpanzees. \answer{$\enot(G \eor C).$} %See p.34, sentence 19, and p. 156
\item Unless those creatures are men in suits, they are either chimpanzees or they are gorillas. \answer{$M \eor (C \eor G)$} 
\end{exercises}

%If					X
%only if				X
%if and only if		X
%but				X
%unless				X
%not both      		X
%neither nor			X

%added and changed problems to get a better distribution of kinds of problems. 

\noindent\problempart Using the symbolization key given, translate each English-language sentence into SL.
\begin{ekey}
\item[A:] Mister Ace was murdered.
\item[B:] The butler did it.
\item[C:] The cook did it.
\item[D:] The Duchess is lying.
\item[E:] Mister Edge was murdered.
\item[F:] The murder weapon was a frying pan.
\end{ekey}
\begin{exercises}
\item Either Mister Ace or Mister Edge was murdered. % {\color{red} $A \eor E$}  \vspace{1ex}
\item If Mister Ace was murdered, then the cook did it. % {\color{red} $A \eif C$} \vspace{1ex}
\item If Mister Edge was murdered, then the cook did not do it. % {\color{red} $E \eif \enot C} \vspace{1ex}
\item Either the butler did it, or the Duchess is lying. % {\color{red} $B \eor D$} \vspace{1ex}
\item The cook did it only if the Duchess is lying. % {\color{red} $C \eif D$} \vspace{1ex}
\item If the murder weapon was a frying pan, then the culprit must have been the cook. % {\color{red} $F \eif C$} \vspace{1ex}
\item If the murder weapon was not a frying pan, then the culprit was neither the cook nor the butler. % {\color{red} $\enot F \eif \enot(C \or B) \vspace{1ex}
\item Mister Ace was murdered if and only if Mister Edge was not murdered. % {\color{red} $A \eiff \enot E$} \vspace{1ex}
\item The Duchess is lying, unless it was Mister Edge who was murdered. % {\color{red} $D \eor A$} \vspace{1ex}
\item Mister Ace was murdered, but not with a frying pan. % {\color{red} $A \eand \enot F$} \vspace{1ex}
\item The butler and the cook did not both do it. % {\color{red} $\enot(B \enad C)$} \vspace{1ex}
\item Of course the Duchess is lying! % {\color{red}$D$} \vspace{1ex}
\end{exercises}

%If  			x
%only if             x
%if and only if   x
%but			x
%unless			x
%not both		x
%neither nor		x

%changed problems to get a better distribution of kinds of problems. 


\noindent\problempart Using the symbolization key given, translate each English-language sentence into SL.
\label{pr.avacareer}
\begin{ekey}
\item[E$_1$:] Ava is an electrician.
\item[E$_2$:] Harrison is an electrician.
\item[F$_1$:] Ava is a firefighter.
\item[F$_2$:] Harrison is a firefighter.
\item[S$_1$:] Ava is satisfied with her career.
\item[S$_2$:] Harrison is satisfied with his career.
\end{ekey}
\begin{exercises}
\item Ava and Harrison are both electricians. \answer{$E_1 \eand E_2$} 
\item If Ava is a firefighter, then she is satisfied with her career. \answer{$F_1 \eif S_1$}  
\item Ava is a firefighter, unless she is an electrician. \answer{$F_1 \eor E_1$}  
\item Harrison is an unsatisfied electrician. \answer{$E_2 \eand \enot S_2$}  
\item Neither Ava nor Harrison is an electrician. \answer{$\enot(E_1 \eor E_2)$}  
\item Both Ava and Harrison are electricians, but neither of them find it satisfying. \answer{$(E_1 \eand E_2) \eand \enot (S_1 \eor S_2)$} 
\item Harrison is satisfied only if he is a firefighter. \answer{$S_2 \eif F_2$} 
\item If Ava is not an electrician, then neither is Harrison, but if she is, then he is too. \answer{$(\enot E_1 \eif \enot E_2) \eand (E_1 \eif E_2)$} 
\item Ava is satisfied with her career if and only if Harrison is not satisfied with his. \answer{$S_1 \eiff \enot S_2$} 
\item If Harrison is both an electrician and a firefighter, then he must be satisfied with his work. \answer{$(E_2 \eand F_2) \eif S_2$} 
\item It cannot be that Harrison is both an electrician and a firefighter. \answer{$\enot (E_2 \eand F_2)$} 
\item Harrison and Ava are both firefighters if and only if neither of them is an electrician. \answer{$(F_1 \eand F_2) \eiff \enot (E_1 \eor E_2)$} \vspace{1ex}
\end{exercises}

%If					x	
%only if				x
%if and only if		x
%but				x
%unless				x
%not both			x
%neither nor			x

\noindent\problempart Using the symbolization key given, translate each English-language sentence into SL.
\label{pr.jazzinstruments}
\begin{ekey}
\item[J$_1$:] John Coltrane played tenor sax.
\item[J$_2$:] John Coltrane played soprano sax.
\item[J$_3$:] John Coltrane played tuba
\item[M$_1$:] Miles Davis played trumpet
\item[M$_2$:]Miles Davis played tuba
\end{ekey}

\begin{exercises}
\item John Coltrane played tenor and soprano sax. %{\color{red} $J_1 \eand J_2$} \vspace{1ex}
\item Neither Miles Davis nor John Coltrane played tuba. %{\color{red} $\enot(M_2 \eor J_3)$ or $\enot M_2 \eand \enot J_3$} \vspace{1ex}
\item John Coltrane did not play both tenor sax and tuba.  %{\color{red} $\enot(J_1 \eand J_3)$ or $\enot J_1 \eor \enotJ_3$} \vspace{1ex}
\item John Coltrane did not play tenor sax unless he also played soprano sax. %{\color{red} $\enot J_1 \eor J_2$} \vspace{1ex}
\item John Coltrane did not play tuba, but Miles Davis did. %{\color{red} $\enotJ_3 \eand M_2$} \vspace{1ex}
\item Miles Davis played trumpet only if he also played tuba. %{\color{red} $M_1 \eiff M_2$} \vspace{1ex}
\item If Miles Davis played trumpet, then John Coltrane played at least one of these three instruments: tenor sax, soprano sax, or tuba. %{\color{red} $M_1 \eif (J_1 \eor (J_2 \eor J_3))&} \vspace{1ex}
\item If John Coltrane played tuba then Miles Davis played neither trumpet nor tuba. %{\color{red} $J_3 \eif \enot(M_1 \eor M_2)$ or $J_3 \eif (\enot M_1 \eand \enot M_2)$  } \vspace{1ex}
\item Miles Davis and John Coltrane both played tuba if and only if Coltrane did not play tenor sax and Miles Davis did not play trumpet. %{\color{red} $(J_3 \eand M_2) \eiff \enotJ_1 & \enot M_1)$ or $(J_3 \eand M_2) \eiff \enot (J_1 \eor M_1)$} \vspace{1ex}
\end{exercises}
%If					x					
%only if				x		
%if and only if		x
%but				x
%unless				x
%not both			x
%neither nor			x


\noindent\problempart
\label{pr.spies}
Give a symbolization key and symbolize the following sentences in SL. \\
\answer{
\begin{ekey}
\item[A:] Alice is a spy
\item [B:] Bob is a spy
\item [C:] The code has been broken
\item [D:] The German embassy is in an uproar
\end{ekey}
}
\begin{exercises}
\item Alice and Bob are both spies. \answer{$A \eand B$ }
\item If either Alice or Bob is a spy, then the code has been broken. \answer{$(A \eor B) \eif C$}
\item If neither Alice nor Bob is a spy, then the code remains unbroken. \answer{$\enot(A \eor B) \eif \enot C$}
\item The German embassy will be in an uproar, unless someone has broken the code. \answer{$D \eor C$}
\item Either the code has been broken or it has not, but the German embassy will be in an uproar regardless. \answer{$(C \eor \enot C) \eand D$}
\item Either Alice or Bob is a spy, but not both. \answer{$(A \eor B) \eand \enot (A \eand B)$}
\end{exercises}

%If
%only if
%if and only if
%but
%unless
%not both
%neither nor

\noindent\problempart Give a symbolization key and symbolize the following sentences in SL.
%\begin{ekey}
%\item[A:] Gregor plays first base
%\item[B:] The team will lose
%\item[C:] There is a miracle
%\item[D:] Gregor's mom will bake cookies.
%\end{ekey}

\begin{exercises}
\item If Gregor plays first base, then the team will lose. %{\color{red} $A \eif B$ \vspace{1ex}}
\item The team will lose unless there is a miracle. %{\color{red}$B \eor C$ \vspace{1ex}}
\item The team will either lose or it won't, but Gregor will play first base regardless. % {\color{red}$(B \eor \enot B) \eand A$ \vspace{1ex}}
\item Gregor's mom will bake cookies if and only if Gregor plays first base.% {\color{red}$C \eiff A$ \vspace{1ex}}
\item If there is a miracle, then Gregor's mom will not bake cookies. %{\color{red} $C \eif \enot D$}
\end{exercises}


\noindent\problempart For each argument, write a symbolization key and translate the argument  into SL, putting the argument in canonical form.

\begin{longtabu}{p{.1\linewidth}p{.9\linewidth}}
\textbf{Example}: &  If Dorothy plays the piano in the morning, then Roger wakes up cranky. Dorothy plays piano in the morning unless she is distracted. So if Roger does not wake up cranky, then Dorothy must be distracted. \\
\textbf{Answer}: & {\color{white}.} \vspace{-20pt} \begin{ekey}
\item[A:] Dorothy plays the piano in the morning
\item[B:] Roger wakes up cranky
\item[C:] Dorothy is distracted
\end{ekey}\\
& {\color{white}.} \vspace{-22pt} \begin{earg*}
\item $A \eif B$
\item  $A \eor C$
\itemc[.1] $\enot B \eif C$
\end{earg*}\\
\end{longtabu}

\begin{exercises}

\item It will either rain or snow on Tuesday. If it rains on Tuesday, Neville will be sad. If it snows on Tuesday, Neville will be cold. Therefore, Neville will either be sad or cold on Tuesday.

\answer{
\begin{ekey}
\item[A:]  It will rain on Tuesday
\item[B:]  It will snow on Tuesday
\item[C:]  Neville will be sad
\item[D:]  Neville will be cold
\end{ekey}

\begin{earg*}
\item $A \eor B$
\item $A \eif C$
\item $B \eif D$
\itemc[.1]  $C \eor D$
\end{earg*}
}

\item If Zoog remembered to do his chores, then things are clean but not neat. If he forgot, then things are neat but not clean. Therefore, things are either neat or clean---but not both.

\answer{
\begin{ekey}
\item[A:] Zoog remembered to do his chores. 
\item[B:] Things are clean 
\item[C:] Things are neat %\end{ekey}
\end{ekey}

\begin{earg*}
\item  $A \eif (B \eand \enot C)$
\item  $\enot A \eif (\enot B \eand C)$
\itemc[.2]  $(B \eor C) \eand \enot (B \eand C)$ 
\end{earg*}
}

\end{exercises}


\noindent\problempart For each argument, write a symbolization key and translate the argument as well as possible into SL. The part of the passage in italics is there to provide context for the argument, and doesn't need to be symbolized.

\begin{exercises}
\item It is going to rain soon. I know because my leg is hurting, and my leg hurts if it’s going to rain. 

%{\color{red}
%\begin{ekey}
%\item[A:]  
%\item[B:]  
%\item[C:]  %\end{ekey}

%begin{\earg}
%\item[1.]  
%\item[2.]  
%\item[$\therefore$]  
%}

\item  \emph{Spider-man tries to figure out the bad guy’s plan.} If Doctor Octopus gets the uranium, he will blackmail the city. I am certain of this because if Doctor Octopus gets the uranium, he can make a dirty bomb, and if he can make a dirty bomb, he will blackmail the city.

%{\color{red}
%\begin{ekey}
%\item[A:]  
%\item[B:]  
%\item[C:]  %\end{ekey}

%begin{\earg}
%\item[1.]  
%\item[2.]  
%\item[$\therefore$]  
%}

\item \emph{A westerner tries to predict the policies of the Chinese government.} If the Chinese government cannot solve the water shortages in Beijing, they will have to move the capital. They don’t want to move the capital. Therefore they must solve the water shortage. But the only way to solve the water shortage is to divert almost all the water from the Yangzi river northward. Therefore the Chinese government will go with the project to divert water from the south to the north.       



%{\color{red}
%\begin{ekey}
%\item[A:]  
%\item[B:]  
%\item[C:]  %\end{ekey}

%begin{\earg}
%\item[1.]  
%\item[2.]  
%\item[$\therefore$]  
%}

\end{exercises}




\noindent\problempart
\begin{exercises}
\item Are there any sentences of SL that contain no sentence letters? Why or why not? \answer{\\ No, because the rules for creating sentences begin with sentence letters and then apply connectives and more sentence letters. There is no way to remove the sentence letters that you start with.} 
\item In the chapter, we symbolized an \emph{exclusive or} using \eor, \eand, and \enot. How could you translate an \emph{exclusive or} using only two connectives? Is there any way to translate an \emph{exclusive or} using only one kind of connective? \answer{ \\ The exclusive or (sometimes written xor) is true whenever the two sides of it have opposite truth values. This is the reverse of what the biconditional does. Thus you can represent the xor like this: \enot(A \eiff B). You can't get rid of any more connectives, though. If you had a single connective in the sentence, it would have to be equivalent on its own to the exclusive or, and none of our connectives work like that. Some systems do introduce a separate symbol for the exclusive or, often a plus sign: +.}
\end{exercises}




%\solutions
%\problempart
%\label{pr.wiffSL}
%For each of the following: (a) Is it a wff of SL? (b) Is it a sentence of SL, allowing for notational conventions?
%\begin{earg}
%\item $(A)$
%\item $J_{374} \eor \enot J_{374}$
%\item $\enot \enot \enot \enot F$
%\item $\enot \eand S$
%\item $(G \eand \enot G)$
%\item $\script{A} \eif \script{A}$
%\item $(A \eif (A \eand \enot F)) \eor (D \eiff E)$
%\item $[(Z \eiff S) \eif W] \eand [J \eor X]$
%\item $(F \eiff \enot D \eif J) \eor (C \eand D)$
%\end{earg}


%%%%    Key term list
\section*{Key Terms}
\begin{multicols}{2}
\begin{sortedlist}
\sortitem{Sentence letter}{} 	
\sortitem{Symbolization key}{} 	
\sortitem{Atomic statement}{}
\sortitem{Sentential connective}{}
\sortitem{Negation}{}
\sortitem{Conjunction}{}
\sortitem{Conjunct}{}
\sortitem{Disjunction}{}
\sortitem{Disjunct}{}
\sortitem{Conditional}{}
\sortitem{Antecedent}{}
\sortitem{Consequent}{}
\sortitem{Biconditional}{}
\sortitem{Syntax}{}
\sortitem{Semantics}{}
\sortitem{Object language}{}
\sortitem{Metalanguage}{}
\sortitem{Metavariables}{}
\sortitem{Sentence of SL}{}
\sortitem{Main connective}{}
\sortitem{Recursive definition}{}
\sortitem{Scope}{}
\sortitem{Nonlogical symbol}{}
\sortitem{Logical constant}{}
\sortitem{Exclusive or}{}
\sortitem{Inclusive or}{}
\sortitem{Necessary condition}{}
\sortitem{Sufficient condition}{}
\sortitem{Translation key}{}
\end{sortedlist}
\end{multicols}





	
\chapter{Truth Tables}
\label{chap:truth_tables}
\markright{Chap \ref{chap:truth_tables}: Truth Tables}

% The long comment below is material from the old semantics chapter that I am gradually folding in to this chapter.

This chapter introduces a way of evaluating sentences and arguments of SL called the truth table method. As we shall see, the truth table method is \emph{semantic} because it involves one aspect of the meaning of sentences, whether those sentences are true or false. As we saw on page \pageref{def:semantics}, semantics is the study of aspects of language related to meaning, including truth and falsity. Although it can be laborious, the truth table method is a purely mechanical procedure that requires no intuition or special insight. \iflabelexists{part:quant_logic}{When we get to Chapter \ref{chap:semantics_for_ql},we will provide a parallel semantic method for QL; however, this method will not be purely mechanical.}{}


% *********************************************
% *   Basic Concepts								*
% *********************************************

\section{Basic Concepts}

\newglossaryentry{logical constant}
{
name=logical constant,
description={A symbol whose meaning is fixed by a formal language. Sometimes these are just called ``logical symbols.'' They are contrasted with \textsc{non-logical symbols}.}
}

\newglossaryentry{nonlogical symbol}
{
name=nonlogical symbol,
description={A symbol whose meaning is not fixed by a formal language.}
}



In the previous chapter, we said that a formal language is built from two kinds of elements: logical constants and nonlogical symbols. The \textsc{\glspl{logical constant}}\label{def:logical_constant} have their meaning fixed by the formal language, while the \textsc{\glspl{nonlogical symbol}} \label{def:nonlogical_symbol} get their meaning in the symbolization key. The logical constants in SL are the sentential connectives and the parentheses, while the nonlogical symbols are the sentence letters. 

\newglossaryentry{interpretation}
{
name=interpretation,
description={A correspondence between nonlogical symbols of the object language and elements of some other language or logical structure.}
}

When we assign meaning to the nonlogical symbols of a language using a dictionary, we say we are giving an ``interpretation'' of the language. More formally an \textsc{\gls{interpretation}\label{def:interpretation}} of a language is a correspondence between elements of the object language and elements of some other language or logical structure. The symbolization keys we defined in Chapter \ref{chap:SL} (p. \pageref{def:translation_key}) are one sort of interpretation. Fancier languages will have more complicated kinds of interpretations.

\newglossaryentry{truth value}
{
  name=truth value,
  description={The status of a statement with relationship to truth. For  this textbook, this means the status of a statement as true or false}
}

The truth table method will also involve giving an interpretation of sentences, but they will be much simpler than the translation keys we used in Chapter \ref{chap:SL}. We will not be concerned with what the individual sentence letters mean. We will only care whether they are true or false. In other words, our interpretations will assign \glspl{truth value} to the sentence letters. (See page \pageref{def:Truth_value}.)

\newglossaryentry{truth-functional connective}
{
name=truth-functional connective,
description={an operator that builds larger sentences out of smaller ones and fixes the truth value of the resulting sentence based only on the truth value of the component sentences.}
}

We can get away with only worrying about the truth values of sentence letters because of the way that the meaning of larger sentences is generated by the meaning of their parts. Any larger sentence of SL is composed of atomic sentences with sentential connectives. The truth value of the compound sentence depends only on the truth value of the atomic sentences that it comprises. In order to know the truth value of $D\eiff E$, for instance, you only need to know the truth value of $D$ and the truth value of $E$. Connectives that work in this way are called truth functional. More technically, we define a \textsc{\gls{truth-functional connective}} \label{def:truth-functional_connective}as an operator that builds larger sentences out of smaller ones, and fixes the truth value of the resulting sentence based only on the truth value of the component sentences. 

\newglossaryentry{truth assignment}
{
name=truth assignment,
description={A function that maps the sentence letters in SL onto truth values.}
}

Because all of the logical symbols in SL are truth functional, we can study the the semantics of SL looking only at truth and falsity. If we want to know about the truth of the sentence $A \eand B$, the only thing we need to know is whether $A$ and $B$ are true. It doesn't actually matter what else they mean. So if $A$ is false, then $A \eand B$ is false no matter what false sentence $A$ is used to represent. It could be ``I am the Pope'' or ``Pi is equal to 3.19.'' The larger sentence $A \eand B$ is still false. So to give an interpretation of sentences in SL, all we need to do is create a truth assignment. A \textsc{\gls{truth assignment}} \label{def:truth_assignment} is a function that maps the sentence letters in SL onto our two truth values. In other words, we just need to assign Ts and Fs to all our sentence letters.

It is worth knowing that most languages are not built only out of truth functional connectives. In English, it is possible to form a new sentence from any simpler sentence \script{X} by saying ``It is possible that \script{X}.'' The truth value of this new sentence does not depend directly on the truth value of \script{X}. Even if \script{X} is false, perhaps in some sense \script{X} \emph{could} have been true---then the new sentence would be true. Some formal languages, called \emph{modal logics}, have an operator for possibility. In a modal logic, we could translate ``It is possible that \script{X}'' as {\large $\diamond$}\script{X}. However, the ability to translate sentences like these comes at a cost: The {\large $\diamond$} operator is not truth-functional, and so modal logics are not amenable to truth tables.

% *********************************************
% *   Complete Truth Tables							*
% *********************************************
\section{Complete Truth Tables}

In the last chapter we introduced the characteristic truth tables for the different connectives. To put them all in one place, the truth tables for the connectives of SL are repeated in Table \ref{table.CharacteristicTTs}. On the left is the truth table for negation, and on the right is the truth table for the other four connectives. Notice that the truth table for the negation is shorter than the other table. This is because there is only one metavariable here, \script{A}, which can either be true or false. The other connectives involve two metavariables, which give us four possibilities of true and false. The columns to the left of the double line in these tables are called the reference columns. They just specify the truth values of the individual sentence letters. Each row of the table assigns truth values to all the variables. Each row is thus a truth assignment---a kind of interpretation---for that sentence. Because the full table gives all the possible truth assignments for the sentence, it gives all the possible interpretations of it. 


\begin{table}
\begin{mdframed}[style=mytableclearbox]
\begin{center}
\begin{longtabu}{cccc|c||c|c|c|c}
\multicolumn{1}{r||}{\script{A}}&\enot\script{A}&	&	\script{A} & \script{B} & \script{A}\eand\script{B} & \script{A}\eor\script{B} & \script{A}\eif\script{B} & \script{A}\eiff\script{B}\\
\cline{1-2} \cline{4-9}
\multicolumn{1}{r||}{T}	&	F	&	&	T & T & T & T & T & T\\
\multicolumn{1}{r||}{F}	&	T	&	&	T & F & F & T & F & F\\
	&		&	&	F & T & F & T & T & F\\
	&		&	&	F & F & F & F & T & T
\end{longtabu}
\end{center}
\end{mdframed}
\caption{The characteristic truth tables for the connectives of SL.}
\label{table.CharacteristicTTs}
\end{table}

The truth table of sentences that contain only one connective is given by the characteristic truth table for that connective. So the truth table for the sentence $P \eand Q$ looks just like the characteristic truth table for \eand, with the sentence letters $P$ and $Q$ substituted in. The truth tables for more complicated sentences can simply be built up out of the truth tables for these basic sentences. Consider the sentence $(H\eand I)\eif H$. This sentence has two sentence letters, so we can represent all the possible truth assignments using a four line truth table. We can start by writing out all the possible combinations of true and false for $H$ and $I$ in the reference columns. We then copy the truth values for the sentence letters and write them underneath the letters in the sentence.

\begin{center}
\tabulinesep=.5ex
\begin{tabu}{c|c||@{\TTon}*{5}{c}@{\TToff}}
$H$&$I$&$(H$&\eand&$I)$&\eif&$H$\\
\hline
 T & T & T & & T & & T\\
 T & F & T & & F & & T\\
 F & T & F & & T & & F\\
 F & F & F & & F & & F
\end{tabu}
\end{center}

Now consider just one part of the sentence above, the subsentence $H\eand I$. This is a conjunction \script{A}\eand\script{B} with $H$ as \script{A} and with $I$ as \script{B}. $H$ and $I$ are both true on the first row. Since a conjunction is true when both conjuncts are true, we write a T underneath the conjunction symbol. We continue for the other three rows and get this:

\begin{center}
\begin{tabu}{c|c||ccccc}%{c|c||@{\TTon}*{5}{c}@{\TToff}}
\multicolumn{1}{r}{} &\multicolumn{1}{r}{} & \multicolumn{3}{c}{\script{A} \eand  \script{B}} & & \\
\multicolumn{1}{r}{} &\multicolumn{1}{r}{} & \multicolumn{3}{c}{\downbracefill} & & \\
$H$	&	$I$	&	$(H$	&\eand	&	$I)$	&	\eif	&	$H$\\
\hline
 T & T & T & \TTbf{T} & T & & T\\
 T & F & T & \TTbf{F} & F & & T\\
 F & T & F & \TTbf{F} & T & & F\\
 F & F & F & \TTbf{F} & F & & F
\end{tabu}
\end{center}

Next we need to fill in the final column under the conditional. The conditional is the main connective of the sentence, so the whole sentence is of the form $\script{A}\eif\script{B}$ with $(H \eand I)$ as \script{A} and with $H$ as \script{B}. So to fill the final column, we just need to look at the characteristic truth table for the conditional. For the first row, the sentence $(H \eand I)$ is true and the sentence $H$ is also true. The truth table for he conditional tells us this means that the whole sentence is true. Filling out the rest of the column gives us this:

\begin{center}
\begin{tabu}{c|c||ccccc}%{c|c||@{\TTon}*{5}{c}@{\TToff}}
\multicolumn{1}{r}{} &\multicolumn{1}{r}{} & \multicolumn{3}{c}{\script{A}}		& \eif 					&	\script{B} \\
\multicolumn{1}{r}{} &\multicolumn{1}{r}{} & \multicolumn{3}{c}{\downbracefill}	& \downbracefill	&	\downbracefill \\
$H$&$I$&$(H$&\eand&$I)$&\eif&$H$\\
\hline
 T & T & T  & {T} & T &\TTbf{T} & T\\
 T & F & T & {F} &  F &\TTbf{T} & T\\
 F & T & F & {F} & T &\TTbf{T} & F\\
 F & F & F & {F} & F &\TTbf{T} & F
\end{tabu}
\end{center}

The column of Ts underneath the conditional tells us that the sentence $(H \eand I)\eif H$ is true regardless of the truth values of $H$ and $I$. They can be true or false in any combination, and the compound sentence still comes out true. It is crucial that we have considered all of the possible combinations. If we only had a two-line truth table, we could not be sure that the sentence was not false for some other combination of truth values.

In this example, the script letters over the table have just been there to indicate how the columns get filled in. We won't need them in the final product. Also, the reference columns are redundant with the columns under the individual sentence letters, so we can eliminate those as well. Most of the time, when you see truth tables, we will just write them out this way:
\begin{center}
\begin{tabu}{ccccc}
$(H$	&	\eand	&	$I)$	& \eif	\tikz[overlay, shift={(-1.25ex,-33pt)}, gray] \draw (0pt,0pt) ellipse (2ex and 48pt);			&$H$\\
\hline
T 		& 	{T} 	& 	T 		& T 	& T\\
T 		& 	{F} 	& 	F 		& T 	& T\\
F 		& 	{F} 	&	T 		& T 	& F\\
F 		& 	{F} 	& 	F 		& T 	& F
\end{tabu}
\end{center}
\label{tautology3.1} 
%Whether the elipise is drawn correctly depends on whether you are typsetting just the chapter or the full book, because it shifts where the table falls on the page. As of 5/23/18 \tikz[overlay, shift={(-1ex,-27pt)}, gray] \draw (0pt,0pt) ellipse (2ex and 44pt); gives the right size when you are just typsetting the chapter, but to get the size for the full book  -27pt --> -33pt. and  44pt --> 50pt



The truth value of the sentence on each row is just the column underneath the \emph{main connective} (see p. \pageref{def:main_connective}) of the sentence, in this case, the column underneath the conditional.

\newglossaryentry{complete truth table}
{
name=complete truth table,
description={A table that gives all the possible interpretations for a sentence or set of sentences in SL.}
}

A \textsc{\gls{complete truth table}} \label{def:complete_truth_table} is a table that gives all the possible interpretations for a sentence or set of sentences in SL. It has a row for each possible assignment of T and F to all of the sentence letters. The size of the complete truth table depends on the number of different sentence letters in the table. A sentence that contains only one sentence letter requires only two rows, as in the characteristic truth table for negation. This is true even if the same letter is repeated many times, as in this sentence: $$[(C\eiff C) \eif C] \eand \enot(C \eif C).$$ The complete truth table requires only two lines because there are only two possibilities: $C$ can be true, or it can be false. A single sentence letter can never be marked both T and F on the same row. The truth table for this sentence looks like this:
\begin{center}
\begin{tabu}{cccccccccc}%{c@{\TTon}*{13}{c}@{\TToff}}
[($C$	&\eiff	&	$C)$	&	\eif	&	$C]$	&	\eand	\tikz[overlay, shift={(-1.25ex,-14pt)}, gray] \draw (0pt,0pt) ellipse (2ex and 27pt);		&\enot	&	$(C$	&	\eif	&	$C)$\\
\hline
	T 	&  T  	& 	T 		&  T  		& 	T 		&	F	&  F		& T 		&  T 		& T   \\
	F 	&  T  	& 	F		&  F  		&	 F 		&	F	&  F		& F 		&  T  		& F  
\end{tabu}
\end{center}
\label{contradiction3.1}
%Whether the elipise is drawn correctly depends on whether you are typsetting just the chapter or the full book, because it shifts where the table falls on the page.

Looking at the column underneath the main connective, we see that the sentence is false on both rows of the table; i.e., it is false regardless of whether $C$ is true or false.

A sentence that contains two sentence letters requires four lines for a complete truth table, as we saw above in the table for $(H \eand I)\eif I$.

A sentence that contains three sentence letters requires eight lines, as in this example. Here the reference columns are included so you can see how to arrange the truth values for the individual sentence letters so that all the possibilities are covered.

\begin{center}
\begin{tabu}{c|c|c|@{\TTon}*{5}{c}@{\TToff}}
$M$	&	$N$	&	$P$	&	$M$	&	\eand	\tikz[overlay, shift={(-1.33ex,-68pt)}, gray] \draw (0pt,0pt) ellipse (2ex and 85pt);			&	$(N$	&	\eor	&	$P)$\\
\hline
%           M        &     N   v   P
T		& T 		& T 		& T 		& T & T & T & T\\
T 		& T 		& F 		& T 		& T & T & T & F\\
T 		& F 		& T 		& T 		& T & F & T & T\\
T 		& F 		& F 		& T 		& F & F & F & F\\
F 		& T 		& T 		& F 		& F & T & T & T\\
F 		& T 		& F 		& F 		& F & T & T & F\\
F 		& F 		& T 		& F 		& F & F & T & T\\
F 		& F 		& F 		& F 		& F & F & F & F
\end{tabu}
\end{center}
\label{contingentsentence3.1}
From this table, we know that the sentence $M\eand(N\eor P)$ might be true or false, depending on the truth values of $M$, $N$, and $P$.

A complete truth table for a sentence that contains four different sentence letters requires 16 lines. For five letters, 32 lines are required. For six letters, 64 lines, and so on. To be perfectly general: If a complete truth table has $n$ different sentence letters, then it must have $2^n$ rows.

By convention, the reference columns are filled in with the right most row alternating Ts and Fs. The next column over alternates sets of two Ts and two Fs. For the third column from the right, you have sets of four Ts and four Fs. This continues until you reach the leftmost column, which will always have the top have all Ts and the bottom half all Fs. This convention is completely arbitrary. There are other ways to be sure that all the possible combinations are covered, but everything is easier if we all stick to the same pattern.

%%%%%%%%%%%%%%%%% practice problems


\practiceproblems
\noindent\noindent\problempart Identify the main connective in the each sentence.

\begin{longtabu}{p{.1\linewidth}p{.9\linewidth}}
\textbf{Example}: & $(A \eif C) \eand \enot D$ \\
\textbf{Answer}: & $(A \eif C) \circled[gray, shape=circle]{\eand} \enot D$\\
\end{longtabu}



\begin{exercises}

\item \iflabelexists{showanswers}{$\circled[red, shape=circle]{\enot}(A \eor \enot B)$}{$\enot(A \eor \enot B) $}

\item \iflabelexists{showanswers}{$\enot(A \eor \enot B) \circled[red, shape=circle]{\eor} \enot(A \eand D)$}	{$\enot(A \eor \enot B) \eor \enot(A \eand D)$}
	
\item \iflabelexists{showanswers}{$[\enot(A \eor \enot B) \eor \enot (A \eand D)] \circled[red, shape=circle]{\eif} E$}{$[ \enot(A \eor \enot B) \eor \enot (A \eand D)] \eif E$}	

\item \iflabelexists{showanswers}{$[(A \eif B) \eand C]$ \circled[red, shape=circle]{\eiff} $[A \eor (B \eand C)]$}{$[(A \eif B) \eand C] \eiff [A \eor (B \eand C)]$ }  

\item \iflabelexists{showanswers}{\circled[red, shape=circle]{\enot} $\enot \enot [A \eor (B \eand (C \eor D))]$}{$\enot \enot \enot [A \eor (B \eand (C \eor D))] $} 
\end{exercises}

\noindent\problempart Identify the main connective in the each sentence.
\begin{exercises}

\item $[(A \eiff B) \eand C] \eif D$  %$[(A \eiff B) \eand C] $\framebox[1.1\width]{\eif}$ D$ 

\item $[(D \eand (E \eand F)) \eor G] \eiff  \enot [A \eif (C \eor G)] $ %$[(D \eand (E \eand F)) \eor G] $\framebox[1.1\width]{\eiff}$  \enot [A \eif (C \eor G)] $

\item $\enot (\enot Z \eor \enot H) $ %\framebox[1.1\width]{\enot}  $(\enot Z \eor \enot H) $

\item $(\enot (P \eand S) \eiff G) \eand Y $ %$(\enot (P \eand S) \eiff G) $\framebox[1.1\width]{\eand} $ Y $

\item $(A \eand (B \eif C)) \eor \enot D	$  %$(A \eand (B \eif C)) $\framebox[1.1\width]{\eor} $\enot D	$ 

\end{exercises}

\noindent\problempart Assume A, B, and C are true and X, Y, and Z are false and evaluate the truth of the each sentence by writing a one-line truth table.

\begin{longtabu}{p{.1\textwidth}p{.01\textwidth}p{.01\textwidth}p{.01\textwidth}p{.01\textwidth}p{.01\textwidth}p{.01\textwidth}p{.01\textwidth}p{.01\textwidth}p{.01\textwidth}}
\textbf{Example}: & \multicolumn{9}{p{.9\textwidth}}{$(A \eand \enot X) \eiff (B \eor Y)$ }\\
\textbf{Answer}: & (A &\eand &\enot& X)& \eiff	\tikz[overlay, shift={(-1ex,-6pt)}, gray] \draw (0pt,0pt) ellipse (2ex and 18pt); & (B& \eor& Y)&\\
\cline{2-9} 
& T  &    T    &  T    &  F&	T	&	 T&	T   & F&\\
\tabuphantomline
\end{longtabu}

\begin{exercises}
\item $\enot ((A \eand B) \eif X) $

\answer{
\begin{tabu}{c c c c c c}
\enot \tikz[overlay, shift={(-1ex,-6pt)}, red] \draw (0pt,0pt) ellipse (2ex and 18pt);	 &((A &	\eand&	B)&	\eif&	X) \\
\cline{1-6}
 T    &	T&	T&	T&	F&		F \\
\end{tabu}
}

\item $(Y \eor Z) \eiff	 (\enot X \eiff B)$

\answer{
\begin{tabu}{cccccccc}
(Y	&\eor &	Z)	& \eiff \tikz[overlay, shift={(-1ex,-6pt)}, red] \draw (0pt,0pt) ellipse (2ex and 18pt);	&(\enot	&X	&\eiff	&B)\\	
\cline{1-8}
F &	F &	F	&	F	&	T &	F &  	T  &	T\\
\end{tabu}
}

\item $[(X \eif A) \eor (A \eif X)] \eand Y$

\answer{
\begin{tabu}{ccccccccc}
[(X &\eif& A)& \eor& (A& \eif& X)] &\eand	\tikz[overlay, shift={(-1ex,-6pt)}, red] \draw (0pt,0pt) ellipse (2ex and 18pt);	& Y	\\
\cline{1-9}
F&  T&  T&   T&  T&  F& F &   F&  F\\
\end{tabu}
}

\item $(X  \eif  A) \eor  (A \eif X)$

\answer{
\begin{tabu}{ccccccc}
(X & \eif & A) &\eor \tikz[overlay, shift={(-1ex,-6pt)}, red] \draw (0pt,0pt) ellipse (2ex and 18pt);	& (A & \eif& X)\\	
\cline{1-7}
F&  T&  T&   T&  T&  F& F\\   
\end{tabu}
}

\item $[A \eand (Y \eand Z)] \eor A $

\answer{
\begin{tabu}{ccccccc}
[A& \eand &(Y &\eand &Z)]& \eor	\tikz[overlay, shift={(-1ex,-6pt)}, red] \draw (0pt,0pt) ellipse (2ex and 18pt);	& A \\
\cline{1-7}
T&  F&  F&  F& F&   T& T\\
\end{tabu}
}
\end{exercises}


\noindent\problempart
Assume A, B, and C are true and X, Y, and Z are false and evaluate the truth of the each sentence by writing a one-line truth table.. 


\begin{exercises}
\item $\enot  \enot  (\enot  \enot  \enot A  \eor  X) $

%\begin{tabular}{c|c|ccccccc}
%\cline{2-2}
%1. &	\enot & \enot & (\enot & \enot & \enot &A & \eor & X) \\	
%&F	&T	& F& T& F& T & F & F \\
%\cline{2-2}
%\end{tabular}
%\vspace{1em}

\item $(A \eif B) \eif X$	

%\begin{tabular}{cccc|c|c}
%\cline{5-5}
%2. &	(A& \eif& B)& \eif& X	\\
%&T &T&T&F&F\\
%\cline{5-5}
%\end{tabular}

\item $((A \eor B) \eand (C \eiff X)) \eor Y$	

%\begin{tabular}{cccccccc|c|c}
%\cline{9-9}
%3.&	((A &\eor& B)& \eand& (C& \eiff& X))& \eor& Y	\\
%&T&T&T&F&T&F&F&F&F\\
%\cline{9-9}
%\end{tabular}

\item $(A \eif 	B)	\eor 	(X 	\eand 	(Y 	\eand 	Z))$	

%\begin{tabular}{cccc|c|ccccc}
% \cline{5-5}
%4.&	(A&	\eif &	B)&	\eor &	(X &	\eand &	(Y &	\eand &	Z)) \\
%&	T &	T &	T &	T &	F &	F &	F &	F &	F \\
%\cline{5-5}
%\end{tabular}

\item $((A  	\eor 	X) \eif Y) 	\eand B $

%\begin{tabular}{cccccc|c|c}
%\cline{7-7}
%5.&	((A  &	\eor &	X) &	\eif &	Y) &	\eand &	B \\
%&	T &	T &	F &	F &	F &	F &	T \\
%\cline{7-7}
%\end{tabular}

\end{exercises}

\noindent\problempart Write complete truth tables for the following sentences and mark the column that represents the possible truth values for the whole sentence.

\begin{longtabu}{p{.1\linewidth}p{.9\linewidth}}
\textbf{Example}: & $D \eif (D \eand (\enot F \eor F))$ \\
\textbf{Answer}: & \vspace{-8pt} \begin{tabular}[t]{cccccccc}
D 	&\eif 	\tikz[overlay, shift={(-1ex,-20pt)}, gray] \draw (0pt,0pt) ellipse (2ex and 33pt);	&(D 	&\eand 	& (\enot	& F 	& \eor 	&  F))\\
\cline{1-8}
T	&	T	&	T	&	T		&	F	  	&	T	&	T		& T	\\
T	&	T	&	T	&	T		&	T	  	&	F	&	T		& F	\\
F	&	T	&	F	&	F		&	F	  	&	T	&	T		& T	\\
F	&	T	&	F	&	F		&	T	  	&	F	&	T		& F 	\\
\end{tabular}\\
\end{longtabu}



\begin{exercises}

\item $\enot (S \eiff (P \eif S))$

\answer{
\begin{longtabu}{cccccc}
\enot \tikz[overlay, shift={(-1ex,-27pt)}, red] \draw (0pt,0pt) ellipse (2ex and 44pt);	&	(S 	&	\eiff	&	(P 	&	\eif	&	S))	\\ 
\cline{1-6}
F 		&	T	&	T	&	T	&	T	&	T	\\
F 		&	T	&	T	&	F	&	T	&	T	\\
F 		&	F	&	T	&	T	&	F	&	F	\\
T 		&	F	&	F	&	F	&	T	&	F	\\
\end{longtabu}
}

 \item $\enot [(X \eand Y) \eor (X \eor Y)]$

\answer{
\begin{longtabu}{cccccccc}
\enot	\tikz[overlay, shift={(-1ex,-27pt)}, red] \draw (0pt,0pt) ellipse (2ex and 44pt);	&	 [(X 	&	\eand& 	Y) 	&	\eor 	&	(X 	&	\eor 	&	Y)] \\
\cline{1-8}
F	&	T	&	T	&	T	&	T	&	T	&	T	&	T	\\
F	&	T	&	F	&	F	&	T	&	T	&	T	&	F	\\
F	&	F	&	F	&	T	&	T	&	F	&	T	&	T	\\
T	&	F	&	F	&	F	&	F	&	F	&	F	&	F	\\
\end{longtabu}
}

\item $(A \eif B) \eiff (\enot B\eiff \enot A)$

\answer{
\begin{longtabu}{ccccccccc}
(A 	&	\eif	&	B)	&	 \eiff 	\tikz[overlay, shift={(-1ex,-27pt)}, red] \draw (0pt,0pt) ellipse (2ex and 44pt);	&	(\enot&	B 	&	\eiff 	&	 \enot 	& 	 A) \\
\cline{1-9}
T	&	T	&	T	&	T		&	F	 &	T	&	T	&	F		&	T	\\	
T	&	F	&	F	&	T		&	T	 &	F	&	F	&	F		&	T	\\
F	&	T	&	T	&	F		&	F	 &	T	&	F	&	T		&	F	\\
F	&	T	&	F	&	T		&	T	 &	F	&	T	&	T		&	F	\\
\end{longtabu}
}

\item $[C \eiff (D \eor E)] \eand \enot C$

\answer{
\begin{longtabu}{cccccccc}
[C 	&	\eiff 	&	(D 	&	\eor 	&	E)] 	&	\eand 	\tikz[overlay, shift={(-1ex,-52pt)}, red] \draw (0pt,0pt) ellipse (2ex and 77pt);	&	 \enot 	&	 C \\
\cline{1-8}
T	&	T	&	T	&	T	&	T	&	F		&	F		&	T	\\
T	&	T	&	T	&	T	&	F	&	F		&	F		&	T	\\
T	&	T	&	F	&	T	&	T	&	F		&	F		&	T	\\
T	&	F	&	F	&	F	&	F	&	F		&	F		&	T	\\
F	&	F	&	T	&	T	&	T	&	F		&	T		&	F	\\
F	&	F	&	T	&	T	&	F	&	F		&	T		&	F	\\
F	&	F	&	F	&	T	&	T	&	F		&	T		&	F	\\
F	&	T	&	F	&	F	&	F	&	T		&	T		&	F	\\
\end{longtabu}
}

\item $\enot(G \eand (B \eand H)) \eiff (G \eor (B \eor H))$

\answer{
\begin{longtabu}{cccccccccccc}
\enot&	(G 	&\eand &	(B 	&	 \eand 	&	 H))	&	\eiff \tikz[overlay, shift={(-1ex,-52pt)}, red] \draw (0pt,0pt) ellipse (2ex and 77pt); 	&	(G 	& \eor 	& (B 	& \eor	& H))	\\
\cline{1-12}
F	   &	T	&	  T &	T	&	T		&	T	&	F	&	T	&	T	&	T	&	T	&	T	\\
T	   &	T	&	  F &	T	&	F		&	F	&	T	&	T	&	T	&	T	&	T	&	F	\\	
T	   &	T	&	 F  &	F	&	F		&	T	&	T	&	T	&	T	&	F	&	T	&	T	\\
T	   &	T	&	 F  &	F	&	F		&	F	&	T	&	T	&	T	&	F	&	F	&	F	\\
T	   &	F	&	F   &	T	&	T		&	T	&	T	&	F	&	T	&	T	&	T	&	T	\\
T	   &	F	&	F   &	T	&	F		&	F	&	T	&	F	&	T	&	T	&	T	&	F	\\
T	   &	F	&	F   &	F	&	F		&	T	&	T	&	F	&	T	&	F	&	T	&	T	\\
T	   &	F	&	F   &	F	&	F		&	F	&	F	&	F	&	F	&	F	&	F	&	F	\\
\end{longtabu}
}

\end{exercises}

\noindent\problempart Write complete truth tables for the following sentences and mark the column that represents the possible truth values for the whole sentence.

\begin{exercises}

\item	$(D \eand \enot D) \eif G $

%\vspace{1em}

%\begin{tabular}{ccccc|c|c}
%\cline{6-6}
%1.	&	(D 	&	 \eand 	& 	 \enot	&	 D) 	&	 \eif 	&	 G \\
%	&	T	&	F		&	F		&	T	&	T	&	T	\\
%	&	T	&	F		&	F		&	T	&	T	&	F	\\
%	&	F	&	F		&	T		&	F	&	T	&	T	\\
%	&	F	&	F		&	T		&	F	&	T	&	F	\\
%\cline{6-6}
%\end{tabular}
%\vspace{1em}


\item	$(\enot P \eor \enot M) \eiff M $

%\begin{tabular}{cccccc|c|c}
%\cline{7-7}
%2.	&	(\enot 	&	P 	&	\eor 	&	\enot 	& 	 M) 	& 	\eiff 	&	 M \\
%	&	F		&	T	&	F	&	F		&	T	&	T	&	T	\\
%	&	F		&	T	&	T	&	T		&	F	&	F	&	F	\\
%	&	T		&	F	&	T	&	F		&	T	&	T	&	T	\\
%	&	T		&	F	&	T	&	T		&	F	&	T	&	F	\\
%\cline{7-7}
%\end{tabular}
%\vspace{1em}



\item	$\enot \enot (\enot A \eand \enot B)  $

%\begin{tabular}{c|c|cccccc}
%\cline{2-2}
%3.	&	\enot		&	 \enot 	&	(\enot 	& 	 A 	& \eand 	& 	\enot 	&	 B)  \\
%	&	F		&	T		&	F		&	T	&	F	&	F		&	T	\\
%	&	F		&	T		&	F		&	T	&	F	&	T		&	F	\\
%	&	F		&	T		&	T		&	F	&	F	&	F		&	T	\\
%	&	T		&	F		&	T		&	F	&	T	&	T		&	F	\\
%\cline{2-2}
%\end{tabular}
%\vspace{1em}



\item 	$[(D \eand R) \eif I] \eif \enot(D \eor R) $

%\begin{tabular}{cccccc|c|cccc}
%\cline{7-7}
%4.	&	[(D 	& 	 \eand 	& 	 R)	& 	\eif 	&	I] 	&	\eif 	&	 \enot 	&	(D 	&	 \eor 	& R) \\
%	&	T	&	T		&	T	&	T	&	T	&	F	&	F		&	T	&	T		&T	\\
%	&	T	&	T		&	T	&	F	&	F	&	T	&	F		&	T	&	T		&T	\\
%	&	T	&	F		&	F	&	T	&	T	&	F	&	F		&	T	&	T		&F	\\
%	&	T	&	F		&	F	&	T	&	F	&	F	&	F		&	T	&	T		&F	\\
%	&	F	&	F		&	T	&	T	&	T	&	F	&	F		&	F	&	T		&T	\\
%	&	F	&	F		&	T	&	T	&	F	&	F	&	F		&	F	&	T		&T	\\
%	&	F	&	F		&	F	&	T	&	T	&	T	&	T		&	F	&	F		&F	\\
%	&	F	&	F		&	F	&	T	&	F	&	T	&	T		&	F	&	F		&F	\\
%\cline{7-7}
%\end{tabular}
%	
%\vspace{1em}


\item	$\enot [(D \eiff O) \eiff A] \eif (\enot D \eand O) $

%\begin{tabular}{ccccccc|c|cccc}
%\cline{8-8}
%5.	&	\enot 	&	[(D 	&	\eiff 	&	O) 	&	\eiff 	&	 A]	& 	\eif 	 &	(\enot 	& 	D 	 & 	 \eand &O) \\ 
%	&	F		&	T	&	T	&	T	&	T	&	T	&	T	&	F		&	T	&	F	&T	\\
%	&	T		&	T	&	T	&	T	&	F	&	F	&	F	&	F		&	T	&	F	&T	\\
%	&	T		&	T	&	F	&	F	&	F	&	T	&	F	&	F		&	T	&	F	&F	\\
%	&	F		&	T	&	F	&	F	&	T	&	F	&	T	&	F		&	T	&	F	&F	\\
%	&	T		&	F	&	F	&	T	&	F	&	T	&	T	&	T		&	F	&	T	&T	\\
%	&	F		&	F	&	F	&	T	&	T	&	F	&	T	&	T		&	F	&	T	&T	\\
%	&	F		&	F	&	T	&	F	&	T	&	T	&	T	&	T		&	F	&	F	&F	\\
%	&	T		&	F	&	T	&	F	&	F	&	F	&	T	&	T		&	F	&	F	&F	\\
%\cline{8-8}
%\end{tabular}
%\vspace{1em}


\end{exercises}


% *********************************************
% *   Using Truth Tables								*
% *********************************************

\section{Using Truth Tables}

Because truth table show all the possible interpretations of a sentence or set of sentences we can use them to explore the logical properties we first introduced in Chapter \ref{chap:whatisformallogic}.

\subsection{Tautologies, contradictions, and contingent sentences}
We defined a tautology as a statement that must be true as a matter of logic, no matter how the world is (p. \pageref{def:tautology}). A statement like ``Either it is raining or it is not raining'' is always true, no matter what the weather is like outside. Something similar goes on in truth tables. With a complete truth table, we consider all of the ways that the world might be. Each line of the truth table corresponds to a way the world might be. This means that if the sentence is true on every line of a complete truth table, then it is true as a matter of logic, regardless of what the world is like.

We can use this fact to create a test for whether a sentence is a tautology: if the column under the main connective of a sentence is a T on every row, the sentence is a tautology. We already have seen an example of this. On page \pageref{tautology3.1} that the sentence $(H \eand I)\eif H$ had only T's under its main connective, so it is a tautology.

Not every tautology in English will correspond to a tautology in SL. The sentence ``All bachelors are unmarried'' is a tautology in English, but we cannot represent it as a tautology in SL, because it just translates as a single sentence letter, like $B$. On the other hand, if something is a tautology in SL, it will also be a tautology in English. No matter how you translate $A \eor \enot A$, if you translate the $A$s consistently, the statement will be a tautology. 

\newglossaryentry{semantic tautology in SL}
{
name=semantic tautology in SL,
description={A statement that has only Ts in the column under the main connective of its complete truth table.}
}

\label{semantic_definitions_in_SL}
Rather than thinking of complete truth tables as an imperfect test for the English notion of a tautology, we can define a separate notion of a tautology in SL based on truth tables. A statement is a \textsc{\gls{semantic tautology in SL}} \label{def:semantic_tautology_in_sl} if and only if the column  under the main connective in the complete truth table for the sentence contains only Ts. This is the semantic definition of a tautology in SL, because it uses truth tables. Later we will create a separate, syntactic definition and show that it is equivalent to the semantic definition. We will be doing the same thing for all the concepts defined in this section. 

\newglossaryentry{semantic contradiction in SL}
{
name=semantic contradiction in SL,
description={A statement that has only Fs in the column under the main connective of its complete truth table.}
}

We defined a contradiction as a sentence that is false no matter how the world is (p. \pageref{def:contradiction}). This means we can define a \textsc{\gls{semantic contradiction in SL}} \label{def:semantic_contradiction_in_sl} as a sentence that has only Fs in the column under them main connective of its complete truth table. We saw on page \pageref{contradiction3.1} that the sentence $[(C\eiff C) \eif C] \eand \enot(C \eif C)$ was a contradiction in this sense. As with the definition of a semantic tautology, this is a semantic definition because it uses truth tables. 
		
\newglossaryentry{semantically contingent in SL}
{
name=semantically contingent in SL,
description={A property held by a sentence in SL if and only if the complete truth table for that sentence has both Ts and Fs under its main connective.}
}

Finally, a sentence is contingent if it is sometimes true and sometimes false (p. \pageref{def:contingent_statement}). Similarly, a sentence is \textsc{\gls{semantically contingent in SL}} \label{def:semantically_contingent_in_sl} if and only if its complete truth table for has both Ts and Fs under the main connective. We saw on page \pageref{contingentsentence3.1} that the sentence $M \eand (N \eor P)$ was contingent.

\subsection{Logical equivalence}

\newglossaryentry{semantically logically equivalent in SL}
{
name=semantically logically equivalent in SL,
description={A property held by pairs of sentences if and only if the complete truth table for those sentences has identical columns under the two main connectives.}
}

Two sentences are logically equivalent in English if they have the same truth value as a matter of logic (p. \pageref{def:logical_equivalence}). Once again, we can use truth tables to define a similar property in SL: Two sentences are \textsc{\gls{semantically logically equivalent in SL}} \label{def:semantically_logically_equivalent_in_sl} if they have the same truth value on every row of a complete truth table.

Consider the sentences $\enot(A \eor B)$ and $\enot A \eand \enot B$. Are they logically equivalent? To find out, we construct a truth table.
\begin{center}
\begin{tabu}{ccccc|cccccc}
\enot	\tikz[overlay, shift={(-1ex,-30pt)}, gray] \draw (0pt,0pt) ellipse (2ex and 44pt);		&	$(A$	&	\eor	&	$B)$	&	&	&	\enot	&	$A$	&	\eand	\tikz[overlay, shift={(-1ex,-30pt)}, gray] \draw (0pt,0pt) ellipse (2ex and 44pt); &	\enot	&	$B$\\
\hline
F	& 	T 		& T 		& T 		& 	&	&	F & T & F & F & T\\
F 	&	T 		& T 		& F 		& 	&	&	F & T & F & T & F\\
F 	& 	F 		& T		& T 		& 	&	&	T & F & F & F & T\\
T 	& 	F 		& F 		& F 		& 	&	&	T & F & T & T & F
\end{tabu}
\end{center}
Look at the columns for the main connectives; negation for the first sentence, conjunction for the second. On the first three rows, both are F. On the final row, both are T. Since they match on every row, the two sentences are logically equivalent.

\subsection{Consistency}

\newglossaryentry{semantically consistent in SL}
{
name=semantically consistent in SL,
description={A property held by sets of sentences if and only if the complete truth table for that set contains one line on which all the sentences are true}
}

A set of sentences in English is consistent if it is logically possible for them all to be true at once (p. \pageref{def:inconsistency}).
This means that a sentence is \textsc{\gls{semantically consistent in SL}} \label{def:semantically_consistent_in_sl} if and only if there is at least one line of a complete truth table on which all of the sentences are true. It is semantically inconsistent otherwise.

Consider the three sentences $A \eif B$, $B \eif C$ and $C \eif A$. Since we are considering them as a set, we will put curly braces around them, as is done in 
set theory: \{$A \eif B, B \eif C, C \eif A$\}. The conditionals in this set form a little loop, but it is possible for all the sentences to be true at the same time, as this truth table shows.

\begin{longtabu}{cccc|ccccc|cccc}
A	&	\eif	&	B	&	&	&	B	&	\eif	&	C	&	&	&	C	&	\eif	&	A	\\
\cline{1-13}
T	\tikz[overlay, shift={(100pt,1ex)}, gray] \draw (0pt,0pt) ellipse (132pt and 2ex); &	T		&	T	&	&	&	T	&	T		&	T	&	&	&	T	&		T	&	T	\\
T	&	T		&	T	&	&	&	T	&	F		&	F	&	&	&	F	&		T	&	T	\\
T	&	F		&	F	&	&	&	F	&	T		&	T	&	&	&	T	&		T	&	T	\\
T	&	F		&	F	&	&	&	F	&	T		&	F	&	&	&	F	&		T	&	T	\\
F	&	T		&	T	&	&	&	T	&	T		&	T	&	&	&	T	&		F	&	F	\\
F	&	T		&	T	&	&	&	T	&	F		&	F	&	&	&	F	&		T	&	F	\\
F	&	T		&	F	&	&	&	F	&	T		&	T	&	&	&	T	&		F	&	F	\\
F	\tikz[overlay, shift={(100pt,1ex)}, gray] \draw (0pt,0pt) ellipse (132pt and 2ex); &	T		&	F	&	&	&	F	&	T		&	F	&	&	&	F	&		T	&	F	\\			
\end{longtabu}

\subsection{Validity}

\newglossaryentry{semantically valid in SL}
{
name=semantically valid in SL,
description={A property held by arguments if and only if the complete truth table for the argument contains no rows where the premises are all true and the conclusion false.}
}

Logic is the study of argument, so the most important use of truth tables is to test the validity of arguments. An argument in English is valid if it is logically impossible for the premises to be true and for the conclusion to be false at the same time (p. \pageref{def:valid}). So we can define an argument as \textsc{\gls{semantically valid in SL}} \label{def:semantically_valid_in_sl} if there is no row of a complete truth table on which the premises are all marked ``T'' and the conclusion is marked ``F.'' An argument is invalid if there is such a row.

Consider this argument:
\begin{earg}
\item[1.] $\enot L \eif (J \eor L)$
\item[2.] $\enot L$
\item[] \textcolor{white}{.}\sout{\hspace{.2\linewidth}} \textcolor{white}{.} 
\item[$\therefore$] $J$
\end{earg}
Is it valid? To find out, we construct a truth table.

\begin{center}
\tabulinesep=.5ex
\begin{longtabu}{c|c|@{\TTon}*{6}{c}@{\TToff}|@{\TTon}*{2}{c}@{\TToff}|@{\TTon}c@{\TToff}}
$J$&$L$&\enot&$L$&\eif \tikz[overlay, shift={(-1.25ex,-24pt)}, gray] \draw (0pt,0pt) ellipse (2ex and 36pt);&$(J$&\eor&$L)$&\enot\tikz[overlay, shift={(-1ex,-24pt)}, gray] \draw (0pt,0pt) ellipse (2ex and 36pt);&L&J\tikz[overlay, shift={(-.75ex,-24pt)}, gray] \draw (0pt,0pt) ellipse (2ex and 36pt);\\
\hline
%J   L   -   L      ->     (J   v   L)
 T & T & F & T & T & T & T & T & F & T & T\\
 T & F & T & F & T & T & T & F & T & F & T\\
 F & T & F & T & T & F & T & T & F & T & F\\
 F & F & T & F & F & F & F & F & T & F & F
\end{longtabu}
\end{center}

Yes, the argument is valid.
The only row on which both the premises are T is the second row, and on that row the conclusion is also T.

In Chapters 1 and 2 we used the three dots $\therefore$ to represent an inference in English. We used this symbol to represent any kind of inference. The truth table method gives us a more specific notion of a valid inference. We will call this semantic entailment and represent it using a new symbol, $\sdtstile{}{},$ called the ``double turnstile.'' \label{defDoubleTurnstile} The $\sdtstile{}{}$ is like the $\therefore$, except for arguments verified by truth tables. When you use the double turnstile, you write the premises as a set, using curly brackets, \{ and \}, which mathematicians use in set theory. The argument above would be written  $ \{ \enot L \eif (J \eor L), \enot L \} \sdtstile{}{} J$.

More formally, we can define the double turnstile this way: $ \{ \script{A_1}\ldots \script{A_n} \} \sdtstile{}{} \script{B} $ if and only if there is no truth value assignment for which \script{A_1}\ldots \script{A_n} are true and \script{B} is false. Put differently, it means that \script{B} is true for any and all truth value assignments for which \script{A_1}\ldots \script{A_n} are true.

We can also use the double turnstile to represent other logical notions. Since a tautology is always true, it is like the conclusion of a valid argument with no premises. The string $\sdtstile{}{}\script{C}$ means that \script{C} is true for all truth value assignments. This is equivalent to saying that the sentence is entailed by anything. We can represent logical equivalence by writing the double turnstile in both directions: $\script{A} \ndststile{}{} \hspace{.5em} \sdtstile{}{} \script{B}$ For instance, if we want to point out that the sentence $A \eand B$ is equivalent to $B \eand A$ we would write this: $A \eand B \ndststile{}{} \hspace{.5em} \sdtstile{}{} B \eand A$. 

%%%%%%%%%%%%%%%%%%Practice Problems

\practiceproblems

If you want additional practice, you can construct truth tables for any of the sentences and arguments in the exercises for the previous chapter.


\noindent\problempart Determine whether each sentence is a tautology, a contradiction, or a contingent sentence, using a complete truth table.

\begin{longtabu}{p{.1\linewidth}p{.9\linewidth}}
\textbf{Example}: & $(A \eif B) \eor (B \eif A)$ \\
\textbf{Answer}: & \vspace{-8pt}\begin{tabular}[t]{cccccccc} 
	 (A 	 	 & 	 \eif 	& 	 B) 	 	 & 	 \eor \tikz[overlay, shift={(-.75ex,-24pt)}, gray] \draw (0pt,0pt) ellipse (2ex and 36pt);	 & 	(B 	 	 & 	 \eif	 	 	 & 	 A)	 	 & 	 Tautology\\ 
\cline{1-7}
 T 	 	 & 	 T 		& 	T 	 	 & 	 T 		 & 	 T 	 	 & 	 T 	 	 & 	T 	 	 & 	 \\ 
 T 	 	 & 	 F 		& 	F 	 	 & 	 T 	 	 & 	 F 	 	 & 	 T 	 	 & 	T 	 	 & 	  \\ 
 F 	 	 & 	 T 		& 	T 	 	 & 	 T 	 	 & 	 T 	 	 & 	 F 	 	 & 	F 	 	 & 	 \\ 
 F 	 	 & 	 T		& 	F 	 	 & 	 T 	 	 & 	 F 	 	 & 	 T 	 	 & 	F 	 	 & 	 \\ 

\end{tabular}\\
\end{longtabu}

\begin{exercises}
\item $A \eif A$ 

\answer{
\begin{longtabu}{ccccc}
A 	&\eif \tikz[overlay, shift={(-1ex,-12pt)}, red] \draw (0pt,0pt) ellipse (2ex and 24pt);	& A & Tautology\\
\cline{1-3}
T		&	T	& T	 &			\\
F		&	T	& F	 &			\\
\end{longtabu}
}


\item $C \eif\enot C$ 

\answer{
\begin{longtabu}{ccccc}

C 	& \eif \tikz[overlay, shift={(-1ex,-12pt)}, red] \draw (0pt,0pt) ellipse (2ex and 24pt);	& \enot 	& C 	& Contingent \\
\cline{1-4}
T	&	F	&	F	& 	T	&			\\
F	&	T	&	T	& 	F	&	\\

\end{longtabu}
}

\item $(A \eiff B) \eiff \enot(A\eiff \enot B)$ %tautology

\answer{
\begin{longtabu}{cccccccccc}
(A 	& \eiff 	& B) 	& \eiff \tikz[overlay, shift={(-1ex,-30pt)}, red] \draw (0pt,0pt) ellipse (2ex and 44pt);	& \enot 	& (A 	& \eiff 	& \enot 	& B) 	& Tautology \\
\cline{1-9}
T	&	T	&	T 	&	T	&	T	&	T	&	F	&	F	& 	T	&	\\	
T	&	F	&	F	&	T	&	 F	&	T	&	T	&	T	& 	F	&	\\
F	&	F	&	T	&	T	&	 F	&	F	&	T	&	F	& 	T	&	\\
F	&	T	&	F	&	T	&	 T	&	F	&	F	&	T	& 	F	&	\\
\end{longtabu}
}


\item $(A \eand B) \eif (B \eor A)$  %taut

\answer{
    \begin{longtabu}{cccccccc} 
(A  	 	 & 	 \eand  	  & 	 B)  & 	 \eif  \tikz[overlay, shift={(-1ex,-30pt)}, red] \draw (0pt,0pt) ellipse (2ex and 44pt);	 & 	 (B 	 	 & 	 \eor  	 & 	 A)   	 	 & 	 Tautology\\ 
\cline{1-7} 
T 	 	 & 	 T 	 	 & 	 T 	& 	 T 	 	 & 	 T 	 	 & 	 T 	 	 & 	T 	 	 & 	   \\ 
T 	 	 & 	 F 	 	 & 	 F 	& 	 T 	 	 & 	 F 	 	 & 	 T 	 	 & 	T 	 	 & 	   \\ 
F 	 	 & 	 F 	 	 & 	 T 	& 	 T 	 	 & 	 T 	 	 & 	 T 	 	 & 	F 	 	 & 	   \\ 
F 	 	 & 	 F 	 	 & 	 F 	& 	 T 	 	 & 	 F 	 	 & 	 F 	 	 & 	F 	 	 & 	   \\ 
\end{longtabu}
}


\item $[(\enot A \eor A) \eor B] \eif B$ %taut.

\answer{
 \begin{longtabu}{ccccccccc} 
[(\enot  	  & 	 A  	 	 & 	 \eor  	 & 	A)	  	 & 	\eor	 	 & 	B]   	 & 	 \eif	\tikz[overlay, shift={(-1ex,-30pt)}, red] \draw (0pt,0pt) ellipse (2ex and 44pt);	 & 	 B 	 	 & 	 Contingent sentence \\ 
\cline{1-8} 
F 	 	 & 	 T 	 	 & 	 T 	 	 & 	 T 	 	 & 	 T 	 	 & 	 T 	 	 & 	 T 	 	 & 	 T 	 	 & 	 \\ 
F 	 	 & 	 T 	 	 & 	 T 	 	 & 	 T 	 	 & 	 T 	 	 & 	 F 	 	 & 	 F 	 	 & 	 F 	 	 & 	 \\ 
T 	 	 & 	 F 	 	 & 	 T 	 	 & 	 F 	 	 & 	 T 	 	 & 	 T 	 	 & 	 T 	 	 & 	 T 	 	 & 	 \\ 
T 	 	 & 	 F 	 	 & 	 T 	 	 & 	 F 	 	 & 	 T 	 	 & 	 F 	 	 & 	 F 	 	 & 	 F 	 	 & 	 \\ 

\end{longtabu}
}

\item $[(A \eor B) \eand \enot A] \eand (B \eif A)$ %Contradiction. 

\answer{
\begin{longtabu}{ccccccccccc} 
[(A  	 &	\eor  & 	 B) 	 & 	 \eand 	& 	\enot & 	 A] 	 	 & 	 \eand \tikz[overlay, shift={(-1ex,-30pt)}, red] \draw (0pt,0pt) ellipse (2ex and 44pt); & (B 	 	 & 	 \eif  	 & 	 A) 	 	& Contradiction. \\ 
\cline{1-10} 
T 	 & 	 T 	 & 	T 	 & 	 F 	 	 & 	 F 	  & 	 T 	 	 & 	 F 	  & 	 T 	 	 & 	 T 	 	 & 	 T 	 	 & 	  \\ 
T 	 & 	 T 	 & 	F 	 & 	 F 	 	 & 	 F 	 & 	 T 	 	 & 	 F 	 & 	 F 	 	 & 	 T 	 	 & 	 T 	 	 & 	  \\ 
F 	 & 	 T 	 & 	T 	 & 	 T 	 	 & 	 T 	 & 	 F 	 	 & 	 F 	 & 	 T 	 	 & 	 F 	 	 & 	 F 	 	 & 	 \\ 
F 	 & 	 F 	 & 	F 	 & 	 F 	 	 & 	 T 	 & 	 F 	 	 & 	 F 	 & 	 F 	 	 & 	 T 	 	 & 	 F 	 	 & 	 \\ 
\end{longtabu}
}
\end{exercises}

\noindent\problempart Determine whether each sentence is a tautology, a contradiction, or a contingent sentence, using a complete truth table.
\begin{exercises}
\item $\enot B \eand B$ \vspace{.5ex}%contra


\item $\enot D \eor D$ \vspace{.5ex}%taut


\item $(A\eand B) \eor (B\eand A)$\vspace{.5ex} %contingent


\item $\enot[A \eif (B \eif A)]$\vspace{.5ex} %contra


\item $A \eiff [A \eif (B \eand \enot B)]$ \vspace{.5ex}%contra


\item $[(A \eand B) \eiff B] \eif (A \eif B)$ \vspace{.5ex}% contingent. 

\end{exercises}



\noindent\problempart \label{pr.TT.equiv} Determine whether each the following statements are equivalent using complete truth tables. If the two sentences really are logically equivalent, write "Logically equivalent." Otherwise write, "Not logically equivalent." 

\begin{longtabu}{p{.1\linewidth}p{.9\linewidth}}
\textbf{Example}: & $A \eor B  \ndststile{}{} \hspace{.5em} \sdtstile{}{} \enot A \eif B $\\
\textbf{Answer}: & \vspace{-8pt}\begin{tabular}[t]{ccccccccc}
A	&	\eor \tikz[overlay, shift={(-.75ex,-24pt)}, gray] \draw (0pt,0pt) ellipse (2ex and 36pt);	&	B	&	&	\enot	&	A	&	\eif \tikz[overlay, shift={(-.75ex,-24pt)}, gray] \draw (0pt,0pt) ellipse (2ex and 36pt);&	B	&	Logically Equivalent\\ 
\cline{1-3} \cline{5-7}
T	&	T		&	T	&	&		F	&	T	&	T		&	T	&	\\
T	&	T		&	F	&	&		F	&	T	&	T		&	F	&	\\
F	&	T		&	T	&	&		T	&	F	&	T		&	T	&	\\
F	&	F		&	F	&	&		T	&	F	&	F		&	F	&	\\
\end{tabular}\\
\end{longtabu}


\begin{exercises}
\item $A\ndststile{}{} \hspace{.5em} \sdtstile{}{} \enot A$\vspace{.5ex} %No

\answer{
\begin{longtabu}{ccccc} 
A 	 \tikz[overlay, shift={(-1.25ex,-24pt)}, red] \draw (0pt,0pt) ellipse (2ex and 36pt);	 & 	  	 	 & 	 \enot \tikz[overlay, shift={(-.75ex,-24pt)}, red] \draw (0pt,0pt) ellipse (2ex and 36pt);	 	 & 	 A 	 	 & 	 Not logically equivalent \\ 
\cline{1-1} \cline{3-4}
T 	 	 & 	   	 	 & 	 F 	 	 & 	 T 	 	 & 	  \\ 
F 	 	 & 	   	 	 & 	 T 	 	 & 	 F 	 	 & 	  \\ 
\end{longtabu}
}

\item $A \eand \enot A\ndststile{}{} \hspace{.5em} \sdtstile{}{} \enot B \eiff B$\vspace{.5ex} %Yes

\answer{
\begin{longtabu}{cccccccccc} 

A	 & 	 	\eand \tikz[overlay, shift={(-1.25ex,-24pt)}, red] \draw (0pt,0pt) ellipse (2ex and 36pt);	 & 	 \enot	  & 	 A	 	 & 	 	 	 & 	 \enot	 & 	 B 	 	& 	\eiff \tikz[overlay, shift={(-1.25ex,-24pt)}, red] \draw (0pt,0pt) ellipse (2ex and 36pt);	 & 	 B 	 & 	 Logically equivalent \\ 
\cline{1-4} \cline{6-9} 
T 	 	 & 	 F 	 	 & 	 F 	 	 & 	 T 	 	 & 	  	 	 & 	 F 	 	 & 	 T 	 	 & 	 F 	 	 & 	 T 	 	 & 	  \\ 
T 	 	 & 	 F 	 	 & 	 F 	 	 & 	 T 	 	 & 	  	 	 & 	 T 	 	 & 	 F 	 	 & 	 F 	 	 & 	 F 	 	 & 	  \\ 
F 	 	 & 	 F 	 	 & 	 T 	 	 & 	 F 	 	 & 	  	 	 & 	 F 	 	 & 	 T 	 	 & 	 F 	 	 & 	 T 	 	 & 	  \\ 
F 	 	 & 	 F 	 	 & 	 T 	 	 & 	 F 	 	 & 	  	 	 & 	 T 	 	 & 	 F 	 	 & 	 F 	 	 & 	 F 	 	 & 	  \\ 
\end{longtabu}
}

\item $[(A \eor B) \eor C]\ndststile{}{} \hspace{.5em} \sdtstile{}{} [A \eor (B \eor C)]$\vspace{.5ex} %Yes

\answer{
\begin{longtabu}{cccccccccccc} 
(A		 & 	 \eor	 & 	 	B) 	 & 	\eor	\tikz[overlay, shift={(-1ex,-52pt)}, red] \draw (0pt,0pt) ellipse (2ex and 77pt); 	 	 & 	 C	 	 & 	 	 	 & A 	 	 & 	\eor	\tikz[overlay, shift={(-1ex,-52pt)}, red] \draw (0pt,0pt) ellipse (2ex and 77pt); 	 	 & 	(B 	 	 & 	 \eor 	 & 	C) 	 	 & 	Logically equivalent  \\ 
\cline{1-5} \cline{7-11} 
T	 	 & 	 T	 	 & 	 	T 	 & 	 T	 	 & 	T 	 	 & 	 	 	 & 	 T	 	 & 	T 	 	 & 	 T	 	 & 	 T	 	 & 	 T	 	 & 	  \\ 
T	 	 & 	 T	 	 & 	 	 T	 & 	 T	 	 & 	 F	 	 & 	 	 	 & 	 T	 	 & 	 T	 	 & 	 T	 	 & 	 T	 	 & 	 F	 	 & 	  \\ 
T	 	 & 	 T	 	 & 	 	 F	 & 	 T	 	 & 	 T	 	 & 	 	 	 & 	 T	 	 & 	 T	 	 & 	 F	 	 & 	 T	 	 & 	 T	 	 & 	  \\ 
T	 	 & 	 T	 	 & 	 	 F	 & 	 T	 	 & 	 F	 	 & 	 	 	 & 	 T	 	 & 	 T	 	 & 	 F	 	 & 	 T	 	 & 	 F	 	 & 	  \\ 
F	 	 & 	 T	 	 & 	 	 T	 & 	 T	 	 & 	 T	 	 & 	 	 	 & 	 F	 	 & 	 T	 	 & 	 T	 	 & 	 T	 	 & 	 T	 	 & 	  \\ 
F	 	 & 	 T	 	 & 	 	 T	 & 	 T	 	 & 	 F	 	 & 	 	 	 & 	 F	 	 & 	 T	 	 & 	 T	 	 & 	 T	 	 & 	 F	 	 & 	  \\ 
F	 	 & 	 F	 	 & 	 	 F	 & 	 T	 	 & 	 T	 	 & 	 	 	 & 	 F	 	 & 	 T	 	 & 	 F	 	 & 	 F	 	 & 	 T	 	 & 	  \\ 
F	 	 & 	 F	 	 & 	 	 F	 & 	 F	 	 & 	 F	 	 & 	 	 	 & 	 F	 	 & 	 F	 	 & 	 F	 	 & 	 F	 	 & 	 F	 	 & 	  \\ 

\end{longtabu}
}

\item $A \eor (B \eand C)\ndststile{}{} \hspace{.5em} \sdtstile{}{} (A \eor B) \eand (A \eor C)$\vspace{.5ex} %Equivalent

\answer{
\begin{longtabu}{cccccccccccccc} 
A	 & 	 \eor \tikz[overlay, shift={(-1ex,-52pt)}, red] \draw (0pt,0pt) ellipse (2ex and 77pt); 		 & 	(B 	 	 & 	 \eand 	 & 	 C)	 	 & 	 	 	 & 	 (A	 	 & 	 	\eor	 & 	 	B) 	 & 	 \eand \tikz[overlay, shift={(-1ex,-52pt)}, red] \draw (0pt,0pt) ellipse (2ex and 77pt); 		 & 	 (A	 	 & 	 \eor 	 & 	 C) 	& Logically equivalent\\ 
\cline{1-13} 
T	 & 	 T	 	 & 	 T	 	 & 	 	T 	 & 	T 	 	 & 	 	 	 & 	 T	 	 & 	 T	 	 & 	 	T 	 & 	 T	 	 & 	 T	 	 & 	T 	 	 & 	 T 	 & \\ 
T	 & 	 T	 	 & 	 T	 	 & 	 	 F	 & 	 F	 	 & 	 	 	 & 	 T	 	 & 	 T	 	 & 	 	 T	 & 	 T	 	 & 	 T	 	 & 	 T	 	 & 	 F 	 &  \\ 
T	 & 	 T	 	 & 	 F	 	 & 	 F	 	 & 	 T	 	 & 	 	 	 & 	 T	 	 & 	 T	 	 & 	 F	 	 & 	 T	 	 & 	 T	 	 & 	 T	 	 & 	 T 	 &  \\ 
T	 & 	 T	 	 & 	 F	 	 & 	 F	 	 & 	 F	 	 & 	 	 	 & 	 T	 	 & 	 T	 	 & 	 F	 	 & 	 T	 	 & 	 T	 	 & 	 T	 	 & 	 F 	 &   \\ 
F	 & 	 T	 	 & 	 T	 	 & 	 T	 	 & 	 T	 	 & 	 	 	 & 	 F	 	 & 	 T	 	 & 	 T	 	 & 	 T	 	 & 	 F	 	 & 	 T	 	 & 	 T 	 &  \\ 
F	 & 	 F	 	 & 	 T	 	 & 	 F	 	 & 	 F	 	 & 	 	 	 & 	 F	 	 & 	 T	 	 & 	 T	 	 & 	 F	 	 & 	 F	 	 & 	 F	 	 & 	 F 	 &   \\ 
F	 & 	 F	 	 & 	 F	 	 & 	 F	 	 & 	 T	 	 & 	 	 	 & 	 F	 	 & 	 F	 	 & 	 F	 	 & 	 F	 	 & 	 F	 	 & 	 T	 	 & 	 T 	 &  \\ 
F	 & 	 F	 	 & 	 F	 	 & 	 F	 	 & 	 F	 	 & 	 	 	 & 	 F	 	 & 	 F	 	 & 	 F	 	 & 	 	F 	 & 	 F	 	 & 	 F	 	 & 	F 	 &    \\ 
\end{longtabu}
}

\item $[A \eand (A \eor B)] \eif B\ndststile{}{} \hspace{.5em} \sdtstile{}{} A \eif B$\vspace{.5ex} %Equivalent. 

\answer{
\begin{longtabu}{cccccccccccc} 
[A	 & 	\eand 	 & 	 (A	 	 & 	 \eor	 & 	B)] 	 	 & \eif 	
\tikz[overlay, shift={(-.75ex,-24pt)}, red] \draw (0pt,0pt) ellipse (2ex and 36pt); 	 & 	 	B 	 & 	 	 	 & 	 A	 	 & \eif 	
\tikz[overlay, shift={(-.75ex,-24pt)}, red ] \draw (0pt,0pt) ellipse (2ex and 36pt); 	 & 	 	B 	 & 	Logically equivalent  \\ 
\cline{1-7} \cline{9-11}
T	  & 	 	T 	 & 	 T	 	 & 	 T	 	 & 	T 	 	 & 	 T	 	 & 	 T	 	 & 	 	 	 & 	 T	 	 & 	T 	 	 & 	T 	 	 & 	  \\ 
T	  & 	 	T 	 & 	 T	 	 & 	 T	 	 & 	 F	 	 & 	 F	 	 & 	 F	 	 & 	 	 	 & 	 T	 	 & 	 F	 	 & 	 F	 	 & 	  \\ 
F	  & 	 	 F	 & 	 F	 	 & 	 T	 	 & 	 T	 	 & 	 T	 	 & 	 T	 	 & 	 	 	 & 	 F	 	 & 	 T	 	 & 	 T	 	 & 	  \\ 
F	  & 	 	 F	 & 	 F	 	 & 	 F	 	 & 	 F	 	 & 	 T	 	 & 	 F	 	 & 	 	 	 & 	 F	 	 & 	 T	 	 & 	 F	 	 & 	  \\ 
\end{longtabu}
}

\end{exercises}


\noindent\problempart
\label{pr.TT.equiv}
Determine whether each the following statements of equivalence are true or false using complete truth tables. If the two sentences really are logically equivalent, write "Logically equivalent." Otherwise write, "Not logically equivalent." 
\begin{exercises}
\item $A\eif A\ndststile{}{} \hspace{.5em} \sdtstile{}{} A \eiff A$ \vspace{.5ex}%No
\item $\enot(A \eif B)\ndststile{}{} \hspace{.5em} \sdtstile{}{} \enot A \eif \enot B$\vspace{.5ex} %No
\item $A \eor B\ndststile{}{} \hspace{.5em} \sdtstile{}{} \enot A \eif B$ \vspace{.5ex}%equivalent. 
\item$(A \eif B) \eif C\ndststile{}{} \hspace{.5em} \sdtstile{}{} A \eif (B \eif C)$\vspace{.5ex} %not equivalent.
\item $A \eiff (B \eiff C)\ndststile{}{} \hspace{.5em} \sdtstile{}{} A \eand (B \eand C)$ \vspace{.5ex}%not equivalent. 
\end{exercises}


\noindent\problempart \label{pr.TT.consistent} Determine whether each set of sentences is consistent or inconsistent using a complete truth table. 

\begin{longtabu}{p{.1\linewidth}p{.9\linewidth}}
\textbf{Example}: & \{$\enot(A \eor B)$, $\enot A \eor B$, $A \eor \enot B$\}\\
\textbf{Answer}: & \begin{tabular}[t]{ccccccccccccccc}
\enot	&	(A	&	\eor	&	B),	&	&	\enot	&	A	&	\eor	&	B,	&	&	A	&	\eor	&	\enot	&	B	&	Consistent \\	
\cline{1-4}	\cline{6-9}	\cline{11-14}
F		&	T	&	T		&	T	&	&		F	&	T	&	T		&	T	&	&	T	&	T		&	F		&	T	&\\
F		&	T	&	T		&	F	&	&		F	&	T	&	F		&	F	&	&	T	&	T		&	T		&	F	&\\
F		&	F	&	T		&	T	&	&		T	&	F	&	T		&	T	&	&	F	&	F		&	F		&	T	&\\
\textbf{T}	\tikz[overlay, shift={(120pt,1ex)}, gray] \draw (0pt,0pt) ellipse (154pt and 2ex);	&	F	&	F		&	F	&	&		T	&	F	&	\textbf{T}		&	F	&	&	F	&	\textbf{T}		&	T		&	F	&\\		
\end{tabular}\\
\end{longtabu}


\begin{exercises}
\item \{$A \eand \enot B$, $\enot(A \eif B)$, $B \eif A$\}\vspace{.5ex} %Consistent

\answer{
\begin{longtabu}{cccccccccccccc} 
A 					 & \eand 		&  \enot & B & & \enot  		& 	 (A	  & 	 \eif	 	 & 	 B)		 & 	 & 	 B	 	 & 	\eif 	 	 & 	A 	 	 & 	 Consistent \\ 
\cline{1-4} \cline{6-9}\cline{11-13} 
T 					 & 	 F	 		&  F	 & T & & F	 		& 	 T	  & 	 T	 	 & 	T 	 	 & 	 & 	 T	 	 & 	 T	 	 & T	 	 	&	  \\ 
T \tikz[overlay, shift={(110pt,1ex)}, red] \draw (0pt,0pt) ellipse (143pt and 2ex); & 	{\color{black}\textbf{T}}	 & T	 & F & & {\color{black}\textbf{T}}	 & 	 T	 & 	 F	 	 & 	 F	 	 & 	 & 	 F	 	 & 	 {\color{black}\textbf{T}}	 	 & 	 T	 	 & 	  \\ 
F	 				 & 	 F	 & 	 F	 & T & 	& 	 F	 & 	 F	 & 	 T	 	 & 	 T	 	 & 	  & 	 T	 	 & 	 F	 	 & 	 F	 	 & 	  \\ 
F	  				& 	 F	 & 	 T	 & 	F&  & 	 F	 & 	 F	 & 	 T	 	 & 	 F	 	 & 	  & 	 F	 	 & 	 T	 	 & 	 F	 	 & 	  \\ 
\end{longtabu}
}
\item \{$A \eor B$, $A \eif \enot A$, $B \eif \enot B\}$ \vspace{.5ex}%inconsistent. 

\answer{
\begin{longtabu}{cccccccccccccc} 
A	 & \eor 	 & B 	 & 	 	 & A 	 & \eif 	 & 	\enot & A 	 & 	 	 & B 	 & \eif 	 & \enot	 & 	B 	 & 	Inconsistent \\ 
\cline{1-3}\cline{5-8} \cline{10-13}
T	 & 	 T	 &T  	 & 	 	 & T	 & 	 F	 & 	F 	 & T 	 & 	 	 & 	T 	 & 	F 	 & 	 F	 & 	T 	 & 	 \\ 
T	& 	 T	 & F 	 & 	 	 & 	T 	 & 	 F	 & 	 F	 & 	 T	 & 	 	 & 	F 	 & 	 T	 & 	 T	 & 	 F	 & 	 \\ 
F	& 	 T	 & 	 T	 & 	 	 & 	F 	 & 	 T	 & 	 T	 & 	F 	 & 	 	 & 	 T	 & 	 F	 & 	 F	 & 	 T	 & 	 \\ 
F	& 	 F	 & 	 F	 & 	 	 & 	 F	 & 	 T	 & 	 T	 & 	 F	 & 	 	 & 	 F	 & 	 T	 & 	 T	 & 	 F	 & 	 \\ 
\end{longtabu}
}

\item \{$\enot(\enot A \eor B) $, $A \eif \enot C$, $A \eif (B \eif C)\}$\vspace{.5ex} %Inconsistent

\answer{
\begin{longtabu}{ccccccccccccccccc} 
\enot & (\enot & A & \eor &B) &  &A  & \eif 	 &\enot 	 &C & 	 & A &\eif 	& (B 	 &\eif 	& C)	 &Consistent \\ 
\cline{1-5}\cline{7-10} \cline{12-16} 
F 	& 	F	 & 	T & T	 & T & 	  & T & F	 & 	 F&T 	 & 	 &T & T	 & T	 &T 	 &T 	 & \\ 
F	& 	F	 & 	T & T	 & T & 	  & T & T	 & 	 T& F	 & 	 &T & F	 & T	 & F	 &F 	 & \\ 
T & 	F 	& 	T & F	 & F & 	  & T & F	 & 	 F& T	 & 	 &T & T	 & F	 & T	 &T 	 & \\ 
\color{black}\textbf{T}	\tikz[overlay, shift={(140pt,1ex)}, red] \draw (0pt,0pt) ellipse (179pt and 2ex);	&  F	 & 	T & F	 & 	F &  & 	T & {\color{black}\textbf{T}}	 & 	 T&F 	& 	 &T & {\color{black}\textbf{T}}	 & F	 & T	&  F 	 & \\ 
 F	& 	T	 & 	F & T	 & 	T &  & 	F & T	 & 	 F& T	 & 	 &F	 & F	 & T	 & T	 &T 	 & \\ 
 F	& 	 T	& 	F & T	 & 	T &  & 	F & T	 & 	T & F & 	 &F	 & T	 & T	 &F 	 &F 	 & \\ 
 F	& 	 T	& 	F & T	 & 	F &  & 	F & T	 & 	F & T	 & 	 &F	 & T	 & F	 & T	 &T 	 & \\ 
 F	& 	 T	& 	F & T	 & 	F &  & 	F & T	 & 	T & F	 & 	 &F	 & T	 & F	 & T	 &F 	 & \\ 
\end{longtabu}
}


\item \{$A \eif B$, $A \eand \enot B$\}\vspace{.5ex} %Inconsistent

\answer{
\begin{longtabu}{ccccccccc}
A	&	\eif	 &	B,	&	&	A	&	\eand	&	\enot	&	B	&	Inconsistent\\
\cline{1-3} \cline{5-8}
T	&	T		&	T	&	&	T	&		F	&		F	&	T	&	\\
T	&	F		&	F	&	&	T	&		T	&		T	&	F	&	\\
F	&	T		&	T	&	&	F	&		F	&		F	&	T	&	\\
F	&	T		&	F	&	&	F	&		F	&		T	&	F	&	\\
\end{longtabu}
}


\item \{$A \eif (B \eif C)$, $(A \eif B) \eif C$, $A \eif C$\}\vspace{.5ex} % consistent. 

\answer{
\begin{longtabu}{cccccccccccccccc}
A	&	\eif	&	(B	&	\eif	&	 C)	&	&	(A	&	\eif	&	B)	&	\eif	&C	&	&	A	&	\eif	&	C	&	Consistent\\
\cline{1-5} \cline{7-11} \cline{13-15}
T	&	T		&	T	&	T		&	T	&	&	T	&	T		&	T	&	T		&	T	&	&	T	&	T		&	T	&	\\
T	&	F		&	T	&	F		&	F	&	&	T	&	T		&	T	&	F		&	F	&	&	T	&	F		&	F	&	\\
T	&	T		&	F	&	T		&	T	&	&	T	&	F		&	F	&	T		&	T	&	&	T	&	T		&	T	&	\\
T	&	T		&	F	&	T		&	F	&	&	T	&	F		&	F	&	T		&	F	&	&	T	&	F		&	F	&	\\
F	&	T		&	T	&	T		&	T	&	&	F	&	T		&	T	&	T		&	T	&	&	F	&	T		&	T	&	\\
F	&	T		&	T	&	F		&	F	&	&	F	&	T		&	T	&	T		&	F	&	&	F	&	T		&	F	&	\\
F	&	T		&	F	&	T		&	T	&	&	F	&	T		&	F	&	T		&	T	&	&	F	&	T		&	T	&	\\
F	&	{\color{black}\textbf{T}}	\tikz[overlay, shift={(125pt,1ex)}, red] \draw (0pt,0pt) ellipse (165pt and 2ex);	&	F	&	T		&	F	&	&	F	&	T		&	F	&	{\color{black}\textbf{T}}		&	F	&	&	F	&	{\color{black}\textbf{T}}		&	T	&	\\
\end{longtabu}
}

\end{exercises}

\noindent\problempart
\label{pr.TT.consistent}
Determine whether each set of sentences is consistent or inconsistent, using a complete truth table. 
\begin{exercises}
\item \{$\enot B$, $A \eif B$, $A$\} \vspace{.5ex}%inconsistent.
\item \{$\enot(A \eor B)$, $A \eiff B$, $B \eif A$\}\vspace{.5ex} %Consistent
\item \{ $A \eor B$, $\enot B$, $\enot B \eif \enot A$\}\vspace{.5ex} %Inconsistent
\item \{$A \eiff B$, $\enot B \eor \enot A$, $A \eif B$\}\vspace{.5ex} %consistent. 
\item \{$(A \eor B) \eor C$, $\enot A \eor \enot B$, $\enot C \eor \enot B$\}\vspace{.5ex} %consistent
\end{exercises}




\noindent\problempart \label{pr.TT.valid} Determine whether each argument is valid or invalid, using a complete truth table. 

\begin{longtabu}{p{.1\linewidth}p{.9\linewidth}}
\textbf{Example}: & $A \eor B$, $C \eif A$, $C \eif B \sdtstile{}{} C$   \\
\textbf{Answer}: & \begin{tabular}[t]{cccccccccccccc}
A	&	\eor	&	B	&	&	C	&	\eif	&	A	&	&	C	&	\eif	&	B	&	&	C	&	Invalid	\\
\cline{1-3} \cline{5-7}	\cline{9-11}	\cline{13-13}
T	&	T		&	T	&	&	T	&	T		&	T	&	&	T	&	T		&	T	&	&	T	&	\\

T \tikz[overlay, shift={(100pt,1ex)}, gray] \draw (0pt,0pt) ellipse (132pt and 2ex);	&	\textbf{T}		&	T	&	&	F	&	\textbf{T}		&	T	&	&	F	&	\textbf{T}		&	T	&	&	\textbf{F}	&	\\

T	&	T		&	F	&	&	T	&	T		&	T	&	&	T	&	F		&	F	&	&	T	&	\\
T	&	T		&	F	&	&	F	&	T		&	T	&	&	F	&	T		&	F	&	&	F	&	\\
F	&	T		&	T	&	&	T	&	F		&	F	&	&	T	&	T		&	T	&	&	T	&	\\
F	&	T		&	T	&	&	F	&	T		&	F	&	&	F	&	T		&	T	&	&	F	&	\\
F	&	F		&	F	&	&	T	&	F		&	F	&	&	T	&	F		&	F	&	&	T	&	\\
F	&	F		&	F	&	&	F	&	T		&	F	&	&	F	&	T		&	F	&	&	F	&	\\						
\end{tabular}\\
\end{longtabu}


\begin{exercises}
\item $A\eif A \sdtstile{}{} A$ \vspace{.5ex}%invalid

\answer{
\begin{longtabu}{cccccc}
A	&	\eif	&	A	&	&	A	&	Invalid \\	
\cline{1-3}	\cline{5-5}
T	&	T		&	T	&	&	T	&		\\

F \tikz[overlay, shift={(33pt,1ex)}, red] \draw (0pt,0pt) ellipse (66pt and 2ex);	&	{\color{black}\textbf{T}}		&	F	&	&	{\color{black}\textbf{F}}	&		\\
\end{longtabu}
}

\item $A\eif B$, $B \sdtstile{}{} A$ %invalid
\answer{
\begin{longtabu}{cccccccc}
A	&	\eif	&	B	&	&	B	&	&	A	&	Invalid \\
\cline{1-3} \cline{5-5}	\cline{7-7}
T	&	T		&	T	&	&	T	&	&	T	&	\\
T	&	F		&	F	&	&	F	&	&	T	&	\\
F \tikz[overlay, shift={(44pt,1ex)}, red] \draw (0pt,0pt) ellipse (88pt and 2ex);	&	{\color{black}\textbf{T}}		&	T	&	&	{\color{black}\textbf{T}}	&	&	{\color{black}\textbf{F}}	&	\\
F	&	T		&	F	&	&	F	&	&	F	&	\\
\end{longtabu}
}

\item $A\eiff B$, $B\eiff C \sdtstile{}{}A\eiff C$ %valid

\answer{
\begin{longtabu}{cccccccccccc}
A	&	\eiff	&	B	&	&	B	&	\eiff	&	C	&	&	A	&	\eiff	&	C 	&	Valid \\
\cline{1-3} \cline{5-7} \cline{9-11}
T	&	T		&	T	&	&	T	&	T		&	T	&	&	T	&	T		&	T	&	\\
T	&	T		&	T	&	&	T	&	F		&	F	&	&	T	&	F		&	F	&	\\	
T	&	F		&	F	&	&	F	&	F		&	T	&	&	T	&	T		&	T	&	\\
T	&	F		&	F	&	&	F	&	T		&	F	&	&	T	&	F		&	F	&	\\
F	&	F		&	T	&	&	T	&	T		&	T	&	&	F	&	F		&	T	&	\\
F	&	F		&	T	&	&	T	&	F		&	F	&	&	F	&	T		&	F	&	\\	
F	&	T		&	F	&	&	F	&	F		&	T	&	&	F	&	F		&	T	&	\\
F	&	T		&	F	&	&	F	&	T		&	F	&	&	F	&	T		&	F	&	\\
\end{longtabu}
}

\item $A \eif B$, $A \eif C\sdtstile{}{}B \eif C$ %invalid. 

\answer{
\begin{longtabu}{cccccccccccc}
A	&	\eif	&	B	&	&	A	&	\eif	&	C	&	&	B	&	\eif	&	C	&	Invalid \\
\cline{1-3} \cline{5-7} \cline{9-11}
T	&	T		&	T	&	&	T	&	T		&	T	&	&	T	&	T		&	T	&	\\
T	&	T		&	T	&	&	T	&	F		&	F	&	&	T	&	F		&	F	&	\\
T	&	F		&	F	&	&	T	&	T		&	T	&	&	F	&	T		&	T	&	\\
T	&	F		&	F	&	&	T	&	F		&	F	&	&	F	&	T		&	F	&	\\
F	&	T		&	T	&	&	F	&	T		&	T	&	&	T	&	T		&	T	&	\\	

F \tikz[overlay, shift={(100pt,1ex)}, red] \draw (0pt,0pt) ellipse (132pt and 2ex);	&	{\color{black}\textbf{T}}		&	T	&	&	F	&	{\color{black}\textbf{T}}		&	F	&	&	T	&	{\color{black}\textbf{F}}		&	F	&	\\

F	&	T		&	F	&	&	F	&	T		&	T	&	&	F	&	T		&	T	&	\\
F	&	T		&	F	&	&	F	&	T		&	F	&	&	F	&	T		&	F	&	\\		
\end{longtabu}
}

\item $A \eif B$, $B \eif A\sdtstile{}{}A \eiff B$ %valid. 

\answer{
\begin{longtabu}{cccccccccccc}
A	&	\eif	&	B	&	&	B	&	\eif	&	A	&	&	A	&	\eiff	&	B	&	Valid	\\
\cline{1-3} \cline{5-7} \cline{9-11}
T	&	T		&	T	&	&	T	&	T		&	T	&	&	T	&	T		&	T	&	\\
T	&	F		&	F	&	&	F	&	T		&	T	&	&	T	&	F		&	F	&	\\
F	&	T		&	T	&	&	T	&	F		&	F	&	&	F	&	F		&	T	&	\\
F	&	T		&	F	&	&	F	&	T		&	F	&	&	F	&	T		&	F	&	\\		
\end{longtabu}
}

\end{exercises}

\noindent\problempart
\label{pr.TT.valid}
Determine whether each argument is valid or invalid, using a complete truth table. 
\begin{exercises}
\item $A\eor\bigl[A\eif(A\eiff A)\bigr] \sdtstile{}{} A $\vspace{.5ex}%invalid
\item $A\eor B$, $B\eor C$, $\enot B \sdtstile{}{}A \eand C$\vspace{.5ex} %valid
\item $A \eif B$, $\enot A\sdtstile{}{}\enot B$ \vspace{.5ex}%invalid
\item $A$, $B\sdtstile{}{}\enot(A\eif \enot B)$ \vspace{.5ex}%valid
\item $\enot(A \eand B)$, $A \eor B$, $A \eiff B\sdtstile{}{}C$ \vspace{.5ex}%valid 
\end{exercises}


% *********************************************
% *   Partial Truth Tables							*
% *********************************************
\section{Partial Truth Tables}

In order to show that a sentence is a tautology, we need to show that it is T on every row. So we need a complete truth table. To show that a sentence is \emph{not} a tautology, however, we only need one line: a line on which the sentence is F. Therefore, in order to show that something is not a tautology, it is enough to provide a one-line \emph{partial truth table}---regardless of how many sentence letters the sentence might have in it.

Consider, for example, the sentence $(U \eand T) \eif (S \eand W)$. We want to show that it is \emph{not} a tautology by providing a partial truth table. To begin, we fill in F for the entire sentence, the reverse of how we started when we were doing complete truth tables.

\begin{center}
\begin{tabu}{c|c|c|c|@{\TTon}*{7}{c}@{\TToff}}
$S$&$T$&$U$&$W$&$(U$&\eand&$T)$&\eif  \tikz[overlay, shift={(-1.25ex,-6pt)}, gray] \draw (0pt,0pt) ellipse (2ex and 18pt);  &$(S$&\eand&$W)$\\
\hline
   &   &   &   &    &    &    &F&    &    &   
\end{tabu}
\end{center}

 The main connective of the sentence is a conditional. In order for the conditional to be false, the antecedent must be true (T) and the consequent must be false (F). So we fill these in on the table:

\begin{center}
\begin{tabu}{c|c|c|c|@{\TTon}*{7}{c}@{\TToff}}
$S$&$T$&$U$&$W$&$(U$&\eand&$T)$&\eif  \tikz[overlay, shift={(-1.25ex,-6pt)}, gray] \draw (0pt,0pt) ellipse (2ex and 18pt);  &$(S$&\eand&$W)$\\
\hline
   &   &   &   &    &  T  &    &F&    &   F &   
\end{tabu}
\end{center}

In order for the $(U\eand T)$ to be true, both $U$ and $T$ must be true.

\begin{center}
\begin{tabu}{c|c|c|c|@{\TTon}*{7}{c}@{\TToff}}
$S$&$T$&$U$&$W$&$(U$&\eand&$T)$&\eif  \tikz[overlay, shift={(-1.25ex,-6pt)}, gray] \draw (0pt,0pt) ellipse (2ex and 18pt);  &$(S$&\eand&$W)$\\
\hline
   & T & T &   &  T &  T  & T  &F&    &   F &   
\end{tabu}
\end{center}

Now we just need to make $(S\eand W)$ false. To do this, we need to make at least one of $S$ and $W$ false. We can make both $S$ and $W$ false if we want. All that matters is that the whole sentence turns out false on this line. Making an arbitrary decision, we finish the table in this way:

\begin{center}
\begin{tabu}{c|c|c|c|@{\TTon}*{7}{c}@{\TToff}}
$S$&$T$&$U$&$W$&$(U$&\eand&$T)$&\eif  \tikz[overlay, shift={(-1.25ex,-6pt)}, gray] \draw (0pt,0pt) ellipse (2ex and 18pt);  &$(S$&\eand&$W)$\\
\hline
 F & T & T & F &  T &  T  & T  &F&  F &   F & F  
\end{tabu}
\end{center}

Showing that something is a contradiction requires a complete truth table. Showing that something is \emph{not} a contradiction requires only a one-line partial truth table, where the sentence is true on that one line.

A sentence is contingent if it is neither a tautology nor a contradiction. So showing that a sentence is contingent requires a \emph{two-line} partial truth table: The sentence must be true on one line and false on the other. For example, we can show that the sentence above is contingent with this truth table:
\begin{center}
\begin{tabu}{c|c|c|c|@{\TTon}*{7}{c}@{\TToff}}
$S$&$T$&$U$&$W$&$(U$&\eand&$T)$&\eif  \tikz[overlay, shift={(-1.25ex,-14pt)}, gray] \draw (0pt,0pt) ellipse (2ex and 28pt);  &$(S$&\eand&$W)$\\
\hline
 F & T & T & F &  T &  T  & T  &F&  F &   F & F\\
 F & T & F & F &  F &  F  & T  &T&  F &   F & F
\end{tabu}
\end{center}
Note that there are many combinations of truth values that would have made the sentence true, so there are many ways we could have written the second line.

Showing that a sentence is \emph{not} contingent requires providing a complete truth table, because it requires showing that the sentence is a tautology or that it is a contradiction.  If you do not know whether a particular sentence is contingent, then you do not know whether you will need a complete or partial truth table. You can always start working on a complete truth table. If you complete rows that show the sentence is contingent, then you can stop. If not, then complete the truth table. Even though two carefully selected rows will show that a contingent sentence is contingent, there is nothing wrong with filling in more rows.

Showing that two sentences are logically equivalent requires providing a complete truth table. Showing that two sentences are \emph{not} logically equivalent requires only a one-line partial truth table: Make the table so that one sentence is true and the other false.

Showing that a set of sentences is consistent requires providing one row of a truth table on which all of the sentences are true. The rest of the table is irrelevant, so a one-line partial truth table will do. Showing that a set of sentences is inconsistent, on the other hand, requires a complete truth table: You must show that on every row of the table at least one of the sentences is false.

Showing that an argument is valid requires a complete truth table. Showing that an argument is \emph{invalid} only requires providing a one-line truth table: If you can produce a line on which the premises are all true and the conclusion is false, then the argument is invalid.

\begin{table}
\begin{mdframed}[style=mytablebox]
\begin{center}
\begin{tabu}{X[1,l,b] X[1,l,b] X[1,l,b]}
\tabulinesep=1ex
\underline{Property}	& Truth table required \newline \underline{to show presence}		&	Truth table required \newline \underline{to show absence} \\ 
being a tautology		& complete 													 		& one-line partial \\ 
being a contradiction 	& complete 													 		& one-line partial \\ 
contingency				& two-line partial 													& complete truth \\ 
equivalence				& complete 													 		& one-line partial\\ 
consistency				& one-line partial 													& complete \\ 
validity					& complete 															& one-line partial \\ 
\end{tabu}
\end{center}
\end{mdframed}
\caption{Complete or partial truth tables to test for different properties}
\label{table.CompleteVsPartial}
\end{table}

Table \ref{table.CompleteVsPartial} summarizes when a complete truth table is required and when a partial truth table will do. 

%\section{The material conditional}
%\label{MaterialConditional}

%The material conditional has some odd properties. For one thing, it does not require that the antecedent and consequent are related in any way.

%contradiction in the antecedent

%tautology in the consequent


%%%%%%%%%%%%%%%%% practice problems 

\practiceproblems
\noindent\problempart \label{pr.TT.TTorC} Determine whether each sentence is a tautology, a contradiction, or a contingent sentence. Justify your answer with a complete or partial truth table where appropriate.


\begin{exercises}
\item  $A \eif \enot A$ \vspace{.5ex}							

\answer{

\begin{longtabu}{cccc}
A&\eif \tikz[overlay, shift={(-1ex,-12pt)}, red] \draw (0pt,0pt) ellipse (2ex and 24pt);&\enot&A\\\hline
T&F&F&T\\
F&T&T&F
\end{longtabu}
Contingent	 \vspace{6pt}
}
%	T letter, 2 connectives

\item $A \eif (A \eand (A \eor B))$ \vspace{.5ex}	

\answer{


\begin{longtabu}{ccc@{}ccc@{}ccc@{}c@{}c}
A&\eif \tikz[overlay, shift={(-1ex,-30pt)}, red] \draw (0pt,0pt) ellipse (2ex and 44pt);
 &(&A&\eand&(&A&\eor&B&)&)\\\hline
T&T&&T&T&&T&T&T&&\\
T&T&&T&T&&T&T&F&&\\
F&T&&F&F&&F&T&T&&\\
F&T&&F&F&&F&F&F&&
\end{longtabu}


Tautology \vspace{6pt}
}
%			2 letters, 3 connectives

\item $(A \eif B) \eiff (B \eif A)$ 	\vspace{.5ex}				%

\answer{
 
\begin{longtabu}{ccccccc}
(A	&	\eif 	&	B) 	&	\eiff \tikz[overlay, shift={(-1ex,-12pt)}, red] \draw (0pt,0pt) ellipse (2ex and 24pt);	&	(B 	&	\eif 	&	A) \\ \hline
T	&	T		&	T	&	T		&	T	&	T		&	T	\\
T	&	F		&	F	&	F		&	F	&	T		&	T	\\
\end{longtabu}
Contingent \vspace{6pt}
}
%		2 letters, 3 connectives

\item $A \eif \enot(A \eand (A \eor B)) $	\vspace{.5ex}	

\answer{
 
\begin{longtabu}{cccccccc}
A	&	\eif	&	 \enot	&	(A	&	\eand	&	(A	&	\eor	&	B)) \\ \hline
	&			&			&		&			&		&			&	\\
	&			&			&		&			&		&			&	\\				
\end{longtabu}
}


%
% 2 letters, 4 connectives

\item $\enot B \eif [(\enot A \eand A) \eor B]$\vspace{.5ex} 

\answer{
 
\begin{longtabu}{ccccccc}

\end{longtabu}
}

%{\color{red}
%$
%\begin{array}{cc|cccc@{}c@{}cccc@{}ccc@{}c}
%a&b&\enot&b&\rightarrow&(&(&\enot&a&\eand&a&)&\eor&b&)\\\hline
%T&T&F&T&\mathbf{T}&&&F&T&F&T&&T&T&\\
%T&F&T&F&\mathbf{F}&&&F&T&F&T&&F&F&\\
%F&T&F&T&\mathbf{T}&&&T&F&F&F&&T&T&\\
%F&F&T&F&\mathbf{F}&&&T&F&F&F&&F&F&
%\end{array}
%$
%Contingent	 \vspace{6pt}
%
%}
%	2 letters, 5 connectives

\item $\enot(A \eor B) \eiff (\enot A \eand \enot B)$ \vspace{.5ex}

\answer{
 
\begin{longtabu}{ccccccc}

\end{longtabu}
}

%{\color{red}
%$
%\begin{array}{cc|cc@{}ccc@{}ccc@{}ccccc@{}c}
%a&b&\enot&(&a&\eor&b&)&\leftrightarrow&(&\enot&a&\eand&\enot&b&)\\\hline
%T&T&F&&T&T&T&&\mathbf{T}&&F&T&F&F&T&\\
%T&F&F&&T&T&F&&\mathbf{T}&&F&T&F&T&F&\\
%F&T&F&&F&T&T&&\mathbf{T}&&T&F&F&F&T&\\
%F&F&T&&F&F&F&&\mathbf{T}&&T&F&T&T&F&
%\end{array}
%$
%
%Tautology \vspace{6pt}
%}
%2 letters, 6 connectives

\item $[(A \eand B) \eand C] \eif B$\vspace{.5ex}		

\answer{
 
\begin{longtabu}{ccccccc}

\end{longtabu}
}
					
%
%{\color{red}
%$
%\begin{array}{ccc|c@{}c@{}ccc@{}ccc@{}ccc}
%a&b&c&(&(&a&\eand&b&)&\eand&c&)&\rightarrow&b\\\hline
%T&T&T&&&T&T&T&&T&T&&\mathbf{T}&T\\
%T&T&F&&&T&T&T&&F&F&&\mathbf{T}&T\\
%T&F&T&&&T&F&F&&F&T&&\mathbf{T}&F\\
%T&F&F&&&T&F&F&&F&F&&\mathbf{T}&F\\
%F&T&T&&&F&F&T&&F&T&&\mathbf{T}&T\\
%F&T&F&&&F&F&T&&F&F&&\mathbf{T}&T\\
%F&F&T&&&F&F&F&&F&T&&\mathbf{T}&F\\
%F&F&F&&&F&F&F&&F&F&&\mathbf{T}&F
%\end{array}
%$
%
%Tautology \vspace{6pt}
%}
%
%3 letters, 3 connectives

\item $\enot\bigl[(C\eor A) \eor B\bigr]$\vspace{.5ex} 						

\answer{
 
\begin{longtabu}{ccccccc}

\end{longtabu}
}


%
%{\color{red}
%$
%\begin{array}{ccc|cc@{}c@{}ccc@{}ccc@{}c}
%a&b&c&\enot&(&(&c&\eor&a&)&\eor&b&)\\\hline
%T&T&T&\mathbf{F}&&&T&T&T&&T&T&\\
%T&T&F&\mathbf{F}&&&F&T&T&&T&T&\\
%T&F&T&\mathbf{F}&&&T&T&T&&T&F&\\
%T&F&F&\mathbf{F}&&&F&T&T&&T&F&\\
%F&T&T&\mathbf{F}&&&T&T&F&&T&T&\\
%F&T&F&\mathbf{F}&&&F&F&F&&T&T&\\
%F&F&T&\mathbf{F}&&&T&T&F&&T&F&\\
%F&F&F&\mathbf{T}&&&F&F&F&&F&F&
%\end{array}
%$
%
%Contingent \vspace{6pt}
%
%}
%	 	3 letters, 3 connectives

\item $\bigl[(A\eand B) \eand\enot(A\eand B)\bigr] \eand C$ \vspace{.5ex}	


\answer{
 
\begin{longtabu}{ccccccc}

\end{longtabu}
}

%
%{\color{red}
%$
%\begin{array}{ccc|c@{}c@{}ccc@{}cccc@{}ccc@{}c@{}ccc}
%a&b&c&(&(&a&\eand&b&)&\eand&\enot&(&a&\eand&b&)&)&\eand&c\\\hline
%T&T&T&&&T&T&T&&F&F&&T&T&T&&&\mathbf{F}&T\\
%T&T&F&&&T&T&T&&F&F&&T&T&T&&&\mathbf{F}&F\\
%T&F&T&&&T&F&F&&F&T&&T&F&F&&&\mathbf{F}&T\\
%T&F&F&&&T&F&F&&F&T&&T&F&F&&&\mathbf{F}&F\\
%F&T&T&&&F&F&T&&F&T&&F&F&T&&&\mathbf{F}&T\\
%F&T&F&&&F&F&T&&F&T&&F&F&T&&&\mathbf{F}&F\\
%F&F&T&&&F&F&F&&F&T&&F&F&F&&&\mathbf{F}&T\\
%F&F&F&&&F&F&F&&F&T&&F&F&F&&&\mathbf{F}&F
%\end{array}
%$
%
%Contradiction \vspace{6pt}
%
%}
%
%% 	3 letters, 5 connectives
%
\item $(A \eand B) ]\eif[(A \eand C) \eor (B \eand D)]$ \vspace{.5ex}		


\answer{
 
\begin{longtabu}{ccccccc}

\end{longtabu}
}


%
%{\color{red}
%$
%\begin{array}{cccc|c@{}c@{}ccc@{}c@{}ccc@{}c@{}ccc@{}ccc@{}ccc@{}c@{}c}
%a&b&c&d&(&(&a&\eand&b&)&)&\eif&(&(&a&\eand&c&)&\eor&(&b&\eand&d&)&)\\\hline
%T&T&T&T&&&T&T&T&&&\mathbf{T}&&&T&T&T&&T&&T&T&T&&\\
%T&T&F&F&&&T&T&T&&&\mathbf{F}&&&T&F&F&&F&&T&F&F&&\\
%\end{array}
%$
%
%Contingent \vspace{6pt}
%}
%
%	4 letters, 5 connectives
\end{exercises}

\noindent\problempart
\label{pr.TT.TTorC}
Determine whether each sentence is a tautology, a contradiction, or a contingent sentence. Justify your answer with a complete or partial truth table where appropriate.
\begin{exercises}
\item  $\enot (A \eor A)$\vspace{.5ex}							%	Contradiction		1 letter, 2 connectives
\item $(A \eif B) \eor (B \eif A)$\vspace{.5ex}					%	Tautology			2 letters, 2 connectives
\item $[(A \eif B) \eif A] \eif A$\vspace{.5ex}					%	Tautology			2 letters, 3 connectives
\item $\enot[( A \eif B) \eor (B \eif A)]$\vspace{.5ex}			%	Contradiction		2 letters, 4 connectives
\item $(A \eand B) \eor (A \eor B)$\vspace{.5ex} 				%	Contingent		2 letters, 5 connectives
\item $\enot(A\eand B) \eiff A$\vspace{.5ex} 					%contingent			2 letters, 3 connectives
\item $A\eif(B\eor C)$\vspace{.5ex} 							%contingent			3 letters, 2 connectives
\item $(A \eand\enot A) \eif (B \eor C)$\vspace{.5ex} 			%tautology			3 letters, 4 connectives 
\item $(B\eand D) \eiff [A \eiff(A \eor C)]$\vspace{.5ex}			%contingent			4 letters, 4 connectives
\item $\enot[(A \eif B) \eor (C \eif D)]$\vspace{.5ex} 			% Contingent. 		4 letters, 4 connectives
\end{exercises}


\noindent\problempart
\label{pr.TT.equiv}
Determine whether each the following statements of equivalence are true or false using complete truth tables. If the two sentences really are logically equivalent, write "Logically equivalent." Otherwise write, "Not logically equivalent." 
\begin{exercises}
\item $A\ndststile{}{} \hspace{.5em} \sdtstile{}{}\enot A$\vspace{.5ex} 											%No		1 letter, 1 connective, matching
\item $A\eif A\ndststile{}{} \hspace{.5em} \sdtstile{}{}A \eiff A$\vspace{.5ex} 									%No		1 letter, 1 connectives, matching	
\item $A	\eand (B	\eand C)\ndststile{}{} \hspace{.5em} \sdtstile{}{} A \eand \enot A$\vspace{.5ex}  					%No		2 letters, 4 connectives, matching
\item $A \eand \enot A\ndststile{}{} \hspace{.5em} \sdtstile{}{}\enot B \eiff B$ \vspace{.5ex}						%Yes		2 letters, 4 connectives, matching
\item $\enot(A \eif B)\ndststile{}{} \hspace{.5em} \sdtstile{}{}\enot A \eif \enot B$\vspace{.5ex}					%No		2 letters, 5 connectives, matching
\item $A \eiff B\ndststile{}{} \hspace{.5em} \sdtstile{}{}\enot[(A \eif B) \eif \enot (B\eif A)]$\vspace{.5ex}			%yes		2 letters, 6 connectives, matching
\item $(A \eand B) \eif (\enot A \eor \enot B)\ndststile{}{} \hspace{.5em} \sdtstile{}{} \enot(A\eand B)$\vspace{.5ex}	%Yes		2 letters, 7 connectives, matching
\item $[(A \eor B) \eor C]\ndststile{}{} \hspace{.5em} \sdtstile{}{}[A \eor (B \eor C)]$\vspace{.5ex} 				%Yes		3 letters, 4 connectives, matching
\item $ (Z \eand (\enot R \eif O))\ndststile{}{} \hspace{.5em} \sdtstile{}{} \enot (R \eif \enot O) $\vspace{.5ex}		%			3 letters, 6 connectives, not matching	

\end{exercises}

\noindent\problempart
Determine whether each the following statements of equivalence are true or false using complete truth tables. If the two sentences really are logically equivalent, write "Logically equivalent." Otherwise write, "Not logically equivalent." 
\begin{exercises}
\item $A\ndststile{}{} \hspace{.5em} \sdtstile{}{}A \eor A$\vspace{.5ex} 												%Yes		1 letter, 1 connective, matching
\item $A\ndststile{}{} \hspace{.5em} \sdtstile{}{}A \eand A$\vspace{.5ex} 												%Yes		1 letter, 1 connective, matching
\item $A \eor \enot B\ndststile{}{} \hspace{.5em} \sdtstile{}{}A\eif B$\vspace{.5ex} 									%No		2 letters, 3 connectives, matching
\item $(A \eif B)\ndststile{}{} \hspace{.5em} \sdtstile{}{}(\enot B \eif \enot A)$\vspace{.5ex} 							%Yes		2 letters, 4 connectives, matching
\item $\enot(A \eand B)\ndststile{}{} \hspace{.5em} \sdtstile{}{}\enot A \eor \enot B$ \vspace{.5ex}						%Yes		2 letters, 5 connectives, matching
\item $ ((U \eif (X \eor X)) \eor U) \ndststile{}{} \hspace{.5em} \sdtstile{}{} \enot (X \eand (X \eand U)) $\vspace{.5ex}	% 			2 letters, 6 connectives, matching
\item $ ((C \eand (N \eiff C)) \eiff C) \ndststile{}{} \hspace{.5em} \sdtstile{}{} (\enot \enot \enot N \eif C) $\vspace{.5ex}	% 			2 letters, 7 connectives, matching
\item $[(A \eor B) \eand C]\ndststile{}{} \hspace{.5em} \sdtstile{}{}[A \eor (B \eand C)]$\vspace{.5ex} 					%No		3 letters, 4 connectives, matching
\item $((L \eand C) \eand I)\ndststile{}{} \hspace{.5em} \sdtstile{}{}L \eor C$\vspace{.5ex}								%No		3 letters, not matching	
\end{exercises}


\noindent\problempart
\label{pr.TT.consistent}
Determine whether each set of sentences is consistent or inconsistent. Justify your answer with a complete or partial truth table where appropriate.
\begin{exercises}
\item \{$A\eif A$, $\enot A \eif \enot A$, $A\eand A$, $A\eor A$\} \vspace{.5ex}%consistent
\item \{$A \eif \enot A$, $\enot A \eif A$\}\vspace{.5ex}%inconsistent. 
\item \{$A\eor B$, $A\eif C$, $B\eif C$\}\vspace{.5ex} %consistent
\item \{$A \eor B$, $A \eif C$, $B \eif C$, $\enot C$\}\vspace{.5ex} %	Inconsistent
\item \{$B\eand(C\eor A)$, $A\eif B$, $\enot(B\eor C)$\}\vspace{.5ex}  %inconsistent
\item \{$(A \eiff B) \eif B$,  $B \eif \enot (A \eiff B)$, $A \eor B$\} \vspace{.5ex} %	Consistent
\item \{$A\eiff(B\eor C)$, $C\eif \enot A$, $A\eif \enot B$\}\vspace{.5ex} %consistent
\item  \{$A \eiff B$,  $\enot B \eor \enot A$,  $A \eif  B$\} \vspace{.5ex}% Consistent
\item \{$A \eiff B$, $A \eif C$, $B \eif D$, $\enot(C \eor D)$\}\vspace{.5ex} %consitent
\item \{$\enot (A \eand \enot B)$,  $B \eif \enot A$, $\enot B$ \} \vspace{.5ex} %Consistent
\end{exercises}

\noindent\problempart
\label{pr.TT.consistent}
Determine whether each set of sentences is consistent or inconsistent. Justify your answer with a complete or partial truth table where appropriate.
\begin{exercises}
\item \{$A \eand B$, $C\eif \enot B$, $C$\} \vspace{.5ex}%inconsistent
\item \{$A\eif B$, $B\eif C$, $A$, $\enot C$\}\vspace{.5ex} %inconsistent
\item \{$A \eor B$, $B\eor C$, $C\eif \enot A$\}\vspace{.5ex} %consistent
\item \{$A$, $B$, $C$, $\enot D$, $\enot E$, $F$\}\vspace{.5ex} %consistent
\item \{$A \eand (B \eor C)$, $\enot(A \eand C)$, $\enot(B \eand C)$\} \vspace{.5ex}%consistent
\item \{$A \eif B$, $B \eif C$, $\enot(A \eif C)$\} \vspace{.5ex} %inconsistent

%\begin{tabular}{ccc|ccc|ccccc}
%A 	&\eif	&B 	&B 	&\eif 	&C 	&\enot 	&(A 	&\eif 	&C) 	&Inconsistent\\
%\cline{1-10} 
%T 	&T 	&T 	&T 	&T 	&T 	&F 		&T 	&T 	&T 	& \\
%T&T&T&T&F&F&T&T&F&F&\\
%T&F&F&F&T&T&F&T&T&T&\\
%T&F&F&F&T&F&T&T&F&F&\\
%F&T&T&T&T&T&F&F&T&T&\\
%F&T&T&T&F&F&F&F&T&F&\\
%F&T&F&F&T&T&F&F&T&T&\\
%F&T&F&F&T&F&F&F&T&F&\\
%\end{tabular}

\end{exercises}


\noindent\problempart Determine whether each argument is valid or invalid. Justify your answer with a complete or partial truth table where appropriate.
\label{pr.TT.valid} 
\begin{exercises}

\item $A\eif(A\eand\enot A)\sdtstile{}{}\enot A$% valid

\answer{
 \begin{longtabu}{ccccccc|cc}
A	&	\eif	&	(A	&	\eand	&	\enot	&	A)	&	&	\enot	&	A	\\ \hline
T	&	F		&	T	&	F		&	F		&	T	&	&		F	&	T\\ 
F	&	T		&	F	&	F		&	T		&	F	&	&		T	&	F\\ 
\end{longtabu}
Valid
}



\item $A \eor B$, $A \eif B$, $B \eif A \sdtstile{}{} A \eiff B$  % Valid

\answer{
 
\begin{longtabu}{cccc|cccc|cccc|ccc}
A	&	\eor 	&	B	&		&	 A	&	\eif	&	B	&		&	B	&	\eif	&	A	&		&	A	&	\eiff	&	B\\ \hline
T	&			&	T	&		&	T	&			&	T	&		&	T	&			&	T	&		&	T	&			&	T	\\
T	&			&	F	&		&	T	&			&	F	&		&	F	&			&	T	&		&	T	&			&	F	\\
F	&			&	T	&		&	F	&			&	T	&		&	T	&			&	F	&		&	F	&			&	T	\\
F	&			&	F	&		&	F	&			&	F	&		&	F	&			&	F	&		&	F	&			&	F	\\			
\end{longtabu}
}


\item $A\eor(B\eif A)\sdtstile{}{}\enot A \eif \enot B$ %valid

\answer{
 
\begin{longtabu}{ccccccc}

\end{longtabu}
}


\item $A \eor B$, $A \eif B$, $ B \eif A \sdtstile{}{} A \eand B$ %valid

\answer{
 
\begin{longtabu}{ccccccc}

\end{longtabu}
}


\item $(B\eand A)\eif C$, $(C\eand A)\eif B\sdtstile{}{}(C\eand B)\eif A$ % invalid

\answer{
 
\begin{longtabu}{ccccccc}

\end{longtabu}
}


\item $\enot (\enot A \eor \enot B)$, $A \eif \enot C \sdtstile{}{} A \eif (B \eif C)$ % invalid.

\answer{
 
\begin{longtabu}{ccccccc}

\end{longtabu}
}


\item $A \eand (B \eif C)$, $\enot C \eand (\enot B \eif \enot A)\sdtstile{}{}C \eand \enot C$ % valid

\answer{
 
\begin{longtabu}{ccccccc}

\end{longtabu}
}


\item $A \eand B$, $\enot A \eif \enot C$, $B \eif \enot D \sdtstile{}{} A \eor B$ % Invalid

\answer{
 
\begin{longtabu}{ccccccc}

\end{longtabu}
}


\item $A \eif B\sdtstile{}{}(A \eand B) \eor (\enot A \eand \enot B)$ % invalid

\answer{
 
\begin{longtabu}{ccccccc}

\end{longtabu}
}


\item $\enot A \eif B$,$ \enot B \eif C $,$ \enot C \eif A \sdtstile{}{} \enot A \eif (\enot B \eor \enot C) $% Invalid

\answer{
 
\begin{longtabu}{ccccccc}

\end{longtabu}
}


\end{exercises}

\noindent\problempart Determine whether each argument is valid or invalid. Justify your answer with a complete or partial truth table where appropriate.
\label{pr.TT.valid} 
\begin{exercises}
\item $A\eiff\enot(B\eiff A)\sdtstile{}{}A$ % invalid

\answer{
 
\begin{longtabu}{ccccccc}

\end{longtabu}
}


\item $A\eor B$, $B\eor C$, $\enot A\sdtstile{}{}B \eand C$ % invalid

\answer{
 
\begin{longtabu}{ccccccc}

\end{longtabu}
}


\item $A \eif C$, $E \eif (D \eor B)$, $B \eif \enot D\sdtstile{}{}(A \eor C) \eor (B \eif (E \eand D))$ % invalid

\answer{
 
\begin{longtabu}{ccccccccccccccccccccc}

\end{longtabu}
}


\item $A \eor B$, $C \eif A$, $C \eif B\sdtstile{}{}A \eif (B \eif C)$ % invalid

\answer{
 
\begin{longtabu}{ccccccc}

\end{longtabu}
}


\item $A \eif B$, $\enot B \eor A\sdtstile{}{}A \eiff B$ % valid

\answer{
 
\begin{longtabu}{ccccccc}

\end{longtabu}
}


\end{exercises}

\noindent\problempart
\label{pr.TT.concepts}
Answer each of the questions below and justify your answer.
\begin{exercises}
\item Suppose that \script{A} and \script{B} are logically equivalent. What can you say about $\script{A}\eiff\script{B}$?
%\script{A} and \script{B} have the same truth value on every line of a complete truth table, so $\script{A}\eiff\script{B}$ is true on every line. It is a tautology.
\item Suppose that $(\script{A}\eand\script{B})\eif\script{C}$ is contingent. What can you say about the argument \script{A}, \script{B}, $\sdtstile{}{}$ \script{C}?
%The sentence is false on some line of a complete truth table. On that line, \script{A} and \script{B} are true and \script{C} is false. So the argument is invalid.
\item Suppose that $\{\script{A},\script{B}, \script{C}\}$ is inconsistent. What can you say about $(\script{A}\eand\script{B}\eand\script{C})$?
%Since there is no line of a complete truth table on which all three sentences are true, the conjunction is false on every line. So it is a contradiction.
\item Suppose that \script{A} is a contradiction. What can you say about the argument \{\script{A}, \script{B}\} $\sdtstile{}{}$  \script{C}?
%Since \script{A} is false on every line of a complete truth table, there is no line on which \script{A} and \script{B} are true and \script{C} is false. So the argument is valid.
\item Suppose that \script{C} is a tautology. What can you say about the argument \{\script{A}, \script{B}\} $\sdtstile{}{}$ \script{C}?
%Since \script{C} is true on every line of a complete truth table, there is no line on which \script{A} and \script{B} are true and \script{C} is false. So the argument is valid.
%\item Suppose that \script{A} and \script{B} are logically equivalent. What can you say about $(\script{A}\eor\script{B})$?
%Not much. $(\script{A}\eor\script{B})$ is a tautology if \script{A} and \script{B} are tautologies; it is a contradiction if they are contradictions; it is contingent if they are contingent.
\item Suppose that \script{A} and \script{B} are \emph{not} logically equivalent. What can you say about $(\script{A}\eor\script{B})$?
%\script{A} and \script{B} have different truth values on at least one line of a complete truth table, and $(\script{A}\eor\script{B})$ will be true on that line. On other lines, it might be true or false. So $(\script{A}\eor\script{B})$ is either a tautology or it is contingent; it is \emph{not} a contradiction.
\end{exercises}

% *********************************************
% *   Expressive Completeness	      						*
% *********************************************

\section{Expressive Completeness}
\label{sec:expressive_completeness}

We could leave the biconditional (\eiff) out of the language. If we did that, we could still write ``$A\eiff B$'' so as to make sentences easier to read, but that would be shorthand for $(A\eif B) \eand (B\eif A)$. The resulting language would be formally equivalent to SL, since $A\eiff B$ and $(A\eif B) \eand (B\eif A)$ are logically equivalent in SL. If we valued formal simplicity over expressive richness, we could replace more of the connectives with notational conventions and still have a language equivalent to SL. 

There are a number of equivalent languages with only two connectives. You could do logic with only the negation and the material conditional. Alternately you could just have the negation and the disjunction. You will be asked to prove that these things are true in the last problem set. You could even have a language with only one connective, if you designed the connective right. The \emph{Sheffer stroke} is a logical connective with the following characteristic truth table:
\begin{center}
\begin{tabular}{c|c|c}
\script{A} & \script{B} & \script{A}$|$\script{B}\\
\hline
T & T & F\\
T & F & T\\
F & T & T\\
F & F & T
\end{tabular}
\end{center}
The Sheffer stroke has the unique property that it is the only connective you need to have a complete system of logic. You will be asked to prove that this is true in the last problem set also.  


%\fix{Summary of test conditions}

\practiceproblems
\noindent\problempart
\begin{exercises}
\item In section \ref{sec:expressive_completeness}, we said that you could have a language that only used the negation and the material conditional. Prove that this is true by writing sentences that are logically equivalent to each of the following using only parentheses, sentence letters, negation (\enot), and the material conditional (\eif).
\begin{enumerate}
\item $A\eor B$
%$\enot A \eif B$
\item $A\eand B$
%$\enot(A \eif \enot B)$
\item $A\eiff B$
%$\enot [(A\eif B) \eif \enot(B\eif A)]$
\end{enumerate}

\item We also said in section 3.5 that you could have a language which used only the negation and the disjunction. Show this: Using only parentheses, sentence letters, negation (\enot), and disjunction (\eor), write sentences that are logically equivalent to each of the following.
\begin{enumerate}

\item $A \eand B$
%$\enot(\enot A \eor \enot B)$
\item $A \eif B$
%$\enot A \eor B$
\item $A \eiff B$
%$\enot(\enot A \eor \enot B) \eor \enot(A \eor B)$
\end{enumerate}

\item Write a sentence using the connectives of SL that is logically equivalent to $(A|B)$.
\item Every sentence written using a connective of SL can be rewritten as a logically equivalent sentence using one or more Sheffer strokes. Using only the Sheffer stroke, write sentences that are equivalent to each of the following. 
%...
\begin{enumerate}
\setcounter{eargnum}{\arabic{OLDeargnum}}
\item $\enot A$
\item $(A\eand B)$
\item $(A\eor B)$
\item $(A\eif B)$
\item $(A\eiff B)$
\end{enumerate}
\end{exercises}


%%%% A recursive definition of truth in SL

%[go back and explicitly mark the section on a recursive definition of a sentence in SL as optional and then rework it so it is parallel to this passage.]

%In the optional later sections of the chapter, we gave a recursive definition of what it meant to be a sentence in SL. We will also end this chapter with a recursive definition that summarizes the material in the chapter. In this case, we are going to give a recursive definition of truth in SL. [sources: magnus's original treatment. Hodges in the logic handbook. Tarski. Be sure to motivate this and explain its relationship to a recursive definition of a truth table.]



%%%Below here is Magnus's original version of recursive definition of truth in SL  


%
%Formally, what we want is a function that assigns a 1 or 0 to each of the sentences of SL. We can interpret this function as a definition of truth for SL if it assigns 1 to all of the true sentences of SL and 0 to all of the false sentences of SL. Call this function ``$v$'' (for ``valuation''). We want $v$ to a be a function such that for any sentence \script{A}, $v(\script{A})=1$ if \script{A} is true and $v(\script{A})=0$ if \script{A} is false.
%
%Recall that the recursive definition of a wff for SL had two stages: The first step said that atomic sentences (solitary sentence letters) are wffs. The second stage allowed for wffs to be constructed out of more basic wffs. There were clauses of the definition for all of the sentential connectives. For example, if \script{A} is a wff, then \enot\script{A} is a wff.
%
%Our strategy for defining the truth function, $v$, will also be in two steps. The first step will handle truth for atomic sentences; the second step will handle truth for compound sentences.
%
%
%\subsection{Truth in SL}
%How can we define truth for an atomic sentence of SL? Consider, for example, the sentence $M$. Without an interpretation, we cannot say whether $M$ is true or false. It might mean anything. If we use $M$ to symbolize ``The moon orbits the Earth'', then $M$ is true. If we use $M$ to symbolize ``The moon is a giant turnip'', then $M$ is false.
%
%Moreover, the way you would discover whether or not $M$ is true depends on what $M$ means. If $M$ means ``It is Monday,'' then you would need to check a calendar. If $M$ means ``Jupiter's moon Io has significant volcanic activity,'' then you would need to check an astronomy text---and astronomers know because they sent satellites to observe Io.
%
%When we give a symbolization key for SL, we provide an {interpretation} of the sentence letters that we use. The key gives an English language sentence for each sentence letter that we use. In this way, the interpretation specifies what each of the sentence letters \emph{means}. However, this is not enough to determine whether or not that sentence is true. The sentences about the moon, for instance, require that you know some rudimentary astronomy. Imagine a small child who became convinced that the moon is a giant turnip. She could understand what the sentence ``The moon is a giant turnip'' means, but mistakenly think that it was true.
%
%Consider another example: If $M$ means ``It is morning now'', then whether it is true or not depends on when you are reading this. I know what the sentence means, but---since I do not know when you will be reading this---I do not know whether it is true or false.
%
%So an interpretation alone does not determine whether a sentence is true or false. Truth or falsity depends also on what the world is like. If $M$ meant ``The moon is a giant turnip'' and the real moon were a giant turnip, then $M$ would be true. To put the point in a general way, truth or falsity is determined by an interpretation \emph{plus} a way that the world is.
%
%\begin{center}
%INTERPRETATION + STATE OF THE WORLD $\Longrightarrow$ TRUTH/FALSITY
%\end{center}
%
%In providing a logical definition of truth, we will not be able to give an account of how an atomic sentence is made true or false by the world. Instead, we will introduce a \emph{truth value assignment}. Formally, this will be a function that tells us the truth value of all the atomic sentences. Call this function ``$a$'' (for ``assignment''). We define $a$ for all sentence letters \script{P}, such that
%\begin{displaymath}
%a(\script{P}) =
%\left\{
%	\begin{array}{ll}
%	1 & \mbox{if \script{P} is true},\\
%	0 & \mbox{otherwise.}
%	\end{array}
%\right.
%\end{displaymath}
%This means that $a$ takes any sentence of SL and assigns it either a one or a zero; one if the sentence is true, zero if the sentence is false. The details of the function $a$ are determined by the meaning of the sentence letters together with the state of the world. If $D$ means ``It is dark outside'', then $a(D)=1$ at night or during a heavy storm, while $a(D)=0$ on a clear day.
%
%You can think of $a$ as being like a row of a truth table. Whereas a truth table row assigns a truth value to a few atomic sentences, the truth value assignment assigns a value to every atomic sentence of SL. There are infinitely many sentence letters, and the truth value assignment gives a value to each of them. When constructing a truth table, we only care about sentence letters that affect the truth value of sentences that interest us. As such, we ignore the rest. Strictly speaking, every row of a truth table gives a \emph{partial} truth value assignment.
%
%It is important to note that the truth value assignment, $a$, is not part of the language SL. Rather, it is part of the mathematical machinery that we are using to describe SL. It encodes which atomic sentences are true and which are false.
%
%
%We now define the truth function, $v$, using the same recursive structure that we used to define a wff of SL.
%
%\begin{enumerate}
%\item If \script{A} is a sentence letter, then $v(\script{A})=a(\script{A})$.
%%\setcounter{Example}{\arabic{enumi}}\end{enumerate}
%%...
%% Break out of the {enumerate} environment to say something about what is
%% going on. Using \setcounter in this way preserves the numbering, so
%% that the list can resume after the comments.
%
%%This is a mathematical equals sign, not the identity predicate we defined for QL.
%
%% Resume the {enumerate} environment and restore the counter.
%%...
%%\begin{enumerate}\setcounter{enumi}{\arabic{Example}}
%\item If \script{A} is ${\enot}\script{B}$ for some sentence \script{B}, then
%\begin{displaymath}v(\script{A}) =
%	\left\{\begin{array}{ll}
%	1 & \mbox{if $v(\script{B}) = 0$},\\
%	0 & \mbox{otherwise.}
%	\end{array}\right.
%\end{displaymath}
%
%\item If \script{A} is $(\script{B}\eand\script{C})$ for some sentences \script{B,C}, then
%\begin{displaymath}v(\script{A}) =
%	\left\{\begin{array}{ll}
%	1 & \mbox{if $v(\script{B}) = 1$ and $v(\script{C}) = 1$,}\\
%	0 & \mbox{otherwise.}
%	\end{array}\right.
%\end{displaymath}
%\setcounter{Example}{\arabic{enumi}}\end{enumerate}
%%...
%
%It might seem as if this definition is circular, because it uses the word ``and'' in trying to define ``and.'' Notice, however, that this is not a definition of the English word ``and''; it is a definition of truth for sentences of SL containing the logical symbol ``\eand.'' We define truth for object language sentences containing the symbol ``\eand'' using the metalanguage word ``and.'' There is nothing circular about that.
%
%%...
%\begin{enumerate}\setcounter{enumi}{\arabic{Example}}
%\item If \script{A} is $(\script{B}\eor\script{C})$ for some sentences \script{B,C}, then
%\begin{displaymath}v(\script{A}) =
%	\left\{\begin{array}{ll}
%	0 & \mbox{if $v(\script{B}) = 0$ and $v(\script{C}) = 0$,}\\
%	1 & \mbox{otherwise.}
%	\end{array}\right.
%\end{displaymath}
%%\setcounter{Example}{\arabic{enumi}}\end{enumerate}
%%...
%%Notice that this defines truth for sentences containing the symbol ``\eor''' using the word ``and.''
%%...
%%\begin{enumerate}\setcounter{enumi}{\arabic{Example}}
%\item If \script{A} is $(\script{B}\eif\script{C})$ for some sentences \script{B,C}, then
%\begin{displaymath}v(\script{A}) =
%	\left\{\begin{array}{ll}
%	0 & \mbox{if $v(\script{B}) = 1$ and $v(\script{C}) = 0$,}\\
%	1 & \mbox{otherwise.}
%	\end{array}\right.
%\end{displaymath}
%
%\item If \script{A} is $(\script{B}\eiff\script{C})$ for some sentences \script{B,C}, then
%\begin{displaymath}v(\script{A}) =
%	\left\{\begin{array}{ll}
%	1 & \mbox{if $v(\script{B}) = v(\script{C})$},\\
%	0 & \mbox{otherwise.}
%	\end{array}\right.
%\end{displaymath}
%\end{enumerate}
%
%Since the definition of $v$ has the same structure as the definition of a wff, we know that $v$ assigns a value to \emph{every} wff of SL. Since the sentences of SL and the wffs of SL are the same, this means that $v$ returns the truth value of every sentence of SL.
%
%Truth in SL is always truth \emph{relative to} some truth value assignment, because the definition of truth for SL does not say whether a given sentence is true or false. Rather, it says how the truth of that sentence relates to a truth value assignment.
%

%%%%%%%%%%%%%%%%%%%%%%%%		 Key Terms

\section*{Key Terms}
\begin{multicols}{2}
\begin{sortedlist}
\sortitem{Semantically contingent in SL}{}
\sortitem{Semantically logically equivalent in SL}{}
\sortitem{Semantically consistent in SL}{}
\sortitem{Semantically valid in SL}{}
\sortitem{Semantic contradiction in SL}{}
\sortitem{Semantic tautology in SL}{}
\sortitem{Complete truth table}{}
\sortitem{Truth assignment}{}
\sortitem{Truth-functional connective}{} 
\sortitem{Nonlogical symbol}{}
\sortitem{Logical constant}{}
\sortitem{Interpretation}{}
\end{sortedlist}
\end{multicols}









\chapter{Proofs in Sentential Logic}
\label{chap:proofsinSL}
\markright{Chap. \ref{chap:proofsinSL}: Proofs in SL}
\setlength{\parindent}{1em}

\newcounter{theorem}
\setcounter{theorem}{1}

\label{whole_slproof_chap} %uncomment and typeset twice to print the whole chapter.


%rob: This chapter is based on the original Chapter 6. I took all the material on proof in SL and moved it earlier in the book so the students would have a chance to start doing derivations earlier. I have also expanded the opening material that wasn't in a Chapter section into its own Chapter section explaining the basic idea of a proof. 

% *******************************************
% *		Substitution Instances and Proofs			   *	
% *******************************************

\section{Substitution Instances and Proofs}
\label{sec:substitution_instances}

% rob: Changed opening to add big picture stuff. What we did last chapter, what we will do this chapter. The ability to use deduction as an important mental skill

In the last chapter, we introduced the truth table method, which allowed us to check to see if various logical properties were present, such as whether a statement is a tautology or whether an argument is valid. The method in that chapter was semantic, because it relied on the meaning of symbols, specifically, whether they were interpreted as true or false. The nice thing about that method was that it was completely mechanical. If you just followed the rules like a robot, you would eventually get the right answer. You didn't need any special insight and there were no tough decisions to make. The downside to this method was that the tables quickly became way too long. It just isn't practical to make a 32 line table every time you have to deal with five different sentence letters. 

In this chapter, we are going to introduce a new method for checking for validity and other logical properties. This time our method is going to be purely syntactic. We won't be at all concerned with what our symbols mean. We are just going to look at the way they are arranged. Our method here will be called a system of natural deduction. When you use a system of natural deduction, you won't do it mechanically. You will need to understand the logical structure of the argument and employ your insight. This is actually one of the reasons people like systems of natural deduction. They let us represent the logical structure of arguments in a way we can understand. Learning to represent and manipulate arguments this way is a core mental skill, used in fields like mathematics and computer programming. 

Consider two arguments in SL:
\begin{quotation}
\begin{tabu}{X[1,p,m]X[1,p,m]}
\textbf{Argument A} & \textbf{Argument B} \\
\begin{earg*}
\item $P \eor Q$
\item  $\enot P$
\itemc[.2] Q
\end{earg*}
&

\begin{earg*}
\item $P \eif Q$
\item $P$
\itemc[.2] Q
\end{earg*}

\end{tabu}
\end{quotation}

These are both valid arguments. Go ahead and prove that for yourself by constructing the four-line truth tables. These particular valid arguments are examples of important kinds of 
arguments that are given special names. Argument A is an example of a kind of argument traditionally called \emph{disjunctive syllogism}. In the system of proof we will develop later in 
the chapter, it will be given a newer name, \emph{disjunction elimination} (\eor-E). Given a disjunction and the negation of one of the disjuncts, the other disjunct follows as a valid 
consequence. Argument B makes use of a different valid form: Given a conditional and its antecedent, the consequent follows as a valid consequence. This is traditionally called 
\emph{modus ponens}. In our system it will be called \emph{conditional elimination} (\eif-E).

Both of the arguments above remain valid even if we substitute different sentence letters. You don't even need to run the truth tables again to see that these arguments are valid: 
\begin{quotation}
\begin{tabu}{X[1,p,m]X[1,p,m]}
\textbf{Argument A*} & \textbf{Argument B*} \\
\begin{earg*}
\item $A \eor B$
\item $\enot A$
\itemc[.2] B
\end{earg*}

&

\begin{earg*}
\item $A \eif B$
\item $A$
\itemc[.2] B
\end{earg*}
\end{tabu}
\end{quotation}

Replacing $P$ with $A$ and $Q$ with $B$ changes nothing (so long as we are sure to replace \emph{every} $P$ with an $A$ and every $Q$ with a $B$). What's more interesting is that we can replace the individual sentence letters in Argument A and Argument B with longer sentences in SL and the arguments will still be valid, as long as we do the substitutions consistently. Here are two more perfectly valid instances of disjunction and conditional elimination. 
\begin{quotation}
\begin{tabu}{X[1,p,m]X[1,p,m]}
\textbf{Argument A**} & \textbf{Argument B**} \\
\begin{earg*}
\item  $(C \eand D) \eor (E \eor F)$
\item  $\enot (C \eand D)$
\itemc[.2] $E \eor F$
\end{earg*}

&

\begin{earg*}
\item $(G \eif H) \eif (I \eor J)$
\item $(G \eif H)$
\itemc[.2] $I \eor J$
\end{earg*}
\end{tabu}
\end{quotation}
Again, you can check these using truth tables, although the 16 line truth tables begin to get tiresome. All of these arguments are what we call \emph{substitution instances} of the same two logical forms. We call them that because you get them by replacing the sentence letters with other sentences, either sentence letters or longer sentences in SL. A substitution instance cannot change the sentential connectives of a sentence, however. The sentential connectives are what make the \emph{logical form} of the sentence. We can write these logical forms using fancy script letters.

\begin{quotation}
\begin{tabu}{X[1,p,m]X[1,p,m]}
\textbf{Disjunction Elimination} \newline (Disjunctive Syllogism) &
\textbf{Conditional Elimination} \newline (Modus Ponens) \\


\begin{earg*}
\item $\script{A} \eor \script{B}$
\item $\enot \script{A}$
\itemc[.2] \script{B}
\end{earg*}

&

\begin{earg*}
\item  $\script{A} \eif \script{B}$
\item  $\script{A}$
\itemc[.2] \script{B}
\end{earg*}
\end{tabu}
\end{quotation}

As we explained in Chapter \ref{chap:SL}, the fancy script letters are \emph{metavariables}.  They are a part of our metalanguage and can refer to single sentence letters like $P$ or longer sentences like $A \eiff (B \eand (C \eor D))$. 

\newglossaryentry{sentence form}
{
name=sentence form,
description={A sentence in SL that contains one or more metavariables in place of sentence letters.}
}



\newglossaryentry{substitution instance}
{
name=substitution instance,
description={A sentence that is created by consistently substituting sentences for one or more of the metavariables in a sentence form..}
}


\newglossaryentry{argument form}
{
name=argument form,
description={An argument that includes one or more sentence forms.}
}


\newglossaryentry{substitution instance of an argument form}
{
name=substitution instance of an argument form,
description={An argument obtained by consistently replacing the sentence forms in the argument form with their substitution instances..}
}



Formally, we can define a \textsc{\gls{sentence form}}\label{def:sentence_form} as a sentence in SL that contains one or more metavariables in place of sentence letters. A \textsc{\gls{substitution instance}}\label{def:substitution_instance} of that sentence form is then a sentence created by consistently substituting sentences for one or more of the metavariables in the sentence form. Here ``consistently substituting'' means replacing all instances of the metavariable with the same sentence. You cannot replace instances of the same metavariable with different sentences, or leave a metavariable as it is, if you have replaced other metavariables of that same type. An \textsc{\gls{argument form}}\label{def:argument_form}
 is an argument that includes one or more sentence forms, and a \textsc{\gls{substitution instance of an argument form}}\label{def:substitution instance_of_an_argument_form} of the argument form is the argument obtained by consistently replacing the sentence forms in the argument form with their substitution instances.

Once we start identifying valid argument forms like this, we have a new way of showing that longer arguments are valid. Truth tables are fun, but doing the 1028 line truth table for an argument with 10 sentence letters would be tedious. Worse, we would never be sure we hadn't made a little mistake in all those Ts and Fs. Part of the problem is that we have no way of knowing  \emph{why} the argument is valid. The table gives you very little insight into how the premises work together. 

The aim of a \emph{proof system} is to show that particular arguments are valid in a way that allows us to understand the reasoning involved in the argument. Instead of representing all the premises and the conclusion in one table, we break the argument up into steps. Each step is a basic argument form of the sort we saw above, like disjunctive syllogism or modus ponens. Suppose we are given the premises $\enot L \eif (J \eor L)$ and $\enot L$ and wanted to show $J$. We can break this up into two smaller arguments, each of which is a substitution inference of a form we know is correct.

\begin{quotation}
\begin{tabu}{X[1,p,m]X[1,p,m]}
\textbf{Argument 1} & \textbf{Argument 2} \\
\begin{earg*}
\item $\enot L \eif (J \eor L)$
\item $\enot L$
\itemc[.2] $J \eor L$
\end{earg*}

&

\begin{earg*}
\item $J \eor L$
\item $\enot L$
\itemc[.2] $J$
\end{earg*}
\end{tabu}
\end{quotation}

The first argument is a substitution instance of modus ponens and the second is a substitution instance of disjunctive syllogism, so we know they are both valid. Notice also that the conclusion of the first argument is the first premise of the second, and the second premise is the same in both arguments. Together, these arguments are enough to get us from $\enot L \eif (J \eor L)$ and $\enot L$ to $J$.

These two arguments take up a lot of space, though. To complete our proof system, we need a system for showing clearly how simple steps can combine to get us from premises to conclusions. The system we will use in this book was devised by the American logician Frederic Brenton Fitch (1908--1987). We begin by writing our premises on numbered lines with a bar on the left and a little bar underneath to represent the end of the premises. Then we write ``Want'' on the side followed by the conclusion we are trying to reach. If we wanted to write out arguments 1 and 2 above, we would begin like this.

\begin{proof}
	\hypo{1}{\enot L \eif (J \eor\ L)}
	\hypo{2}{\enot L} \by{Want: $J$}{}			
\end{proof}

We then add the steps leading to the conclusion below the horizontal line, each time explaining off to the right why we are allowed to write the new line. This explanation consists of citing a rule and the prior lines the rule is applied to. In the example we have been working with we would begin like this

\begin{proof}
	\hypo{1}{\enot L \eif (J \eor L)}
	\hypo{2}{\enot L} \by{Want: $J$}{}
	\have{3}{J \eor L} \ce{1, 2}
\end{proof}

and then go like this

\begin{proof}
	\hypo{1}{\enot L \eif (J \eor L)}
	\hypo{2}{\enot L} \by{Want: $J$}{}
	\have{3}{J \eor L} \ce {1, 2}
	\have{4}{J} \oe{2, 3}
\end{proof}

\newglossaryentry{proof}
{
name=proof,
description={A sequence of sentences, where the first sentences of the sequence are assumptions, and all sentences after the assumptions follow from sentences earlier in the sequence according to the rules of derivation.}
}



The little chart above is a \emph{proof} that $J$ follows from $\enot L \eif (J \eor L)$ and $\enot L$. We will also call proofs like this \emph{derivations}. Formally, a \textsc{\gls{proof}}\label{def:proof} is a sequence of sentences. The first sentences of the sequence are assumptions; these are the premises of the argument. Every sentence later in the sequence follows from earlier sentences by one of the rules of proof. The final sentence of the sequence is the conclusion of the argument.

\iflabelexists{chap:proofsinQL}{In the remainder of this chapter, we will develop a system for proving sentences in SL. Later, in Chapter \ref{chap:proofsinQL}, this will be expanded to cover Quantified Logic (QL). First, though, you should practice identifying substitution instances of sentences and longer rules.}{} 

%I added exercises for identifying substitution inferences, because many students need practice with this really basic form of pattern recognition. 

%%%%%  PRACTICE PROBLEMS %%%%%%%%%%%%%

\practiceproblems
\noindent\problempart For each problem, a sentence form is given in metavariables. Identify which of the sentences after it are legitimate substitution instances of that form. 

\begin{exercises}
\begin{longtabu}{X[1,p,m]X[1,p,m]} 

\item $\script{A} \eand \script{B}$: 
	\begin{enumerate}[label=\alph*.]
	\item $P \eor Q$
	\iflabelexists{showanswers}{{\color{red}\item [\circled{\emph{\color{red}{b.}}}]$(A \eif B) \eand C$}}{\item $(A \eif B) \eand C$}
	\iflabelexists{showanswers}{{\color{red}\item [\circled{\emph{\color{red}{c.}}}] \begin{flushleft}$[(A \eand B) \eif (B \eand A)] \linebreak \eand (\enot A \eand \enot B)$\end{flushleft}}}{\item \begin{flushleft}$[(A \eand B) \eif (B \eand A)] \linebreak \eand (\enot A \eand \enot B)$\end{flushleft}}
	\iflabelexists{showanswers}{{\color{red}\item [\circled{\emph{\color{red}{c.}}}]$[((A \eand B) \eand C) \eand D] \eand F$}}{\item $[((A \eand B) \eand C) \eand D] \eand F$ } 
	\item[e.] $(A \eand B) \eif C$
	\end{enumerate}

&

\item $\enot(\script{P} \eand \script{Q})$
	\begin{enumerate}[label=\alph*.]
	\iflabelexists{showanswers}{{\color{red}\item [\circled{\emph{\color{red}{b.}}}]$\enot(A \eand B)$}}{\item $\enot(A \eand B)$}
	\iflabelexists{showanswers}{{\color{red}\item [\circled{\emph{\color{red}{b.}}}]$\enot(A \eand A)$}}{\item $\enot(A \eand A)$}
	\item[c.] $\enot A \eand B$
	\iflabelexists{showanswers}{{\color{red}\item [\circled{\emph{\color{red}{d.}}}]\begin{flushleft}$\enot((\enot A \eand B) \eand (B \eand \enot A))$\end{flushleft}}}{\item \begin{flushleft}$\enot((\enot A \eand B) \eand (B \eand \enot A))$\end{flushleft}}
	\item[e.] $\enot(A \eif B)$
	\end{enumerate}


\\

\item $\enot \script{A}$
	\begin{enumerate}[label=\alph*.]
	\item $\enot A \eif B$
	\iflabelexists{showanswers}{{\color{red}\item [\circled{\emph{\color{red}{b.}}}]$\enot (A \eif B)$}}{\item $\enot (A \eif B)$}
	\iflabelexists{showanswers}{{\color{red}\item [\circled{\emph{\color{red}{c.}}}]$\enot[(G \eif (H \eor I)) \eif G]$}}{\item $\enot[(G \eif (H \eor I)) \eif G]$}
	\item $\enot G \eand (\enot B \eand \enot H)$
	\iflabelexists{showanswers}{{\color{red}\item [\circled{\emph{\color{red}{e.}}}]$\enot(G \eand (B \eand H))$}}{\item $\enot(G \eand (B \eand H))$}
	\end{enumerate}
&

\item $\enot \script{A} \eif  \script{B}$
	\begin{enumerate}[label=\alph*.]
	\item $\enot A \eand B$
	\iflabelexists{showanswers}{{\color{red}\item [\circled{\emph{\color{red}{b.}}}]$\enot B \eif A$}}{\item $\enot B \eif A$}
	\iflabelexists{showanswers}{{\color{red}\item [\circled{\emph{\color{red}{c.}}}]$\enot(X \eand Y) \eif (Z \eor B)$}}{\item $\enot(X \eand Y) \eif (Z \eor B)$}
	\item $\enot(A \eif B)$
	\item $A \eif \enot B$
	\end{enumerate}
\\

\item $\enot \script{A} \eiff \enot \script{Z}$
	\begin{enumerate}[label=\alph*.]
	\item $\enot (P \eiff Q)$
	\iflabelexists{showanswers}{{\color{red}\item [\circled{\emph{\color{red}{b.}}}]$\enot(P \eiff Q) \eiff \enot (Q \eiff P)$}}{\item $\enot(P \eiff Q) \eiff \enot (Q \eiff P)$}
	\item $\enot H \eif \enot G$
	\item $\enot (A \eand B) \eiff C$
	\iflabelexists{showanswers}{{\color{red}\item [\circled{\emph{\color{red}{e.}}}]\begin{flushleft} $\enot [\enot (P \eiff Q) \eiff R] \eiff \enot S$ \end{flushleft}}}{\item \begin{flushleft} $\enot [\enot (P \eiff Q) \eiff R] \eiff \enot S$ \end{flushleft}}
	\end{enumerate}

&

\item $(\script{A} \eand \script{B}) \eor \script{C}$
	\begin{enumerate}[label=\alph*.]
	\item $(P \eor Q) \eand R$
	\iflabelexists{showanswers}{{\color{red}\item [\circled{\emph{\color{red}{b.}}}]$(\enot M \eand \enot D) \eor C$}}{\item $(\enot M \eand \enot D) \eor C$}
	\item $(D \eand R) \eand (I \eor D)$
	\item $[(D \eif O) \eor A] \eand D$
	\iflabelexists{showanswers}{{\color{red}\item [\circled{\emph{\color{red}{e.}}}]$[(A \eand B) \eand C] \eor (D \eor A)$}}{\item $[(A \eand B) \eand C] \eor (D \eor A)$}
	\end{enumerate}
%\factoidbox{B, E}


\\

\item $(\script{A} \eand \script{B}) \eor \script{A}$							
	\begin{flushleft}
	\begin{enumerate}[label=\alph*.]
	\item$((C \eif D) \eand E) \eor A$
	\iflabelexists{showanswers}{{\color{red}\item [\circled{\emph{\color{red}{b.}}}]$(A \eand A) \eor A$}}{\item$(A \eand A) \eor A$}
	\iflabelexists{showanswers}{{\color{red}\item [\circled{\emph{\color{red}{c.}}}]$((C \eif D) \eand E) \eor (C \eif D)$}}{\item$((C \eif D) \eand E) \eor (C \eif D)$}
	\iflabelexists{showanswers}{{\color{red}\item [\circled{\emph{\color{red}{d.}}}]$((G \eand B) \eand (Q \eor R)) \eor (G \eand B)$}}{\item$((G \eand B) \eand (Q \eor R)) \eor (G \eand B)$}
	\item$(P \eor Q) \eand P$
	\end{enumerate}
	\end{flushleft}

&
\item $\script{P} \eif (\script{P} \eif \script{Q})$
	\begin{flushleft}
	\begin{enumerate}[label=\alph*.]
	\item $A \eif (B \eif C)$
	\iflabelexists{showanswers}{{\color{red}\item [\circled{\emph{\color{red}{b.}}}]$(A \eand B) \eif [(A \eand B) \eif C]$}}{\item $(A \eand B) \eif [(A \eand B) \eif C]$}
	\iflabelexists{showanswers}{{\color{red}\item [\circled{\emph{\color{red}{c.}}}]$(G \eif B) \eif [(G \eif B) \eif (G \eif B)]$}}{\item $(G \eif B) \eif [(G \eif B) \eif (G \eif B)]$}
	\iflabelexists{showanswers}{{\color{red}\item [\circled{\emph{\color{red}{d.}}}]$M \eif [M \eif (D \eand (C \eand M))]$}}{\item $M \eif [M \eif (D \eand (C \eand M))]$}
	\item $(S \eor O) \eif [(O \eor S) \eif A]$
	\end{enumerate}
	\end{flushleft}



\\
\item $\enot \script{A} \eor (\script{B} \eand \enot \script{B})$
	\begin{flushleft}
	\begin{enumerate}[label=\alph*.]
	\item $\enot P \eor (Q \eand \enot P)$
	\iflabelexists{showanswers}{{\color{red}\item [\circled{\emph{\color{red}{b.}}}]$\enot A \eor (A \eand \enot A)$}}{\item $\enot A \eor (A \eand \enot A)$}
	\item $(P \eif Q) \eor [(P \eif Q) \eand \enot R]$
	\item $\enot E \eand (F \eand \enot F)$
	\iflabelexists{showanswers}{{\color{red}\item [\circled{\emph{\color{red}{e.}}}]$\enot G \eor [(H \eif G) \eand \enot (H \eif G)]$}}{\item $\enot G \eor [(H \eif G) \eand \enot (H \eif G)]$}
	\end{enumerate}
	\end{flushleft}

&


\item	$(\script{P} \eor \script{Q}) \eif \enot(\script{P} \eand \script{Q})$
\begin{flushleft} 	
\begin{enumerate}[label=\alph*.]
	\item	$A \eif \enot B$
	\iflabelexists{showanswers}{{\color{red}\item [\circled{\emph{\color{red}{b.}}}]$(A \eor B) \eif \enot(A \eand B)$}}{\item	$(A \eor B) \eif \enot(A \eand B)$}
	\iflabelexists{showanswers}{{\color{red}\item [\circled{\emph{\color{red}{c.}}}]$(A \eor A) \eif \enot(A \eand A)$}}{\item	$(A \eor A) \eif \enot(A \eand A)$}
	\iflabelexists{showanswers}{{\color{red}\item [\circled{\emph{\color{red}{d.}}}]$[(A \eand B) \eor (D \eif E)] \eif $ \linebreak[4]$ \enot[(A \eand B) \eand (D \eif E)]$}}{$[(A \eand B) \eor (D \eif E)] \eif $ \linebreak[4]$ \enot[(A \eand B) \eand (D \eif E)]$}
	\item	$(A \eand B) \eif \enot(A \eor B)$
	\end{enumerate}
\end{flushleft} 

\end{longtabu}
\end{exercises}
\noindent\problempart For each problem, a sentence form is given in sentence variables. Identify which of the sentences after it are legitimate substitution instances of that form. 

\begin{exercises}
\begin{longtabu}{p{2.5in}p{2.5in}}

\item $ \script{P} \eand \script{P} $ 
\begin{flushleft} 	
\begin{enumerate}[label=\alph*.]
\item 	$A \eand B$
\item 	$D \eor D$
\item 	$Z \eand Z$
\item 	$(Z \eor B) \eand (Z \eand B)$
\item 	$(Z \eor B) \eand (Z \eor B)$
\end{enumerate}
\end{flushleft}
%\begin{flushleft} 	
%\begin{enumerate}[label=\alph*.]
%\item 	$A \eand B$
%\item 	$D \eor D$
%\item 	\framebox{$Z \eand Z$}
%\item 	$(Z \eor B) \eand (Z \eand B)$
%\item 	\framebox{$(Z \eor B) \eand (Z \eor B)$}
%\end{enumerate}
%\end{flushleft}
&
\item $ \script{O} \eand (\script{N} \eand \script{N}) $ 
\begin{flushleft} 	
\begin{enumerate}[label=\alph*.]
\item 	$A \eand (B \eand C)$
\item 	$A \eand (A \eand B)$
\item 	$(A \eand B) \eand B$
\item 	$A \eand (B \eand B)$
\item 	$(C\eif D) \eand (Q \eand Q)$
\end{enumerate}
\end{flushleft}
%\begin{flushleft} 	
%\begin{enumerate}[label=\alph*.]
%\item 	$A \eand (B \eand C)$
%\item 	$A \eand (A \eand B)$
%\item 	$(A \eand B) \eand B$
%\item 	\framebox{$A \eand (B \eand B)$}
%\item 	\framebox{$(C\eif D) \eand (Q \eand Q)$}
%\end{enumerate}
%\end{flushleft}
\\ 
\item $ \script{H} \eif \script{Z} $ 
\begin{flushleft} 	
\begin{enumerate}[label=\alph*.]
\item 	$E \eif E$
\item 	$G \eif H$
\item 	$G \eif (I \eif K)$
\item 	$[(I \eif K) \eif G] \eif A$
\item 	$G \eand (I \eif K)$
\end{enumerate}
\end{flushleft}
%\begin{flushleft} 	
%\begin{enumerate}[label=\alph*.]
%\item 	\framebox{$E \eif E$}
%\item 	\framebox{$G \eif H$}
%\item 	\framebox{$G \eif (I \eif K)$}
%\item 	\framebox{$[(I \eif K) \eif G] \eif A$}
%\item 	$G \eand (I \eif K)$
%\end{enumerate}
%\end{flushleft}
&
\item $ \enot \script{H} \eand \script{C} $ 
\begin{flushleft} 	
\begin{enumerate}[label=\alph*.]
\item 	$H \eand C$
\item 	$\enot (H \eand C)$
\item 	$\enot Q \eand R$
\item 	$R \eand \enot Q$
\item 	$\enot (X \eiff Y) \eand (Y \eif Z)$
\end{enumerate}
\end{flushleft}
%\begin{flushleft} 	
%\begin{enumerate}[label=\alph*.]
%\item 	$H \eand C$
%\item 	$\enot (H \eand C)$
%\item 	\framebox{$\enot Q \eand R}$
%\item 	$R \eand \enot Q$
%\item 	\framebox{$\enot (X \eiff Y) \eand (Y \eif Z)$}
%\end{enumerate}
%\end{flushleft}
\\
\item $ \enot (\script{G} \eiff \script{M}) $ 
\begin{flushleft} 	
\begin{enumerate}[label=\alph*.]
\item 	$\enot (K \eiff K) $
\item 	$\enot K \eiff K$
\item 	$\enot ((I \eiff K) \eiff (S \eand S)) $
\item 	$\enot (H \eif (I \eor J)$
\item 	$\enot ((H \eor F)  \eiff (Z \eif D) ) $
\end{enumerate}
\end{flushleft}
&
\item $ (\script{I} \eif \script{W}) \eor \script{W} $ 
\begin{flushleft} 	
\begin{enumerate}[label=\alph*.]
\item 	$(D \eor E) \eif E$
\item 	$(D \eif E) \eor E$
\item 	$ D \eif (E \eor E)	$
\item 	$ ((W \eand L) \eif L) \eor W$
\item 	$((W \eand L) \eif J) \eor J$
\end{enumerate}
\end{flushleft}

\\

\item $ \script{M} \eor (\script{A} \eor \script{A}) $ 
\begin{flushleft} 	
\begin{enumerate}[label=\alph*.]
\item 	$ A \eor (A \eor A) 			$
\item 	$ (A \eor A) \eor A			$
\item 	$ C \eor (C \eor D)			$
\item 	$ (R \eif K) \eor ((D \eand G) \eor (D \eand G)) 			$
\item 	$ (P \eand P)  \eor ((\enot H \eand C) \eor (\enot H \eand C)) 			$
\end{enumerate}
\end{flushleft}
&
\item $ \script{A} \eif \enot (\script{G} \eand \script{G}) $ 
\begin{flushleft} 	
\begin{enumerate}[label=\alph*.]
\item 	$B \eiff \enot (G \eand G) 			$
\item 	$O \eif \enot (R \eand D) 			$
\item 	$(H \eif Z) \eif (\enot D	\eand D)		$
\item 	$ (O \eand (N \eand N))  \eif \enot (F \eand F)			$
\item 	$\enot D \eand \enot( (J \eif J) \eand (O \eiff O) $ 
\end{enumerate}
\end{flushleft}
\\
\item $ \enot ((\script{K} \eif \script{K}) \eor \script{K}) \eand \script{G} $ 
\begin{flushleft} 	
\begin{enumerate}[label=\alph*.]
\item 	$\enot (D \eif D) (\eor D \eand L)	 			$
\item 	$ \enot (D \eif (D \eor (D \eand L))				$
\item 	$ \enot ((D \eif D) \eor D) \eand L			$
\item 	$((\enot K \eif \enot K) \eor K) \eand L 			$
\item 	$ \enot ((D \eif D) \eor D) \eand ((D \eif D) \eor D)			$
\end{enumerate}
\end{flushleft}
&
\item $ (\script{B} \eiff (\script{N} \eiff \script{N})) \eor \script{N} $ 
\begin{flushleft} 	
\begin{enumerate}[label=\alph*.]
\item 	$(B \eiff (N \eiff (N \eand N))) \eor N  			$ %nope
\item 	$((E \eand T) \eiff (V \eiff V )) \eor V  			$  %yup
\item 	$ (B \eiff (N \eand N)) \eor B			$  %nope
\item 	$A \eiff (N \eiff (N \eor N)))			$ %nope
\item 	$((X \eiff N) \eiff N) \eor N			$ %nope
\end{enumerate}
\end{flushleft}
\end{longtabu}
\end{exercises}

\noindent\problempart Use the following symbolization key in the gray bubble to create substitution instances of the sentences below.

\begin{mdframed}[style=mytablebox] 
\begin{longtabu}{X[.5]X[.5]X[1]X[1]X[1]} 
$\script{A}: B$ 	& 	$\script{B}: \enot C$  	& $\script{C}: A \eif B$ &
$\script{D}:\enot (B \eand C)$  & $\script{E}: D \eiff E$
\end{longtabu}
\end{mdframed}

\begin{exercises}
\begin{longtabu}{X[1,l,m]X[1,p,m]} 
\item $\enot( \script{A} \eiff \script{B})$ 
\answer{$\enot(B \eiff  \enot{C})$}
&

\item $(\script{B} \eif \script{C}) \eand \script{D}$ 


\answer{$(\enot{C} \eif (A \eif B)) \eand \enot(B \eand C)$}
\\
\item $\script{D} \eif (\script{B} \eand \enot \script{B}) $ 


\answer{$\enot(B \eand C) \eif (\enot{C} \eand \enot \enot{C}) $}
&
\item $\enot \enot (\script{C} \eor \script{E})$ 


\answer{$\enot \enot ((A \eif B) \eor (D \eiff E))$}
\\
\item $\enot \script{C} \eiff (\enot \enot \script{D} \eand \script{E})$ 


\answer{$\enot (A \eif B) \eiff (\enot \enot \enot(B \eand C) \eand (D \eiff E))$}

\end{longtabu}
\end{exercises}



\noindent\problempart Use the following symbolization key in the gray bubble to create substitution instances of the sentences below.

\begin{mdframed}[style=mytablebox] 
\begin{longtabu}{X[1]X[1]X[1]X[.5]X[.5]} 
$\script{A}: I \eor (I \eiff V)  $ 
&	$\script{B}: C \eiff V$ 
&	$\script{C}: L \eif X$  
&	$\script{D}: V$  
&	$\script{E}: U$ 
\end{longtabu}
\end{mdframed}

\begin{exercises}
\begin{longtabu}{X[1,l,m]X[1,p,m]} 
\item $\enot \script{A} \eif \enot \script B$ 
&
\item $\enot(\script{B} \eand \script{D})$ 
\\
\item $(\script{A} \eif \script{A}) \eor (\script{C} \eif \script{A})$ 
&
\item $[(\script{A} \eif \script{B}) \eif \script{A}] \eif \script{A}$ 
\\
\item $\script{A} \eand (\script{B} \eand (\script{C} \eand (\script{D} \eand \script{E})))$
&\\
\end{longtabu}
\end{exercises}

%%%%%%%%%%%%%%%%%% Part E


\noindent\problempart \label{sec4.1partC} Decide whether the following are examples of $\eif$E (modus ponens).

\begin{exercises}
\begin{longtabu}{X[1,p,m]X[1,p,m]X[1,p,m]} 

\item \begin{earg*}
\item $A \eif B$ 
\item $B \eif C$ 
\itemc[.3] $A \eif C$
\end{earg*}
\answer{\framebox{Not MP}}
	
&

\item \begin{earg*}	
\item$P \eand Q$ 
\item 	$P$ 
\itemc[.3] 	 $Q$
\end{earg*}

\answer{\framebox{Not MP}}
	
&
\item \begin{earg*}	
\item $P \eif Q$ 
\itemc[.3] 	$Q$
\end{earg*}
\answer{\framebox{Not MP}}

\\
\item \begin{earg*}	
\item $D \eif E$ 
\item 	$E$ 
\itemc[.3] 	$D$
\end{earg*}
\answer{\framebox{Not MP}}

&

\item \begin{earg*}
\item $(P \eand Q) \eif (Q \eand V)$
\item 	$P \eand Q$
\itemc[.3] 	 $Q \eand V$
\end{earg*}
\answer{\framebox{MP}}
\end{longtabu}
\end{exercises}
	

\noindent\problempart \label{sec4.1partC} Decide whether the following are examples of $\eif$E (modus ponens).

\begin{exercises}
\begin{longtabu}{X[1]X[1]} 
\item \begin{earg*}
\item	$C \eif D$  
\itemc[.3] 	 $C$
\end{earg*}
%\frame{Not MP}\\
	
&

\item \begin{earg*}
\item $(C \eand L) \eif (E \eor C)$ 
\item $C \eand L$ 
\itemc[.3] 	  $E \eor C$
\end{earg*}
%\framebox{MP}
	
\\
\item \begin{earg*}
\item  $\enot A \eif B$ 
\item $\enot B$ 
\itemc[.3] 	 $B$
\end{earg*}
	%\framebox{Not MP}\\
&

\item \begin{earg*}
\item	$X \eif \enot Y$ 
\item  	$\enot Y$ 
\itemc[.3] 	 $\therefore$\ $X$
\end{earg*}
%\framebox{Not MP}\\
\\
\item \begin{earg*}
\item $G \eif H$ 
\item  $\enot H$ 
\itemc[.3] 	  $\enot G$
\end{earg*}
%\framebox{Not MP}\\

\end{longtabu}
\end{exercises}


%%%% part G

\noindent\problempart Decide whether the following are examples of \eor-E (disjunctive syllogism). 

\begin{exercises}
\begin{longtabu}{X[1]X[1]} 
\item \begin{earg*}
\item $(A \eif B) \eor (X \eif Y)$  
\item $\enot A$  
\itemc[.3]  $X \eif Y$
\end{earg*}

\answer{\framebox{Not DS}}

&	

\item \begin{earg*}
\item $[(S \eor T) \eor U] \eor V$  
\item $\enot[(S \eor T) \eor U]$  
\itemc[.3] $V$
\end{earg*}
\answer{\framebox{DS}}

\\
\item \begin{earg*}
\item $P \eor Q$  
\item $P$  
\itemc[.3] \enot $Q$
\end{earg*}
\answer{\framebox{Not DS}}

&
\item \begin{earg*}
\item $\enot (A \eor B)$  
\item $\enot A$  
\itemc[.3] $B$
\end{earg*}
\answer{\framebox{Not DS}}
\\

\item \begin{earg*}
\item $(P \eor Q) \eor R$  
\itemc[.3]  $R$
\answer{\framebox{Not DS}}
\end{earg*}

\end{longtabu}
\end{exercises}

\noindent\problempart Decide whether the following are examples of \eor-E (disjunctive syllogism).

\begin{exercises}
\begin{longtabu}{X[1]X[1]} 
\item \begin{earg*} 
\item $(C \eand D) \eor E$  
\item $(C \eand D)$  
\itemc[.3] $E$
\end{earg*}
%\framebox{Not DS}\\
&

\item \begin{earg*} 
\item $(P \eor Q) \eif R$  
\item $\enot(P \eor Q)$  
\itemc[.3] $R$
\end{earg*}
%\framebox{Not DS}\\
\\

\item \begin{earg*} 
\item  $X \eor (Y \eif Z)$  
\item $\enot X$  
\itemc[.3] $Y \eif Z$
\end{earg*}
%\framebox{DS}\\

&
\item \begin{earg*} 
\item $(P \eor Q) \eor R$  
\item  $\enot P$  
\itemc[.3] $Q$
\end{earg*}
%\framebox{Not DS}\\

\\
\item \begin{earg*} 
\item $A \eor (B \eor C)$  
\item $\enot A$   
\itemc[.3]  $B \eor C$	
\end{earg*}
%\framebox{DS}\\
\end{longtabu}
\end{exercises}




% *******************************************
% *				Basic Rules for Sentential Logic	   *	
% *******************************************

\section{Basic Rules for Sentential Logic}
\setlength{\parindent}{1em}
%rob: I removed indirect and conditional proof from this section, so that they would have practice just doing direct proofs before they moved on to the fancy stuff. 

In designing a proof system, we could just start with disjunctive syllogism and modus ponens. Whenever we discovered a valid argument that could not be proved with rules we already had, we could introduce new rules. Proceeding in this way, we would have an unsystematic grab bag of rules. We might accidentally add some rules, and we would surely end up with more rules than we need.

Instead, we will develop what is called a \define{system of natural deduction}. In a natural deduction system, there will be two rules for each logical operator: an introduction, and an elimination rule. The introduction rule will allow us to prove a sentence that has the operator you are ``introducing'' as its main connective. The elimination rule will allow us to prove something given a sentence that has the operator we are ``eliminating'' as the main logical operator.

In addition to the rules for each logical operator, we will also have a reiteration rule. If you already have shown something in the course of a proof, the reiteration rule allows you to repeat it on a new line. We can define the rule of reiteration like this

Reiteration (R)
\begin{proof}
	\have[m]{a}{\script{A}}
	\have[n]{b}{\script{A}} \by{R}{a}
\end{proof}

This diagram shows how you can add lines to a proof using the rule of reiteration. As before, the script letters represent sentences of any length. The upper line shows the sentence that 
comes earlier in the proof, and the bottom line shows the new sentence you are allowed to write and how you justify it. The reiteration rule above is justified by one line, the line that 
you are reiterating. So the ``R $m$'' on line 2 of the proof means that the line is justified by the reiteration rule (R) applied to line $m$. The letters $m$ and $n$ are variables, not 
real line numbers. In a real proof, they might be lines 5 and 7, or lines 1 and 2, or whatever. When we define the rule, however, we use variables to underscore the point that the rule 
may be applied to any line that is already in the proof.

Obviously, the reiteration rule will not allow us to show anything \emph{new}. For that, we will need more rules. The remainder of this section will give six basic introduction and 
elimination rules. This will be enough to do some basic proofs in SL. Sections \ref{sec:conditional_proof} through \ref{sec:indirect_proof} will explain introduction rules involved in 
fancier kinds of derivation called conditional proof and indirect proof. The remaining sections of this chapter will develop our system of natural deduction further and give you tips for 
playing in it.

All of the rules introduced in this chapter are summarized starting on p.~\pageref{sec:proof_rules}.

%%%%%%%%%%%%%
%rcr I added the proofrules label above -- added the rules at the end of the chapter
%%%%%%%%%%%%

\subsection{Conjunction}

Think for a moment: What would you need to show in order to prove $E \eand F$?

Of course, you could show $E \eand F$ by proving $E$ and separately proving $F$. This holds even if the two conjuncts are not atomic sentences. If you can prove $[(A \eor J) \eif V]$ and  $[(V \eif L) \eiff (F \eor N)]$, then you have effectively proved $[(A \eor J) \eif V] \eand [(V \eif L) \eiff (F \eor N)].$
So this will be our conjunction introduction rule, which we abbreviate {\eand}I:

\begin{multicols}{2}

\begin{proof}
	\have[m]{a}{\script{A}}
	\have[n]{b}{\script{B}}
	\have[\ ]{c}{\script{A}\eand\script{B}} \ai{a, b}
\end{proof}

\begin{proof}
	\have[m]{a}{\script{A}}
	\have[n]{b}{\script{B}}
	\have[\ ]{c}{\script{B}\eand\script{A}} \ai{a, b}
\end{proof}

\end{multicols}

A line of proof must be justified by some rule, and here we have ``{\eand}I $m$, $n$.'' This means: Conjunction introduction applied to line $m$ and line $n$. Again, these are variables, not real line numbers; $m$ is some line and $n$ is some other line. If you have $K$ on line 8 and $L$ on line 15, you can prove $(K\eand L)$ at some later point in the proof with the justification ``{\eand}I 8, 15.'' 

We have written two versions of the rule to indicate that you can write the conjuncts in any order. Even though $K$ occurs before $L$ in the proof, you can derive $(L \eand K)$ from them using the right-hand version {\eand}I. You do not need to mark this in any special way in the proof.

Now, consider the elimination rule for conjunction. What are you entitled to conclude from a sentence like $E \eand F$? Surely, you are entitled to conclude $E$; if $E \eand F$ were true, then $E$ would be true. Similarly, you are entitled to conclude $F$. This will be our conjunction elimination rule, which we 
abbreviate {\eand}E:

\begin{multicols}{2}
\begin{proof}
	\have[m]{ab}{\script{A}\eand\script{B}}
	\have[\ ]{a}{\script{A}} \ae{ab}
\end{proof}

\begin{proof}
	\have[m]{ab}{\script{A}\eand\script{B}}
	\have[\ ]{a}{\script{B}} \ae{ab}
\end{proof}
\end{multicols}

When you have a conjunction on some line of a proof, you can use {\eand}E to derive either of the conjuncts. Again, we have written two versions of the rule to indicate that it can be applied to either side of the conjunction. The {\eand}E rule requires only one sentence, so we write one line number as the justification for applying it. For example, both of these moves are acceptable in derivations. 

\begin{multicols}{2}
\begin{proof}
\have[4]{4}{A \eand (B \eor C)}
\have[5]{5}{A} \ae{4}
\end{proof}

\begin{proof}
\have[10]{10}{A \eand (B \eor C)}
\have[\ldots]{...}{\ldots}
\have[15]{15}{(B \eor C)} \by {\eand E}{10}
\end{proof}
\end{multicols}
Some textbooks will only let you use \eand E on one side of a conjunction. They then make you \emph{prove} that it works for the other side. We won't do this, because it is a pain in the neck. 

Even with just these two rules, we can provide some proofs. Consider this argument.
\begin{earg}
\item[] $[(A\eor B)\eif(C\eor D)] \eand [(E \eor F) \eif (G\eor H)]$
\item[$\therefore$] $[(E \eor F) \eif (G\eor H)] \eand [(A\eor B)\eif(C\eor D)]$
\end{earg}
The main logical operator in both the premise and conclusion is a conjunction. Since the conjunction is symmetric, the argument is obviously valid. In order to provide a proof, we begin by writing down the premise. After the premises, we draw a horizontal line---everything below this line must be justified by a rule of proof. So the beginning of the proof looks like this:

\begin{proof}
	\hypo{ab}{{[}(A\eor B)\eif(C\eor D){]} \eand {[}(E \eor F) \eif (G\eor H){]}}
\end{proof}

From the premise, we can get each of the conjuncts by {\eand}E. The proof now looks like this:

\begin{proof}
	\hypo{ab}{{[}(A\eor B)\eif(C\eor D){]} \eand {[}(E \eor F) \eif (G\eor H){]}}
	\have{a}{{[}(A\eor B)\eif(C\eor D){]}} \ae{ab}
	\have{b}{{[}(E \eor F) \eif (G\eor H){]}} \ae{ab}
\end{proof}

The rule {\eand}I requires that we have each of the conjuncts available somewhere in the proof. They can be separated from one another, and they can appear in any order. So by applying the {\eand}I rule to lines 3 and 2, we arrive at the desired conclusion. The finished proof looks like this:

\begin{proof}
	\hypo{ab}{{[}(A\eor B)\eif(C\eor D){]} \eand {[}(E \eor F) \eif (G\eor H){]}}

	\have{a}{{[}(A\eor B)\eif(C\eor D){]}} \ae{ab}
	\have{b}{{[}(E \eor F) \eif (G\eor H){]}} \ae{ab}
	\have{ba}{{[}(E \eor F) \eif (G\eor H){]} \eand {[}(A\eor B)\eif(C\eor D){]}} \ai{b,a}
\end{proof}

This proof is trivial, but it shows how we can use rules of proof together to demonstrate the validity of an argument form. Also: Using a truth table to show that this argument is valid would have required a staggering 256 lines, since there are eight sentence letters in the argument.

%When we defined a wff, we did not allow for conjunctions with more than two conjuncts. If we had done so, then we could define a more general version of the rules of proof for conjunction.


\subsection{Disjunction}
If $M$ were true, then $M \eor N$ would also be true. So the disjunction introduction rule ({\eor}I) allows us to derive a disjunction if we have one of the two disjuncts:

\begin{multicols}{2}

\begin{proof}
	\have[m]{a}{\script{A}}
	\have[\ ]{ab}{\script{A}\eor\script{B}}\oi{a}
\end{proof}

\begin{proof}
	\have[m]{a}{\script{A}}
	\have[\ ]{ab}{\script{B}\eor\script{A}}\oi{a}
\end{proof}

\end{multicols}

Like the rule of conjunction elimination, this rule can be applied two ways. Also notice that \script{B} can be \emph{any} sentence whatsoever. So the following is a legitimate proof:

\begin{proof}
	\hypo{m}{M}
	\have{mmm}{M \eor ([(A\eiff B) \eif (C \eand D)] \eiff [E \eand F])}\oi{m}
\end{proof}

This might seem odd. How can we prove a sentence that includes $A$, $B$, and the rest, from the simple sentence $M$---which has nothing to do with the other letters? The secret here is to remember that all the new letters are on just one side of a disjunction, and nothing on that side of the disjunction has to be true. As long as $M$ is true, we can add whatever we want after a disjunction and the whole thing will continue to be true. 

Now consider the disjunction elimination rule. What can you conclude from $M \eor N$? You cannot conclude $M$. It might be $M$'s truth that makes $M \eor N$ true, as in the example above, 
but it might not. From $M \eor N$ alone, you cannot conclude anything about either $M$ or $N$ specifically. If you also knew that $N$ was false, however, then you would be able to 
conclude $M$.

\begin{multicols}{2}
\begin{proof}
	\have[m]{ab}{\script{A}\eor\script{B}}
	\have[n]{nb}{\enot\script{B}}
	\have[\ ]{a}{\script{A}} \oe{ab,nb}
\end{proof}

\begin{proof}
	\have[m]{ab}{\script{A}\eor\script{B}}
	\have[n]{na}{\enot\script{A}}
	\have[\ ]{b}{\script{B}} \oe{ab,nb}
\end{proof}
\end{multicols}

We've seen this rule before: it is just disjunctive syllogism. Now that we are using a system of natural deduction, we are going to make it our rule for disjunction elimination ({\eor}E). Once again, the rule works on both sides of the sentential connective. 

\subsection{Conditionals and biconditionals}

The rule for conditional introduction is complicated because it requires a whole new kind of proof, called conditional proof. We will deal with this in the next section. For now, we will 
only use the rule of conditional elimination.

Nothing follows from $M\eif N$ alone, but if we have both $M \eif N$ and $M$, then we can conclude $N$. This is another rule we've seen before: modus ponens. It now enters our system of 
natural deduction as the conditional elimination rule ({\eif}E).

\begin{proof}
	\have[m]{ab}{\script{A}\eif\script{B}}
	\have[n]{a}{\script{A}}
	\have[\ ]{b}{\script{B}} \ce{ab,a}
\end{proof}

Biconditional elimination ({\eiff}E) will be a double-barreled version of conditional elimination. If you have the left-hand subsentence of the biconditional, you can derive the 
right-hand subsentence. If you have the right-hand subsentence, you can derive the left-hand subsentence. This is the rule:

\begin{multicols}{2}
\begin{proof}
	\have[m]{ab}{\script{A}\eiff\script{B}}
	\have[n]{a}{\script{A}}
	\have[\ ]{b}{\script{B}} \be{ab,a}
\end{proof}

\begin{proof}
	\have[m]{ab}{\script{A}\eiff\script{B}}
	\have[n]{a}{\script{B}}
	\have[\ ]{b}{\script{A}} \be{ab,a}
\end{proof}
\end{multicols}

\subsection{Invalid argument forms}

%rob: I added this brief subsection to make clear what was going to happen in the first problem part, and to re-emphasize the idea of invalid arguments

In section \ref{sec:substitution_instances}, in the last two problem parts (p. \pageref{sec4.1partC}), we saw that sometimes an argument looks like a legitimate substitution instance of a 
valid argument form, but really isn't.  For instance, the problem set C asked you to identify instances of modus ponens. Below I'm giving you two of the answers.
 
\begin{multicols}{2}
(5) Modus ponens
	\begin{earg}
	\item[1.] $(C \eand L) \eif (E \eor C)$
	\item[2.] $C \eand L$
	\item[] \textcolor{white}{.}\sout{\hspace{.5\linewidth}} \textcolor{white}{.} 
	\item[$\therefore$] $E \eor C$
	\end{earg}
(7) \emph{Not} modus ponens.
	\begin{earg} 
	\item[1.] $D \eif E$
	\item[2.] $E$
\item[] \textcolor{white}{.}\sout{\hspace{.2\linewidth}} \textcolor{white}{.} 
	\item[$\therefore$] $D$
	\end{earg}
\end{multicols}
The argument on the left is an example of a valid argument, because it is an instance of modus ponens, while the argument on the right is an example of an invalid argument, because it is not an example of modus ponens. (We originally defined the terms valid and invalid on p. \pageref{def:valid}). Arguments like the one on the right, which try to trick you into thinking that they are instances of valid arguments, are called \define{deductive fallacies}. The argument on the right is specifically called the fallacy of \define{affirming the consequent}. In the system of natural deduction we are using in this textbook, modus ponens has been renamed ``conditional elimination,'' but it still works the same way. So you will need to be on the lookout for deductive fallacies like affirming the consequent as you construct proofs. 

\subsection{Notation}

The rules we have learned in this chapter give us enough to start doing some basic derivations in SL. This will allow us to prove things syntactically which would have been too cumbersome to prove using the semantic method of truth tables. We now need to introduce a few more symbols to be clear about what methods of proof we are using. 

In Chapter 1, we used the three dots $\therefore$ to indicate generally that one thing followed from another. In chapter 3 we introduced the double turnstile, $\sdtstile{}{}$, to indicate that one statement could be proven some others using truth tables. Now we are going to use a single turnstile, $\sststile{}{}$, to indicate that we can derive a statement from a bunch of premises, using the system of natural deduction we have begun to introduce in this section. Thus we will write $\{\script{A}, \script{B}, \script{C}\} \sststile{}{} \script{D}$, to indicate that there is a derivation going from the premises \script{A}, \script{B}, and \script{C} to the conclusion \script{D}. Note that these are metavariables, so I could be talking about any sentences in SL.

The single turnstile will work the same way the double turnstile did. So, in addition to the uses of the single turnstile above we can write  $\sststile{}{} \script{A}$ to indicate that \script{A} can be proven a tautology using syntactic methods. We can write $\script{A}\nsststile{}{} \hspace{.5em}  \sststile{}{}\script{B}$ to say that \script{A} and \script{B} can be proven logically equivalent using these derivations. You will learn how to do these later things at the end of the chapter. In the meantime, we need to practice our basic rules of derivation.

%%%%%%%%%% PRACTICE PROBLEMS %%%%%%
\practiceproblems
\noindent\problempart Some of the following arguments are legitimate instances of our six basic inference rules. The others are either invalid arguments or valid arguments that are still illegitimate because they would take multiple steps using our basic inference rules. For those that are legitimate, mark the rule that they are instances of. Mark those that are not ``Not a single inference.'' 

\begin{exercises}
\begin{longtabu}{X[1]X[1]} 

\item %1
	\begin{earg*}
	\item $R \eor S$ 
\itemc[.3] $S$
	\end{earg*}
\answer{\factoidbox{Not a single inference}}

&

\item %2
	\begin{earg*}
	\item $(A \eif B) \eor (B \eif A)$
	\item $A \eif B$
\itemc[.3] $B \eif A$
	\end{earg*}

\answer{\factoidbox{Not a single inference}}

\\

\item %3
	\begin{earg*}
	\item $P \eand (Q \eor R)$
\itemc[.3] $R$
	\end{earg*}
\answer{\factoidbox{Not a single inference}}

&
\item %4
	\begin{earg*}
	\item $P \eand (Q \eand R)$
\itemc[.3] $P$
	\end{earg*}
\answer{		\factoidbox{\eand-Elimination}}

\\
\item %5
	\begin{earg*}
	\item  $A$
\itemc[.3] $P \eand (Q \eif A)$
	\end{earg*}
\answer{\factoidbox{Not a single inference}}

&	
	
\item %6
	\begin{earg*}
	\item  $A$
	\item  $B \eand C$
\itemc[.3] $(A \eand B) \eand C$
	\end{earg*}
\answer{\factoidbox{\begin{flushleft}Not a single inference. You need the associativity of \eand to infer this.\end{flushleft}}}
\\
\item %7
	\begin{earg*}
	\item $(X \eand Y) \eiff (Z \eand W)$
	\item $Z \eand W$
\itemc[.3] $X \eand Y$
	\end{earg*}
\answer{	\factoidbox{\eiff-Elimination}}
&
\item %8
	\begin{earg*}
	\item $((L \eif M) \eif N) \eif O$
	\item $L$
\itemc[.3] $M$
	\end{earg*}
\answer{	\factoidbox{Not a single inference}}

\end{longtabu}
\end{exercises}


\noindent\problempart Some of the following arguments are legitimate instances of our six basic inference rules. The others are either invalid arguments or valid arguments that are still illegitimate because they would take multiple steps using our basic inference rules. For those that are legitimate, mark the rule that they are instances of. Mark those that are not ``Not a single inference.''

\begin{exercises} \vspace{-.5cm}
\begin{longtabu}{X[1]X[1]} 

\item %1
	\begin{earg*}
	\item  $A \eand B$
 
	\itemc[.3]$A$ 	
	\end{earg*}
%\factoidbox{\eand-Elimination}
&

\item %2
	\begin{earg*}
	\item $A \eif (B \eand (C \eor D))$
	\item $A$
 
	\itemc[.3]$B \eand (C \eor D)$
	\end{earg*}
%\factoidbox{\eif-Elimination}
\\

\item %3
	\begin{earg*}
	\item $P \eand (Q \eand R)$
 
	\itemc[.3]$R$
	\end{earg*}
%	\factoidbox{\begin{flushleft}Not a single inference. You need two uses of \eand-elim. to do this\end{flushleft}}
&

\item %4
	\begin{earg*}
	\item  $P$
 
	\itemc[.3] $P \eor [A \eand (B \eiff C)]$
	\end{earg*}
%	\factoidbox{\eor-Introduction}
\\

\item %5
	\begin{earg*}
	\item  $M$
	\item  $D \eand C$
 
	\itemc[.3] $M \eand (D \eand C)$
	\end{earg*}
%	\factoidbox{\eand-Introduction}
&

\item %6
	\begin{earg*}
	\item $(X \eand Y) \eif (Z \eand W)$
	\item $Z \eand W$
 
	\itemc[.3]$X \eand Y$
	\end{earg*}
%	\factoidbox{Not a single inference}
\\

\item %7
	\begin{earg*}
	\item $(X \eand Y) \eif (Z \eand W)$
	\item $\enot (X \eand Y)$
 
	\itemc[.3]$\enot(Z \eand W)$
	\end{earg*}
%	\factoidbox{Not a single inference}
&

\item %8
	\begin{earg*}
	\item $((L \eif M) \eif N) \eif O$
	\item $(L \eif M) \eif N$
 
	\itemc[.3]$O$
	\end{earg*}
%	\factoidbox{\eif-Elimination}

\end{longtabu}
\end{exercises}

\vspace{-8pt}

\noindent\problempart \label{pr.justifySLproof} Fill in the missing pieces in the following proofs. Some are missing the justification column on the right. Some are missing the left column that contains the actual steps, and some are missing lines from both columns.
%rob: problem one was in the original problem section at the end of Chapter 6. 

\begin{exercises}
\vspace{-.5cm}
\begin{longtabu}{X[1.4]X[1]} 

\item \textcolor{white}{.}  
\vspace{-16pt}
\begin{proof}
	\hypo{1}{W \eif \enot B}
	\hypo{2}{A \eand W}
	\hypo{3}{B \eor (J \eand K)} \by{Want: $K$}{}
	\have{4}{W}{} \iflabelexists{showanswers}{\by{\color{red}\eand E,}{2}}{}
	\have{5}{\enot B} {} \iflabelexists{showanswers}{\by{\color{red}\eif E,} {1,4}}{}
	\have{6}{J \eand K} {} \iflabelexists{showanswers}{\by{\color{red}\eor E}{3,5}}{}
	\have{7}{K}{} \iflabelexists{showanswers}{\by{\color{red}\eand E}{6}}{}
	\end{proof}

&

\item \textcolor{white}{.} 
\vspace{-16pt}

	\begin{proof}
	\hypo{1}{W \eand B}
	\hypo{2}{E \eand Z} \by{Want: $W \eand Z$}{}
	\have{3}{\iflabelexists{showanswers}{\color{red}W}{}} \by{\eand E}{1}
	\have{4}{\iflabelexists{showanswers}{\color{red}Z}{}} \by{\eand E}{2}
	\have{5}{W \eand Z} \iflabelexists{showanswers}{\by{\color{red}\eand I}{3, 4}}{}
	\end{proof}



\\

\item \textcolor{white}{.} 
\vspace{-16pt}
	\begin{proof}
	\hypo{1}{(A \eand B) \eand C} \by{Want: $A \eand (B \eand C)$}{}
	\have{2}{A \eand B} \iflabelexists{showanswers}{\by{\color{red}\eand E}{1}}{}	
	\have{3}{C} \iflabelexists{showanswers}{\by{\color{red}\eand E}{1}}{}
	\have{4}{A} \iflabelexists{showanswers}{\by{\color{red}\eand E}{2}}{}
	\have{5}{B} \iflabelexists{showanswers}{\by{\color{red}\eand E}{2}}{}
	\have{6}{B \eand C} \iflabelexists{showanswers}{\by{\color{red}\eand I}{3,5}}{}
	\have{7}{A \eand (B \eand C)} \iflabelexists{showanswers}{\by{\color{red}\eand I}{4,6}}{}
	\end{proof}

&

\item \textcolor{white}{.}  
\vspace{-16pt}
\begin{proof}
	\hypo{1}{(\enot A \eand B) \eif C}
	\hypo{2}{\enot A}
	\hypo{3}{A \eor B} \by{Want: $C$}{}
	\have{4}{\iflabelexists{showanswers}{\color{red}B}{}} \by{\eor E}{2, 3}
	\have{5}{\iflabelexists{showanswers}{\color{red}\enot A \eand B}{}} \by{\eand I}{2, 4}
	\have{6}{\iflabelexists{showanswers}{\color{red}C}{}} \by{\eif E}{1, 5}
	\end{proof} 


%\iflabelexists{showanswers}{\by{\color{red}foo}{bar}}{}
%\iflabelexists{showanswers}{\color{red}Foo}{}
\\
\vspace{-1cm}
\item \textcolor{white}{.}  
\vspace{-16pt}
	\begin{proof}
	\hypo{1}{\enot A \eand (\enot B \eand C)}
	\hypo{2}{C \eif (D \eand (B \eor E))}
	\hypo{3}{(E \eand \enot A) \eif F}	\by{Want: $D \eand F$}{} 
	\have{4}{\iflabelexists{showanswers}{\color{red}\enot A}{}} \by{\eand E}{1} 
	\have{5}{\enot B \eand C} \iflabelexists{showanswers}{\by{\color{red}\eand E}{1}}{} %
	\have{6}{\iflabelexists{showanswers}{\color{red}\enot B}{}} \by{\eand E}{5}
	\have{7}{C} \iflabelexists{showanswers}{\by{\color{red}\eand E}{5}}{} %
	\have{8}{\iflabelexists{showanswers}{\color{red}D \eand (B \eor E)}{}} \by{\eif E}{2, 7} 
	\have{9}{D}\iflabelexists{showanswers}{\by{\color{red}\eand E}{8}}{} %
	\have{10}{\iflabelexists{showanswers}{\color{red}B \eor E}{}} \by{\eand E}{8} 
	\have{11}{E} \iflabelexists{showanswers}{\by{\color{red}\eor  E}{6, 10}}{}
	\have[12]{12}{\iflabelexists{showanswers}{\color{red}E \eand \enot A}{}} \by{\eand I}{4, 11} 
	\have[13]{13}{F} \iflabelexists{showanswers}{\by{\color{red}\eif E}{3, 12}}{}
	\have[14]{14}{\iflabelexists{showanswers}{\color{red}D \eand F}{}} \by{\eand I}{9, 13} 
	\end{proof}

\end{longtabu}
\end{exercises}

\noindent\problempart \label{pr.justifySLproof} Fill in the missing pieces in the following proofs. Some are missing the justification column on the right. Some are missing the left column that contains the actual steps, and some are missing lines from both columns.

\begin{exercises}
\begin{longtabu}{X[1]X[1]} 

\item \textcolor{white}{.}  
\vspace{-16pt}
	\begin{proof}
	\hypo{1}{A \eand \enot B}
	\hypo{2}{A \eif \enot C}
	\hypo{3}{B \eor (C \eor D)}	 \by{Want: $D$}{}
	\have{4}{} \by{\eand E}{1}
	\have{5}{} \by{\eand E}{1}
	\have{6}{} \by{\eif E}{2, 4}
	\have{7}{} \by{\eor E}{3, 5}
	\have{8}{} \by{\eor E}{6,7}
	\end{proof}
&
\item \textcolor{white}{.}  
\vspace{-16pt}
	\begin{proof}
	\hypo{1}{W \eor V}
	\hypo{2}{I \eand (\enot Z \eif \enot W)}
	\hypo{3}{I \eif \enot Z} \by{Want: $I \eand V$}{}
	\have{4}{} \ae{2}
	\have{5}{} \ae{2}
	\have{6}{\enot Z} \by{}{}
	\have{7}{} \by{\eif E}{5, 6}
	\have{8}{V} \by{}{}
	\have{9}{} \ai{4,8}
	\end{proof}
\\
\item \textcolor{white}{.}  
\vspace{-16pt}
		
\begin{proof}
\hypo{1}{\enot P \eand S) \eiff S}
\hypo{2}{S \eand (P \eor Q)} \by{Want: Q}{}
\have{3}{S} \nix{\by{\eand E}{2}}
\have{4}{P \eor Q} \nix{\by{\eand E}{2}}
\have{5}{\enot P \eand S} \nix{\by{\eiff E}{1, 3}}
\have{6}{\enot P} \nix{\by{\eand E}{5}}
\have{7}{Q} \nix{\by{\eor E}{4, 6}}
\end{proof}

&
\item \textcolor{white}{.}  
\vspace{-16pt}
\begin{proof}
\hypo{1}{C \eif (A \eif B)}
\hypo{2}{D \eor C}
\hypo{3}{\enot D} \by{Want: A \eif B}{}
\have{4}{\nix{C}} \by{\eor E}{2, 3}
\have{5}{\nix{A \eif B}} \by{\eif E}{1, 4}
\end{proof}
\\

\item \textcolor{white}{.}  
\vspace{-16pt}

	\begin{proof}
	\hypo{1}{X \eand (Y \eand Z)} \by{Want: $(X \eor A) \eand [(Y \eor B) \eand (Z \eand C)]$} {}
	\have{2}{} \ae{1}
	\have{3}{} \ae{1}
	\have{4}{} \ae{3}
	\have{5}{} \ae{3}
	\have{6}{} \oi{2}
	\have{7}{} \oi{4}
	\have{8}{} \oi{5}
	\have{9}{} \by{\eand I}{7,8}
	\have{10}{} \ai{6,9}
	\end{proof}

\end{longtabu}
\end{exercises}

%%%%%PART E

\noindent\problempart Derive the following.

\begin{enumerate}[label=(\arabic*)]
\item \{$A \eif B, A\} \sststile{}{} A \eand B$

\answer{
	\begin{proof}
	\hypo{1}{A \eif B}
	\hypo{2}{A} \by{Want: A \eand B}{}
	\have{3}{B} \by{\eif E}{1,2}
	\have{4}{A \eand B} \ai{2,3}
	\end{proof}
}
\item \{$A \eiff D, C, [(A \eiff D) \eand C] \eif (C \eiff B)\} \sststile{}{} B$

\answer{
\begin{proof}
\hypo{1}{A \eiff D}
\hypo{2}{C}
\hypo{3}{((A \eiff D) \eand C) \eif (C \eiff B)} \by{Want: B}{} 
\have{4}{(A \eiff D) \eand C} \by{\eand I}{1, 2}
\have{5}{C \eiff B} \by{\eif E}{3, 4}
\have{6}{B} \by{\eiff E}{2, 5}
\end{proof}
}

\item \{$A \eiff B, B \eiff C, C \eif D, A\} \sststile{}{} D$

\answer{
	\begin{proof}
	\hypo{1}{A \eiff B}
	\hypo{2}{B \eiff C}
	\hypo{3}{C \eiff D} 
	\hypo{4}{A}	\by{Want: D}{}
	\have{5}{B} \be{1, 4}
	\have{6}{C} \be{2, 5}
	\have{7}{D} \be{3, 6}
	\end{proof}
}

\item $\{(A \eif \enot B) \eand A, B \eor C\} \sststile{}{} C$

\answer{
\begin{proof}
\hypo{1}{(A \eif \enot B) \eand A}
\hypo{2}{B \eor C} \by{Want: C}{}
\have{3}{A \eif \enot B} \by{\eand E}{1}
\have{4}{A}\by{\eand E}{1}
\have{5}{\enot B} \by{\eif E}{3, 4}
\have{6}{C} \by{\eor E}{2, 5}
\end{proof} 
}

\item $\{(A \eif B) \eor (C \eif (D \eand E)), \enot (A \eif B), C\} \sststile{}{} D$

\answer{
	\begin{proof}
	\hypo{1}{(A \eif B) \eor (C \eif (D \eand E))} 
	\hypo{2}{\enot (A \eif B)}
	\hypo{3}{C} \by{Want: D}{}
	\have{4}{C \eif (D \eand E)} \oe{1, 3}
	\have{5}{D \eand E} \ce{3, 4}
	\have{6}{D} \ae{5}
	\end{proof}
}

\item $\{C \eor (B \eand  A),  \enot C\} \sststile{}{} A \eor A$		%requires \eorI

\answer{
\begin{proof}
\hypo{1}{C \eor (B \eand A)}
\hypo{2}{\enot C} \by{Want: A \eor A}{}
\have{3}{B \eand A} \oe{1, 2}
\have{4}{A} \ae{3}
\have{5}{A \eor A} \oe{4}
\end{proof}
}

\item $\{A \eor B, \enot A, \enot B\} \sststile{}{} C$				%\eorIE trick

\answer{
	\begin{proof}
	\hypo{1}{A \eor B}
	\hypo{2}{\enot A}
	\hypo{3}{\enot B} \by{Want: C}{}
	\have{4}{B} \oe{1, 2}
	\have{5}{B \eor C} \oi{4}
	\have{6}{C} \oe{3, 5}
	\end{proof}
}
\end{enumerate}

%%%%PART F

\noindent\problempart Derive the following.
\begin{enumerate}[label=(\arabic*)]

\item $\{A \eand B, B \eif C\} \sststile{}{} A \eand (B \eand C) $ %1

%\begin{proof}
%\hypo{1}{A \eand B}
%\hypo{2}{B \eif C}	\by{Want: A \eand (B \eand C)}{}
%\have{3}{A} \ae{1}
%\have{4}{B} \ae{1}
%\have{5}{C} \ce{2, 4}
%\have{6}{B \eand C} \ai{4, 5}
%\have{7}{A \eand (B \eand C)} \ai{3, 6}
%\end{proof}

\item $\{(P \eor R) \eand (S \eor R), \enot R \eand Q\} \sststile{}{} P \eand (Q \eor R)$		%2
\item $\{(X \eand Y) \eif Z, X \eand W, W \eif Y\} \sststile{}{} Z$ 	%3

%\begin{proof}
%\hypo{1}{(X \eand Y) \eif Z}
%\hypo{2}{X \eand W}
%\hypo{3}{W \eif Y} \by{Want: Z}{}
%\have{4}{X} \ae{2}
%\have{5}{W} \ae{2}
%\have{6}{Y} \ce{3, 5}
%\have{7}{X \eand Y} \ai{4, 6}
%\have{8}{Z} \ce{1, 7}
%\end{proof}

\item $\{A \eor  (B \eor  G), A \eor  (B \eor  H), \enot A \eand \enot B\} \sststile{}{} G \eand H $		%4

\item $\{P \eand (Q \eand \enot R), R \eor T\} \sststile{}{} T \eor S$ 		%requires \eorI
\item $\{((A \eif D) \eor B) \eor C, \enot C, \enot B, A\} \sststile{}{} D$
\item $\{A \eor \enot\enot B, \enot B \eor \enot C, C \eor A, \enot A\} \sststile{}{}D		$			%\eorIE trick
\end{enumerate}

%%%%%%%% PART G
\noindent\problempart Derive the following.
\begin{enumerate}[label=(\arabic*)]

\item $H \eand A \sststile{}{} A \eand H	$

\answer{
\begin{proof}
\hypo{1}{H \eand A} \by{Want: A \eand H}{}
\have{2}{H} \by{\eand E}{1}
\have{3}{A} \by{\eand E}{1}
\have{4}{A \eand H} \by{\eand I}{2, 3}
\end{proof}
}

\item $\{{P \eor Q, D \eif E, \enot P \eand D} \} \sststile{}{} E \eand Q$

\answer{
\begin{proof}
\hypo{1}{P \eor Q}
\hypo{2}{D \eif E}
\hypo{3}{~P \eand D}  \by{Want: E \eand Q}{}
\have{4}{~P} \by{\eand E}{3}
\have{5}{D} \by{\eand E 3}{}
\have{6}{Q} \by{\eor E}{1, 4}
\have{7}{E	} \by{\eif E}{2, 5}
\have{8}{E \eand Q} \by{\eand E}{6, 7}
\end{proof}
}

\item $\{\enot A \eif (A \eor \enot C), \enot A, \enot C \eiff D \} \sststile{}{} D$

\answer{
\begin{proof}
\hypo{1}{~A \eif (A \eor ~C)}
\hypo{2}{~A}
\hypo{3}{~C \eiff D} \by{Want: D}{}
\have{4}{A \eor ~C} \by{\eif E}{1, 2}
\have{5}{~C} \by{\eor E} {2, 4}
\have{6}{D} \by{\eiff E}{3, 5}
\end{proof}
}

\item $\{\enot A \eand C, A \eor B, (B \eand C) \eif (D \eand E) \} \sststile{}{} D$

\answer{
\begin{proof}
\hypo{1}{~A \eand C}
\hypo{2}{A \eor B}
\hypo{3}{(B \eand C) \eif (D \eand E)} \by{Want: D}{}
\have{4}{~A} \by{\eand E}{1}
\have{5}{C} \by{\eand E}{1}
\have{6}{B} \by{\eor E}{2, 4}
\have{7}{B \eand C} \by{\eand E}{5, 6}
\have{8}{D \eand E} \by{\eif E}{3, 7}
\have{9}{D}\by{\eand E}{8}
\end{proof}
}

\item $\{A \eif (B \eif (C \eif D)), A \eand (B \eand C) \} \sststile{}{} D$

\answer{
\begin{proof}
\hypo{1}{A \eif (B \eif (C \eif D))}
\hypo{2}{A \eand (B \eand C)} \by{Want: D}{}
\have{3}{A} \by{\eand E}{2}
\have{4}{B \eand C} \by{\eand E}{2}
\have{5}{B} \by{\eand E}{4}
\have{6}{C} \by{\eand E}{4}
\have{7}{B \eif (C \eif D)} \by{\eif E}{1, 3}
\have{8}{C \eif D} \by{\eif E}{5, 7}
\have{9}{D} \by{\eif E}{6, 8}
\end{proof}
}


\item $\{E \eor F, F \eor G, \enot F\} \sststile{}{} E \eand G$

\answer{
\begin{proof}
\hypo{1}{E \eor F}
\hypo{2}{F\eor G}
\hypo{3}{\enot F} \by{Want: $E \eand G$}{}
\have{4}{E} \by{\eor E}{1, 3}
\have{5}{G} \by{\eor E}{2, 3}
\have{6}{E \eand G} \by{\eand I}{4, 5}
\end{proof}
}


\item $\{X \eand (Z \eor Y), \enot Z, Y \eif \enot X\} \sststile{}{} A$  %\eorIE trick

\answer{
\begin{proof}
\hypo{1}{X \eand (Z \eor Y)}
\hypo{2}{\enot Z}
\hypo{3}{Y \eif \enot X} \by{Want: $A$}{}
\have{4}{X} \by{\eand E}{1}
\have{5}{Z \eor Y} \by{\eand E}{1}
\have{6}{Y} \by{\eor E}{2, 5}
\have{7}{\enot X} \by{\eif E}{3, 6}
\have{8}{X \eor A} \by{\eor I}{4}
\have{9}{A} \by{\eor E}{7, 8}
\end{proof}
}

\end{enumerate}

%%%%%%PART H


\noindent\problempart Derive the following.
\begin{enumerate}[label=(\arabic*)]

\item $\{P \eiff (Q \eiff R)$,$ P$,$ P \eif R\} \sststile{}{} Q$

\item $\{A \eif (B \eif C), A, B\} \sststile{}{}C$
\item $\{(X \eor A) \eif \enot Y, Y \eor (Z \eand Q), X\} \sststile{}{}Z	$
\item $\{A \eand (B \eand C), A \eand D, B \eand E\} \sststile{}{}D \eand (E \eand C)		$
\item $\{A \eand (B \eor \enot C), \enot B \eand (C \eor E), E \eif D \} \sststile{}{} D$

%1.             A & (B ˅ ~C)
%2.             ~B & (C ˅ E)
%3.             E → D                                    Want: D
%4.             A                                                             &-E 1      
%5.             B ˅ ~C                                   &-E 1
%6.             ~B                                                           &-E 2
%7.             C ˅ E                                      &-E 2
%8              ~C                                                            ˅E 5, 6
%9.             E                                                               ˅E 7, 8
%10.           D                                                             →E 3,10

\item $\{A \eif B, B \eif C, C \eif A, B, \enot A\} \sststile{}{}D	$  %\eorIE trick
\item $\{\enot A \eand B, A \eor P, A \eor Q, B \eif R \} \sststile{}{} P \eand (Q\eand R)$


\end{enumerate}

\noindent\problempart Translate the following arguments into SL and then show that they are valid. Be sure to write out your dictionary. 
\begin{enumerate}[label=(\arabic*)]
\item If Professor Plum did it, he did it with the rope in the kitchen. Either Professor Plum or Miss Scarlett did it, and it wasn't Miss Scarlett. Therefore the murder was in the kitchen.  
%rob: note to self: problem 1 taken from test 4 SP08.

\answer{
A: Professor Plum did it \\
B: The murder was committed in the kitchen \\
C: The murder was committed with the rope\\
D: Miss Scarlett did it


\begin{proof}
\hypo{1}{A \eif (B \eand C)}
\hypo{2}{(A \eor D) \eand \enot D} \by{Want: B}{}
\have{3}{A \eor D} \ae{2}
\have{4}{\enot D} \ae{2}
\have{5}{A} \oe{3, 4}
\have{6}{B \eand C} \ce{1, 5}
\have{7}{B} \ae{6}
\end{proof}
}

\item If you are going to replace the bathtub, you might as well redo the whole bathroom. If you redo the whole bathroom, you will have to replace all the plumbing on the north side of the house. You will spend a lot of money on this project if and only if you replace the plumbing on the north side of the house. You are definitely going to replace the bathtub. Therefore you will spend a lot of money on this project. 

\answer{
A: You are going to replace the bathtub \\
B: You redo the whole bathroom.  \\
C: You replace all the plumbing on the north side of the house. \\
D: You will spend a lot of money on this project  \\ 


\begin{proof}
\hypo{1}{A \eif B}
\hypo{2}{B \eif C}
\hypo{3}{C \eiff D}
\hypo{4}{A} \by{Want: D}{}
\have{5}{B} \by{\eif E}{1, 4}
\have{6}{C} \by{\eif E}{2, 5}
\have{7}{D}  \by{\eiff E}{3, 6}
\end{proof}
}

\end{enumerate}

\noindent\problempart
Translate the following arguments into SL and then show that they are valid. Be sure to write out your dictionary. 
\begin{enumerate}[label=(\arabic*)]

\item Either Caroline is happy, or Joey is happy, but not both. If Joey teases Caroline, she is not happy. Joey is teasing Caroline. Therefore, Joey is happy.

%A: Caroline is happy. \hspace{.25in}
%B: Joey is happy.\hspace{.25in}
%C: Joey teases Caroline. \\
%
%\begin{proof}
%\hypo{1}{(A \eor B) \eand \enot(A \eand B)}
%\hypo{2}{C \eif \enot A}
%\hypo{3}{C} \by{Want: B}{}
%\have{4}{A \eor B} \ae{1}
%\have{5}{\enot A} \ce{2, 3}
%\have{6}{B} \oe{4,5}
%\end{proof}

\item Either grass is green or one of two other things: the sky is blue or snow is white. If my lawn is brown, the sky is gray, and if the sky is gray, it is not blue. If my lawn is brown, then grass is not green, and on top of that my lawn is brown. Therefore snow is white.
%rob: note to self: replace this with a better problem sometime.

\end{enumerate}

% *******************************************
% *			Conditional Proof					   *	
% *******************************************
\section{Conditional Proof}
\label{sec:conditional_proof}
\setlength{\parindent}{1em}
%I separated this out from the first section. 

So far we have introduced introduction and elimination rules for the conjunction and disjunction, and elimination rules for the conditional and biconditional, but we have no introduction rules for conditionals and biconditionals, and no rules at all for negations. That's because these other rules require fancy kinds of derivations that involve putting proofs inside proofs. In this section, we will look at one of these kinds of proof, called conditional proof.

%rob: added a transition paragraph.

\subsection{Conditional introduction}
Consider this argument:
\begin{earg*}
\item $R \eor F$
\itemc[.15] $\enot R \eif F$
\end{earg*}
The argument is valid. You can use the truth table to check it. Unfortunately, we don't have a way to prove it in our syntactic system of derivation. To help us see what our rule for 
conditional introduction should be, we can try to figure out what new rule would let us prove this obviously true argument.

Let's start the proof in the usual way, like this:

\begin{proof}
	\hypo{rf}{R \eor F} \by{Want: \enot R \eif F}{}
\end{proof}

If we had $\enot R$ as a further premise, we could derive $F$ by the {\eor}E rule. But sadly, we do not have $\enot R$ as a premise, and we can't derive it directly from the premise we do have---so we cannot simply prove $F$. What we will do instead is start a \emph{subproof}, a proof within the main proof. When we start a subproof, we draw another vertical line to indicate that we are no longer in the main proof. Then we write in an assumption for the subproof. This can be anything we want. Here, it will be helpful to assume $\enot R$. Our proof now looks like this:

\begin{proof}
	\hypo{rf}{R \eor F}\by{Want: \enot R \eif F}{}
	\open
		\hypo{nr}{\enot R}\by{Assumption for CD, Want: F}{}
	\close
\end{proof}

It is important to notice that we are not claiming to have proved $\enot R$. We do not need to write in any justification for the assumption line of a subproof. You can think of the subproof as posing the question: What could we show \emph{if} $\enot R$ were true? For one thing, we can derive $F$. To make this completely clear, I have annotated line 2 ``Assumption for CD,'' to indicate that this is an additional assumption we are making because we are using conditional derivation (CD). I have also added ``Want: F'' because that is what we will want to show during the subderivation. In the future I won't always include all this information in the annotation. But for now we will use it to be completely clear on what we will be doing.

So now let's go ahead and show F in the subderivation. 

\begin{proof}
	\hypo{rf}{R \eor F}\by{Want: \enot R \eif F}{}
	\open
		\hypo{nr}{\enot R}\by{Assumption for CD, Want: F}{}
		\have{f}{F}\oe{rf, nr}
	\close
\end{proof}

This has shown that \emph{if} we had $\enot R$ as a premise, \emph{then} we could prove $F$. In effect, we have proven $\enot R \eif F$. So the 
conditional introduction rule ({\eif}I) will allow us to close the subproof and derive $\enot R \eif F$ in the main proof. Our final proof looks like this:

\begin{proof}
	\hypo{rf}{R \eor F}\by{Want: \enot R \eif F}{}
	\open
		\hypo{nr}{\enot R}\by{Assumption for CD, Want: F}{}
		\have{f}{F}\oe{rf, nr}
	\close
	\have{nrf}{\enot R \eif F}\ci{nr-f}
\end{proof}

Notice that the justification for applying the {\eif}I rule is the entire subproof. Usually that will be more than just two lines.

%rob, I added a paragraph explaining the precise rules that govern subproofs and folded some later material into that paragraph

Now that we have that example, let's lay out more precisely the rules for subproofs and then give the formal schemes for the rule of conditional and biconditional introduction. 

\begin{enumerate}[leftmargin=1.5cm]
\item[\define{Rule 1}] You can start a subproof on any line, except the last one, and introduce any assumptions with that subproof.
\item[\define{Rule 2}] All subproofs must be closed by the time the proof is over.
\item[\define{Rule 3}] Subproofs may closed at any time. Once closed, they can be used to justify \eif I, \eiff I, \enot E, and \enot I.
\item[\define{Rule 4}] Nested subproofs must be closed before the outer subproof is closed.
\item[\define{Rule 5}] Once the subproof is closed, lines in the subproof cannot be used in later justifications.
\end{enumerate}

Rule 1 gives you great power. You can assume anything you want, at any time. But with great power, comes great responsibility, and rules 2--5 explain what your responsibilities are. Making an assumption creates the burden of starting a subproof, and subproofs must end before the proof is done. (That's why we can't start a subproof on the last line.) Closing a subproof is called \emph{discharging} the assumptions of that subproof. So we can summarize your responsibilities this way: You cannot complete a proof until you have discharged all of the assumptions introduced in subproofs. Once the assumptions are discharged, you can use the whole subproof as a justification, but not the individual lines. So you need to know going into the subproof what you are going to use it for once you get out. As in so many parts of life, you need an exit strategy.  

With those rules for subproofs in mind, the {\eif}I rule looks like this:

\begin{proof}
	\open
		\hypo[m]{a}{\script{A}} \by{want \script{B}}{}
		\have[n]{b}{\script{B}}
	\close
	\have[\ ]{ab}{\script{A}\eif\script{B}}\ci{a-b}
\end{proof}

You still might think this gives us too much power. In logic, the ultimate sign you have too much power is that given any premise \script{A} you can prove any conclusion \script{B}. Fortunately, our rules for subproofs don't let us do this. Imagine a proof that looks like this:

\begin{proof}
	\hypo{a}{\script{A}}
	\open
		\hypo{b1}{\script{B}}
\end{proof}

It may seem as if a proof like this will let you reach any conclusion \script{B} from any premise \script{A}. But this is not the case. By rule 2, in order to complete a proof, you must close all of the subproofs, and we haven't done that. A subproof is only closed when the vertical line for that subproof ends. To put it another way, you  can't end a proof and still have two vertical lines going. 

You still might think this system gives you too much power. Maybe we can try closing the subproof and writing \script{B} in the main proof, like this 

\begin{proof}
	\hypo{a}{\script{A}}
	\open
		\hypo{b1}{\script{B}}
		\have{b2}{\script{B}} \by{R}{b1}
	\close
	\have{b}{\script B} \by{R}{b2}
\end{proof}

But this is wrong, too. By rule 5, once you close a subproof, you cannot refer back to individual lines inside it.

Of course, it is legitimate to do this:

\begin{proof}
	\hypo{a}{\script{A}}
	\open
		\hypo{b1}{\script{B}}
		\have{b2}{\script{B}} \by{R}{b1}
	\close
	\have{bb}{\script{B}\eif\script{B}} \ci{b1-b2}
\end{proof}

This should not seem so strange, though. Since \script{B}\eif\script{B} is a tautology, no particular premises should be required to validly derive it. (Indeed, as we will see, a tautology follows from any premises.)

When we introduce a subproof, we typically write what we want to derive in the right column, just like we did in the first example in this section. This is just so that we do not forget why we started the subproof if it goes on for five or ten lines. There is no ``want'' rule. It is a note to ourselves and not formally part of the proof.

Having an exit strategy when you launch a subproof is crucial. Even if you discharge an assumption properly, you might wind up with a final line that doesn't do you any good. In order to derive a conditional by {\eif}I, for instance, you must assume the antecedent of the conditional in a subproof. The last line of the subproof must be the consequent of the conditional, and the whole conditional is the first line after the end of the subproof. Pick your assumptions so that you wind up with a conditional that you actually need. It is always permissible to close a subproof and discharge its assumptions, but it will not be helpful to do so until you get what you want.

%This is also moved from the conditional section

Now that we have the rule for conditional introduction, consider this argument:
\label{proofHS}
\begin{earg*}
\item $P \eif Q$
\item $Q \eif R$
\itemc[.15] $P \eif R$
\end{earg*}
We begin the proof by writing the two premises as assumptions. Since the main logical operator in the conclusion is a conditional, we can expect to use the {\eif}I rule. For that, we need a subproof---so we write in the antecedent of the conditional as an assumption of a subproof:

\begin{proof}
	\hypo{pq}{P \eif Q}
	\hypo{qr}{Q \eif R}
	\open
		\hypo{p}{P}
	\close
\end{proof}

We made $P$ available by assuming it in a subproof, allowing us to use {\eif}E on the first premise. This gives us $Q$, which allows us to use {\eif}E on the second premise. Having derived  $R$, we close the subproof. By assuming $P$ we were able to prove $R$, so we apply the {\eif}I rule and finish the proof.

\label{HSproof}
\begin{proof}
	\hypo{pq}{P \eif Q}
	\hypo{qr}{Q \eif R}
	\open
		\hypo{p}{P}\by{want $R$}{}
		\have{q}{Q}\ce{pq,p}
		\have{r}{R}\ce{qr,q}
	\close
	\have{pr}{P \eif R}\ci{p-r}
\end{proof}


\subsection{Biconditional introduction}

Just as the rule for biconditional elimination was a double-headed version of conditional elimination, our rule for biconditional introduction is a double-headed version of conditional introduction. In order to derive $W \eiff X$, for instance, you must be able to prove $X$ by assuming $W$ \emph{and} prove $W$ by assuming $X$. The biconditional introduction rule ({\eiff}I) requires two subproofs. The subproofs can come in any order, and the second subproof does not need to come immediately after the first---but schematically, the rule works like this:

\begin{proof}
	\open
		\hypo[m]{a1}{\script{A}} \by{want \script{B}}{}
		\have[n]{b1}{\script{B}}
	\close
	\open
		\hypo[p]{b2}{\script{B}} \by{want \script{A}}{}
		\have[q]{a2}{\script{A}}
	\close
	\have[\ ]{ab}{\script{A}\eiff\script{B}}\bi{a1-b1,b2-a2}
\end{proof}

We will call any proof that uses subproofs and either \eif I or \eiff I \define{conditional proof}. By contrast, the first kind of proof you learned, where you only use the six basic 
rules, will be called \define{direct proof}. In section \ref{sec:indirect_proof} we will learn the third and final kind of proof \emph{indirect proof}. But for now you should practice 
conditional proof.

%%%%%%Practice Problems %%%%%%%%%%%%%%%
\practiceproblems
\noindent\problempart Fill in the blanks in the following proofs. Be sure to include the ``Want'' line for each subproof.  %!@#$

\begin{exercises}
\item  \textcolor{white}{.} % $\{\enot P \eif (Q \eor R), P \eor \enot Q\} \sststile{}{} \enot P \eif R$
\vspace{-16pt}
\begin{proof}
\hypo{1}{\enot P \eif (Q \eor R)}
\hypo{2}{P \eor \enot Q}  \by{Want: \enot P \eif R}{}
\open
\hypo{3}{\enot P} \by{Want: \iflabelexists{showanswers}{\color{red} R}{}}{} 
\have{4}{Q \eor R}   \iflabelexists{showanswers}{\by{\color{red} \eif E}{1, 3}}{}
\have{5}{\iflabelexists{showanswers}{\color{red}\enot Q}{}} \oe{2, 3}
\have{6}{R}  \iflabelexists{showanswers}{\by{\color{red} \eor E}{2, 3}}{}
\close
\have{7}{\enot P \eif R} \iflabelexists{showanswers}{\by{\color{red} \eif I}{3-6}}{}
\end{proof}

%\begin{proof}
%\hypo{1}{\enot P \eif (Q \eor R)}
%\hypo{2}{P \eor \enot Q}  \by{Want: \enot P \eif R}{}
%\open
%\hypo{3}{\enot P} \by{Want: R}{}
%\have{4}{Q \eor R} \ce{1, 3}
%\have{5}{\enot Q} \oe{2, 3}
%\have{6}{R} \oe{4,5}
%\close
%\have{7}{\enot P \eif R} \ci{3-6}
%\end{proof}

\item  \textcolor{white}{.} \\ % $\{\enot P \eif (Q \eor R), P \eor \enot Q\} \sststile{}{} \enot P \eif R$
\vspace{-16pt}
\begin{proof}
\hypo{1}{A \eor B}
\hypo{2}{B \eif (B \eif \enot A)} \by{Want: \enot A \eiff B}{}
	\open
	\hypo{3}{\enot A} \by{Want: \iflabelexists{showanswers}{\color{red}B}{}}{}
	\have{4}{\iflabelexists{showanswers}{\color{red}B}{}} \by{\eor E}{1, 3}
	\close
	\open
	\hypo{5}{B} \by{Want: \iflabelexists{showanswers}{\color{red}\enot A}{}}{}
	\have{6}{B \eif \enot A} \iflabelexists{showanswers}{ \by{\color{red}\eif E}{2, 3}}{} % 
	\have{7}{\iflabelexists{showanswers}{\color{red}\enot A}{}} \by{\eif E}{5, 6}
	\close
\have{8}{\enot A \eiff B} \iflabelexists{showanswers}{\by{\color{red}\eiff I}{3-4, 5-7}}{}
\end{proof}
\end{exercises}

 
\noindent\problempart Fill in the blanks in the following proofs. Be sure to include the ``Want'' line for each subproof. 

\begin{exercises}

\item \textcolor{white}{.} \\ % $\{\enot P \eif (Q \eor R), P \eor \enot Q\} \sststile{}{} \enot P \eif R$
\vspace{-16pt}
\begin{proof}
\hypo{1}{B \eif \enot D}
\hypo{2}{A \eif (D \eor C)}  \by{Want: $A \eif (B \eif C)$}{}
\open
\hypo{3}{A} \by{ }{}
\open 
\hypo{4}{} \by{Want: C}{}
\have{5}{} \ce{2, 3}
\have{6}{} \ce{1, 4}
\have{7}{} \oe{5, 6}
\close
\have{8}{} \ci{4-7}
\close
\have{9}{} \ci{3-8}
\end{proof}

%\item $B \eif \enot D, A \eif (D \eor C) $\therefore$ A \eif (B \eif C)$
%\begin{proof}
%\hypo{1}{B \eif \enot D}
%\hypo{2}{A \eif (D \eor C)}  \by{Want: A \eif (B \eif C)}{}
%\open
%\hypo{3}{A} \by{Want: B \eif C}{}
%\open 
%\hypo{4}{B} \by{Want: C}{}
%\have{5}{D \eor C} \ce{2, 3}
%\have{6}{\enot D} \ce{1, 4}
%\have{7}{C} \oe{5, 6}
%\close
%\have{8}{B \eif C} \ci{4-7}
%\close
%\have{9}{A \eif (B \eif C)} \ce{3-8}
%\end{proof}


\item \textcolor{white}{.} \\ 
\vspace{-16pt}

\begin{proof}
\hypo{1}{(G \eor H) \eif (S \eand T)}
\hypo{2}{(T \eor U) \eif (C \eand D)}	\by{Want: $G \eif C$}{}
	\open
	\hypo{3}{\nix{G}} \by{Want: C}{}
	\have{4}{G \eor H} \nix{\by{\eor I}{3}}
	\have{5}{\nix{S \eor T}} \by{\eif E}{1, 4}
	\have{6}{T} \nix{\by{\eand E}{5}}
	\have{7}{\nix{T \eor U}} \by{\eor I}{6}
	\have{8}{C \eand D} \nix{\by{\eif E}{2, 7}}
	\have{9}{\nix{C}}	\by{\eand E}{8}
	\close
\have{10}{\nix{G \eif C}} \by{\eif I}{3-9}
\end{proof}
\end{exercises}

\noindent\problempart Derive the following 
\begin{enumerate}[label=(\arabic*)]

\item	$\{S \eor Q, Q \eif P \}\sststile{}{} \enot S \eif P  $ %Basic conditional									

\answer{
\begin{proof}
\hypo{1}{S \eor Q}
\hypo{2}{Q \eif P} \by{Want: $\enot S \eif P$}{}
\open
\hypo{3}{\enot S} \by{Want: P}{}
\have{4}{Q} \oe{1, 3}
\have{5}{P} \ce{2, 4}
\close
\have{6}{\enot S \eif P} \by{\eif E}{3-5}
\end{proof}
}


\item	$\{A \eif C, B \eif D\}\sststile{}{}  (A \eand B) \eif (C \eand D)$ %Basic conditional					


\answer{
\begin{proof}
\hypo{1}{A \eif C}	
\hypo{2}{B \eif D} \by{Want: $(A \eand B) \eif (C \eand D)$}{}
\open
\hypo{3}{A \eand B} \by{Want: C \eand D}{}
\have{4}{A} \ae{3}
\have{5}{B} \ae{3}
\have{6}{C} \ce{1, 4}
\have{7}{D} \ce{2, 5}
\have{8}{C \eand D} \ai{6, 7}
\close
\have{9}{(A \eand B) \eif (C \eand D)} \by{\eif I}{3-8}
\end{proof}
}


\item $\{K\eand L\} \sststile{}{} K\eiff L$ %Basic biconditional
%originally Chapter 6, part B, number 1
\answer{
\begin{proof}
\hypo{1}{K \eand L} \by{want: $K \eiff L$}{}
\open
\hypo{2}{K} \by{Want: L}{}
\have{3}{L} \ae{1}
\close
\open
\hypo{4}{L} \by{Want: K}{}
\have{5}{K} \ae{1}
\close
\have{6}{K \eiff L} \by{\eiff E}{4-5, 6-7}
\end{proof}
}

\item $\{A\eif (B\eif C)\} \sststile{}{} (A\eand B)\eif C$ % Basic conditional, tempted to do the wrong thing
%originally Chapter 6, part B, number 2

\answer{
\begin{proof}
\hypo{1}{A\eif (B\eif C)} \by{$(A\eand B)\eif C$}{}
\open
\hypo{2}{A \eand B} \by{Want: C}{}
\have{3}{A} \ae{2}
\have{4}{B} \ae{2}
\have{5}{B \eif C} \ce{1, 3}
\have{6}{C} \ce{4, 5}
\close
\have{7}{(A \eand B) \eif C} \by{\eif I}{2--6}
\end{proof}
}

\item $\{A\eiff B, B\eiff C\} \sststile{}{} A\eiff C$ %Basic biconditional
%originally Chapter 6, part C, number 4

\answer{
\begin{proof}
\hypo{1}{A \eiff B}
\hypo{2}{B \eiff C} \by{Want: $A \eiff C$}{}
\open
\hypo{3}{A} \by{Want: C}{}
\have{4}{B} \by{\eiff E}{1, 3}
\have{5}{C} \by{\eiff E}{2, 4}
\close
\open
\hypo{6}{C} \by{Want: A}{}
\have{7}{B} \by{\eiff E}{2, 6}
\have{8}{A} \by{\eiff E}{1, 7}
\close
\have{9}{A \eiff C} \by{\eiff I}{3--5, 6--8}
\end{proof}
}

\item $\{P \eif (Q \eif R)\} \sststile{}{} Q \eif (P \eif R)$ %two subproofs

\answer{
\begin{proof}
\hypo{1}{P \eif (Q \eif R)} \by{Want: $Q \eif (P \eif R)$}{}
\open
\hypo{2}{Q} \by{Want: $P \eif R$}{}
\open
\hypo{3}{P} \by{Want: $R$}{}
\have{4}{Q \eif R} \ce{1, 3}
\have{5}{R} \ce{2, 4}
\close
\have{6}{P \eif R} \by{\eif I}{3-5}
\close 
\have{7}{Q \eif (P \eif R)} \by{\eif I}{2-6}
\end{proof}
}


\item $\{\enot A, (B \eand C) \eif D\} \sststile{}{}(A \eor B) \eif (C \eif D)$ %two subproofs  Modified from KMR T107 p. 82.

\answer{
\begin{proof}

\hypo{1}{\enot A}
\hypo{2}{(B \eand C) \eif D} \by{Want: (A \eor B) \eif (C \eif D}{}
\open
\hypo{3}{A \eor B} \by{Want: C \eif D}{}
\open
\hypo{4}{C}  \by{Want: D}{}
\have{5}{B} \oe{1, 4}
\have{6}{B \eand C} \ai{4, 5}
\have{7}{D} \ce{2, 6}
\close
\have{8}{C \eif D} \ci{4-7}
\close
\have{9}{(A \eor B) \eif (C \eif D)} \ci{3-8}
\end{proof}
}





\end{enumerate}	

\noindent\problempart Derive the following 
\begin{enumerate}[label=(\arabic*)]

\item $\{X \eiff (A \eand B), B \eiff Y, B \eif A\} \sststile{}{} X \eiff Y$

\answer{
\begin{proof}
\hypo{1}{X \eiff (A \eand B)}
\hypo{2}{B \eiff Y}
\hypo{3}{B \eif A} \by{Want: X \eiff Y}{} 
\open
\hypo{4}{X} \by{Want: Y}{}
\have{5}{A \eand B} \by{\eiff E}{1, 4}
\have{6}{B} \ae{5}
\have{7}{Y} \by{\eiff E}{2, 6}
\close
\open
\hypo{8}{Y} \by{Want: X}{}
\have{9}{B} \by{\eiff E}{2, 8}
\have{10}{A} \ce{3, 9}
\have{11}{A \eand B} \ai{9, 10}
\have{12}{X} \by{\eiff E}{1, 12}
\close
\have{13}{X \eiff Y} \by{\eiff I}{4-7, 8-12} 
\end{proof}
}

\item $\{B \eif \enot E, A \eif \enot D, D \eor (E \eor R), (R \eand A) \eif C\} \sststile{}{} A \eif (B \eif C)$ 

\answer{
\begin{proof}
\hypo{}{B \eif \enot E}
\hypo{}{A \eif \enot D}
\hypo{}{D \eor (E \eor R)}
\hypo{}{(R \eand A) \eif C}	\by{Want: A \eif (B \eif C)}{} 
\open
\hypo{}{A}	\by{Want: B \eif C}{}
\open
\hypo{}{B}	\by{Want: C}{}
\have{}{\enot E} \by{	\eif E 1, 6}{}
\have{}{\enot D} \by{\eif E 2, 5}{}
\have{}{E \eor R} \by{\eor E 3, 7}{}
\have{}{R} \by{\eor E 7, 9}{}
\have{}{R \eand} \by{\eand I 5, 10}{}
\have{}{C	} \by{\eif E 4, 11}{}
\close
\have{}{B \eif C} \by{\eif I 6-12}{}
\close
\have{}{A \eif (B \eif C)} \by{\eif I 5-13}{}
\end{proof}
}


\item $\{\enot W \eand \enot E, Q \eiff D\} \sststile{}{} (W \eor Q) \eiff (E \eor D)$

\answer{
\begin{proof}
\hypo{1}{\enot W \eand \eand E} 
\hypo{2}{Q \eiff D} \by{Want: (W \eor Q) \eiff (E \eor D)}{}
\have{3}{\enot W} \ae{1}
\have{4}{\enot E} \ae{1}
\open
\hypo{5}{W \eor Q} \by{Want: E \eor D}{}
\have{6}{Q} \oe{3, 5}
\have{7}{D} \by{\eiff E}{2, 6}
\have{8}{E \eor D} \oi{7}
\close
\open
\hypo{9}{E \eor D} \by{Want: W \eor Q}{}
\have{10}{D} \oe{4, 9}
\have{11}{Q} \by{\eiff E}{2, 10}
\have{12}{W \eor Q} \oi{11}
\close
\have{13}{(W \eor Q) \eiff (E \eor D)} \ci{5-8, 9-12}
\end{proof}
}


\item $\{(A \eand B) \eiff D, D \eiff (X \eand Y), C \eiff Z\} \sststile{}{} A \eand (B \eand C) \eiff X \eand (Y \eand Z)$ %long biconditional

\answer{
\begin{proof}
\hypo{1}{(A \eand B) \eiff D}
\hypo{2}{D \eiff (X \eand Y)}
\hypo{3}{C \eiff Z} \by{A \eand (B \eand C) \eiff X \eand (Y \eand Z)}{}
	\open
	\hypo{4}{A \eand (B \eand C)} \by{Want: X \eand (Y \eand Z)}{}
	\have{5}{A} \ae{4}
	\have{6}{B \eand C} \ae{4}
	\have{7}{B} \ae{6}
	\have{8}{C} \ae{6}
	\have{9}{Z} \by{\eiff-E}{3, 8}
	\have{10}{A \eand B} \ai{5, 7}
	\have{11}{D} \by{\eiff-E}{1, 10}
	\have{12}{X \eand Y}  \by{\eiff-E}{2, 11}
	\have{13}{X} \ae{12}
	\have{14}{Y} \ae{12}
	\have{15}{Y \eand Z} \ai{13, 14}
	\have{16}{X \eand (Y \eand Z)} \ai{13, 15}
	\close
	
	\open
	\hypo{17}{X \eand (Y \eand Z)} \by{Want: A \eand (B \eand C)}{} 
	\have{18}{X} \ae{17}
	\have{19}{Y \eand Z} \ae{17}
	\have{20}{Y} \ae{19}
	\have{21}{Z} \ae{19}
	\have{22}{C}  \by{\eiff-E}{3, 22}
	\have{23}{X \eand Y} \ai{18, 20}
	\have{24}{D}  \by{\eiff-E}{2, 23}
	\have{25}{A \eand B}  \by{\eiff-E}{1, 24}
	\have{26}{A} \ae{25}
	\have{27}{B} \ae{25}
	\have{28}{B \eand C} \ai{22, 27}
	\have{29}{A \eand (B \eand C)} \ai{26, 28}
	\close
\have{30}{A \eand (B \eand C) \eiff (X \eand (Y \eand Z)} \by{\eiff I, 4-16, 17-29}{}
\end{proof}
}


\end{enumerate}

%\noindent\problempart 
%Translate the following arguments in to SL and then show that they are valid. Be sure to write out your dictionary. 


% *******************************************
% *				Indirect Proof					   *	
% *******************************************

\section{Indirect Proof}
\label{sec:indirect_proof}

%signposting paragraph added

The last two rules we need to discuss are negation introduction (\enot I)  and negation elimination (\enot E). As with the rules of conditional and biconditional introduction, we have put off explaining the rules, because they require launching subproofs. In the case of negation introduction and elimination, these subproofs are designed to let us perform a special kind of derivation classically known as \emph{reductio ad absurdum}, or simply \emph{reductio}. 

%rob: changed the example to something less exact but more familiar
A \emph{reductio} in logic is a variation on a tactic we use in ordinary arguments all the time. In arguments we often stop to imagine, for a second, that what our opponent is saying is true, and then realize that it has unacceptable consequences. In so-called ``slippery slope'' arguments or ``arguments from consequences,'' we claim that doing one thing will will lead us to doing another thing which would be horrible. For instance, you might argue that legalizing physician assisted suicide for some patients might lead to the involuntary termination of lots of other sick people. These arguments are typically not very good, but they have a basic pattern whcih we can make rigorous  in our logical system. These arguments say ``if my opponent wins, all hell will break loose.'' In logic the equivalent of all hell breaking loose is asserting a contradiction. The worst thing you can do in logic is contradict yourself. The equivalent of our opponent being right in logic would be that a sentence we are trying to prove true turns out to be false (or alternately, that a sentence we are trying to prove false turns out to be true.) So, in developing the rules for reductio ad absurdum, we need to find a way to say ``if this sentence were false (or true), we would have to assert a contradiction.'' 

%The example from Magnus's version
%
%Here is a simple mathematical argument in English for the conclusion that there is no largest number:
%\begin{earg}
%\item[] Assume there \emph{is} some greatest natural number. Call it $A$.
%\item[] That number plus one is also a natural number.
%\item[] Obviously, $A+1 > A$.
%\item[] So there is a natural number greater than $A$.
%\item[] This is impossible, since $A$ is assumed to be the greatest natural number.
%\item[$\therefore$] There is no greatest natural number.
%\end{earg}
%This argument form is traditionally called a \emph{reductio}. Its full Latin name is \emph{reductio ad absurdum}, which means ``reduction to absurdity.'' In a reductio, we assume something for the sake of argument---for example, that there is a greatest natural number. Then we show that the assumption leads to two contradictory sentences---for example, that $A$ is the greatest natural number and that it is not. In this way, we show that the original assumption must have been false.

In our system of natural deduction, this kind of proof will be known as \define{indirect proof}.The basic rules for negation will allow for arguments like this. If we assume something and show that it leads to contradictory sentences, then we have proven the negation of the assumption. This is the negation introduction ({\enot}I) rule:

\begin{multicols}{2}

\begin{proof}
\open
	\hypo[m]{na}{\script{A}}\by{for reductio}{}
	\have[n]{b}{\script{B}}
	\have{nb}{\enot\script{B}}
\close
\have{a}[\ ]{\enot\script{A}}\ni{na-nb}
\end{proof}

\begin{proof}
\open
	\hypo[m]{na}{\script{A}}\by{for reductio}{}
	\have[n]{b}{\enot\script{B}}
	\have{nb}{\script{B}}
\close
\have{a}[\ ]{\enot\script{A}}\ni{na-nb}
\end{proof}

\end{multicols}


For the rule to apply, the last two lines of the subproof must be an explicit contradiction: either the second sentence is the direct negation of the first, or vice versa. We write ``for 
reductio'' as a note to ourselves, a reminder of why we started the subproof. It is not formally part of the proof, and you can leave it out if you find it distracting.

To see how the rule works, suppose we want to prove a law of double negation $A$ $\therefore$ $\enot \enot A$
\label{DN1}
%doublenegation

\begin{proof}
\hypo{1}{A} \by{Want: \enot \enot A}{}
	\open
	\hypo{2}{\enot A} \by{for reductio}{}
	\have{3}{A} \by{R}{1}
	\have{4}{\enot A} \by{R}{2}
	\close
\have{5}{\enot \enot A} \ni{2-4}
\end{proof}

The {\enot}E rule will work in much the same way. If we assume \enot\script{A} and show that it leads to a contradiction, we have effectively proven \script{A}. So the rule looks like this:

\begin{multicols}{2}
\begin{proof}
\open
	\hypo[m]{na}{\enot\script{A}}\by{for reductio}{}
	\have[n]{b}{\script{B}}
	\have{nb}{\enot\script{B}}
\close
\have{a}[\ ]{\script{A}}\ne{na-nb}
\end{proof}


\begin{proof}
\open
	\hypo[m]{na}{\enot\script{A}}\by{for reductio}{}
	\have[n]{b}{\enot\script{B}}
	\have{nb}{\script{B}}
\close
\have{a}[\ ]{\script{A}}\ne{na-nb}

\end{proof}
\end{multicols}

Here is a simple example of negation elimination at work. We can show $L \eiff \enot O, L \eor \enot O \therefore L$ by assuming \enot L, deriving a contradiction, and then using \enot E.

\begin{proof}
\hypo{1}{L \eiff \enot O}
\hypo{2}{L \eor \enot O}\by{Want: $L$}{}
\open
	\hypo{3}{\enot L}\by{for reductio}{}
	\have{4}{\enot O} \oe{2, 3}
	\have{5}{L} \be{1, 4}
	\have{6}{\enot L} \by{R}{3}
\close
\have{7}{L}\ne{3-6}
\end{proof}

With the addition of \enot E and \enot I, our system of natural deduction is complete. We can now prove that any valid argument is actually valid. This is really where the fun begins. 

One important bit of strategy. Sometimes, you will launch a subproof right away by assuming the negation of the conclusion to the whole argument. Other times, you will use a subproof to get a piece of the conclusion you want, or some stepping stone to the conclusion you want. Here's a simple example. Suppose you were asked to show that this argument is valid: $\enot(A \eor B) \therefore \enot A \eand \enot B$. (The argument, by the way, is part of DeMorgan's Laws, some very useful equivalences which we will see more of later on.)

You need to set up the proof like this.

\begin{proof}
\hypo{1}{\enot(A \eor B)} \by{Want $\enot A \eand \enot B$}{}
\end{proof}

Since you are trying to show $\enot A \eand \enot B$, you could open a subproof with $\enot(\enot A \eand \enot B$) and try to derive a contradiction, but there is an easier way to do things. Since you are tying to prove a conjunction, you can set out to prove each conjunct separately. Each conjunct, then, would get its own reductio. Let's start by assuming $A$ in order to show $\enot A$

\begin{proof}
\hypo{1}{\enot(A \eor B)} \by{Want $\enot A \eand \enot B$}{}
	\open
	\hypo{2}{A}	\by{for reductio}{}
	\have{3}{A \eor B} \oi{2}
	\have{4}{\enot (A \eor B)} \by{R}{1}
	\close
\have{5}{\enot A} \ni{2-4}
\end{proof}

%\pagebreak[4]

We can then finish the proof by showing $\enot B$ and putting it together with $\enot A$ and conjunction introduction. 

% Demorgan's negated disjunction to conjunction of negations.
\label{DeM1}
\begin{proof}
\hypo{1}{\enot(A \eor B)} \by{Want $\enot A \eand \enot B$}{}
	\open
	\hypo{2}{A}	\by{for reductio}{}
	\have{3}{A \eor B} \oi{2}
	\have{4}{\enot (A \eor B)} \by{R}{1}
	\close
\have{5}{\enot A} \ni{2-4}
	\open
	\hypo{6}{B}	\by{for reductio}{}
	\have{7}{A \eor B} \oi{6}
	\have{8}{\enot (A \eor B)} \by{R}{1}
	\close
\have{10}{\enot B} \ni{7-9}
\have{11}{\enot A \eand \enot B} \ai{6,10}
\end{proof}


%%%%%%%%Practice problems  %%%%%%%%%%%%%%%%%  !@#$ 
 	
\practiceproblems

\noindent\problempart Fill in the blanks in the following proofs.

\begin{multicols}{2}
\begin{enumerate}[label=(\arabic*)]

\item \textcolor{white}{.} \\ 
\vspace{-16pt}
%DeMorgans, conjunction of negations to negated disjunction
\label{DeM2}
\begin{proof}
\hypo{1}{\enot A \eand \enot B} \by{Want: \iflabelexists{showanswers}{\color{red}$\enot (A \eor B)$}{}}{}
\have{2}{\enot A} \iflabelexists{showanswers}{ \by{\color{red}\eand E}{1}}{}
\have{3}{\enot B} \iflabelexists{showanswers}{ \by{\color{red}\eand E}{1}}{}
\open
	\hypo{4}{A \eor B} \by{for reductio}{}
	\have{5}{B} \iflabelexists{showanswers}{ \by{\color{red}\eor E}{2, 4}}{}
	\have{6}{\enot B} \iflabelexists{showanswers}{ \by{\color{red}R}{3}}{}
\close
\have{7}{\enot(A \eor B)} \iflabelexists{showanswers}{\by{\color{red} \enot I}{4-6}}{}
\end{proof}

\vspace{5cm}

\item \textcolor{white}{.} \\ 
\vspace{-16pt}
%Demorgans disjunction of negations to a negated conjunction.
\label{DeM3}
\begin{proof}
\hypo{1}{\enot A \eor \enot B} \by{Want: $\enot(A \eand B)$}{}
	\open
	\hypo{2}{A \eand B} \by{for reductio}{}
	\have{3}{\iflabelexists{showanswers}{\color{red}A}{}} \ae{2}	
	\have{4}{\iflabelexists{showanswers}{\color{red}B}{}} \ae{2}
		\open
		\hypo{5}{\iflabelexists{showanswers}{\color{red} \enot A}{}}\by{for reductio}{}
		\have{6}{\iflabelexists{showanswers}{\color{red}A}{}} \by{R}{3}
		\have{7}{\iflabelexists{showanswers}{\color{red}\enot A}{}} \by{R}{5}
		\close
	\have{8}{\enot \enot A} \iflabelexists{showanswers}{ \by{\color{red}\enot I}{5-7}}{} 
	\have{9}{B} \iflabelexists{showanswers}{ \by{\color{red}R}{4}}{} 
	\have{10}{\enot B} \iflabelexists{showanswers}{ \by{\color{red}\eor E}{1, 8}}{} 
	\close
\have{11}{\enot(A \eand B)} \iflabelexists{showanswers}{ \by{\color{red}\enot I}{2-10}}{} 
\end{proof}

%Demorgans disjunction of negations to a negated conjunction.
%\begin{proof}
%\have{1}{\enot A \eor \enot B} \by{Want: \enot(A \eand B)}{}
%	\open
%	\hypo{2}{A \eand B} \by{for reductio}{}
%	\have{3}{A} \ae{2}	
%	\have{4}{B} \ae{2}
%		\open
%		\hypo{5}{\enot A} \by{for reductio}{}
%		\have{6}{A} \by{R}{3}
%		\have{7}{\enot A} \by{R}{5}
%		\close
%	\have{8}{\enot \enot A} \ni{5-7}
%	\have{9}{B} \by{R}{4}
%	\have{10}{\enot B} \oe{1, 8}
%	\close
%\have{11}{\enot(A \eand B} \ni{2-10}
%\end{proof}
\end{enumerate}
\end{multicols}

\noindent\problempart Fill in the blanks in the following proofs.

\begin{enumerate}[label=(\arabic*)]
\item \textcolor{white}{.}  
\vspace{-20pt} %$P \eif Q \therefore \enot P \eor Q$
%conditional disjunction, from conditional to disjunction
\begin{proof}
\hypo{1}{P \eif Q} \by{Want: $\enot P \eor Q$}{}
	\open
	\hypo{2}{\hspace{1cm}} \by{for reductio}{}
		\open
		\hypo{3}{\hspace{1cm}} \by{for reductio}{}
		\have{4}{} \ce{1, 3}
		\have{5}{} \oi{4}
		\have{6}{} \by{R}{2}
		\close
	\have{7}{\enot P} \ni{3-6}
	\have{8}{ } \oi{7}
	\have{9}{ } \by{R}{2}
	\close
\have{10}{\enot P \eor Q} \ne{2-9}			
\end{proof}

%\begin{proof}
%\hypo{1}{P \eif Q} \by{Want: \enot P \eor Q}{}
%	\open
%	\hypo{2}{\enot(\enot P \eor Q)} \by{for reductio}{}
%		\open
%		\hypo{3}{P} \by{for reductio}{}
%		\have{4}{Q} \ce{1, 3}
%		\have{5}{\enot P \eor Q} \oi{4}
%		\have{6}{\enot(\enot P \eor Q} \by{R}{2}
%		\close
%	\have{7}{\enot P} \ni{3-6}
%	\have{8}{\enot P \eor Q} \oi{7}
%	\have{9}{\enot(\enot P \eor Q)} \by{R}{2}
%	\close
%\have{10}{\enot P \eor Q} \ne{2-9}			
%\end{proof}


\item \textcolor{white}{.}  
\vspace{-18pt} %$(X\eand Y)\eor(X\eand Z)$, $\enot(X\eand D)$, $D\eor M$ $\therefore$ $M$

\begin{proof}
\hypo{1}{(X\eand Y)\eor(X\eand Z)}
\hypo{2}{\enot(X\eand D)}
\hypo{3}{D \eor M} \by{Want: M}{}
	\open
	\hypo{4}{\hspace{1cm}} \by{for reductio}{}
	\have{5}{D} \oe{ }
		\open
		\hypo{6}{\hspace{1cm}} \by{for reductio}{}
		\have{7}{\enot X \eor \enot Y} 
			\open
			\hypo{8}{\hspace{1cm}}	\by{for reductio}{}
			\have{9}{X}	
			\have{10}{\enot X}	
			\close
		\have{11}{\enot(X \eand Y)} \ni{8-10}
		\have{12}{} \oe{1, 11}
		\have{13}{} \ae{12}
		\have{14}{} \by{R}{6}
		\close
	\have{15}{X} 
	\have{16}{X \eand D} 
	\have{17}{\enot (X \eand D)} 
	\close
\have{18}{M} \ne{4-17}
\end{proof}

%\begin{proof}
%\have{1}{(X\eand Y)\eor(X\eand Z)}
%\have{2}{\enot(X\eand D)}
%\have{3}{D \eor M} \by{Want: M}{}
%	\open
%	\hypo{4}{\enot M} \by{for reductio}{}
%	\hypo{5}{D} \oe{3, 4}
%		\open
%		\hypo{6}{\enot X} \by{for reductio}{}
%		\have{7}{\enot X \eor \enot Y} \oi{6}
%			\open
%			\hypo{8}{X \eand Y}	\by{for reductio}{}
%			\have{9}{X}	\ae{8}
%			\have{10}{\enot X}	\by{R}{6}
%			\close
%		\have{11}{\enot(X \eand Y)} \ni{8-10}
%		\have{12}{X \eand Z} \oe{1, 11}
%		\have{13}{X} \ae{12}
%		\have{14}{\enot X} \by{R}{6}
%		\close
%	\have{15}{X} \ne{6-14}
%	\have{16}{X \eand D} \ai{5, 15}
%	\have{17}{\enot (X \eand D)} \by{R}{2}
%	\close
%\have{18}{M} \ne{4-17}
%\end{proof}

\end{enumerate}


%%%%%%%%%%%%%%%%%%%%%%%   Part C: Derive the following using indirect derivation %%%%%
\noindent\problempart Derive the following using indirect derivation. You may also have to use conditional derivation.
\begin{enumerate}[label=(\arabic*)]

\item $\enot \enot A  \sststile{}{}  A$
%Double negation removing negations
\label{DN2}

\answer{
\begin{proof}
\hypo{1}{\enot \enot A} \by{Want: $A$}{}
	\open
	\hypo{2}{\enot A} \by{for reductio}{}
	\have{3}{\enot \enot A} \by{R}{1}
	\have{4}{\enot A} \by{R}{2}
	\close
\have{5}{A} \ne{2-4}
\end{proof}
}

\item $\{A \eif B, \enot B\} \sststile{}{} \enot A$
%modus tollens
\label{ModusTollens}

\answer{
\begin{proof}
	\hypo{1}{A \eif B}
	\hypo{2}{\enot B} \by{Want: $\enot A$}{}
		\open
		\hypo{3}{A} \by{for reductio}{}
		\have{4}{B} \ce{1, 3}
		\have{5}{\enot B} \by{R}{2}
		\close
	\have{6}{\enot A} \ni{3-5}
\end{proof}
}

\item $A \eif (\enot B \eor \enot C) \sststile{}{} A \eif \enot (B \eand C)$
%
\answer{
\begin{proof}
\hypo{1}{A \eif (\enot B \eor \enot C)} \by{Want: $A \eif \enot (B \eand C)$}{}
	\open
	\hypo{2}{A}	\by{Want: \enot (B \eand C)}{}
	\have{3}{\enot B \eor \enot C} \ce{1,2}		
		\open
		\hypo{4}{B \eand C} \by{for reductio}{}
		\have{5}{B} \ae{3}
		\have{6}{C} \ae{3}
			\open
			\hypo{7}{\enot B} \by{for reductio}{}
			\have{8}{B} \by{R}{4}
			\have{9}{\enot B} \by{R}{6}
			\close
		\have{10}{\enot \enot B} \ni{6-8}
		\have{11}{\enot C} \oe{3, 10}
		\have{12}{C} \by{R}{5}
		\close
	\have{13}{\enot (B \eand C)} \ni{3-11}
	\close
\have{14}{A \eif \enot (B \eand C)} \ci{2-12} 
\end{proof}
}
\item $\enot(A \eand B) \sststile{}{} \enot A \eor \enot B$
%DeMorgan's negated conjunction to disjunction of negations

\answer{
\begin{proof}
\hypo{1}{\enot(A \eand B)} \by{Want: \enot A \eor \enot B}{}
	\open
	\hypo{2}{\enot(\enot A \eor \enot B)} \by{for reductio}{}
		\open
		\hypo{3}{\enot A} \by{for reductio}{}
		\have{4}{\enot A \eor  \enot B} \oi{3}
		\have{5}{\enot(\enot A \eor \enot B)} \by{R}{2}
		\close
	\have{6}{A} \ne{3-5}
		\open
		\hypo{7}{\enot B} \by{for reductio}{}
		\have{8}{\enot A \eor \enot B} \oi{7}
		\have{9}{\enot(\enot A \eor \enot B)} \by{R}{2}
		\close
	\have{10}{B} \ne{7-9}
	\have{11}{A \eand B} \ai{6, 10}
	\have{12}{\enot(A \eand B)} \by{R}{1}
	\close
\have{13}{\enot A \eor \enot B} \ne{2-12}
\end{proof}
}

\item $\{\enot F\eif G, F\eif H\} \sststile{}{} G\eor H$

\answer{
\begin{proof}
\hypo{1}{\enot F\eif G}
\hypo{2}{F\eif H} \by{Want: $G\eor H$}{}
	\open
	\hypo{3}{\enot (G \eor H)} \by{for reductio}{}
		\open
		\hypo{4}{F} \by{for reductio}{}
		\have{5}{H} \ce{2, 4}
		\have{6}{G \eor H} \oi{5}
		\have{7}{\enot(G \eor H)} \by{R}{3}
		\close
	\have{8}{\enot F} \ni{4-7}
	\have{9}{G} \oe{1,8}
	\have{10}{G \eor H} \oi{9}
	\have{11}{\enot (G \eor H)} \by{R}{3}
	\close
\have{12}{G \eor H} \ne{3-11}
\end{proof}
}

\item	$\{(T \eand K) \eor (C \eand E), E \eif \enot C\} \sststile{}{}  T \eand K$

\answer{
There are two solutions. In one, you look at the want line to figure out the assumption for the subproof. In the other, you think of another think you might want, and assume the negation of that.

\begin{proof}
\hypo{1.}{(T \eand K) \eor (C \eand E)}
\hypo{2.}{E \eif \enot C} 			\by{Want: T \eand K}{}
\open
\hypo{3.}{\enot (T \eand K)} \by{For Reductio}{}
\have{4.}{C \eand E}	 \by{\eor E}{1, 4}
\have{5.}{E} \by{\eand E}{4}
\have{6.}{C} \by{\eand E}{5}
\have{7.}{\enot C} \by{\enot E}{2, 6}
\close
\have{8.}{T \eand K}	\by{\enot E}{3-7}
\end{proof}

\begin{proof}
\hypo{1.}{(T \eand K) \eor (C \eand E)}
\hypo{2.}{E \eif \enot C} \by{Want: T \eand K}{}
\open
\have{3.}{C \eand  E}	\by{For Reductio}{}
\have{4.}{E}\by{\eand E}{3}
\have{5.}{C}\by{\eand E}{4}
\have{6.}{\enot C}\by{\eif E}{2, 4}
\close
\have{7.}{\enot (C \eand E)}\by{\enot I}{3-4}
\have{8.}{T \eand K}\by{\eor E}{1, 7}
\end{proof}
}

\item $\{(A \eif B) \}\sststile{}{} (A \eif \enot B) \eif \enot A$

\answer{
\begin{proof}
 \hypo{}{A \eif B} \by{Want: (A \eif \enot B) \eif \enot A}{}
\open
\hypo{}{A \eif \enot B} \by{Want: \enot A}{}
\open
\hypo{}{A} \by{Want: A contradiction}{}
\have{}{B} \by{\eif E}{1, 3}
\have{}{\enot B} \by{\eif E}{2, 3}
\close
\have{}{\enot A} \by{\enot I}{3-5}
\close
\have{}{(A \eif \enot B) \eif \enot A} \by{\eif I}{2-6}
\end{proof}
}

\end{enumerate}

%%%%%%%%%%%%%%%%%%%%%%%   Part D: Derive the following using indirect derivation %%%%%
\noindent\problempart Derive the following using indirect derivation. You may also have to use conditional derivation.
\label{derivation_set_with_const_d}
\begin{enumerate}[label=(\arabic*)]

\item $\{P \eif Q, P \eif \enot Q\} \sststile{}{} \enot P$

%\begin{proof}
%\hypo{1}{P \eif Q}
%\hypo{2}{P \eif \enot Q} \by{Want: \enot P}{}
%	\open
%	\hypo{3}{P} \by{for reductio}{}
%	\have{4}{Q} \ce{1, 3}
%	\have{5}{\enot Q} \ce{2, 3}
%	\close
%\have{6}{\enot P} \ni{3-5}
%\end{proof}

\item $(C\eand D)\eor E \sststile{}{} E\eor D$

%\begin{proof}
%\hypo{1}{(C\eand D)\eor E} \by{Want: $E \eor D$}{}
%	\open
%	\hypo{2}{\enot (E \eor D)} \by{for reductio}{}
%		\open
%		\hypo{3}{E} \by{for reductio}{}
%		\have{4}{E \eor D} \oi{3}
%		\have{5}{\enot (E \eor D)} \by{R}{2}
%		\close
%	\have{6}{\enot E} \ni{3-5}
%	\have{7}{C \eand D} \oe{1, 6}
%	\have{8}{D} \ae{7}
%	\have{9}{E \eor D} \oi{8}
%	\have{10}{\enot (E \eor D)}	\by{R}{2}
%	\close
%\have{11}{E \eor D} \ne{2-10}
%\end{proof}

\item $M\eor(N\eif M) \sststile{}{} \enot M \eif \enot N$ \label{DeM4}

%\begin{proof}
%\hypo{1}{M\eor(N\eif M)} \by{want: \enot M \eif \enot N}{}
%\open
%\hypo{2}{\enot M} \by{want: \enot N}{}
%\have{3}{N \eif M}
%\open
%\hypo{4}{N} \by{want: M and \enot M}{}
%\have{5}{M} \ce{3, 4}
%\have{6}{\enot M} \by{R}{2}
%\close
%\have{7}{\enot N} \by{\enot I}{4--6}
%\close
%\have{8}{\enot M \eif \enot N} \ci{2--7}
%\end{proof}



\item \label{itm:const_d} \{$A \eor B, A \eif C, B \eif C\} \sststile{}{} C$

%\begin{proof}
%\hypo{1}{A \eor B}
%\hypo{2}{A \eif C}
%\hypo{3}{B \eif C}  \by{Want: C}{}
%	\open
%	\hypo{4}{\enot C} \by{for reductio}{}
%		\open
%		\hypo{5}{\enot A} \by{for reductio}{}
%		\have{6}{B} \oe{1,5}
%		\have{7}{C} \ce{3, 6}
%		\have{8}{\enot C} \by{R}{7}
%		\close
%	\have{9}{A} \ne{5-8}
%	\have{10}{C} \ce{2, 9}
%	\have{11}{\enot C} \by{R}{4}
%	\close
%\have{12}{C} \ne{4-11}
%\end{proof}

\item	$A \eif (B \eor (C \eor D))  \sststile{}{} \enot[A \eand (\enot B \eand (\enot C \eand \enot D))] $

%1.	A → (B ˅ (C ˅ D)) 			Want: ~[A & (~B & (~C &~D))]
%2.		A & (~B & (~C &~D))	For reductio
%3.		A							&E 2
%4.		B ˅ (C ˅ D)				→E 1, 3
%5.		~B & (~C &~D)			&E 2
%6.		~B							&E 5
%7.		C ˅ D						˅E 4, 6
%8.		~C & ~D					&E 5
%9.		~C							&E 8
%10.		D							˅E 7, 9
%11.		~D							&E 8
%12.	~[A & (~B & (~C &~D))]	~I 2–11	
%





\end{enumerate}

%\noindent\problempart 
%Translate the following arguments in to SL and then show that they are valid. Be sure to write out your dictionary. 

%
%  This is the opening of the conditional formatting tag for typesetting only part of this chapter. Everything from here to the close tag will be skipped 
% unless the {whole_slproof_chap} label at the
%  start of this chapter is uncommented.
%

\iflabelexists{whole_slproof_chap}{

% *******************************************
% *		Tautologies and Equivalences				   *	
% *******************************************

\section{Tautologies, Equivalences and Inconsistencies}
\label{sec:taut-eq-incon}

So far all we've looked at is whether conclusions follow validly from sets of premises. However, as we saw in the chapter on truth tables, there are other logical properties we want to 
investigate: whether a statement is a tautology, a contradiction or a contingent statement, whether two statements are equivalent, and whether sets of sentences are consistent. In this 
section, we will look at using derivations to test for three properties which will be important in later sections, logical equivalence, inconsistency and being a tautology.


\newglossaryentry{syntactically logically equivalent in SL}
{
name=syntactically logically equivalent in SL,
description={A property held by pairs of statements in SL if and only if there is a derivation which takes you from each one to the other one.}
}

We can say that two statements are \textsc{\gls{syntactically logically equivalent in SL}} \label{def:syntactically_logically_equivalent_in_sl} if you can derive each of them from the 
other. We can symbolize this the same way we symbolized semantic equivalence. When we introduced the double turnstile (p. \pageref{defDoubleTurnstile}), we said we would write the symbol 
facing both directions to indicate that two sentences were semantically equivalent, like this: $A \eand B \ndststile{}{} \hspace{.5em} \sdtstile{}{} B \eand A$. We can do the same thing 
with the single turnstile for syntactic equivalence, like this: $A \eand B \nsststile{}{} \hspace{.5em} \sststile{}{} B \eand A$.

For an example of how we can show two sentences to be syntactically equivalent, consider the sentences $P \eif (Q \eif R)$ and $(P \eif Q) \eif (P \eif R)$. \label{theorem_DistributionOfImplicationOverImplication} To prove these logically equivalent using derivations, we simply use derivations to prove the equivalence one way, from P \eif (Q \eif R) to (P \eif Q) \eif (P \eif R). And then we prove it going the other way, from (P \eif Q) \eif (P \eif R) to P \eif (Q \eif R). We set up the proof going left to right like this: 

\begin{proof}
\hypo{1}{P \eif (Q \eif R)}	\by{Want: (P \eif Q) \eif (P \eif R)}{}
\end{proof}

Since our want line is a conditional, we can set this up as a conditional proof. Once we set up the conditional proof, we also have a conditional in next want line, which means that we can put a conditional proof inside a conditional proof, like this.

\begin{proof}
\hypo{1}{P \eif (Q \eif R)}	\by{Want: (P \eif Q) \eif (P \eif R)}{}
	\open
	\hypo{2}{P \eif Q}	\by{Want: P \eif R}{}
		\open
		\hypo{3}{P}	\by{Want: R}{}
\end{proof}

The completed proof for the equivalence going in one direction will look like this.

\begin{proof}
\hypo{1}{P \eif (Q \eif R)}	\by{Want: (P \eif Q) \eif (P \eif R)}{}
	\open
	\hypo{2}{P \eif Q}	\by{Want: P \eif R}{}
		\open
		\hypo{3}{P}	\by{Want: R}{}
		\have{4}{Q \eif R} \by{\eif E}{1, 3}
		\have{5}{Q}	\by{\eif E}{2, 3}
		\have{6}{R} \by{\eif E}{4, 5}
		\close
	\have{7}{P \eif R} \by{\eif I}{3-6}
	\close
\have{8}{(P \eif Q) \eif (P \eif R)} \by{\eif I}{2-7}
\end{proof}

This shows that $P \eif (Q \eif R) \sststile{}{} (P \eif Q) \eif (P \eif R)$. In order to show $P \eif (Q \eif R) \nsststile{}{} \hspace{.5em} \sststile{}{} (P \eif Q) \eif (P \eif R)$, we need to prove the equivalence going the other direction. That proof will look like this:

\begin{proof}
\hypo{1}{(P \eif Q) \eif (P \eif R)}	\by{Want: P \eif (Q \eif R)}{}
	\open
	\hypo{2}{P} \by{Want: Q \eif R}{}
		\open
		\hypo{3}{Q} \by{Want: R}{}
			\open
			\hypo{4}{P} \by{Want: Q}{}
			\have{5}{Q} \by{R}{3}
			\close
		\have{6}{P \eif Q} \by{\eif I}{4-5}
		\have{7}{P \eif R} \by{\eif E}{1, 6}
		\have{8}{R} \by{\eif E}{2, 7}
		\close	
\have{9}{Q \eif R} \by{\eif I}{3-8}
\close
\have{10}{P \eif (Q \eif R)} \by{\eif I}{2-9}
\end{proof}
You might think it is strange that we assume $P$ twice in this proof, but that is the way we have to do it. When we assume $P$ on line 2, our goal is to prove $P \eif (Q \eif R)$. Before we can finish that proof, we also need to know that $P \eif Q$. This requires a different subproof. 
%%%%%
%fixed a little typoe above where {} was instead of ()
%%%%%

These two proofs show that $P \eif (Q \eif R)$ and $(P \eif Q) \eif (P \eif R)$ are equivalent, so we can write $P \eif (Q \eif R) \nsststile{}{} \hspace{.5em} \sststile{}{} (P \eif Q) \eif (P \eif R)$. 

\newglossaryentry{syntactic tautology in SL}
{
name=syntactic tautology in SL,
description={A statement in SL that can be derived without any premises}
}

We can also prove that a sentence is a tautology using a derivation. A tautology is something that must be true as a matter of logic. If we want to put this in syntactic terms, we would say that \textsc{\gls{syntactic tautology in SL}} \label{def:syntactic_tautology_in_sl} is a statement that can be derived without any premises, because its truth doesn't depend on anything else. Now that we have all of our rules for starting and ending subproofs, we can actually do this. Rather than listing any premises, we simply start a subproof at the beginning of the derivation. The rest of the proof can work only using premises assumed for the purposes of subproofs. By the end of the proof, you have discharged all these assumptions, and are left knowing a tautological statement without relying on any leftover premises. Consider this proof of the law of noncontradiction: $\enot(G \eand \enot G)$. \label{theorem_Noncontradiction}

\begin{proof}
	\open
		\hypo{gng}{G\eand \enot G}\by{for reductio}{}
		\have{g}{G}\ae{gng}
		\have{ng}{\enot G}\ae{gng}
	\close
	\have{ngng}{\enot(G \eand \enot G)}\ni{gng-ng}
\end{proof}

This statement simply says that any sentence G cannot be both true and not true at the same time. We prove it by imagining what would happen if G were actually both true and not true, and 
then pointing out that we already have our contradiction.

In the previous chapter, we expressed the fact that something could be proven a tautology using truth tables by writing the double turnstile in front of it. The law of noncontradiction 
above could have been proven using truth tables, so we could write: $\sdtstile{}{} \enot(G \eand \enot G)$ In this chapter, we will use the single turnstile the same way, to indicate that 
a sentence can be proven to be a tautology using a derivation. Thus the above proof entitles us to write $\sststile{}{} \enot(G \eand \enot G)$.

\newglossaryentry{syntactically logically inconsistent in SL}
{
name=syntactically logically inconsistent in SL,
description={A property held by sets of statements in SL if and only if there is a derivation which results in a contradiction.}
}

%%%%%%%%%%
%%rcr. Syntactically inconsistent
%%%%%%%%%%
Finally, the natural deduction method can be used to prove that a set of sentences are inconsistent. Recall our definition from
Section~\ref{sec:consistency}: a set of sentences in English are said to be inconsistent if they cannot all be true at the same time. Otherwise
they are consistent. 
We were able to prove this semantic relationship in Chapter~\ref{chap:truth_tables} by drawing a truth table for
every sentence in the set, and then inspecting the column under the main connective of each. If there was a single row in which every sentence
contained a T, then it was determined to be logically consistent. If there is not a single row in which all members of the set contains a T, then it was
determined to be logically inconsistent.

When we say that a set of sentences is inconsistent, we sometimes also say that it ``contains a contradiction''. By this, we mean that logically, it is not possible for all of the
statements to be true, taken together. However, more precisely, if a set of sentences is logically inconsistent, it does not ``contain" a contradiction: rather, from a set of inconsistent
sentences one can derive a contradiction. It is this relationship that we will exploit in a derivation, using either the indirect proof, or conditional proof method. We can say that two
statements are \textsc{\gls{syntactically logically inconsistent in SL}} \label{def:syntactically_logically_inconsistent_in_sl} if you can derive a contradiction from them.



Like invalid arguments, the natural deduction method cannot be used to show that a set of sentences is logically consistent, since the test for such a relationship
would involve showing that a contradiction \emph{cannot} be derived. Of course, because of the strategic nature of the derivation rules, there is no independent
rigorous test to distinguish between an attempted derivation in which no contradiction is logically possible, and one in which no derivation could be found.
Since there are an infinite number of combinations of application of the derivation rules, but each derivation is finite, no such test can be performed.

Consider this proof that the following set of sentences are logically inconsistent: \{$P \eif Q, P, \enot Q$\}.

\begin{proof}
        \hypo{1}{P \eif Q}
        \hypo{2}{P}
        \hypo{3}{\enot Q} \by{Want: $P \eand \enot P$}{}
        \open
                \hypo{4}{P}\by{for reductio}{}
                \have{5}{Q}\by{\eif I}{1, 4} 
                \have{6}{\enot Q} \by{R}{3}
        \close
                \have{7}{\enot P} \by{IP}{4-6}
                \have{8}{P}{} \by{R}{2}
                \have{9}{P \eand \enot P} \by{\eand I}{7, 8}
\end{proof}

By listing the set of sentences among our assumptions, we were able to derive a contradiction \{$P, \eand \enot P$\} using the indirect proof method. Since there 
is a derivation in which a contradiction can be derived from the set of sentences, we conclude that they are logically inconsistent.

\practiceproblems
 	

\noindent\problempart
Prove each of the following equivalences
\begin{enumerate}[label=(\arabic*)]

\item $J \nsststile{}{} \hspace{.5em} \sststile{}{} J\eor (L\eand\enot L)$
%\vspace{5pt}
%$ \sststile{}{}$
%\vspace{5pt}
%\begin{proof}
%		\hypo{1}{J} \by{Want: J \eor(L \eand \enot L)}{}
%		\have{2}{J \eor (L \eand \enot L)} \oi{1}
%\end{proof}
%\vspace{5pt}
%$\nsststile{}{}$
%\vspace{5pt}
%\begin{proof}	
%		\hypo{1}{J \eor (L \eand \enot L)} \by{Want: J}{}
%			\open
%			\hypo{2}{L \eand \enot L} \by{for reductio}{}
%			\have{3}{L} \ae{4}
%			\have{4}{\enot L} \ae{4}
%			\close
%		\have{5}{\enot(L \eand \enot L)} \ni{4-6}
%		\have{6}{J} \oe{3, 7}
%\end{proof}


\item $P \eif (Q \eif R) \nsststile{}{} \hspace{.5em} \sststile{}{} Q \eif (P \eif R)$

%Modified from KMM T107 p. 82.

%\vspace{5pt}
%$ \sststile{}{}$
%\vspace{5pt}
%
%\begin{proof}
%\hypo{1.}{P \eif (Q \eif R)} \by{Want: Q \eif (P \eif R)}{}
%\open
%\hypo{2.}{Q} \by{Want: P \eif R}{}
%\open
%\hypo{3.}{P} \by{Want: R}{}
%\have{4}{Q \eif R } \by{\eif E}{1,3}
%\have{5.}{R} \by{ \eif E }{2, 4}
%\close
%\have{6}{P \eif R } \by{\eif I}{3-5}
%\close
%\have{7.}{Q \eif (P \eif R)} \by{\eif I}{2-6}
%\end{proof}
%
%\vspace{5pt}
%$\nsststile{}{}$
%\vspace{5pt}
%
%
%\begin{proof}
%\hypo{1.}{Q \eif (P \eif R)} \by{Want: P \eif (Q \eif R)}{}
%\open
%\hypo{2.}{P} \by{Want: Q \eif R}{}
%\open
%\hypo{3.}{Q} \by{Want: R}{}
%\have{4}{P \eif R } \by{\eif E}{1,3}
%\have{5.}{R} \by{ \eif E }{2, 4}
%\close
%\have{6}{Q \eif R } \by{\eif I}{3-5}
%\close
%\have{7.}{P \eif (Q \eif R)} \by{\eif I}{2-6}
%\end{proof}

\item $P \eif \enot P \nsststile{}{} \hspace{.5em} \sststile{}{}  \enot P $ %(KMM T115, p. 111)

%\vspace{5pt}
%$ \sststile{}{}$
%\vspace{5pt}
%
%
%\begin{proof}
%\hypo{1}{P \eif \enot P} \by{Want:  \enot P}{}
%	\open
%	\hypo{2}{P} \by{Want: A contradiction}{}
%	\have{3}{\enot P} \by{\eif E}{1, 2}
%	\have{4}{P} \by{R}{2}
%	\close
%\have{5}{P} \by{\enot E}{2-4}
%\end{proof}
%
%\vspace{5pt}
%$\nsststile{}{}$
%\vspace{5pt}
%
%\begin{proof}
%\hypo{1}{ \enot P } \by{Want: P \eif \enot P}{}
%	\open
%	\hypo{2}{P} \by{Want: \enot P}{}
%	\have{3}{\enot P} \by{R}{1}
%	\close
%\have{4}{P \eif \enot P} \by{\eif I}{2-3}
%\end{proof}
%

\item $\enot (P \eiff Q) \nsststile{}{} \hspace{.5em} \sststile{}{} (P \eiff \enot Q) $ %(KMM T90 p. 110)

%\vspace{5pt}
%$ \sststile{}{}$
%\vspace{5pt}
%
%\begin{proof}
%\hypo{1.}{\enot (P \eiff Q)} \by{Want: $P \eiff \enot Q$}{}
%	\open
%	\hypo{2.}{P} \by{Want: $\enot Q$}{}
%		\open
%		\hypo{3.}{Q}	\by{Want: A contradiction}{}
%			\open
%			\hypo{4.}{P} \by{Want: Q}{}
%			\have{5.}{Q} \by{R}{3}
%			\close
%			\open
%			\hypo{6.}{Q} \by{Want: P}{}
%			\have{7.}{P} \by{R}{2}
%			\close
%		\have{8.}{P \eiff Q} \by{\eiff I}{4-5, 6-7}
%		\have{9.}{\enot(P \eiff Q)} \by{R}{1}
%		\close
%	\have{10.}{\enot Q} \by{\enot I}{2-9}
%	\close
%	\open
%	\hypo{a}{\enot Q} \by{Want: P}{}
%		\open
%		\hypo{b}{\enot P} \by{Want: A contradiction}{}
%			\open
%			\hypo{c}{P} \by{Want: Q}{}
%			\have{d}{P \eor Q} \by{\eor I}{13}
%			\have{e}{Q} \by{\eor E}{12, 14}
%			\close
%			\open
%			\hypo{f}{Q} \by{Want: P}{}
%			\have{g}{Q \eor P} \by{\eor I}{f}
%			\have{h}{P} \by{\eor E}{f, g}
%			\close
%		\have{i}{P \eiff Q} \by{\eiff I}{c-e, f-h}
%		\have{j}{\enot (P \eiff Q)} \by{R}{b}
%		\close
%	\have{21.}{P} \by{\enot I}{a-j}
%	\close
%\have{22.}{P \eiff \enot Q} \by{\eiff I}{}
%\end{proof}
%
%\vspace{5pt}
%$\nsststile{}{}$
%\vspace{5pt}
%
%\begin{proof}
%\hypo{1}{P \eiff \enot Q} \by {Want: \enot (P \eiff Q)}{}
%	\open
%	\hypo{2}{P \eiff Q} \by{Want: A contradiction}{}
%		\open
%		\hypo{3}{Q} \by{Want: A contradiction}{}
%		\have{4}{P} \by{\eiff E}{2, 3}
%		\have{5}{\enot Q} \by{\eiff E}{1, 4}
%		\have{6}{Q} \by{R}{3}
%		\close
%	\have{7}{\enot Q} \by{\enot I}{3-6}
%	\have{8}{P} \by{\eiff}{1, 7}
%	\have{9}{Q}\by{\eiff}{2, 8}
%	\have{10}{\enot Q} \by{R}{7}
%	\close
%\have{11}{\enot (P \eiff Q)} \by{\enot I}{2-10}
%\end{proof}
%\vspace{15pt}




\end{enumerate}

\noindent\problempart
Prove each of the following equivalences
\begin{enumerate}[label=(\arabic*)]

\item $(P \eif R) \eand (Q \eif R) \nsststile{}{} \hspace{.5em} \sststile{}{}(P \eor Q) \eif R $ %(KMM T50 p.109)
\item $(P \eif (Q \eor R)) \nsststile{}{} \hspace{.5em} \sststile{}{} (P \eif Q) \eor (P \eif R)$ %(KMM T55 p.109)
\item $(P \eiff Q)  \nsststile{}{} \hspace{.5em} \sststile{}{} \enot P \eiff \enot Q		$ %(KMM T96 p.110)
\end{enumerate}

%\item $P \eif Q \nsststile{}{} \hspace{.5em} \sststile{}{}(R \eor P) \eif (R \eor Q)$ %(KMM T56 p.109)
% ^ removed because it doesn't work right to left. Check to see if this is really in KRR.

\noindent\problempart
Prove each of the following tautologies
\begin{enumerate}[label=(\arabic*)]

\item $\sststile{}{} O \eif O$		%KMM T1, p.41
%
%	\begin{proof}
%
%	\open
%	\hypo{1}{O}\by{Want: O}{}
%	\have{2}{O}\by{R}{1}
%	\close
%	\have{3}{O \eif O} \ci{1-2}
%
%	\end{proof}

\item $\sststile{}{} N \eor \enot N$ \label{theorem_ExcludedMiddle}

%	\begin{proof}
%
%	\open
%	\hypo{1}{\enot (N \eor \enot N)} \by{for reductio}{}
%	\open
%	\hypo{2}{N} \by{for reductio}{}
%	\have{3}{N \eor \enot N} \oi{2}
%	\have{4}{\enot (N \eor \enot N)} \by{R}{1}
%	\close
%	\have{5}{\enot N} \ni{2-4}
%	\have{6}{N \eor \enot N} \oi{5}
%	\have{7}{\enot (N \eor \enot N)} \by{R}{1}
%	\close
%	\have{8}{N \eor \enot N} \ne{2-7}
%
%	\end{proof}

\item $\sststile{}{} \enot(A \eif \enot C) \eif (A \eif C)$

%	\begin{proof}
%	
%		\open
%		\hypo{1}{\enot(A \eif \enot C)} \by{Want: A \eif C}{}
%			\open
%			\hypo{2}{A} \by{Want: C}{}
%				\open
%				\hypo{3}{\enot C} \by{for reductio}{}
%					\open
%					\hypo{4}{A} \by{Want: \enot C}{}
%					\have{5}{\enot C} \by{R}{3}
%					\close
%				\have{6}{A \eif \enot C} \ci{4-5}
%				\have{7}{\enot (A \eif \enot C)} \by{R}{1}
%				\close
%			\have{8}{C} \ne{3-7}
%			\close
%		\have{9}{A \eif C} \ci{2-9}
%		\close
%	\have{10}{\enot(A \eif \enot C) \eif (A \eif C)} \ci{1-9}
%	
%	\end{proof}

\item $\sststile{}{} P \eiff (P \eor (Q \eand P))$ 

%
%1.		P				Want: P  (Q & P)
%2.		P  	(Q & P)		I 1
%3.		P  (Q & P)		Want P
%4			~P			For reductio
%5.			Q & P		E 3, 4
%6.			P			&E 5
%8.			~P			R 4
%9.		P				~E 4
%10.	P ↔ (P  (Q & P)	I 1¬–2, 3–9
%
%Appears in Hurley 10, 404 replace ASAP

\end{enumerate}


\noindent\problempart
Prove each of the following tautologies
\begin{enumerate}[label=(\arabic*)]
\item $\sststile{}{} (B \eif \enot B) \eiff \enot B$

%1.		B \eif ~B		Want: ~B
%2.			B			For reductio
%3.			~B			\eif E 1, 2
%4.			B			R2
%5.		~B				~I 2–4
%6.		~B				Want: B \eif ~B
%7.			B			Want: ~B
%8.			~B			R6
%9.		B \eif ~B
%10	(B \eif ~B) \eiff ~B

\item $\sststile{}{} (P \eif [P \eif Q]) \eif (P \eif Q)$ %(KMM T9 p. 42)

\item $\sststile{}{} (P \eor \enot P) \eand (Q \eiff Q) $ %(KMM T119 p.111)

\item $\sststile{}{} (P \eand \enot P) \eor  (Q \eiff Q)$%(KMM T120 p.111)

\end{enumerate}

%%%%%%%%%%
%rcr incon problems
%%%%%%%%%%
\noindent\problempart   
\label{pr.derivation.inconsistent}
Show that each set of sentences is inconsistent by deriving a contradiction.
\begin{enumerate}
\item \{$A \eif \enot A$, $\enot A \eif A$\}\vspace{.5ex} %inconsistent. 
\item \{$A \eor B$, $A \eif C$, $B \eif C$, $\enot C$\}\vspace{.5ex} %  Inconsistent
\item \{$B\eand(C\eor A)$, $A\eif B$, $\enot(B\eor C)$\}\vspace{.5ex}  %inconsistent
\item \{$A \eand B$, $C\eif \enot B$, $C$\} \vspace{.5ex} %inconsistent
\item \{$A\eif B$, $B\eif C$, $A$, $\enot C$\}\vspace{.5ex} %inconsistent
\item \{$A \eif B$, $B \eif C$, $\enot(A \eif C)$\} \vspace{.5ex} %inconsistent
\end{enumerate}



\noindent\problempart   
\label{pr.derivation2.inconsistent2}
Show that each set of sentences is inconsistent by deriving a contradiction.
\begin{enumerate}
\item \{$P \eif P$, $\enot (P \eif P)$  \}\vspace{.5ex}
\item \{$P \iff A$, $Q \eif \enot P$, $P$ \}\vspace{.5ex}
\item \{$P \eif \enot P$, $\enot P \eif P)$ \}\vspace{.5ex}
\item \{$\enot P \eor Q$, $R \eif P$, $\enot R \eif P$  \}\vspace{.5ex}
\end{enumerate}



\noindent\problempart   
\label{pr.derivation2.inconsistent3}
Show that each set of sentences is inconsistent by deriving a contradiction.
\begin{enumerate}
\item \{$C \eiff G$, $G \eiff \enot C$\}\vspace{.5ex}
\item \{$F \eor T$, $(F \eor T) \eif (\enot F \eand \enot T)$\}\vspace{.5ex}
\item \{$J \eor K$, $\enot J \eor \enot K$, $J \eiff K$\}\vspace{.5ex}
\item \{$(G \eor K) \eif A$, $(A \eor H) \eif G$, $G \eand \enot A$\}\vspace{.5ex}
\item \{$D \eiff(\enot P \eand \enot M)$, $P \eiff (J \eand \enot F)$, $\enot F \eor \enot D$, $D \eand J$\}\vspace{.5ex}
\end{enumerate}





%c l ) C=G,      G=-C


%c3) JvK,   -Jv-K,   J=K
%c4)  (GvK)>A,   (AvH)>G,  G&-A
%c.5) D=(-P&-M),    P=(I&-F), -Fv-D, D&J






% *******************************************
% *					Derived Rules				   *	
% *******************************************

\section{Derived Rules}
\setlength{\parindent}{1em}

%rob: new opening paragraph, more ambitions for this section.

Now that we have our five rules for introduction and our five rules for elimination, plus the rule of reiteration, our system is complete. If an argument is valid, and you can symbolize 
that argument in SL, you can \emph{prove} that the argument is valid using a derivation. (We will say a bit more about this in section \ref{sec:rules_of_rep}.) Now that our system is 
complete, we can really begin to play around with it and explore the exciting logical world it creates.

There's an exciting logical world created by these eleven rules? Yes, yes there is. You can begin to see this by noticing that there are a lot of other interesting rules that we could 
have used for our introduction and elimination rules, but didn't. In many textbooks, the system of natural deduction has a disjunction elimination rule that works like this:

\begin{proof}
	\have[m]{ab}{\script{A}\eor\script{B}}
	\have[n]{ac}{\script{A}\eif\script{C}}
	\have[o]{bc}{\script{B}\eif\script{C}}
	\have[\ ]{c}{\script{C}} \by{${\eor}\ast$}{ab,ac,bc}
\end{proof}

You might think our system is incomplete because it lacks this alternative rule of disjunction elimination. Yet this is not the case. If you can do a proof with this rule, you can do a 
proof with the basic rules of the natural deduction system. You actually proved this rule in problem \ref{itm:const_d} of part \ref{derivation_set_with_const_d} in the exercises for 
section \ref{sec:indirect_proof}. Furthermore, once you have a proof of this rule, you can use it inside other proofs whenever you think you would need a rule like $\eor \ast$. Simply use 
the proof you gave in the last homework as a sort of recipe for generating a new series of steps to get you to a line saying $\script{C} \eor \script{D}$

But adding lines to a proof using this recipe all the time would be a pain in the neck. What's worse, there are dozens of interesting possible rules out there, which we could have used 
for our introduction and elimination rules, and which we now find ourselves replacing with recipes like the one above.

Fortunately our basic set of introduction and elimination rules, plus reiteration, was meant to be expanded on. That's part of the game we are playing here. The first system of deduction 
created in the Western tradition was the system of geometry created by Euclid (c 300 BCE). Euclid's \emph{Elements} began with 10 basic laws, along with definitions of terms like 
``point,'' ``line,'' and ``plane.'' He then went on to prove hundreds of different theorems about geometry, and each time he proved a theorem he could use that theorem to help him prove 
later theorems.

We can do the same thing in our system of natural deduction. What we need is a rule that will allow us to make up new rules. The new rules we add to the system will be called \define{derived rules}. Our ten rules for adding and eliminating connectives are then the \define{axioms} of SL. Now here is our rule for adding rules. 

{\narrower \narrower
 
\bf{Rule of Derived Theorem Introduction:} \rm Given a derivation in SL of some argument $A_1$ \ldots $A_n \sststile{}{} B$, create the rule $\script{A}_1$ \ldots $\script{A}_n \sststile{}{} \script{B}$ and assign a name to it of the form ``$T_n$'', to be read ``theorem n.'' Now given a derivation of some theorem $T_m$, where $n < m$, if $\script{A_1}$ \ldots $\script{A_n}$ occur as earlier lines $x_1$ \ldots $x_n$ in a proof, one may infer \script{B}, and justify it ``$T_n$, $x_1$ \ldots $x_n$'', so long as none of lines $x_1$ \ldots $x_n$ are in a closed subproof.
\par
}


Let's make our rule $\eor \ast$ above our first theorem. The proof of $T_1$ is derived simply from the recipe above.

{\narrower
\bf $T_\arabic{theorem} $ (Constructive Dilemma, CD): \rm $ \{ \script{A} \eor \script {B}, \script{A}\eif\script{C}, \script{B}\eif\script{C} \} \sststile{}{} 	\script{C}$
\addtocounter{theorem}{1}
\par
}

Proof:

\begin{proof}
	\hypo{1}{A\eor B}
	\hypo{2}{A\eif C}
	\hypo{3}{B \eif C} \by{want: C}{}
	\open
		\hypo{4}{\enot{C}}\by{for reductio}{}
			\open
			\hypo{5}{\enot A} \by{for reductio}{}
			\have{6}{B} \oe{1, 5}
			\have{7}{C}\ci{3, 6}
			\have{8}{\enot C} \by{R}{4}
			\close
		\have{9}{A}\ne{5-8}
		\have{11}{C} \ce{2, 10}
		\have{12}{\enot C} \by{R}{4}
		\close
	\have{13}{C} \ne{4-13}		
\end{proof} 



Informally, we will refer to $T_1$ as ``Constructive Dilemma'' or by the abbreviation ``CD.'' Most theorems will have names and easy abbreviations like this. We will generally use the abbreviations to refer to the proofs when we use them in derivations, because they are easier to remember. 

Several other important theorems have already appeared as examples or in homework problems. We'll talk about most of them in the next section, when we discuss rules of replacement. In the meantime, there is one important one we need to introduce now

{\narrower
$\mathbf T_\arabic{theorem}$  \bf (Modus Tollens, MT): \rm $\{ \script{A} \eif \script{B}, \enot \script{B} \} \sststile{}{} \hspace{.25em} \enot \script{A}$
\addtocounter{theorem}{1}
\par}

Proof: See page \pageref{ModusTollens}

Now that we have some theorems, let's close by looking at how they can be used in a proof. 


$\mathbf T_\arabic{theorem}$ \bf (Destructive Dilemma, DD): \rm $ \{ \script{A} \eif \script{B}, \script{A} \eif \script{C}, \enot \script{B} \eor \enot \script{C} \} \sststile{}{} \hspace{.25em} \enot \script{A}$
\addtocounter{theorem}{1}

\begin{proof}
\hypo{1}{A \eif B}
\hypo{2}{A \eif C}
\hypo{3}{\enot B \eor \enot C} \by{Want: \enot A}{}
	\open
	\hypo{4}{A} \by{for reductio}{}
	\have{5}{B} \ce{1, 4}
		\open
		\hypo{6}{\enot B} \by{For reductio}{}
		\have{7}{B} \by{R}{5}
		\have{8}{\enot B} \by{R}{6}
		\close
	\have{9}{\enot \enot B} \ni{6-8}
	\have{10}{\enot C} \oe{3, 9}
	\have{11}{A}	\by{R}{4}
	\have{12}{\enot A} \by{MT}{2, 10}
	\close
\have{13}{\enot A} \ni{4-12}
\end{proof}


%%%%%%%%%%%%%%%%%          Practice problems %%%%%%

\practiceproblems
 

\noindent\problempart
Prove the following theorems

\begin{enumerate}[label=(\arabic*)]

\item $T_\arabic{theorem}$ (Hypothetical Syllogism, HS): $ \{ \script{A} \eif \script{B}, \script{B} \eif \script{C} \} \sststile{}{} \hspace{.25em} \script{A} \eif \script{C}$
\addtocounter{theorem}{1}


%\begin{proof}
%\hypo{1}{A \eif B}
%\hypo{2}{B \eif C} \by{Want: A \eif C}{}
%	\open
%	\hypo{3}{A} \by{Want: C}{}
%	\have{4}{B} \ce{1, 3}
%	\have{5}{C} \ce{2, 4}
%	\close
%\have{6}{A \eif C} \ci{3-5}
%\end{proof}

\item $T_\arabic{theorem}$ (Idempotence of \eor, Idem\eor): $  \script{A} \eor \script{A}  \sststile{}{} \script{A} $
\addtocounter{theorem}{1}

\item $T_\arabic{theorem}$ (Idempotence of \eand, Idem\eand): $  \script{A} \sststile{}{} \script{A} \eand \script{A} $
\addtocounter{theorem}{1}

%\begin{proof}
%\hypo{1}{A} \by{Want A \eand A}{}
%\have{2}{A} \by{R}{1}
%\have{3}{A \eand A} \ai{1, 2}
%\end{proof}

\item $ T_\arabic{theorem}$ (Weakening, WK): \rm $\script{A} \sststile{}{} \script{B} \eif \script{A}$ \\
\addtocounter{theorem}{1}

\end{enumerate}

%\item $\enot\enot\enot\enot G$, $G$

\noindent\problempart
Provide proofs using both axioms and derived rules to show each of the following.
\begin{enumerate}[label=(\arabic*)]
\item \{$M \eand (\enot N \eif \enot M) \} \sststile{}{} (N \eand M) \eor \enot M$

%\begin{proof}
%\hypo{1}{M \eand (\enot N \eif \enot M)} \by{Want: $(N \eand M) \eor \enot M$}{}
%\have{2}{M} \ae{1}
%\have{3}{\enot N \eif \enot M} \ae{1}
%	\open
%	\hypo{4}{\enot N} \by{For reductio}{}
%	\have{5}{M} \by{R}{2}
%	\have{6}{\enot M} \by{\eif E}{3, 4}
%	\close
%\have{7}{N} \ne{4-6}
%\have{8}{N \eand M} \ai{2, 7}
%\have{9}{(N \eand M) \eor M} \oi{8}
%\end{proof}



\item \{$C\eif(E\eand G)$, $\enot C \eif G$\} $\sststile{}{}$ $G$
\item \{$(Z\eand K)\eiff(Y\eand M)$, $D\eand(D\eif M)$\} $\sststile{}{}$ $Y\eif Z$

%\begin{proof}
%\hypo{1}{(Z\eand K)\eiff(Y\eand M)}
%\hypo{2}{D\eand(D\eif M)} \by{Want: Y \eif Z}{}
%\have[3]{3}{D} \ae{2}
%\have[4]{4}{D \eif M} \ae{2}
%\have[5]{5}{M} \ce{3-4}
%	\open
%	\hypo[6]{6}{Y} \by{Want: Z}{}
%	\have[7]{7}{Y \eand M} \ai{5, 6}
%	\have[8]{8}{Z \eand K} \by{\eiff E, 1, 7}{}
%	\have[9]{9}{Z} \ae{8}
%	\close
%\have[10]{10}{Y \eif Z} \ci{6-9}
%\end{proof}



\item \{$(W \eor X) \eor (Y \eor Z)$, $X\eif Y$, $\enot Z$\} $\sststile{}{}$ $W\eor Y$
\item \{$(B \eif C) \eand (C \eif D), (B \eif D) \eif A $ \}$ \sststile{}{}$ $A$

%\begin{proof}
%\hypo{1}{(B \eif C) \eand (C \eif D)}
%\hypo{2}{(B \eif D) \eif A} \by{Want: A}{}\
%\have{3}{B \eif C}\by{\eand E}{1}
%\have{4}{C \eif D} \by{\eand E}{1}
%\have{5}{B \eif D} \by{HS}{3, 5}
%\have{6}{A} \by{\eif E}{2, 5}
%\end{proof}


\end{enumerate}

\noindent\problempart
\begin{enumerate}[label=(\arabic*)]

\item If you know that $\script{A}\sststile{}{}\script{B}$, what can you say about $(\script{A}\eand\script{C})\sststile{}{}\script{B}$? Explain your answer.

%1) It is valid. If you know that \script{A} on its own implies \script{B}, then a proof of $(\script{A}\eand\script{C})\sststile{}{}\script{B}$ would only require \eand E and then the proof that $\script{A}\sststile{}{}\script{B}$


\item If you know that $\script{A}\sststile{}{}\script{B}$, what can you say about $(\script{A}\eor\script{C})\sststile{}{}\script{B}$? Explain your answer.
\end{enumerate}





% *******************************************
% *				Rules of Replacement			   *	
% *******************************************
\section{Rules of Replacement}
\label{sec:rules_of_rep}
\setlength{\parindent}{1em}



Very often in a derivation, you have probably been tempted to apply a rule to a part of a line. For instance, if you knew $F\eif(G\eand H)$ and wanted $F\eif G$, you would be tempted to apply \eand E to just the $G \eand H$ part of $F \eif (G \eand H)$. But, of course you aren't allowed to do that. We will now introduce some new derived rules where you can do that. These are called \define{rules of replacement}, because they can be used to replace part of a sentence with a logically equivalent expression. What makes the rules of replacement different from other derived rules is that they draw on only one previous line and are symmetrical, so that you can reverse premise and conclusion and still have a valid argument. Some of the most simple examples are Theorems $8-10$, the rules of commutativity for \eand, \eor, and \eiff. 

{\narrower


$\mathbf T_\arabic{theorem}$  \bf (Commutativity of \eand, Comm\eand): \rm $(\script{A}\eand\script{B}) \nsststile{}{} \hspace{.25em} \sststile{}{}  (\script{B}\eand\script{A})$\\ 
\addtocounter{theorem}{1}
$\mathbf T_\arabic{theorem}$  \bf (Commutativity of \eor, Comm\eor): \rm $(\script{A}\eor\script{B}) \nsststile{}{} \hspace{.25em} \sststile{}{} (\script{B}\eor\script{A})$\\
\addtocounter{theorem}{1}
$\mathbf T_{\arabic{theorem}}$  \bf (Commutativity of \eiff, Comm\eiff): \rm $(\script{A}\eiff\script{B}) \nsststile{}{} \hspace{.25em} \sststile{}{} (\script{B}\eiff\script{A})$
\addtocounter{theorem}{1}

\par}


You will be asked to prove these in the homework. In the meantime, let's see an example of how they work in a proof. Suppose you wanted to prove $(M \eor P) \eif (P \eand M)$, $\therefore$\ $(P \eor M) \eif (M \eand P)$ You could do it using only the basic rules, but it will be long and inconvenient. With the Comm rules, we can provide a proof easily:

\begin{proof}
	\hypo{1}{(M \eor P) \eif (P \eand M)}
	\have{2}{(P \eor M) \eif (P \eand M)}\by{Comm\eand}{1}
	\have{n}{(P \eor M) \eif (M \eand P)}\by{Comm\eor}{2}
\end{proof}

Formally, we can put our rule for deploying rules of replacement like this

{\narrower \narrower
 
\noindent\bf Inserting rules of replacement: \rm Given a theorem T of the form $\script{A} \nsststile{}{} \hspace{.5em} \sststile{}{} \script B$ and a line in a derivation \script{C} which contains in it a sentence \script{D}, where \script{D} is a substitution instance of either \script{A} or \script{B}, replace \script{D} with the equivalent substitution instance of the other side of theorem T. 
\setlength{\parindent}{1em}

\par}

\setlength{\parindent}{1em}

Here are some other important theorems that can act as rules of replacement. Some are theorems we have already proved, while you will be asked to prove others in the homework.

{\narrower

$\mathbf T_{\arabic{theorem}}$ \bf (Double Negation, DN): \rm $\script{A} \nsststile{}{} \hspace{.5em} \sststile{}{} \hspace{.25em} \enot \enot \script{A}$
\addtocounter{theorem}{1}
\par}
Proof: See pages \pageref{DN1} and \pageref{DN2}. 

{\narrower

$\mathbf T_{\arabic{theorem}}$: \rm $\enot(\script{A} \eor \script{B}) \nsststile{}{} \hspace{.5em} \sststile{}{} \hspace{.25em} \enot \script{A} \eand \enot \script{B}$
\addtocounter{theorem}{1}
\par}
Proof: See page \pageref{DeM1}

{\narrower
$\mathbf T_{\arabic{theorem}}$: \rm $\enot (\script{A} \eand \script{B}) \nsststile{}{} \hspace{.5em} \sststile{}{} \hspace{.25em} \enot \script{A} \eor \enot \script{B}$
\addtocounter{theorem}{1}
\par}

Proof: See pages \pageref{DeM3} and \pageref{DeM4}.



$ T_{12}$  and $T_{13}$ are collectively known as \define{DeMorgan's Laws}, and we will use the abbreviation DeM to refer to either of them in proofs.


{\narrower

$ \mathbf T_{\arabic{theorem}}$: \rm $(\script{A}\eif\script{B}) \nsststile{}{} \hspace{.5em} \sststile{}{} \hspace{.25em} (\enot\script{A}\eor\script{B})$ 
\addtocounter{theorem}{1}


$ \mathbf T_{\arabic{theorem}}$: \rm $(\script{A}\eor\script{B}) \nsststile{}{} \hspace{.5em} \sststile{}{} \hspace{.25em} (\enot\script{A}\eif\script{B})$  
\addtocounter{theorem}{1}

\par}

$ T_{14}$ and $T_{15}$ are collectively known as the rule of Material Conditional (MC). You will prove them in the homework. 

 
$ \mathbf T_{\arabic{theorem}}$ \bf (Biconditional Exportation, ex): \rm $\script{A}\eiff \script{B} \nsststile{}{} \hspace{.5em} \sststile{}{} (\script{A} \eif \script{B})\eand(\script{B}\eif \script{\script{A}})$ 
\addtocounter{theorem}{1}
\setlength{\parindent}{1em}

Proof: See the homework.

{\narrower
$ \mathbf T_{\arabic{theorem}}$ \bf (Transposition, trans): \rm $\script{A}\eif \script{B} \nsststile{}{} \hspace{.5em} \sststile{}{} \hspace{.25em} \enot \script{B} \eif \enot \script{A}$ \addtocounter{theorem}{1}
\par}

Proof: See the homework.


To see how much these theorems can help us, consider this argument: $$\enot(P \eif Q) \sststile{}{} P \eand \enot Q$$

As always, we could prove this argument using only the basic rules. With rules of replacement, though, the proof is much simpler:

 

\begin{proof}
	\hypo{1}{\enot(P \eif Q)}
	\have{2}{\enot(\enot P \eor Q)}\by{MC}{1}
	\have{3}{\enot\enot P \eand \enot Q}\by{DeM}{2}
	\have{4}{P \eand \enot Q}\by{DN}{3}
\end{proof}

%Although they don't do it in the book, I've been in the habit of writing $(\script{A}\eand\script{B}\eand\script{C})$ and dropping the inner pair of parentheses. This is fine. If we'd wanted to, we could have defined the basic rules in a more general way:

%\begin{proof}
%	\have[n]{a1}{\script{A}_1}
%	\have{2}{\script{A}_2}
%	\have[\vdots]{1}{\vdots}
%	\have[n]{an}{\script{A}_n}
%	\have[\ ]{aaa}{\script{A}_1~\eand\ldots\eand~\script{A}_n} \ai{}
%\end{proof}

%\bigskip
%\begin{proof}
%	\have{3}{\script{A}_1~\eand\ldots\eand~\script{A}_n}
%	\have{1}{\script{A}_i} \ae{}
%\end{proof}

%\bigskip
%\begin{proof}
%	\have{1}{\script{A}}
%	\have{3}{\script{A}\eor\script{B}_1\eor\script{B}_2\ldots\eor\script{B}_n} \ai{}
%\end{proof}

%We don't need these extended versions, since for any given n we could prove them as a derived rule.




%%%%%%%%%%%%%          Practice problems %%%%%%%%%
 

\practiceproblems
\noindent\problempart
Prove $T_{8}$ through $T_{10}$. You may use $T_{1}$ through $T_7$ in your proofs.

\noindent\problempart
Prove $T_{11}$ through $T_{17}$. You may use $T_{1}$ through $T_{12}$ in your proofs.




% *******************************************
% *				Proof Strategy					   *	
% *******************************************


\section{Proof Strategy}
\setlength{\parindent}{1em}
There is no simple recipe for proofs, and there is no substitute for practice. Here, though, are some rules of thumb and strategies to keep in mind.

\emph{Work backwards from what you want.}
The ultimate goal is to derive the conclusion. Look at the conclusion and ask what the introduction rule is for its main logical operator. This gives you an idea of what should happen \emph{just before} the last line of the proof. Then you can treat this line as if it were your goal. Ask what you could do to derive this new goal. For example: If your conclusion is a conditional $\script{A}\eif\script{B}$, plan to use the {\eif}I rule. This requires starting a subproof in which you assume \script{A}. In the subproof, you want to derive \script{B}. Similarly, if your conclusion is a biconditional, $\script{A} \eiff \script{B}$, plan on using {\eiff}I and be prepared to launch two subproofs. If you are trying to prove a single sentence letter or a negated single sentence letter, you might plan on using indirect proof. 

%Rob: I removed QL examples here and put ih more SL examples

\emph{Work forwards from what you have.}
When you are starting a proof, look at the premises; later, look at the sentences that you have derived so far. Think about the elimination rules for the main operators of these sentences. These will tell you what your options are. For example: If you have $A \eand B$ use \eand E to get $A$ and $B$ separately. If you have $A \eor B$ see if you can find the negation of either $A$ or $B$ and use \eor E.

\emph{Repeat as necessary.} Once you have decided how you might be able to get to the conclusion, ask what you might be able to do with the premises. Then consider the target sentences again and ask how you might reach them.  Remember, a long proof is formally just a number of short proofs linked together, so you can fill the gap by alternately working back from the conclusion and forward from the premises.

%Rob: I deleted a sentence from ``work forward from what you have'' that didn't make sense and then merged other material in the the ``repeat as necessary'' section.. 

\emph{Change what you are looking at.} Replacement rules can often make your life easier. If a proof seems impossible, try out some different substitutions.For example: It is often difficult to prove a disjunction using the basic rules. If you want to show $\script{A}\eor\script{B}$, it is often easier to show $\enot\script{A}\eif\script{B}$ and use the MC rule. Some replacement rules should become second nature. If you see a negated disjunction, for instance, you should immediately think of DeMorgan's rule.

\emph{When all else fails, try indirect proof.} If you cannot find a way to show something directly, try assuming its negation. Remember that most proofs can be done either indirectly or directly. One way might be easier---or perhaps one sparks your imagination more than the other---but either one is formally legitimate.

%Rob: I changed the way the advice is phrased to match the slogan I repeat in class.

%\emph{Persist.} Try different things. If one approach fails, then try something else.
% Rob: deleted ``persist'' in favor of other advice.

\emph{Take a break} If you are completely stuck, put down your pen and paper, get up from your computer, and do something completely different for a while. Walk the dog. Do the dishes. Take a shower. I find it especially helpful to do something physically active. Doing other desk work or watching TV doesn't have the same effect. When you come back to the problem, everything will seem clearer. Of course, if you are in a testing situation, taking a break to walk around might not be advisible. Instead, switch to another problem.

A lot of times, when you are stuck, your mind keeps trying the same solution again and again, even though you know it won't work. ``If I only knew $Q \eif R$,'' you say to yourself, ``it would all work. Why can't I derive $Q \eif R$!'' If you go away from a problem and then come back, you might not be as focused on That One Thing that you were sure you needed, and you can find a different approach.

	
%I added the section on taking a break


%%%%%%%Practice problems %%%%%%%%%

\practiceproblems
 
\noindent\problempart
 
Show the following theorems are valid. Feel free to use $T_{1}$ through $T_{17}$

\begin{enumerate}[label=(\arabic*)]
\item $ T_{\arabic{theorem}}$ (Associativity of \eand, Ass\eand): \rm $(\script{A} \eand \script{B}) \eand \script{C} \nsststile{}{} \hspace{.25em} \sststile{}{} \script{A} \eand (\script{B} \eand \script{C})$ \\ \addtocounter{theorem}{1}
\item $ T_{\arabic{theorem}}$  (Associativity of \eor, Ass\eor): \rm $(\script{A} \eor \script{B}) \eor \script{C} \nsststile{}{} \hspace{.25em} \sststile{}{} \script{A} \eor (\script{B} \eor \script{C})$ 	\\ \addtocounter{theorem}{1}
\item $ T_{\arabic{theorem}}$  (Associativity of \eiff, Ass\eiff): \rm $(\script{A} \eiff \script{B}) \eiff \script{C} \nsststile{}{} \hspace{.25em} \sststile{}{} \script{A} \eiff (\script{B} \eiff \script{C})$ 	\addtocounter{theorem}{1}
\end{enumerate}


% *******************************************
% *			Soundness and completeness			   *	
% *******************************************

%I merged sections 6.7, 6.8, and 6.9 and restricted the material to SL to create this section. 
\section{Soundness and completeness}
\label{sec:soundness_and_completeness}
 In section \ref{sec:taut-eq-incon}, we saw that we could use derivations to test for the same concepts we used truth tables to test for. Not only could we use derivations to prove that an argument is valid, 
we could also use them to test if a statement is a tautology, if a pair of statements are equivalent, or if a set of sentences are inconsistent. We also started using the single turnstile 
the same way we used the double turnstile. If we could prove that \script{A} was a tautology with a truth table, we wrote $\sdtstile{}{}\script{A}$, and if we could prove it using a 
derivation, we wrote $\sststile{}{}\script{A}.$

You may have wondered at that point if the two kinds of turnstiles always worked the same way. If you can show that \script{A} is a tautology using truth tables, can you also always show 
that it is true using a derivation? Is the reverse true? Are these things also true for tautologies and pairs of equivalent sentences? As it turns out, the answer to all these questions 
and many more like them is yes. We can show this by defining all these concepts separately and then proving them equivalent. That is, we imagine that we actually have two notions of 
validity, $valid_{\models}$ and $valid_{\vdash}$ and then show that the two concepts always work the same way.

\newglossaryentry{syntactic contradiction in SL}
{
name=syntactic contradiction in SL,
description={A statement in SL whose negation can be derived without any premises.}
}


   
\newglossaryentry{syntactically contingent in SL}
{
name=syntactically contingent in SL,
description={A property held by a statement in SL if and only if it is not a syntactic tautology or a syntactic contradiction.}
}




To begin with, we need to define all of our logical concepts separately for truth tables and derivations. A lot of this work has already been done. We handled all of the truth table definitions in Chapter \ref{chap:truth_tables}. We have also already given syntactic definitions for a tautologies and pairs of logically equivalent sentences. The other definitions follow naturally. For most logical properties we can devise a test using derivations, and those that we cannot test for directly can be defined in terms of the concepts that we can define.

For instance, we defined a syntactic tautology as a statement that can be derived without any premises (p. \pageref{def:syntactic_tautology_in_sl}). Since the negation of a contradiction is a tautology, we can define a \textsc{\gls{syntactic contradiction in SL}} \label{def:syntactic_contradiction_in_sl} as a sentence whose negation can be derived without any premises. The syntactic definition of a contingent sentence is a little different. We don't have any practical, finite method for proving that a sentence is contingent using derivations, the way we did using truth tables. So we have to content ourselves with defining ``contingent sentence'' negatively. A sentence is \textsc{\gls{syntactically contingent in SL}} \label{def:syntactically_contingent_in_sl} if it is not a syntactic tautology or contradiction. 
 
\newglossaryentry{syntactically inconsistent in SL}
{
name=syntactically inconsistent in SL,
description={A property held by sets of sentences in SL if and only if one can derive a contradiction from them.}
}

\newglossaryentry{syntactically consistent in SL}
{
name=syntactically consistent in SL,
description={A property held by sets of sentences in SL if and only if they are not syntactically inconsistent.}
}

A set of sentences is \textsc{\gls{syntactically inconsistent in SL}} \label{def:syntactically_inconsistent_ in_sl} if and only if one can derive a contradiction from them. Consistency, on the other hand, is like contingency, in that we do not have a practical finite method to test for it directly. So again, we have to define a term negatively. A set of set of sentences is \textsc{\gls{syntactically consistent in SL}} \label{def:syntactically consistent in SL} if and only if they are not syntactically inconsistent.
    
\newglossaryentry{syntactically valid in SL}
{
name=syntactically valid in SL,
description={A property held by arguments in SL if and only if there is a derivation that goes from the premises to the conclusion.}
}

Finally, an argument is \textsc{\gls{syntactically valid in SL}} \label{def:syntactically_valid_in_SL} if and only if there is a derivation of it. All of these definitions are given in Table \ref{table:truth_tables_or_derivations}.


\begin{sidewaystable}
\begin{mdframed}[style=mytablebox]
\tabulinesep=1ex
\begin{tabu}{X[.5,c,m] ||X[1,l,m] |X[1,l,m]}
\bf{Concept} 		&	\bf{Truth table (semantic) definition} 	&	\bf{Derivation (syntactic) definition} \\ \hline \hline

Tautology  &	A statement whose truth table only has Ts under the main connective & A statement that can be derived without any premises.	 \\ \hline
 
Contradiction		&	A statement whose truth table only has Fs under the main connective  &	A statement whose negation can be derived without any premises\\ \hline

Contingent sentence	&	A statement whose truth table contains both Ts and Fs under the main connective & A statement that is not a syntactic tautology or contradiction \\ \hline

Equivalent sentences &	The columns under the main connectives are identical.& The statements can be derived from each other	\\ \hline

Inconsistent sentences	&	Sentences which do not have a single line in their truth table where they are all true.	& Sentences which one can derive a contradiction from \\ \hline

Consistent sentences	&	Sentences which have at least one line in their truth table where they are all true. & Sentences which are no inconsistent	\\ \hline

Valid argument		&	An argument whose truth table has no lines where there are all Ts under main connectives for the premises and an F under the main connective for the conclusion.  & An argument where can derive the conclusion from the premises	\\ 
\end{tabu}
\end{mdframed}
\caption{Two ways to define logical concepts.}
\label{table:truth_tables_or_derivations}
\end{sidewaystable}

All of our concepts have now been defined both semantically and syntactically. How can we prove that these definitions always work the same way? A full proof here goes well beyond the scope of this book. However, we can sketch what it would be like. We will focus on showing the two notions of validity to be equivalent.  From that the other concepts will follow quickly. The proof will have to go in two directions. First we will have to show that things which are syntactically valid will also be semantically valid. In other words, everything that we can prove using derivations could also be proven using truth tables. Put symbolically, we want to show that $valid_{\vdash}$ implies $valid_{\models}$. Afterwards, we will need to show things in the other directions,  $valid_{\models}$ implies $valid_{\vdash}$

\newglossaryentry{soundness}
{
name=soundness,
description={A property held by logical systems if and only if $\sststile{}{}$ implies $\sdtstile{}{}$}
}

This argument from $\sststile{}{}$ to $\sdtstile{}{}$ is the problem of \textsc{\gls{soundness}}. \label{def:soundness} A proof system is \define{sound} if there are no derivations of arguments that can be shown invalid by truth tables. \label{def_Soundness} Demonstrating that the proof system is sound would require showing that \emph{any} possible proof is the proof of a valid argument. It would not be enough simply to succeed when trying to prove many valid arguments and to fail when trying to prove invalid ones.

The proof that we will sketch depends on the fact that we initially defined a sentence of SL using a recursive definition (see p. \pageref{def:recursive_definition}). We could have also used recursive definitions to define a proper proof in SL and a proper truth table. \nix{Later this will be a truth assignment}(Although we didn't.) If we had these definitions, we could then use a \emph{recursive proof} to show the soundness of SL. A recursive proof works the same way a recursive definition does.With the recursive definition, we identified a group of base elements that were stipulated to be examples of the thing we were trying to define. In the case of a well formed formula, the base class was the set of sentence letters A, B, C \ldots{}. We just announced that these were sentences. The second step of a recursive definition is to say that anything that is built up from your base class using certain rules also counts as an example of the thing you are defining. In the case of a definition of a sentence, the rules corresponded to the five sentential connectives (see p. \pageref{def:sentence_of_SL}). Once you have established a recursive definition, you can use that definition to show that all the members of the class you have defined have a certain property. You simply prove that the property is true of the members of the base class, and then you prove that the rules for extending the base class don't change the property. This is what it means to give a recursive proof.

Even though we don't have a recursive definition of a proof in SL, we can sketch how a recursive proof of the soundness of SL would go. Imagine a base class of one-line proofs, one for each of our eleven rules of inference. The members of this class would look like this $\{\script{A}, \script{B}\} \sststile{}{} \script{A} \eand \script{B}$; $\script{A} \eand \script{B} \sststile{}{}\script{A}$; $\{\script{A} \eor \script{B}, \enot\script{A}\} \sststile{}{} \script{B}$ \ldots{} etc. Since some rules have a couple different forms, we would have to have add some members to this base class, for instance $\script{A} \eand \script{B} \sststile{}{} \script{B}$ Notice that these are all statements in the metalanguage. The proof that SL is sound is not a part of SL, because SL does not have the power to talk about itself. 

You can use truth tables to prove to yourself that each of these one-line proofs in this base class is $valid_{\models}$. For instance the proof $\{\script{A}, \script{B}\} \sststile{}{} \script{A} \eand \script{B}$ corresponds to a truth table that shows $\{\script{A}, \script{B}\} \sdtstile{}{} \script{A} \eand \script{B}$ This establishes the first part of our recursive proof. 

The next step is to show that adding lines to any proof will never change a $valid_{\models}$ proof into an $invalid_{\models}$ one. We would need to this for each of our eleven basic rules of inference. So, for instance, for \eand{I} we need to show that for any proof $\script{A}_{1} \ldots{} \script{A}_{n} \sststile{}{} \script {B}$ adding a line where we use \eand{I} to infer $\script{C} \eand \script{D}$, where $\script{C} \eand \script{D}$ can be legitimately inferred from $\{\script{A}_{1} \ldots{} \script{A}_{n}, \script {B}\}$, would not change a valid proof into an invalid proof. But wait, if we can legitimately derive $\script{C} \eand \script{D}$ from these premises, then $\script{C} and \script{D}$ must be already available in the proof. They are either members of  $\{\script{A}_{1} \ldots{} \script{A}_{n}, \script {B}\}$ or can be legitimately derived from them. As such, any truth table line in which the premises are true must be a truth table line in which \script{C} and \script{D} are true. According to the characteristic truth table for \eand, this means that \script{C}\eand\script{D} is also true on that line. Therefore, \script{C}\eand\script{D} validly follows from the premises. This means that using the {\eand}E rule to extend a valid proof produces another valid proof.

In order to show that the proof system is sound, we would need to show this for the other inference rules. Since the derived rules are consequences of the basic rules, it would suffice to provide similar arguments for the 11 other basic rules. This tedious exercise falls beyond the scope of this book.

So we have shown that $\script{A} \sststile{}{} \script{B}$ implies $\script{A} \sdtstile{}{}\script{B}.$ What about the other direction, that is why think that \emph{every} argument that can be shown valid using truth tables can also be proven using a derivation. 

\newglossaryentry{completeness}
{
name=completeness,
description={A property held by logical systems if and only if $\sdtstile{}{}$ implies $\sststile{}{}$}
}

This is the problem of completeness. A proof system has the property of  \textsc{\gls{completeness}} \label{def:completeness} if and only if there is a derivation of every semantically valid argument. Proving that a system is complete is generally harder than proving that it is sound. Proving that a system is sound amounts to showing that all of the rules of your proof system work the way they are supposed to. Showing that a system is complete means showing that you have included \emph{all} the rules you need, that you haven't left any out. Showing this is beyond the scope of this book. The important point is that, happily, the proof system for SL is both sound and complete. This is not the case for all proof systems and all formal languages. Because it is true of SL, we can choose to give proofs or give truth tables---whichever is easier for the task at hand.

Now that we know that the truth table method is interchangeable with the method of derivation, you can chose which method you want to use for any given problem. Students often prefer to use truth tables, because a person can produce them purely mechanically, and that seems `easier'. However, we have already seen that truth tables become impossibly large after just a few sentence letters. On the other hand, there are a couple situations where using derivations simply isn't possible. We syntactically defined a contingent sentence as a sentence that couldn't be proven to be a tautology or a contradiction. There is no practical way to prove this kind of negative statement. We will never know if there isn't some proof out there that a statement is a contradiction and we just haven't found it yet. We have nothing to do in this situation but resort to truth tables. Similarly, we can use derivations to prove two sentences equivalent, but what if we want to prove that they are \emph{not} equivalent? We have no way of proving that we will never find the relevant proof. So we have to fall back on truth tables again.

Table \ref{table.ProofOrModel} summarizes when it is best to give proofs and when it is best to give truth tables. 

\begin{table}
\tabulinesep=1ex
\begin{mdframed}[style=mytablebox]
\begin{tabu}{X[.5,l,b] X[1,l,b] X[1,l,b]}
\underline{Property}		& \underline{To prove it present} 	&	\underline{To prove it absent} \\ 
Being a tautology 			& Derive the statement  						& Find the false line in the truth table for the sentence \\ 
Being a contradiction 		&  Derive the negation of the statement  		 & Find the true line in the truth table for the sentence\\ 
Contingency			 		& Find a false line and a true line in the truth table for the statement & Prove the statement or its negation\\ 
Equivalence 					& Derive each statement from the other 		 & Find a line in the truth tables for the statements where they have different values\\ 
Consistency	 				& Find a line in truth table for the sentence where they all are true & Derive a contradiction from the sentences\\ 
Validity		 				& Derive the conclusion form the premises & Find a line in the truth table where the premises are true and the conclusion false. \\ 
\end{tabu}
\end{mdframed}
\caption{When to provide a truth table and when to provide a proof.}
\label{table.ProofOrModel}
\end{table}



\practiceproblems
\noindent\problempart Use either a derivation or a truth table for each of the following. 
\begin{enumerate}[label=(\arabic*)]
\item Show that $A \eif [((B \eand C) \eor D) \eif A]$ is a tautology.
\item Show that $A \eif (A \eif B)$ is not a tautology
\item Show that the sentence $A \eif \enot{A}$ is not a contradiction.
\item Show that the sentence $A \eiff \enot A$ is a contradiction. 
\item Show that the sentence $ \enot (W \eif (J \eor J)) $ is contingent
\item Show that the sentence $ \enot(X \eor (Y \eor Z)) \eor (X \eor (Y \eor Z))$ is not contingent
\item Show that the sentence $B \eif \enot S$ is equivalent to the sentence $\enot \enot B \eif \enot S$
\item Show that the sentence $ \enot (X \eor O) $ is not equivalent to the sentence $X \eand O$
\item Show that the set $\{\enot(A \eor B), C, C \eif A\}$ is inconsistent.
\item Show that the set \{\enot(A \eor B), \enot{B}, B \eif A\} is consistent
\item Show that $\enot(A \eor (B \eor C)) $ \therefore $ \enot{C}$ is valid.
\item Show that $\enot(A \eand (B \eor C))$ \therefore $ \enot{C}$ is invalid. 
\end{enumerate}


\noindent\problempart Use either a derivation or a truth table for each of the following. 
\begin{enumerate}[label=(\arabic*)]
\item Show that $A \eif (B \eif A)$ is a tautology
\item Show that $\enot (((N \eiff Q) \eor Q) \eor N)$ is not a tautology
\item Show that $ Z \eor (\enot Z \eiff Z) $ is contingent
\item Show that $ (L \eiff ((N \eif N) \eif L)) \eor H $ is not contingent
\item Show that $ (A \eiff A) \eand (B \eand \enot B)$ is a contradiction
\item Show that $ (B \eiff (C \eor B)) $ is not a contradiction.
\item Show that $ ((\enot X \eiff X) \eor X) $ is equivalent to $X$
\item Show that $F \eand (K \eand R) $ is not equivalent to $ (F \eiff (K \eiff R)) $
\item Show that the set \{$ \enot (W \eif W)$, $(W \eiff W) \eand W$, $E \eor (W \eif \enot (E \eand W))$\} is inconsistent.
\item Show that the set  \{$\enot R \eor C $, $(C \eand R) \eif \not R$, $(\enot (R \eor R) \eif R) $\} is consistent.
\item Show that $\enot \enot (C \eiff \enot C), ((G \eor C) \eor G) \therefore ((G \eif C) \eand G) $ is valid.
\item Show that $ \enot \enot L,  (C \eif \enot L) \eif C) \therefore \enot C$ is invalid. 
\end{enumerate}


%\noindent\problempart
%Show that each of the following is provably inconsistent.
%\begin{earg}
%\item \{$Sa\eif Tm$, $Tm \eif Sa$, $Tm \eand \enot Sa$\}
%\end{earg}
%convert last item to something in SL

% % Below is the closing tag for typesetting only part of the chapter. Everything up to here to the close tag will be skipped unless the {whole_slproof_chap} label at the start of this 
%chapter file is % uncommented.

}{}


\section*{Rules of Inference Summary}
\label{sec:proof_rules}
%%%%%%%%%%%%%%%%%
%%%%%%%%%%rcr start proofrules
%%%%%%%%%%%%%%%%%%

Reiteration (R):
\begin{proof}
        \have[m]{a}{\script{A}}
        \have[n]{b}{\script{A}} \by{R}{a}
\end{proof}

Negation Introduction ({\enot}I):

\begin{multicols}{2}

\begin{proof}
\open
        \hypo[m]{na}{\script{A}}\by{for reductio}{}
        \have[n]{b}{\script{B}}
        \have{nb}{\enot\script{B}}
\close
\have{a}[\ ]{\enot\script{A}}\ni{na-nb}
\end{proof}

\begin{proof}
\open
        \hypo[m]{na}{\script{A}}\by{for reductio}{}
        \have[n]{b}{\enot\script{B}}
        \have{nb}{\script{B}}
\close
\have{a}[\ ]{\enot\script{A}}\ni{na-nb}
\end{proof}

\end{multicols}



Negation Elimination ({\enot}E):

\begin{multicols}{2}
\begin{proof}
\open
        \hypo[m]{na}{\enot\script{A}}\by{for reductio}{}
        \have[n]{b}{\script{B}}
        \have{nb}{\enot\script{B}}
\close
\have{a}[\ ]{\script{A}}\ne{na-nb}
\end{proof}


\begin{proof}
\open
        \hypo[m]{na}{\enot\script{A}}\by{for reductio}{}
        \have[n]{b}{\enot\script{B}}
        \have{nb}{\script{B}}
\close
\have{a}[\ ]{\script{A}}\ne{na-nb}

\end{proof}
\end{multicols}



Conjunction Introduction ({\eand}I):
 
\begin{multicols}{2}

\begin{proof}
        \have[m]{a}{\script{A}}
        \have[n]{b}{\script{B}}
        \have[\ ]{c}{\script{A}\eand\script{B}} \ai{a, b}
\end{proof}

\begin{proof}
        \have[m]{a}{\script{A}}
        \have[n]{b}{\script{B}}
        \have[\ ]{c}{\script{B}\eand\script{A}} \ai{a, b}
\end{proof}

\end{multicols}


Conjunction Elimination ({\eand}E):

\begin{multicols}{2}
\begin{proof}
        \have[m]{ab}{\script{A}\eand\script{B}}
        \have[\ ]{a}{\script{A}} \ae{ab}
\end{proof}

\begin{proof}
        \have[m]{ab}{\script{A}\eand\script{B}}
        \have[\ ]{a}{\script{B}} \ae{ab}
\end{proof}
\end{multicols}




Disjunction Introduction ({\eor}I):

\begin{multicols}{2}
 
\begin{proof}
        \have[m]{a}{\script{A}}
        \have[\ ]{ab}{\script{A}\eor\script{B}}\oi{a}
\end{proof}

\begin{proof}
        \have[m]{a}{\script{A}}
        \have[\ ]{ab}{\script{B}\eor\script{A}}\oi{a}
\end{proof}

\end{multicols}
  

Disjunction Elimination ({\eor}E):

\begin{multicols}{2}
\begin{proof}
        \have[m]{ab}{\script{A}\eor\script{B}}
        \have[n]{nb}{\enot\script{B}}
        \have[\ ]{a}{\script{A}} \oe{ab,nb}
\end{proof}

\begin{proof}
        \have[m]{ab}{\script{A}\eor\script{B}}
        \have[n]{na}{\enot\script{A}}
        \have[\ ]{b}{\script{B}} \oe{ab,nb}
\end{proof}
\end{multicols}


Conditional Introduction({\eif}I):

\begin{proof}
        \open
                \hypo[m]{a}{\script{A}} \by{want \script{B}}{}
                \have[n]{b}{\script{B}}
        \close
        \have[\ ]{ab}{\script{A}\eif\script{B}}\ci{a-b}
\end{proof}



Conditional Elimination ({\eif}E):

\begin{proof}
        \have[m]{ab}{\script{A}\eif\script{B}}
        \have[n]{a}{\script{A}}
        \have[\ ]{b}{\script{B}} \ce{ab,a}
\end{proof}


Biconditional Introduction({\eiff}I):

\begin{proof}
        \open
                \hypo[m]{a1}{\script{A}} \by{want \script{B}}{}
                \have[n]{b1}{\script{B}}
        \close
        \open
                \hypo[p]{b2}{\script{B}} \by{want \script{A}}{}
                \have[q]{a2}{\script{A}}
        \close
        \have[\ ]{ab}{\script{A}\eiff\script{B}}\bi{a1-b1,b2-a2}
\end{proof}



Biconditional Elimination ({\eiff}E):

\begin{multicols}{2}
\begin{proof}
        \have[m]{ab}{\script{A}\eiff\script{B}}
        \have[n]{a}{\script{A}}
        \have[\ ]{b}{\script{B}} \be{ab,a}
\end{proof}

\begin{proof}
        \have[m]{ab}{\script{A}\eiff\script{B}}
        \have[n]{a}{\script{B}}
        \have[\ ]{b}{\script{A}} \be{ab,a}
\end{proof}
\end{multicols}



%%%%%%%%end proofrules

\section*{Key Terms}
\begin{multicols}{2}
\begin{sortedlist}
\sortitem{sentence form}{}

\sortitem{substitution instance}{}

\sortitem{argument form}{}

\sortitem{substitution instance of an argument form}{}

\sortitem{proof}{}

\iflabelexists{def:syntactically_logically_equivalent_in_sl}{\sortitem{Syntactically logically equivalent in SL}{}}{}

\iflabelexists{def:syntactic_tautology_in_sl}{\sortitem{Syntactic tautology in SL}{}}{}

\iflabelexists{syntactic contradiction in SL}{\sortitem{Syntactic contradiction in SL}{}}{}

\iflabelexists{def:syntactically_contingent_in_sl}{\sortitem{Syntactically contingent in SL}{}}{}

\iflabelexists{def:syntactically_inconsistent_ in_sl}{\sortitem{Syntactically inconsistent in SL}{}}{}

\iflabelexists{def:syntactically consistent in SL}{\sortitem{Syntactically consistent in SL}{}}{}

\iflabelexists{def:syntactically_valid_in_SL}{\sortitem{Syntactically valid in SL}{}}{}

\iflabelexists{def:soundness}{\sortitem{Soundness}{}}{}

\iflabelexists{def:completeness}{\sortitem{Completeness}{}}{} 	

\end{sortedlist}
\end{multicols}







%Label for typesetting full chapter is at the start of the file. Uncomment to get the whole thing. 
%\part{Quantificational Logic} \label{part:quant_logic}
%\include{tex/ch09-predicate}
%\include{tex/ch10-semanticsforql}
%\include{tex/ch11-proofsinql}

%
%\part{Critical Thinking} \label{part:CT}	
%\include{tex/ch12-whatiscriticalthinking}
%\include{tex/ch13-substitutes}
%\include{tex/ch14-incompletearguments}
%\include{tex/ch15-emotions}
%\include{tex/ch16-generalizations}
%\include{tex/ch17-analogy}
%\include{tex/ch18-sources}
%\include{tex/ch19-maps}
%\include{tex/ch20-practicalarguments}

%\part{Inductive and Scientific Reasoning}  \label{part:inductive_scientific}
%\include{tex/ch21-whatareinductionandscientificreasoning}
%\include{tex/ch22-inductioninscience}
%\include{tex/ch23-mills-methods}
%\include{tex/ch24-causation-explanation}
%\include{tex/ch25-Analogy-in-Science}
%\include{tex/ch26-experimental-methods}
%\include{tex/ch27-Association-Diagrams-Cross-Tabulations}
%\include{tex/ch28-Explanation-Building}
%\include{tex/ch29-Problems-In-Induction}

\appendix
\iflabelexists{part:formal_logic}{\include{tex/z-app-notation}}{}
%\include{app-solutions}

%Bibstuff
%If the {part:CT} label is found, LaTeX will typeset separate bibliographies for sample passages and logical sources.

\iflabelexists{part:CT}{%text for CT version}

\defbibnote{sample}{\textit{ \large  This bibliography includes all sources except for those that were used as examples for logical analysis, either in the main text or problem sets}}

\printbibliography [keyword=samplepassage, title=Bibliography of Sample Passages, prenote=sample, heading=bibnumbered] %for separate bibs sample passages and general citations

\printbibliography [notkeyword=samplepassage, title=General Bibliography, heading=bibnumbered] %for separate bibs sample passages and general citations

}% End CT version
{\printbibliography[heading=bibnumbered]} %single bib for non-CT version


%%The way I’ve set this up now is that there is one bib for sample passages and one bib for everything else. This means that it would not be possible to put one entry in both bibliographies. (This might be needed for Aristotle.) To do that, you will need to define a separate logicsource category


\setglossarysection{chapter}
\printglossaries

\iflabelexists{part:formal_logic}{\include{tex/zz-quickreference}}{}
\include{tex/zzz-backmatter}	







\end{document}
