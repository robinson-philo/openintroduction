% Generated by GrindEQ Word-to-LaTeX 
\documentclass[oneside, openany]{book} %%% use \documentstyle for old LaTeX compilers

\usepackage[english]{babel} %%% 'french', 'german', 'spanish', 'danish', etc.
\usepackage{amssymb}
\usepackage{amsmath}
\usepackage{txfonts}
\usepackage{mathdots}
\usepackage[classicReIm]{kpfonts}
\usepackage[dvips]{graphicx} %%% use 'pdftex' instead of 'dvips' for PDF output
\usepackage{titlesec}
\titleformat{\chapter}[hang]{\bf\huge}{\thechapter}{2pc}{}


\usepackage{suffix}

\newcommand\chapterauthor[1]{\authortoc{#1}\printchapterauthor{#1}}
\WithSuffix\newcommand\chapterauthor*[1]{\printchapterauthor{#1}}

\makeatletter
\newcommand{\printchapterauthor}[1]{%
 {\parindent0pt\vspace*{-25pt}%
 \linespread{1.1}\large\scshape#1%
 \par\nobreak\vspace*{35pt}}
 \@afterheading%
}
\newcommand{\authortoc}[1]{%
 \addtocontents{toc}{\vskip-10pt}%
 \addtocontents{toc}{%
  \protect\contentsline{chapter}%
  {\hskip1.3em\mdseries\scshape\protect\scriptsize#1}{}{}}
 \addtocontents{toc}{\vskip5pt}%
}
\makeatother




%\usepackage{openintroduction}

%%% graphics packages %%%
\usepackage{tikz}
\usetikzlibrary{shapes,backgrounds,matrix,arrows,decorations,positioning,arrows.new}
\usepackage{graphicx}
\usepackage{xcolor}



% You can include more LaTeX packages here 


\begin{document}
\setlength{\parindent}{1em}
\setlength{\parskip}{1em}	

%\selectlanguage{english} %%% remove comment delimiter ('%') and select language if required
\titlepage
\vfill

\noindent{\huge AUTOBIOGRAPHY IN LOGIC}\\ \vspace{6pt}\\
\noindent\textit{A Multi-Step Assignment for Introductory Logic Courses}

\vfill
\noindent ALLYSON MOUNT 

\noindent with J. Robert Loftis
\vfill

\frontmatter


\noindent \includegraphics[width=66pt, height=23pt, keepaspectratio=true]{img/cc-by-nc-sa.png}\\ \vspace{6pt}

\noindent \textit{Autobiography In Logic} by Allyson Mount is licensed under a Creative Commons Attribution-NonCommercial-ShareAlike 4.0 \\ \vspace{6pt}

\noindent International License, except where otherwise noted.



\tableofcontents


%\begin{tabular}{|p{0.2in}|p{3.0in}|p{1.5in}|} \hline 
% & Note to Instructors\newline Allyson Mount & 1 \\ \hline 
% & Note to Students\newline Allyson Mount & iii \\ \hline 
%\newline 1. & \newline Biographical Information & \newline 4 \\ \hline 
% & Allyson Mount & \\ \hline 
%\newline 2. & \newline Interests & \newline 7 \\ \hline 
% & Allyson Mount & \\ \hline 
%\newline 3. & \newline Goal or Aspiration & \newline 10 \\ \hline 
% & Allyson Mount & \\ \hline 
%\newline 4. & \newline Fallacy & \newline 13 \\ \hline 
% & Allyson Mount & \\ \hline 
%\newline 5. & \newline A Challenge & \newline 16 \\ \hline 
% & Allyson Mount & \\ \hline 
%\newline 6. & \newline Academic Integrity and Permissions & \newline 19 \\ \hline 
% & Allyson Mount & \\ \hline 
%\newline 7. & \newline Final Project Checklist & \newline 20 \\ \hline 
% & Allyson Mount & \\ \hline 
%\newline 8. & \newline Grading Rubric & \newline 21 \\ \hline 
% & Allyson Mount & \\ \hline 
%\end{tabular}




\pagebreak

\mainmatter

\chapter{Note to Instructors}

\chapterauthor{Allyson Mount}

This assignment was designed to enhance understanding of core concepts in an introductory logic course. Each worksheet requires students to apply logical principles they learn in class to 
different aspects of their own lives. The worksheets are then collected into a booklet that provides information (mostly in symbolic form) about their background, goals, and interests. 
After being checked for accuracy, the booklets created by one class can be used as teaching tools in subsequent classes. This allows future students to practice interpreting propositional 
logic sentences, truth tables, proofs, and Venn diagrams as they begin to learn these skills themselves.

The aim of this approach is twofold. The most immediate goal is to bridge the gap between abstract principles and ``real-life'' applications, which students often have difficulty seeing 
even when many current examples are used in class. The second goal is to encourage students to see themselves as creators (not just consumers) of educational content while also giving 
them an opportunity to learn about issues that matter to their peers.

The assignment is designed to be customizable. Worksheets on different topics can be added or omitted to correspond to the logical skills covered in a course. Each worksheet (with 
associated instructions) is independent of the others, so they can be completed in any order. The first time through it may be useful to demonstrate in class how each worksheet could be 
filled out, but in future semesters the work of past students can be used to illustrate various ways of completing the assignment.

In my own course, I used this assignment in conjunction with two free OER texts:

\begin{itemize}

\item Craig DeLancy's \textit{A Concise Introduction to Logic }(Open SUNY Textbooks, 2017. Creative Commons Attribution Non-Commercial ShareAlike 4.0 International License), which can be 
found at http://textbooks.opensuny.org/concise-introduction-to-logic/. Part I covers the skills needed for the propositional logic pages of this assignment.

\item Matthew J. Van Cleave's text \textit{Introduction to Logic and Critical Thinking, }Version 1.4. (Creative Commons Attribution 4.0 International License), which can be found at 
https://www.oercommons.org/courses/introduction-to-logic-and-critical-thinking-2. Sections

\end{itemize}

\noindent 2.14-2.17 and Chapter 4 cover the skills needed for the categorical logic and fallacies pages of this assignment.
 
This assignment was designed with the support of the University System of New Hampshire's Academic Technology Institute in summer 2017.  My hope is that instructors who adapt this 
assignment and develop new sections will share their adaptations freely, expanding the range of possibilities available to others.


\chapter{Note to Students}

%\noindent ALLYSON MOUNT % %\noindent Along with our textbook curriculum and readings, you will be working on an ``autobiography'' that expresses events, interests, and features of your 
life in logical form. Each section of the autobiography corresponds to a different formal or informal logic skill that we will study. We will devote time in class to drafting each part in 
a workshop-style setting. You will then revise the worksheets on your own, print out a clean copy, and put them together in a booklet at the end of the term. % %\noindent % %\noindent 
Important things to know now: % %\noindent % %\noindent Save your work!!! You will need to keep your in-class drafts, complete them at home if necessary, and revise all of the sections to 
turn in on a clean copy of the booklet at the end of the term. % %\noindent % %\noindent If you are absent on a workshop day, it is your responsibility to find out what you missed and 
seek help in office hours as needed. % %\noindent % %\noindent The drafts will not be graded. However, productive use of the workshop time will make it much easier for you to produce a 
high-quality final version. % %\noindent % %\noindent If you keep up with the work and complete the section relevant to each topic throughout the term, putting together the final booklet 
should not be too time-consuming. If you delay work on it until the end, expect to find it very challenging and difficult to complete. % %\noindent % %\noindent This assignment is not 
designed to require anyone to divulge sensitive personal information. The topics are intended to encourage you to apply your logic skills to a topic you know well -- yourself! The hope is 
that you will have fun doing this, in a thoughtful way that is suitable for sharing publicly. If any part of this assignment raises privacy concerns for some reason, please let your 
instructor know. % %\noindent % %\noindent Grading: The final version of this assignment is worth 100 points (10\% of your overall course grade). The rubric that will be used to grade it 
can be found in a later chapter of this book. It would be wise to check the criteria for each section before you turn it in, to ensure that your project meets all of the requirements.

\chapter{Biographical Information}

\chapterauthor{Allyson Mount}

\noindent \textbf{Autobiographical subject matter: }This section will focus on basic background information about you.

\noindent

\noindent \textbf{Logical skills involved:}

\noindent

\noindent Representing factual claims in propositional logic

\noindent

\noindent Constructing a truth table for a complex claim

\noindent

\noindent Identifying which line of a truth table reflects the real-world situation

\noindent

\noindent

\noindent \textbf{Worksheet instructions: }Make a key that contains five biographical claims about yourself: three \textit{true }claims, and two \textit{false }claims. Each letter should 
stand for an atomic sentence of propositional logic. Do not indicate in the key which claims are true and which are false. Mix up the true and false claims in a random order, so that the 
reader can't tell from the key alone which are which. Make the false ones realistic and believable.

\noindent

\noindent Please use the given key letters. Include whatever types of biographical claims you consider important about your background, such as where you're from, when you were born, 
whether you have siblings, what kind of school you went to, etc. There are no specific facts that everyone must include -- so if you would rather not indicate your age, for instance, it 
is fine to focus on other biographical claims.

\noindent

\noindent Then, given your key, provide the information requested below. If you have done the key correctly, there will be multiple correct examples you could give for some of these 
parts. Just give \textit{one }example of each requested type.

\noindent

\noindent Again, please \textit{don't }indicate which two claims are the false ones on this worksheet itself, even though it can be deduced from your examples.

\noindent

\noindent

\noindent \textbf{Advice and things to keep in mind}

\noindent

\noindent

\noindent

\noindent Make sure each letter in your key stands for an atomic sentence -- not just a word or phrase, and not a negation, conditional, conjunction, etc.

\noindent Here and throughout the project, it is fine to use `I' in your key. Alternatively, you can refer to yourself by your name, although that might sound odd in an autobiography.

\noindent Show the work in your truth table. To be complete, it must include a column of values for each logical constant, not just the main column.\textbf{\textit{Biographical 
Information Worksheet}}

\noindent

\noindent \underbar{KEY}

\noindent

\noindent P =

\noindent

\noindent Q = R = S = T =

\noindent A \textit{negation }that expresses a true claim about me: \underbar{ }

\noindent

\noindent A \textit{conjunction }that expresses a true claim about me: \underbar{ }

\noindent

\noindent A \textit{conditional }that expresses a false claim about me: \underbar{ }

\noindent

\noindent The formula \textbf{($\boldsymbol{\mathrm{\neg}}$Q $\boldsymbol{\mathrm{\to}}$ (P $\boldsymbol{\mathrm{\wedge}}$ S)) }represents the following claim, stated in English:


\noindent Overall, this is a true / false (circle one) claim about me.


\noindent A complex formula involving \textit{three }letters and \textit{at least two }logical constants that represents a true claim about me: \underbar{ }


\noindent Truth table of that formula:

\noindent Which row of the table represents the actual, real-world situation with respect to the truth or falsity of the atomic sentences involved?  \underbar{ }

\noindent

\noindent

\noindent

\chapter{Interests} \chapterauthor{Allyson Mount}

\noindent \textbf{Autobiographical subject matter: }This section will focus on your extracurricular activities and interests. Choose a small number of non-academic activities that you 
enjoy pursuing in your free time. They can be hobbies, sports, outdoor activities, or other things of that nature. They do not need to be activities that correspond to an official club or 
organization, and they do not need to be related to college.

\noindent

\noindent \textbf{Logical skills involved:}

\noindent

\noindent Representing the relationship between groups using Venn diagrams

\noindent

\noindent Interpreting the claims represented by Venns

\noindent

\noindent

\noindent \textbf{Worksheet instructions:}

\noindent

\noindent For the 2-circle parts:

\noindent

\noindent Choose two different extracurricular activities, interests, or hobbies that you enjoy. Label the \textit{right- hand }circle in each pair with a term that applies to people who 
engage in that activity. Notice that the \textit{left }circle in each pair is already labeled, so all circles in the top section should be labeled.

\noindent

\noindent In the proposition below, fill in the blank with the same term as your right circle label. Then fill in the diagram to represent that proposition, starring or shading where 
needed.

\noindent

\noindent For the 3-circle part:

\noindent

\noindent Label two of the circles (it doesn't matter which) with the same two group terms as above. Then think of a third interest or activity you enjoy, and label the remaining circle 
with a term for people who engage in that activity.

\noindent

\noindent Given your labels and the diagram content that is already there, interpret the diagram and fill in the blanks below. Each answer should be a categorical propositions. Please 
write out the actual proposition in English (don't just write A, E, I, O).Using common sense and your actual knowledge of the world, indicate whether each of those categorical 
propositions is true or false.

\noindent

\noindent

\noindent \textbf{Advice and things to keep in mind}

\noindent

\noindent

\noindent

\noindent It is crucial that your circles are labeled correctly. Each circle should be labeled with a term that applies to people (including you!) who engage in that activity. For 
example, either ``stamp collectors'' or ``people who collect stamps'' is fine -- but ``stamps'' would not make sense in this context. (If the circle is labeled ``stamps,'' then just 
stamps themselves fall into the category represented.) Likewise, ``collecting stamps'' is an activity, not a group of people, so it wouldn't be an appropriate label for a 
circle.\includegraphics*[width=2.51in, height=1.59in, keepaspectratio=false]{image2}\textit{Interests Worksheet}

\noindent

\noindent

\noindent

\noindent \includegraphics*[width=2.51in, height=1.59in, keepaspectratio=false]{image3}

\noindent

\noindent

\noindent Some college student is a \underbar{ }

\noindent All \underbar{ }

\noindent are people who were born before 2016.

\noindent

\noindent

\noindent

\noindent \includegraphics*[width=2.44in, height=2.29in, keepaspectratio=false]{image4}

\noindent

\noindent

\noindent Proposition represented by the shaded region:

\noindent

\noindent

\noindent

\noindent

\noindent This claim is true/false (circle one).

\noindent

\noindent

\noindent

\noindent

\noindent Proposition represented by the starred region:

\noindent

\noindent

\noindent

\noindent

\noindent This claim is true/false (circle one).

\chapter{GOAL OR ASPIRATION} \chapterauthor{Allyson Mount}


\noindent \textbf{Autobiographical subject matter: }This section will focus on a future goal or aspiration. It can be an educational, professional, or personal goal. Choose a goal you 
have for at least a year or two in the future.

\noindent

\noindent \textbf{Logical skills involved:}

\noindent

\noindent Expressing conditional claims in propositional logic

\noindent

\noindent Using the conditional derivation rule in a proof

\noindent

\noindent

\noindent \textbf{Worksheet instructions: }State your goal as an atomic sentence. Choose a letter to represent it, and add it to the key.

\noindent

\noindent Think of an initial activity or condition that would help you start making progress toward that goal. Choose a letter to represent it, state it as an atomic sentence, and add it 
to the key.

\noindent

\noindent Write a chain of three conditionals that expresses a sequence of steps that you anticipate needing to take in order to accomplish the goal. Make a key and give the form of the 
three conditionals.

\noindent

\noindent The antecedent of the first should be the initial activity that you just identified above.

\noindent

\noindent The consequent of the first should match the antecedent of the second.

\noindent

\noindent The consequent of the second should match the antecedent of the third.

\noindent

\noindent The consequent of the third should be the goal that you stated above.

\noindent

\noindent

\noindent Making this chain of conditionals will require adding two more atomic sentences to your key. Then give the form of those three conditionals.

\noindent

\noindent Give a formal proof in which the three conditionals in the chain are premises. The conclusion of your proof should be a conditional stating that \textit{if }the initial 
activity/condition is met, \textit{then }your goal follows. Derive the conclusion using the conditional derivation rule. No additional premises should be needed.\textbf{Advice and things 
to keep in mind}

\noindent

\noindent

\noindent

\noindent You may find it easier to draft your chain of three conditionals in English first, then put them in symbolic form and fill out the key. Just the key, symbolic forms, and proof 
are on the worksheet, but be sure your chain of conditionals makes conceptual sense before formalizing it!

\noindent Of course, in life there are often many different steps one could take while aiming to accomplish a certain goal. There is no need to capture all the possibilities; just use the 
sequence of conditionals that seems most realistic, sensible, or promising to you.\textit{Goal or Aspiration Worksheet}

\noindent

\noindent

\noindent

\noindent

\noindent GOAL:

\noindent

\noindent

\noindent

\noindent

\noindent INITIAL ACTIVITY:

\noindent

\noindent

\noindent

\noindent

\noindent KEY:

\noindent

\noindent \underbar{ } = \underbar{ }

\noindent

\noindent \underbar{ } = \underbar{ }

\noindent

\noindent \underbar{ } = \underbar{ }

\noindent

\noindent \underbar{ } = \underbar{ }

\noindent

\noindent

\noindent

\noindent

\noindent CONDITIONAL FORMS:

\noindent

\begin{tabular}{|p{0.3in}|p{1.1in}|} \hline 1. & \underbar{ } \\ \hline 2. & \underbar{ } \\ \hline 3. & \underbar{ } \\ \hline \end{tabular}



\noindent

\noindent

\noindent

\noindent PROOF:

\noindent

\noindent

\noindent

\chapter{Fallacy} \chapterauthor{Allyson Mount}


\noindent \textbf{Autobiographical subject matter: }This section will focus on a fallacy or mistaken belief that has impacted your life in some way.

\noindent

\noindent \textbf{Logical skills involved:}

\noindent

\noindent Identifying and analyzing fallacies

\noindent

\noindent Writing a clear and compelling argument

\noindent

\noindent

\noindent \textbf{Worksheet instructions: }You can either focus on a mistaken belief that someone else had about you \underbar{or} a mistaken belief that you had about someone else. On 
the worksheet, state the belief \textit{as a single sentence}. It should be a sentence that you think is false, since that's what makes it a mistaken belief! Then check the appropriate 
box below.

\noindent

\noindent Write a paragraph explaining the reasoning behind the mistaken belief, from the perspective of someone who believes it. In other words, try to express how someone who has that 
belief might attempt to justify it. This paragraph should contain one of the formal or informal fallacies covered in assigned readings for this class, but the error in reasoning should 
not be so blatant that the argument sounds ridiculous. Remember, someone actually did believe it!

\noindent

\noindent Then write a paragraph identifying what the error in reasoning is, and how it most likely came about. Your paragraph should make it clear what fallacy is involved, and exactly 
where the reasoning in the prior paragraph goes wrong. Make sure to explain \textit{the specific problem with that particular argument, }rather than simply naming the fallacy and 
explaining the type of fallacy in general.

\noindent \textbf{Advice and things to keep in mind}

\noindent

\noindent

\noindent

\noindent Whichever option you choose, please avoid including personal information \textit{about anyone else }in a way that would make them uncomfortable. For instance, if a friend or 
teacher made an offensive assumption about you, or if you did so about them, please do not identify the particular person involved. The aim is to identify the fallacy and respond 
effectively to the mistake in reasoning, not to expose sensitive or potentially embarrassing information.

\noindent Likewise, keep in mind that your worksheet should be PG-rated and suitable for a public audience. If the mistaken belief arose based on assumptions that are sexist, racist, 
homophobic, or otherwise hateful, you can convey the serious nature of the issue without expressing it using obscene or degrading language yourself. Recall that responding effectively to 
fallacies requires articulating the source of the error in a way that may change people's minds, rather than simply making them defensive.\textit{Fallacy Worksheet}

\noindent

\noindent

\noindent

\noindent

\noindent MISTAKEN BELIEF:

\noindent

\noindent

\noindent

\noindent

\noindent Check one:

\noindent

\noindent $\mathrm{\lozenge}$ This is a belief that someone once had about me

\noindent

\noindent $\mathrm{\lozenge}$ This is a belief that I once had about someone else

\noindent 

\noindent 

\noindent 

\noindent 

\noindent THE ``ARGUMENT'' FOR THIS BELIEF:

\noindent 

\noindent 

\noindent 

\noindent 

\noindent 

\noindent 

\noindent 

\noindent 

\noindent 

\noindent 

\noindent 

\noindent 

\noindent 

\noindent 

\noindent 

\noindent 

\noindent 

\noindent WHY THAT REASONING IS FALLACIOUS:

\noindent 

\noindent 

\noindent 

\chapter{A CHALLENGE}
\chapterauthor{Allyson Mount}

\noindent \textbf{Autobiographical subject matter:  }This section will focus on a challenge you have faced, and your response to that challenge.

\noindent 

\noindent \textbf{Logical skills involved:}

\noindent 

\noindent  Creating a valid argument

\noindent 

\noindent  Demonstrating validity using appropriate logical methods

\noindent 

\noindent 

\noindent \textbf{Worksheet instructions: }Identify a challenge you have faced at some point in your life. It can be a major challenge or a relatively minor one, although it's best not to pick something too trivial because it will be less interesting. It can be recent, or from childhood. On the top of the worksheet, express the challenge itself in a single sentence, in English.

\noindent 

\noindent Think of a valid argument that expresses some aspect of what your \textit{response }was to that challenge. Your argument should have two or three premises, plus a conclusion. It can be a propositional logic argument or a categorical argument. Write out the argument. If it's in propositional logic, be sure to include a key stating what atomic sentence each letter stands for. If it's a categorical argument, make sure the premises and conclusion are all phrased as categorical propositions having A, E, I, or O form. Aim to make your argument sound too -- or at least reasonably plausible.

\noindent 

\noindent In the bottom box, demonstrate that the argument is valid. Select a method that makes sense given the content of your argument! Recall that you have learned at least three methods of demonstrating validity: truth tables, formal proofs, and Venn diagrams. Select a method that will work for the type of argument you wrote.

\noindent \textbf{Advice and things to keep in mind}

\noindent 

\noindent 

\noindent 

\noindent Thinking of a valid argument (about content you write yourself) can be difficult. For ideas of a valid structure, look back at the many examples we've seen throughout the term in the texts and in class. While the \textit{content }of your argument will need to be your own, validity is a matter of form. That's useful to know! Each of the proof rules has a valid form. There are many exercises in the texts that give valid forms for two-premise arguments. It is perfectly fine if you use a form that was already shown to be valid.

\noindent In choosing whether to use propositional or categorical logic, it is entirely a matter of preference. Use whichever you feel more comfortable with, or whichever seems to provide a more natural phrasing for your argument.

\noindent Please do stick to short arguments, with just two or three premises (no more, no less). If giving a categorical argument, it is okay to have three premises if you wish, even though we've focused on 2-premise syllogisms in class.\textit{A Challenge Worksheet}

\noindent 

\noindent 

\noindent 

\noindent 

\noindent THE CHALLENGE:

\noindent 

\noindent 

\noindent 

\noindent 

\noindent ARGUMENT ABOUT YOUR RESPONSE TO THE CHALLENGE:

\noindent 

\noindent 

\noindent 

\noindent 

\noindent 

\noindent 

\noindent 

\noindent 

\noindent 

\noindent 

\noindent 

\noindent 

\noindent 

\noindent 

\noindent 

\noindent DEMONSTRATION THAT THE ARGUMENT IS VALID:

\noindent 

\noindent 

\noindent 

%\noindent CHAPTER 6 .
%
%\noindent 
%
%\noindent 
%
%\noindent \textbf{ACADEMIC INTEGRITY AND PERMISSIONS}
%
%\noindent 
%
%\noindent 
%
%\noindent 
%
%\noindent 
%
%\noindent 
%
%\noindent 
%
%\noindent 
%
%\noindent 
%
%\noindent ALLYSON MOUNT 
%
%\noindent 
%
%\noindent 
%
%\noindent 
%
%\noindent 
%
%\noindent 
%
%\noindent \textbf{Please print your full name, write in the instructor and course information, and check one of the following boxes. Then sign and date the completed statement. Write in ink.}
%
%\noindent 
%
%\noindent 
%
%\noindent 
%
%\noindent 
%
%\noindent I, \underbar{                                            } certify that the materials in this \textit{Autobiography in Logic }booklet are my original work, which was completed for Professor \underbar{                  }'s course on \underbar{                                    } (course name/number) in the \underbar{              } term of the year 20\underbar{    }. I have checked a box below to indicate how my work may be used:
%
%\noindent 
%
%\noindent $\mathrm{\lozenge}$ I give permission for the instructor to use my work in this booklet in future classes or conferences, \textbf{\textit{with attribution of the work to me by name}}. I understand that s/he would only use it as a good example or to demonstrate various ways of completing the assignment correctly, not as an example of errors.
%
%\noindent 
%
%\noindent $\mathrm{\lozenge}$ I give permission for the instructor to use my work in this booklet in future classes or conferences \textbf{\textit{anonymously, without attribution of the work to me by name}}. I understand that s/ he would only use it as a good example or to demonstrate various ways of completing the assignment correctly, not as an example of errors.
%
%\noindent 
%
%\noindent $\mathrm{\lozenge}$ I do not give permission for the instructor to use my work in this booklet as an example in future classes or conferences.
%
%\noindent 
%
%\noindent 
%
%\noindent 
%
%\noindent Signature: \underbar{ }
%
%\noindent Date: \underbar{ }
%
%\noindent 
%
%\noindent 
%
%\noindent 
%
%\noindent CHAPTER 7 .
%
%\noindent 
%
%\noindent 
%
%\noindent \textbf{FINAL PROJECT CHECKLIST}
%
%\noindent 
%
%\noindent 
%
%\noindent 
%
%\noindent 
%
%\noindent 
%
%\noindent 
%
%\noindent 
%
%\noindent 
%
%\noindent ALLYSON MOUNT 
%
%\noindent 
%
%\noindent 
%
%\noindent 
%
%\noindent 
%
%\noindent 
%
%\noindent $\mathrm{\lozenge}$ Revise the five worksheets that you drafted in class, completing any parts that were not finished before.
%
%\noindent 
%
%\noindent $\mathrm{\lozenge}$ View the rubric (next chapter) to see how it will be graded. Make sure your worksheets meet all the criteria.
%
%\noindent 
%
%\noindent $\mathrm{\lozenge}$ Make and print a cover page. Feel free to use creative fonts or designs. Just make sure your cover page is easy to read and includes all of the following information: Your name, the course title, and the term date.
%
%\noindent 
%
%\noindent $\mathrm{\lozenge}$ Print out clean copies of the five Autobiography in Logic worksheet pages. Fill them out neatly, by hand. Please write in ink.
%
%\noindent 
%
%\noindent $\mathrm{\lozenge}$ Print, sign, and date the Academic Integrity and Permission page. It's totally up to you what permission box you check, but this signed page \underbar{must} be included in your booklet.
%
%\noindent 
%
%\noindent $\mathrm{\lozenge}$ You will not get the booklet back, so make a photocopy for yourself if you would like one.
%
%\noindent 
%
%\noindent $\mathrm{\lozenge}$  Arrange the pages in the following order: Cover page, Biographical Information worksheet, Interests worksheet, Goal or Aspiration worksheet, Fallacy worksheet, A Challenge worksheet, Permission page (signed)
%
%\noindent 
%
%\noindent $\mathrm{\lozenge}$ Staple the pages together with one staple in the upper left corner. The booklets will be collected in class.
%
%\noindent 
%
%\noindent 
%
%\noindent 
%
%\noindent CHAPTER 8 .
%
%\noindent 
%
%\noindent 
%
%\noindent \textbf{GRADING RUBRIC}
%
%\noindent 
%
%\noindent 
%
%\noindent 
%
%\noindent 
%
%\noindent 
%
%\noindent 
%
%\noindent 
%
%\noindent 
%
%\noindent ALLYSON MOUNT 
%
%\noindent 
%
%\noindent 
%
%\noindent 
%
%\noindent 
%
%\noindent 
%
%\noindent 
%
%\noindent 
%
%\noindent 
%
%\noindent \textbf{Score}
%
%\noindent 
%
%\noindent No: 0
%
%\noindent 
%
%\noindent Partly: 1-4
%
%\noindent 
%
%\noindent Yes: 5
%
%\noindent 
%
%\noindent 
%
%\noindent The cover page contains all of the required information
%
%\noindent 
%
%\noindent ``Biographical Information'' worksheet is complete and filled out coherently, following all instructions
%
%\noindent 
%
%\noindent ``Interests'' worksheet is complete and filled out coherently, following all instructions
%
%\noindent 
%
%\noindent ``Goal or Aspiration'' worksheet is compete and filled out coherently, following all instructions
%
%\noindent 
%
%\noindent ``Fallacy'' worksheet is complete and filled out coherently, following all instructions
%
%\noindent 
%
%\noindent ``A Challenge'' worksheet is complete and filled out coherently, following all instructions
%
%\noindent 
%
%\noindent Permissions page is filled out and signed
%
%\noindent 
%
%\noindent Booklet is neat, legible, and stapled
%
%\noindent \textbf{Score}
%
%\noindent 
%
%\noindent Beginning: 1-3
%
%\noindent 
%
%\noindent Developing: 4-7
%
%\noindent 
%
%\noindent Accomplished: 8-10
%
%\noindent 
%
%\noindent 
%
%\noindent Pages involving propositional logic have clear keys and symbolic formulas
%
%\noindent 
%
%\noindent 
%
%\noindent 
%
%\noindent 
%
%\noindent Pages involving propositional logic contain valid proof steps and correct truth tables
%
%\noindent 
%
%\noindent 
%
%\noindent 
%
%\noindent Pages involving Venn diagrams are filled out clearly and accurately (circles labeled appropriately, statements in categorical proposition form, Venns done correctly, etc.)
%
%\noindent 
%
%\noindent 
%
%\noindent 
%
%\noindent Fallacy page is well-written, with a fallacy identified accurately and explained clearly
%
%\noindent 
%
%\noindent 
%
%\noindent 
%
%\noindent Overall, the booklet demonstrates an understanding of the logical principles involved and how to use them appropriately
%
%\noindent 
%
%\noindent 
%
%\noindent 
%
%\noindent Overall, the booklet provides the required information in a thoughtful, creative, and original way
%
%\noindent 
%
%\noindent 
%
%\noindent 
%
%\noindent 
%
%\noindent \textbf{TOTAL SCORE (Out of 100) =}
%

\end{document}

